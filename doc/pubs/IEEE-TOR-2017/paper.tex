\documentclass{itor}

\usepackage{natbib}

\usepackage{amsfonts}
\usepackage{amssymb,amsmath,amsthm}
\usepackage{graphicx}
\usepackage{bm}           % bold math symbols (\boldsymbol)
\usepackage{enumerate}   % para trocar forma de enumeracao dos itens

\usepackage{booktabs}
\usepackage{multirow}

\usepackage{pstricks}
\usepackage{pstricks-add}
%\usepackage[breaklinks=true]{hyperref}
%\usepackage{breakcites}
\usepackage{microtype}    % melhorias no micro-espaçamento de texto
%\usepackage{minibox}   % para caixa de texto com borda
\usepackage{framed}

\usepackage{pgfplots} % graficos
\usepackage{subcaption}

\usepackage{algorithm}
\usepackage{algpseudocode}

%%%   PREDEFINITIONS   %%%%%%%%%%%%%%%%%%%%%%%%%%%%%%%%%%%%%%%%%%%%%%%%%%%%%%%%%
\newcommand{\dtree}[1]{$#1$-d~tree}
\newcommand{\kdtree}{\dtree{k}}
\newcommand{\bsym}[1]{\boldsymbol{#1}}
\newcommand{\sol}[1]{\boldsymbol{#1}}
\newcommand{\pnt}[1]{pnt(\sol{#1})}
\newcommand{\fsol}[1]{f(\sol{#1})}
\newcommand{\np}{m}
\newcommand{\nphard}{$\mathcal{NP}$-Hard}
\newcommand{\missingI}[1]{}
\newcommand{\missing}[1]{
  \begin{framed}
    {\scriptsize \bf #1}
  \end{framed}
}
\newcommand{\weight}[1]{w(\sol{#1})}
\newcommand{\obj}[2]{f_{#1}(\sol{#2})}
\newcommand{\bigweight}[1]{w\big(\sol{#1}\big)}
\newcommand{\dom}[2]{dom(\sol{#1}, \sol{#2})}
\newcommand{\domk}[2]{dom_k(\sol{#1}, \sol{#2})}
\newcommand{\setIN}{\{1, \ldots, n\}}
\newcommand{\ext}[2]{ext(\sol{#1}, \sol{#2})}
\newcommand{\domLess}[2]{ \sol{#1} \prec \sol{#2} }
%\newcommand{\logicAnd}{ \textrm{ and } }
\newcommand{\logicAnd}{ \land }
\newcommand{\logicOr}{ \lor}
\newcommand{\solSetA}{ Q }
\newcommand{\solSetB}{ R }
\newcommand{\solSett}{ S_* }
\newcommand{\solSet}{ S }
\newcommand{\ord}{\mathcal{O}}
\newcommand{\rord}{\mathcal{O}_{rev}}
\newcommand{\cb}[2]{cb^{#1}(#2)}  % cost-benefit function
\renewcommand{\leq}{\leqslant}
\renewcommand{\geq}{\geqslant}
\newcommand{\floor}[1]{\left \lfloor{#1}\right \rfloor}

% bar graphs configs
\newcommand{\cmpH}{4.0cm}
\newcommand{\cmpW}{7cm}
\newcommand{\legX}{0.45}
\newcommand{\legY}{-0.30}

\newtheorem{theorem}{Theorem}

% Metadata Information
\jname{International Transactions in Operational Research}% no need to specify for ITOR
\jvol{XX}
\jyear{20XX}
\doi{xx.xxxx/itor.xxxxx}

\begin{document}

\title{ Multi-dimensional indexing on dynamic programming for multi-objective knapsack problem }

\author[M. D. V. Baroni and F. M. Varej\~ao]{Marcos Daniel Valad\~ao Baroni\affmark{a,$\ast$} and Fl\'avio Miguel Varej\~ao\affmark{a}}

\affil{\affmark{a}Departamento de Inform\'atica/Universidade Federal do Esp\'irito Santo (UFES), Av. Fernando Ferrari, 514, Vit\'oria - ES, Brazil}
\email{marcos.baroni@aluno.ufes.br [Baroni]; fvarejao@inf.ufes.br [Varej\~ao]}

\thanks{\affmark{$\ast$}Research supported by Funda\c c\~ao de Amparo \`a Pesquisa do Esp\'irito Santo.}

% Date
\historydate{Received DD MMMM YYYY; received in revised form DD MMMM YYYY; accepted DD MMMM YYYY}


%%%   OUTLINE   %%%%%%%%%%%%%%%%%%%%%%%%%%%%%%%%%%%%%%%%%%%%%%%%%%%%%%%%%%%%%%%%
% 1. Introduction
%    - The problem and its application
%    - Literature background
%    - The Bazgan algorithm
%    - Our approach
%    - Article structure
% 2. The MOKP
%    - (some definition needed)
%    - Multi-objective problems
% 3. Dyn. Programming Alg. for MOKP
%    - Process overview of DP for MKP
%    - Dom. relations
%    - Application of dom. rel.
% 4. Use of KDTree
%    - (motiuvation: ops. on the algorithms)
%    - The KDTree data structure (...performance issues/dimensions)
%    - Operations
% 5. Experiments
%    - Instances
%    - Results
% 6. Conclusions
%    - (could not achieve bazgan times)


%%%   STRUCTURE %%%%%%%%%%%%%%%%%%%%%%%%%%%%%%%%%%%%%%%%%%%%%%%%%%%%%%%%%%%%%%%%

\begin{abstract}


\begin{resumo}
Diversos problemas reais envolvem a otimização simultânea de múltiplos critérios,
os quais são, geralmente, conflitantes entre si.
Estes problemas são denominados multiobjetivo e
não possuem uma única solução, mas um conjunto de soluções de interesse, denominadas soluções
eficientes ou não dominadas.
Um dos grande desafios a serem enfrentados na resolução deste tipo de problema é o
tamanho do conjunto solução, que tende a crescer rapidamente dado o tamanho da instância,
degradando a performance dos algoritmos.
Dentre os problemas multiobjetivos mais estudados está o problema da mochila multiobjetivo,
pelo qual diversos problemas reais podem ser modelados.
Este trabalho propõe a aceleração do processo de solução do problema da mochila multiobjetivo,
através da utilizando da \kdtree{} como estrutura de indexação multidimensional
para auxiliar a manipulação das soluções.
A performance da abordagem é analisada através de experimentos
computacionais, realizados no contexto exato utilizando um algoritmo estado da arte.
Testes também são realizados no contexto heurístico, utilizando a adaptação
de uma meta-heurística para o problema em questão, sendo esta também uma contribuição do presente trabalho.
Segundo os resultados, para o contexto exato a proposta foi eficaz, apresentam speedup de até $2.3$
para casos bi-objetivo e $15.5$ em casos 3-objetivo, não sendo porém
eficaz no contexto heurístico, apresentando pouco impacto no tempo computacional.
Em todos os casos, porém, houve considerável redução no número de avaliações de soluções.

\vspace{\onelineskip}

\noindent
\textbf{Palavras Chave}:
Problema da Mochila Multiobjetivo,
Indexação Multidimensional,
Meta-heurística,
Algoritmo Exato.
\end{resumo}


\begin{resumo}[Abstract]
\begin{otherlanguage*}{english}
Several real problems involve the simultaneous optimization of multiple criteria,
which are generally conflicting with each other.
These problems are called multiobjective and
do not have a single solution, but a set of solutions of interest, called efficient solutions
or non-dominated solutions.
One of the great challenges to be faced in solving this type of problem is the
size of the solution set, which tends to grow rapidly given the size of the instance,
degrading algorithms performance.
Among the most studied multiobjective problems is the multiobjective knapsack problem,
by which several real problems can be modeled.
This work proposes the acceleration of the resolution process of the multiobjective knapsack problem,
through the use of a \kdtree {} as a multidimensional index structure
to assist the manipulation of solutions.
The performance of the approach is analyzed through computational experiments,
performed in the exact context using a state-of-the-art algorithm.
Tests are also performed in the heuristic context, using the adaptation
of a meta-heuristic for the problem in question, being also a contribution of the present work.
According to the results, the proposal was effective for the exact context, presenting a speedup up to $2.3$
for bi-objective cases and $15.5$ for 3-objective cases, but not
effective in the heuristic context, presenting little impact on computational time.
In all cases, however, there was a considerable reduction in the number of solutions evaluations.
\vspace{\onelineskip}

\noindent

\textbf{Keywords}:
Multiobjective Knapsack Problem,
Multidimensional Indexing,
Metaheuristic,
Exact Algorithm.
\end{otherlanguage*}
\end{resumo}

% Falar brevemente sobre o MOKP
% Falar brevemente sore a estratégia de indexação
% Resumir os resultados obtidos

%\missingt{
%Observar as 5 regras:\\
%1. A general statement introducing the broad research area of the particular topic being investigated;\\
%2. An explanation of the specific problem (difficulty, obstacle, challenge) to be solved;\\
%3. A review of existing or standard solutions to this problem and their limitations;\\
%4. An outline of the proposed new solution;\\
%5. A summary of how the solution was evaluated and what the outcomes of the evaluation were.
%}

% 1. Uma declaração geral introduzindo a área de pesquisa;
% 2. Uma explicação espeficica do problema
% 3.
%
%
\end{abstract}

\maketitle

\section{Introduction}
\label{sec:intro}
% THE MOKP Problem
%  - The problem and its application
%  - Literature background
%  - The Bazgan algorithm
%  - Our approach (motivation and use of KDTRee)
%  - Article structure


\section{The Multiobjective Knapsack Problem}
\label{sec:mokp}
% Breve definicao de multiobjective opt
A general multiobjective optimization problem can be described as a vector
function $f$ that maps a tuple of $n$ parameters (decision variables) to a tuple
of $\np$ objectives.
Formally:
\begin{align*}
  \text{min/max} ~ \sol{y} &= f(\sol{x}) = 
    \big(f_1(\sol{x})
    ,f_2(\sol{x})
    ,\ldots
    ,f_{\np}(\sol{x})\big) \\
  \text{subject to} ~ \sol{x} & = (x_1, x_2, \ldots, x_n) \in X
\end{align*}
where $\sol{x}$ is called the \emph{decision vector} or \emph{solution}, $X$ denotes the set
of feasible solutions, and $\sol{y}$ is the \emph{objective vector} or \emph{criterion vector} where
each objective has to be minimized (or maximized).

Considering two decision vectors $\sol{a}, \sol{b} \in X$, $a$ is said to
\emph{dominate} $b$ if, and only if:
\begin{align*}
    \forall i &\in \{1, 2, \ldots, \np\}: f_i(\sol{a}) \geq f_i(\sol{b}) \\
    \exists j &\in \{1, 2, \ldots, \np\}: f_j(\sol{a}) > f_j(\sol{b})
\end{align*}

A solution $\sol{a} \in X$ is called \emph{efficient} or \emph{non-dominated}
if there is not other feasible solution $\sol{b} \in X$ such that $\sol{b}$ dominates $\sol{a}$.
The set of solutions of a multiobjective optimization problem consists of all efficient solutions.
This set is known as \emph{Pareto optimal}.

The instance of a multiobjective knapsack problem with $\np$
objectives consists of an integer capacity $W > 0$ and $n$ items.
Each item $i$ has a positive weight $w^i$ and $\np$ non negative integer
profits $p_{i}^{1}, \ldots, p_{i}^{\np}$.
A solution is represented by a vector $\sol{x} = (x_1, \ldots, x_n)$ of binary
decision variables $x_i$, such that $x_i = 1$ if item $i$ is included in the
solution and $0$ otherwise, satisfing the capacity of the knapsack.
For any instance of the problem, we aim at determining the set of efficient solutions.

Formally the definition of the problem is:
\begin{align*}
    \text{max   } & f(\sol{x}) = 
      \big(f_1(\sol{x}) ,f_2(\sol{x}) ,\ldots ,f_{\np}(\sol{x})\big) \\
    \text{subject to   } & w(\sol{x}) < W \\
    & x_{i} \in \{0, 1\} \quad i = 1, \ldots, n \\
    \text{where} \phantom{mmmmm} \\
    f_j(\sol{x}) &= \sum_{i=1}^{n} v^j_i x_i \quad j = 1, \ldots, \np \\
    w(\sol{x}) &= \sum_{i=1}^{n} w_i x_i
\end{align*}

The MOKP is considered a \nphard{} problem, since it is a generalization
of the well-known 0$-$1 knapsack problem and
it is quite difficult to determine the Pareto optimal set for the MOKP,
especially for high dimension instances, in which the Pareto set it self tends
to grow exponentially.
For this reason, the development of methods efficiently deal with high-dimension
instances is...

% Definições, Propriedados e teoremas para o MOKP: dominancia



\section{The Dynamic Programming algorithm}
\label{sec:dynprog}
% - Process overview of DP for MKP
% - Dom. relations
% - Application of dom. rel.


% Basic simple Dynamic Programming
% Nemhauser-Ullmann algorithm



%\section{Indexing Multidi-mensional data with \kdtree{}}
\section{Indexed dominance checking}
\label{sec:kdtree}
- Introdução à KDTree: estrutura, inserção, busca \\
- Aplicação da KDTree ao algoritmo: validade das propriedades de dominancia \\


\section{Computational experiments}
\label{sec:exp}
- Base de dados utilizaca \\
- Parametros dos algoritmos \\
- Análise dos resultados (comparação)


\section{Conclusions and future remarks}
\label{sec:conc}
This paper shows
the application of a multi-dimensional indexing structure
for exactly solving the MOKP with a dynamic programming algorithm
and investigating its efficiency.
Through computational experiments we showed that multi-dimensional indexing
is applicable to the problem requiring considerably less solution evaluations,
especially on hard instances,
which resulted in a algorithm speedup $2.3$ for bi-dimensional
cases and up to $15.5$ on 3-dimensional cases.
The multi-dimensional indexing was not efficient on instances for which the set of solutions
is relatively small.
%Unfortunately there was a discrepancy in execution times between our
%implementation and one reported on the original proposal paper which cause could not
%be discovered.
%The original authors was 

A promising line of future research is to investigate the performance
of multi-dimensional indexing on heuristic and approximate approaches
for the MOKP as well as others multi-objective problems
with the same requirement of handling a high number of multi-dimensional
intermediary states.

%- Conclusões dos resultados \\
%  - concluiu se que a kdtree é uma estrutura de dados indicada para a indexação de soluções deste tipo de problema\\
%  - a indexacao for muito eficiente, reduzindo o tempo necessário pelo algoritmo\\
%  - indexacao pode ser aplicados a outros problemas \\
%- Trabalhos futuros \\

\bibliographystyle{plain}
\bibliography{src/refs}

\end{document}