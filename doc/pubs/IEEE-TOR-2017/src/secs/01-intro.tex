Many real applications like project selection,
capital budgeting and cutting stock involves
optimizing multiple objectives that are usually conflicting.
Some of those problems can be modelled as a multi-objective
knapsack problem.
Several exact approaches have been proposed in the literature to solve
the multi-objective knapsack problem.
Examples of such approaches are
a $\varepsilon$-constraint method presented in \cite{chankong2008multiobjective},
a branch and bound algorithm  \cite{visee1998two},
a labeling algorithm  \cite{captivo2003solving} and
a dynamic programming algorithm proposed in \cite{bazgan2009}
which is the approach currently achieving the best results.
Some later contributions for the dynamic programming approach are
an algorithmic improvement for the bi-objective
case~\cite{figueira2013algorithmic} and some techniques
for reducing its memory usage~\cite{correia2018}.

One of the main difficulties on multi-objective optimization problems
is the large cardinality of the set of non-dominated (or \emph{efficient}) solutions,
which has motivated research to provide an approximation of the solution
set~\cite{bazgan2015approximate, vanderpooten2017covers}.
Indeed, it is well-known, in particular, that most multi-objective combinatorial
optimization problems are \emph{intractable}, in the sense that
the number of non-dominated points is exponential in the size of the instance
\cite{ehrgott2013multicriteria}.

It is also well known that the use of a proper data structure
usually has considerable impact on the efficiency of an algorithm,
especially when dealing with large amount of data.
In this work we propose a performance improvement
on the state of art dynamic programming algorithm for
the multi-objective knapsack problem
by applying a multi-dimensional indexing strategy
for handling the large amount of intermediate solutions.

Section \ref{sec:mokp} of this paper
reviews the basic concepts of multi-objective optimization
and defines the multi-objective knapsack problem.
Section~\ref{sec:dynprog} introduces the dynamic programming algorithm.
Section~\ref{sec:kdtree} presents the proposal of multi-dimensional
indexing by using a \kdtree{}.
Section~\ref{sec:exp} analyzes the approach through computational experiments and
Section~\ref{sec:conc} provides conclusions and lines for future research.

%  - The problem and its application
%  - Literature background
%  - Resumo da proposta
