% Breve resumo sobre métodos de programação dynamica
% Explicação overview do método para o MOKP
% - Process overview of DP for MKP
% - Dom. relations
% - Application of dom. rel.

The algorithm addressed in this work
can be seen as a MOKP specialization of the classical Nemhauser
and Ullmann's algorithm proposed in~\cite{nemhauser1969discrete}
for generically solving knapsack problems.
The basic algorithm will be presented in Section~\ref{subsec:nu}.
Three optimizations for reducing the number of partial solutions will then be presented: reordering of items (\ref{subsec:order}), 
the avoidance of deficient solutions (\ref{subsec:deficient}) and
the elimination of unpromising partial solutions (\ref{subsec:unpromissing}).

%uses the concept of knapsack dominance to remove solutions
%that will not lead to efficient ones.

\subsection{The Nemhauser-Ullmann algorithm}
\label{subsec:nu}

The Nemhauser and Ullmann's algorithm generically solves knapsack problems
by applying the concept of knapsack dominance to remove partial solutions that will
not generate efficient ones.
A basic multi-objective version of the algorithm is presented by Algorithm~\ref{alg:nemull}.

\begin{algorithm}
  \caption{Nemhauser and Ullmann's algorithm for MOKP}
  \label{alg:nemull}
  \Kw{$\bsym{p}, \bsym{w}, W$}
\Begin{
  \SetAlgoLined
  $S_0 = \{(0, \ldots, 0)\}$\;
  \For{$k \gets 1, n$}{
    $S_k \gets S_{k-1} \cup \big\{(s^1 + p_k^1, \ldots, s^\np + p_k^\np, s^{\np+1} + w_k)\;$\
    $\phantom{S_k \gets S_{k-1} \cup \;} \big| \; s^{\np+1} + w_k \leq W, \: s \in S_{k-1} \big\}$\;
  }
  $P = \{ s \in S_n \:|\: \nexists (a \in S_n \;|\; a \dom s)\}$\;
  \textbf{return} $P$\;
}

\end{algorithm}

%%% Explicação do algoritmo básico
The algorithm begins by defining an initial solution set $S^0$ containing only
the empty solution (line 2).
At a $k$-th stage the algorithm receives a set $S^{k-1}$
exclusively containing solutions composed by the first ${k-1}$ items, i.e.\,,
$\forall\sol{x} \in S^{k-1}, \sol{x} \subseteq \{1, \ldots, k-1\}$.
The set $S^{k-1}$ is then expanded by adding a copy of its solutions
but now including the $k$-th item (line 4).
This new set is defined by $S^k_*$ having twice the cardinality of $S^{k-1}$.
Set $S^k$ is then defined by selecting from $S^k_*$ all solutions
that are not knapsack dominated by any other existing solution (line 5).
The last step of the algorithm (line 7) consists of selecting the
efficient solutions, i.e., that are not dominated by any other partial solution
according to their objective values.
Any partial solution in the context of stages prior than $n$-th stage
will be considered \emph{efficient} if it is not knapsack dominated by any other
partial solution.

Regarding its simplicity Algorithm~\ref{alg:nemull} is considerably powerful.
However the exponential growth potencial of MOKP solutions sets
may severely compromise its performance.
One way of tackling this is reducing the number of partial solutions
handled through the algorithm stages.
%either by generating the least
%possible number of solution or removing those that are not promising.
Following, three optimizations for achieving this  reduction, proposed in the
Bazgan's algorithm  \cite{bazgan2009}, are described.

%\subsection{Item order}
\subsection{Input order of items}
\label{subsec:order}
%%% Motivação para considerar ordenação
The first important issue to be considered is the order
in which the items are introduced in the algorithm.
It is well-known that good knapsack problem solutions
are generally composed of items with the best cost-benefit rate.
Therefore, the prioritization of those items tends to lead to better solutions.

%%% Problema de ordenação para Multi Objective
An appropriate cost-benefit rate for the single objective case may be directly derived
from the profit/weight ratio of the items.
For the multi-objective case however there is no such natural measure.
Thus, one may mainly consider item orders defined over the aggregation of their ranks
derived from cost-benefit from all objectives.

%%% Definição de ordenação
We denote $\ord^j$ as the set of items ordered by descending order of profit/weight
ratio regarding objective $j$.
Formally:
\begin{align*}
  \ord^j &= (o^j_1, \ldots, o^j_n) , \qquad \cb{j}{o^j_1} \geq \ldots \geq \cb{j}{o^j_n}
\end{align*}
where function $\cb{j}{a} = {p^j_a}/{w_a}$ is the
cost-benefit function of item $a$ regarding objective $j$.
Let $r^j_i$ be the rank of item $i$ in order $\ord^j$ and
$\ord^{sum}$, $\ord^{min}$ and $\ord^{max}$ denotes orders according to increasing
values of the sum, minimum and maximum ranks respectively.
Formally:
\begin{align*}
  r^j_i &= \max\{ k \;|\; o^j_k = i \} \\
  r^{sum}_i &= \textstyle\sum^\np_{j=1} r^j_i \\
  \ord^{sum} &= (o_1, \ldots, o_n) , \qquad r^{sum}_{o_1} \leq \ldots \leq r^{sum}_{o_n}
\end{align*}
The orders $\ord^{min}$ and $\ord^{max}$ are conceived in the same manner as
$\ord^{sum}$, except for the fact that $\frac{r^{sum}_i}{\np}$ is added up
in their ranks as tie breaking criteria.
The notation $\ord(\sol{s})$ will be used to denote order $\ord$
over a restrict set $\sol{s}$ of items and $\ord_{rev}$ to denote the reverse
order of $\ord$.

The order of interest is the one that generates the smallest partial solutions sets.
According to literature, experimental tests have reported $\ord^{max}$
superior to others for inputting items on the algorithm and was adopted
in this work as the cost-benefit order of choice.

\subsection{Deficient solutions avoidance}
\label{subsec:deficient}
At $k$-th stage of the algorithm
a copy of all previous solutions is strictly added to the
new solution set without adding the $k$-th item (line 4).
However preserving solutions with too much available capacity
may generate unnecessary deficient solutions.
%These partial solutions most be detected and discarded.
If a partial solution $\sol{x} \in S^{k-1}$ has enough
space to fit all remaining items, i.e.\,, $\weight{x} + \sum_{i=k}^n w_i \leq W$,
$\sol{x}$ may be discarded and only $\sol{x} \cup \{k\}$ kept, once
keeping $\sol{x}$ will certainly lead to deficient solutions.
It is worth noting that this optimization only regards the available capacities of solutions
and the weights of remaining items.

%\begin{theorem}
%   Considering the $k$-th stage of the algorithm and $\sol{x} \in S^{k-1}$.
%   If $\weight{x} + \sum_{i \in {\{k, \ldots, n\}}} w_i \leq W$ than
%\end{theorem}

\subsection{Elimination of unpromising partial solutions}
\label{subsec:unpromissing}
Another way of reducing the amount of partial solutions
generated is by analyzing their potential to generate an efficient solution
regarding their available capacity, their current quality and the set of remaining items.
This may be done by computing an upper bound for their objective functions
and a set of lower bounds values over the current set of partial solutions.
If the upper bound of a solution is dominated by any known lower bound
this partial solution can be safely discarded.

Lower bound values of partial solutions can be computed by greedly filling
their available capacity with remaining items.
As in this case the order in which the items are inserted is relevant, 
items must be prioritized according to order $\ord^{max}$.
Given a partial solution $\sol{x}$ and the set $\sol{s}$ of remaining items,
the lower bound function $lb(\sol{x}, \sol{s})$ can be defined as:
\begin{align*}
    lb(\sol{x}, \sol{s}) &= \obj{}{x} + \obj{}{y} \\
  \text{where} \phantom{mmmmm} \\
    \sol{y} &= \left\{ o_i \;\middle|\; \textstyle\sum_{j=1}^i w_{o_j} \leq W - \weight{x} \right\} \\
    (o_1, \ldots, o_k) &= \ord^{max}(\sol{s})
\end{align*}
%Set $\sol{y} \subseteq \sol{s}$ represents a greedy extender for $\sol{x}$ 
%according to order $\ord^{max}$.

The upper bound of a partial solution most be an upper limit for each
objective function considering any feasible extender defined on the remaining items.
It must be an optimistic measure, ensuring that any higher value is unfeasible.
For this reason those values will be computed separately for each $j$-th objective,
according to the order $\ord^j$.
Given a partial solution $\sol{x}$ and the set $\sol{s}$ of remaining items,
the upper bound function $ub(\sol{x}, \sol{s})$ can be formally defined as:
\begin{align*}
    ub(\sol{x}, \sol{s}) &= (u_1, \ldots, u_{\np})  \\
  \text{where} \phantom{mmmmm} \\
    u_j &= f_j(\sol{x}) + f_j(\sol{y}) + r.p^j_i\\
    \sol{y} &= \left\{ o_i^j \;\middle|\; \textstyle\sum_{l=1}^i w_{o_l^j} \leq W - \weight{x} \right\} \\
    (o_1^j, \ldots, o_k^j) &= \ord^j(\sol{s}) \\
    r &= \frac{W - \weight{x} - \weight{y}}{w_{o^j_{i+1}}}
\end{align*}
This upper bound definition was presented in~\cite{martello1990knapsack}
and is the same one adopted in~\cite{bazgan2009}.

These three optimization proposals were applied on
Algorithm~\ref{alg:nemull} for defining algorithm 
Algorithm~\ref{alg:bazgan} proposed by \cite{bazgan2009} which
is considerably faster once it handles much less partial solutions.
Further theoretical support for the proposals can be found
in the original paper.

\begin{algorithm}
  \caption{Bazgan's DP algorithm for the MOKP}
  \label{alg:bazgan}
  \begin{algorithmic}[1]
    \Function{BazDP}{$\bsym{p}, \bsym{w}, W$}
    \State $S^0 = [\emptyset ]$
    \For{$k \gets 1, n$}
      \State $S^{k-1} = [\sol{s_1}, \ldots, \sol{s_p}]$
      \State $w_{left} \gets (w_{k} + \ldots + w_n)$
      \State $i \gets min\{i \;| \; \weight{s_i} \geq w_{left}\}$
        \Comment{Avoiding deficient}
      \While{ $\weight{s^k_i} < w_{left}$ }
        \State $i \gets i + 1$
      \EndWhile
      \State $1$
    \EndFor
  %\State $P = \{\sol{x} \:|\: \nexists \sol{a} \in S^n: \dom{a}{x} \}$
  \State \Return $P$
  \EndFunction
\end{algorithmic}

\end{algorithm}
Algorithm~\ref{alg:bazgan} begins by defining an initial solution set $S^0$ containing only the empty solution (line 1).
Then the items order is defined according to the chosen cost-benefit order (line 2).
For each $k$-th stage of the algorithm, a set $S^k$ of partial solutions are generated after the
set $S^{k-1}$ previously defined (lines 5-7).
At line 5 all solutions in the previous set are extended by item $o_k$ except for those which are not feasible.
At line 6 all solutions in the previous set are copied, except for those with too much left capacity (deficient solution avoidance).
At line 7 the knapsack dominance and lower bound filter are applied.
Any solution which is knapsack dominated by any existing solution or has an upper bound
dominated by the lower bound of any existing solution is discarded.

For efficiency reasons the original authors
suggest to apply the knapsack dominance check operation
in parallel with the construction of the new solution set $\solSett^k$.
The elimination of unpromising solutions, which is more
computationally expensive, is implemented as a final step in which
a set of lower bounds is primarily generated
out of the partial solutions in $\solSett^k$.
Then the upper bound of each partial solution is
generated and compared with the lower bound set.
The partial solutions for set $\solSett^k$ are also
generated and stored in ascending order of weight,
which allows us to simplify the knapsack dominance
check operation, once their weights do not need to be
explicitly evaluated.

Despite the effort to reduce the number of partial solutions
Algorithm~\ref{alg:bazgan} still suffers from the size of
efficient solutions set,
especially with higher number of objectives,
since the enumeration of all solutions is necessary.
For this reason it becomes crucial the development of a method that operates
efficiently on large sets of solution.
%For this reason, the implementation of an efficient algorithm for the MOKP
%becomes a challenge.

Operations on lines 2-6 demand linear time and do not present  difficulties.
However the dominance check operation of line 7
may represent the computational bottleneck of the algorithm.
This operation must cover the entire set of solutions and if the right
strategy is not considered, the algorithm will have to compare each solution
individually, which represents a great computational effort.
For the bi-dimensional case~\cite{bazgan2009} proposes to use an AVL tree for 
indexing solutions by its first objective value,
which improves significantly its performance over the use of a list.

This work proposes to increase the performance of dominance check operations
by multi-dimensional indexing partial solutions.
The strategy may reduces the computational complexity of the algorithm,
minimizing the impact of solution growth, which may even bring
viability to previously unfeasible instances.
This proposal is discussed in Section~\ref{sec:kdtree}.
