%\documentclass[journal]{IEEEtran}
\documentclass{IEEEtran}

%\usepackage{lineno,hyperref}
\usepackage{amssymb}
\usepackage{amsmath}
%\usepackage[longend,linesnumbered]{algorithm2e}
\usepackage{algorithmic}
\usepackage{subfig}
\usepackage{pstricks}
\usepackage{setspace}
\usepackage{graphics}
\usepackage{array}

\usepackage{flushend}

%\modulolinenumbers[5]

\usepackage{float}

\hyphenation{op-tical net-works semi-conduc-tor}

\renewcommand{\algorithmicrequire}{\textbf{Input:}}
\renewcommand{\algorithmicensure}{\textbf{Output:}}
\newcommand{\figpar}{\vspace*{-3mm}}
\newcommand{\figspaces}{\vspace*{-7mm}}
\newcommand{\mymathstyle}{\textstyle}

\IEEEoverridecommandlockouts

\begin{document}

\title{An Experimental Analysis of three Computational Approaches for Minimizing Losses on Electricity Distribution}

\author{Jo\~ao Carlos.~H.~Moreira,
	Marcos Daniel.~V.~Baroni\thanks{Research supported by Funda\c c\~ao de Amparo \`a Pesquisa do Esp\'irito Santo.},
	Fl\'avio~M.~Varej\~ao,
}

\markboth{IEEE TRANSACTIONS ON POWER DELIVERY ,Vol.~--, No.~--, --~2015}%
{Moreira \MakeLowercase{\textit{et al.}}: An Experimental Analysis of three Computational Approaches for Minimizing Losses on Electricity Distribution}

\maketitle

\begin{abstract}


\begin{resumo}
Resumo...
\vspace{\onelineskip}

\noindent
\textbf{Palavras Chave}:
Multi-objective Knapsack Problem,
Metaheuristic,
Shuffled Complex Evolution,
Multi-dimensional indexing
\end{resumo}

\begin{resumo}[Abstract]
 \begin{otherlanguage*}{english}
Many real applications like project selection, capital budgeting and cutting stock involves optimizing multiple objectives that are usually conflicting and can be modelled as a multi-objective knapsack problem (MOKP).
Unlike the single-objective case, the MOKP is considered a NP-Hard problem with
considerable intractability.
This work propose a hybrid heuristic for the MOKP based on the
shuffled complex evolution algorithm.
A multi-dimensional indexing strategy for handling large amount of intermediate
solutions are proposed as an optimization, which yields considerable
efficiency, especially on cases with more than two objectives.
A series of computational experiments show the applicability of the proposal
to several types of instances.

\noindent
\textbf{Keywords}:
Multi-objective Knapsack Problem,
Metaheuristic,
Shuffled Complex Evolution,
Multi-dimensional indexing
\end{otherlanguage*}
\end{resumo}

% Falar brevemente sobre o MOKP
% Falar brevemente sore a estratégia de indexação
% Resumir os resultados obtidos

\missingt{
Observar as 5 regras:\\
1. A general statement introduciong the broad reserach area of the particular topic being investigated;\\
2. An explanation of the specific problem (difficulty, obstavle, challange) to be solved;\\
3. A review of existing or standard solutions to this problem and their limitations;\\
4. An outline of the proposed new solution;\\
5. A summary of how the solution was evaluated and what the outcomes of the evaluation were.
}


\end{abstract}

\begin{IEEEkeywords}
OR in energy, combinatorial optimization, metaheuristics
\end{IEEEkeywords}

\IEEEpeerreviewmaketitle


The multidimensional knapsack problem (MKP) is a strongly NP-hard combinatorial
optimization problem which can be viewed as a resource allocation problem and
defined as follows:

\begin{align*}
  \text{maximize} & \sum_{j=1}^n p_j x_j \\
  \text{subject to} & \sum_{j=1}^n w_{ij} x_j \leqslant c_i \quad i \in \{1, \ldots, m\}\\
   & x_j \in \{0, 1\}, \quad j \in \{1, \ldots, n\}.
\end{align*}

% Define the MKP
The problem can be interpreted as a set of $n$ items with profits $p_j$
and a set of $m$ resources with capacities $c_i$.
Each item $j$ consumes an amount $w_{ij}$ from each resource $i$, if selected.
The objective is to select a subset of items with maximum total profit,
not exceeding the defined resource capacities.
The decision variable $x_j$ indicates if $j$-th item is selected.

The multidimensional knapsack problem can be applied on budget planning 
scenarios, subset project selections, cutting stock problems, task scheduling,
allocation of processors and databases in distributed computer programs.
The problem is a generalization of the well-known knapsack problem (KP) in which
$m = 1$.

The MKP is a NP-hard problem significantly harder to solve in practice than the KP.
Despite the existence of a fully polynomial approximation scheme (FPAS) for the KP,
finding a FPAS for the MKP is NP-hard for $m \geqslant 2$~\cite{magazine1984note}.
Due its simple definition but challenging difficulty the MKP is often used to
to verify the efficiency of novel metaheuristics.

Its well known that the hardness of a NP-hard problem grows exponentially over
its size.
Thereupon, a suitable approach for tackling those problems is to reduce their size
through some variable fixing procedure.
Despite not guaranteeing optimality of the solution, an efficient variable
fixing procedure may provide near optimal solutions through a small computational effort.

%A metaheuristic is a set of concepts that can be used to define heuristic methods
%that can be applied to a wide set of different problems.
%In other words, a metaheuristic can be seen as a general algorithmic framework which can be applied to
%different optimization problems with relatively few modifications to make them adapted to a specific problem.”

The SCE is a metaheuristic, proposed by Duan in \cite{duan1992effective},
which combines the ideas of a controlled random search with the concepts
of competitive evolution and shuffling.
The SCE algorithm has been successfully used to solve several problems
like flow shop scheduling~\cite{zhao2014shuffled} and project management~\cite{elbeltagi2007modified}.

In this paper we address the development of a hybrid heuristic for the MKP.
The size of the problem instance is first reduced through a variable fixing
procedure that uses a known concept among knapsack problems called \emph{core}.
The reduced problem is then efficiently solved by a SCE algorithm.

The reminder of the paper is organized as follows:
Section~\ref{sec:core} defines the core concept for the MKP and its application
for the problem reduction.
Section~\ref{sec:sce} presents the shuffled complex evolution algorithm
and proposes its application on the MKP.
Section~\ref{sec:exp} comprises several computational experiments over well-known
instances from literature.
In section~\ref{sec:conc} we make our concluding remarks about the developed
methods and the experimental results.



\section{Problem Definition}
\label{sec:model}

In this section we show the mathematical model defined to tackle the EDCO's problem previously introduced at section~\ref{sec:intro}.
Firstly, section~\ref{sec:model:model} details the mathematical model proposed to solve the problem.
This section is followed by section~\ref{sec:model:knap}, where we show how the model defined relates 
to the available literature, more specificcally, we show that the model is a generalization of the 
well known Knapsack Problem, and discuss about the problem hardness.

\subsection{Mathematical Model}
\label{sec:model:model}

In order to apply optimization methods to the problem, we must first define it formally. 
The model defined in this work considers that the only objective is to maximize the 
\textit{Net Present Value} (NPV) of the investment made, that is, maximize the financial return
of the investments, for a plan of $M$ years, given:

\begin{itemize}
  \item The \textit{yearly budgets} $o_{i,l}$, for a set of $L$ resources, $1 \le i \le M$, $1 \le l \le L$;
  \item the \textit{yearly reduction goals} $g_i$, $1 \le i \le M$, representing the amount of electricity loss 
        that must be reduced on the $i-th$ year;
  \item the \textit{internal return rate} $r$, which represents the annual depreciation of the investment.
	This rate is constant for all years.
\end{itemize}

The investment to be made consists of choosing a subset of actions from a portfolio with $N$ actions.
Each $j-th$ action from this portfolio, $1 \le j \le N$, has a set of characteristics:

\begin{itemize}
  \item The \textit{elecricity value} $v_j$, representing the value of each unit of electricity recovered by the $j-th$ action;
  \item $m_j$, the \textit{market} of the action, i.e. how many times action $j$ can be executed during the whole plan;
  \item $u_{j,i}$, the \textit{yearly market} of the action, or how many times action $j$ can be executed on the plan's $i-th$ year;
  \item $c_{j,l}$, how much each execution of action $j$ consumes from resource $l$, in other words, the \textit{cost} of the action;
  \item $e_{j,k}$, the \textit{electricity recovered} by action $j$ on the $k-th$ year after its execution, given in the form of an electricity recovery curve
		   since an action can recover electricity, i.e. avoid electricity loss, on the year it was executed and on the following years;
  \item a set $D_j$ of pairs $(d, Q_{j,d})$ representing the \textit{dependencies} of action $j$.
    For each execution of action $j$, each action $d \in D_j$ must be executed previously an amount of time defined 
    by $Q_{j,d} \in \mathbb{R^+}$.
\end{itemize}

The objective is to find a solution $\bar{x}$, in other words, a set of values for the decision variables
$x_{j,i}$, $\forall i,j$, $x_{j,i} \in \mathbb{N}$ that maximizes the NPV. This solution represents how many times action $j$ 
will be executed on the $i-th$ year of the plan. In order to present the NPV equation and consequently the problem's objective function, 
three auxiliary equations must be defined first:

\begin{itemize}

  \item The first of them, equation~\ref{eq:rec}, represents the electricity loss reduction caused on the $i-th$ year by action $j$ executed on the $k-th$ year of the plan.
        In other words, it calculates how much electricity loss the execution of action $j$ on the year $k$ will avoid at year $i$:
    \begin{equation}
	\label{eq:rec}
	R_{i,j,k}(\bar{x}) = x_{j, k} \cdot e_{j, i - k + 1} \ %, \forall k \leq i,
    \end{equation}

  \item The second, equation~\ref{eq:lucro}, represents the total yearly profit $V_i$, which is the sum of all energy recovered on year $i$ multiplied by each action's energy value:
    \begin{equation}
      \label{eq:lucro}
	V_{i}(\bar{x}) = \sum_{j=1}^{N} \sum_{k=1}^{i} R_{i, j, k}(\bar{x}) \cdot v_j,
    \end{equation}
    
  \item Finally, equation~\ref{eq:cost} represents the total yearly cost $C_i$, i.e., the sum of the costs on every resource for all actions executed on the $i-th$ year of the plan:
    \begin{equation}
    \label{eq:cost}
    C_{i}(\bar{x}) =  \sum_{j=1}^{N} \sum_{l=1}^{L} x_{j, i} \cdot c_{j,l}
    \end{equation}
    

\end{itemize}

By definition, $V_i - C_i$ is $i-th$ year's total \textit{cash flow}, and the NPV is the sum of all anual cash flows, 
adjusted by the internal return rate for every year. A solution with a bigger NPV means the EDCO will have a greater profit with that solution when compared
to other solutions with lower NPV. So, the problem's objective is to maximize the NPV and the objective function is presented at equation~\ref{eq:objective}:

\begin{equation}
    \label{eq:objective}
    \underset{\bar{x}}{max(O(\bar{x}))} = \underset{\bar{x}}{max}\left(\sum_{i=1}^{M} \frac{V_i(\bar{x}) - C_i(\bar{x})}{(1 + r)^i}\right)
\end{equation}

The problem's objective function must be maximized respecting some constraints. Those are:

\begin{itemize}
  \item The yearly budget constraint (equation~\ref{eq:budget}), which ensures that the solution's cost won't exceed the yearly budgets for each resource, 
    \begin{equation}
	\sum_{j=1}^{N} x_{j, i} \cdot c_{j,l} \le o_{i,l} \, \; \forall i, l,
	    \label{eq:budget}
    \end{equation}
  \item the market constraint (equation~\ref{eq:market}), which prevents the solution from surpassing any action's market,
    \begin{equation}
	\sum_{i=1}^{M} x_{j, i} \le m_j \, \; \forall j,
	    \label{eq:market}
    \end{equation} 
  \item the yearly market constraint (equation~\ref{eq:maxacts}), analogous to the market constraint, but now enforcing each action's yearly market,
    \begin{equation}
	x_{j, i} \le u_{j, i} \, \; \forall j, i,
	    \label{eq:maxacts}
    \end{equation} 
  \item  the yearly reduction goal constraint (equation~\ref{eq:goal}), which ensures that the yearly reduction goals won't be exceeded,
    \begin{equation}
	\label{eq:goal}
	\sum_{j=1}^{N} \sum_{k=1}^{i}R_{i,j,k}(\bar{x}) \leq g_i \, \; \forall i, \\
    \end{equation} 
  \item the dependency restriction (equation~\ref{eq:dep}), which ensures that the solution will respect the dependency relations between actions for all actions of the plan,
    \begin{equation}
	\label{eq:dep}
	\sum_{i=1}^{k} x_{d, i} \ge \sum_{i'=1}^{k} x_{j, i'} \cdot Q_{j, d} \, \; \quad \forall d \in D_j \quad \forall j,k.
    \end{equation}
\end{itemize}

The resulting problem obtained by the concatenation of equations~\ref{eq:rec} to~\ref{eq:dep}, which models the EDCOs problem tackled in this paper, is presented in equation~\ref{eq:pmmlpo++}:

    \begin{equation}
    \label{eq:pmmlpo++}
      \begin{aligned}
	 & \underset{\bar{x}}{\text{max}}  & & \sum_{i=1}^{M} \frac{V_i(\bar{x}) - C_i(\bar{x})}{(1 + r)^i}	\hspace{25ex} \textit{\scriptsize (NPV)} \\
	 & \text{s. to} 			& & \sum_{j=1}^{N} x_{j, i} \cdot c_{j,l} \le o_{i,l} \, \; \forall i, l \hspace{15ex} \textit{\scriptsize (Yearly Budgets)} \\
	 & 					& & \sum_{i=1}^{M} x_{j, i} \le m_j \, \; \forall j 			\hspace{25ex} \textit{\scriptsize (Market)} \\
	 & 					& & x_{j, i} \le u_{j, i} \, \; \forall j, i 			\hspace{22ex} \textit{\scriptsize (Yearly Market)} \\
	 & 					& & \sum_{j=1}^{N} \sum_{k=1}^{i}R_{i,j,k}(\bar{x}) \leq g_i \, \; \forall i    \hspace{18ex}  \textit{\scriptsize (Goals)} \\
	 & 					& & \sum_{i=1}^{k} x_{d, i} \ge \sum_{i'=1}^{k} x_{j, i'} \cdot Q_{j, d} \, \; \forall d \in D_j\;  \forall j,k  \hspace{4ex} \textit{\scriptsize (Deps.)}\\
	 & \text{where}				& &  R_{i,j,k}(\bar{x}) = x_{j, k} \cdot e_{j, i - k + 1} 	\hspace{10ex}  \textit{\scriptsize (Energy Recovery)}\\
	 & 					& & V_{i}(\bar{x}) = \sum_{j=1}^{N} \sum_{k=1}^{i} R_{i, j, k}(\bar{x}) \cdot v_j   \hspace{10ex}  \textit{\scriptsize (Yearly Profit)}\\
	 & 					& & C_{i}(\bar{x}) = \sum_{j=1}^{N} \sum_{l=1}^{L} x_{j, i} \cdot c_{j,l}   \hspace{15ex} \textit{\scriptsize (Yearly Cost)}\\
	 & 					& & x_{j,i} \in N, \quad i \le M, \quad j \le N, \quad l \le L
      \end{aligned}
    \end{equation}

\subsection{The Knapsack Approach}
\label{sec:model:knap}

The exact formulation of the problem has shown to be quite particular and no work addressing 
a similar problem was found in literature. In this work, we face it as a generalization
of the well known Knapsack Problem, which has lots of practical applications. 
Even though our problem is not the exact generalization of the classical knapsack
and its variants, we try to relate the hardness of those problems to our formulation of the
EDCO's problem.

The \textit{Knapsack Problem} (KP) is a well known problem on the literature consisting of the allocation of $N$ items inside a 
knapsack with a maximum capacity $c$. Each item $j$ has its weight $w_j$ and profit $p_j$, and the objecive is to 
maximize the profit of the items in the knapsack, without exceeding its capacity. It can be formulated~\cite{pisinger1995} as:

\begin{equation}
\label{eq:knap01}
  \begin{aligned}
    & \text{maximize}
    & & \sum_{j=1}^{N} p_j \cdot x_j \\
    & \text{subject to} 
    & & \sum_{j=1}^{N} w_j \cdot x_j \leq c, \\
    & & & x_j \in \{0,1\}, &j = 1,...,N.
  \end{aligned}
\end{equation}

One possible generalization of the classical problem is to relax the restrictions in a way that more than one 
copy of the same item can be in the knapsack. That generalization is called \textit{Bounded Knapsack Problem} (BKP)~\cite{pisinger1995}, 
and is usually solved with a transformation of the BKP instance to a KP instance with more variables~\cite{kellerer2004knapsack}, 
solving the resulting problem with KP techniques.

Another possible generalization can be achieved when we consider some kind of ordering between the items, or a dependency relation. 
That generalization is called the \textit{Partially-Ordered Knapsack Problem} (POKP)\cite{pok2002}. In this problem, an item $a$ may depend on item $b$, meaning
that to be able to put item $b$ in the knapsack, item $a$ must also be on the knapsack. A review on Knapsack Problems With Neighbour Constraints, a generalization of the POKP, is presented
in~\cite{borradaile2012}, and a real application is demonstrated in~\cite{lambert2014}, where the Open Pit Block Sequencing problem is modelled using the POKP.

We may also consider a Knapsack Problem with several knapsacks available, each of them with its own capacity. This generalization is called 
\textit{Multiple Knapsack Problem} (MKP)~\cite{kellerer2004knapsack}. In~\cite{amarante2013} the MKP is used to model the allocation of virtual mahines in a 
cloud computing environment, and the resulting model is solved using an Ant Colony Optimization based heuristic. 
In ~\cite{patvardhan2014}, the authors propose a Quantum-Inspired Evolutionary Algorithm to solve the MKP.

Finally, another generalization is obtained when we consider a knapsack with more than one dimension, like weight and volume. In this variation, 
each item has a diferent weigth $w_{j,r}$ on each dimension $c_r$. This generalization is known as the \textit{Mutidimensional Knapsack Problem} 
or \textit{d-Dimensional Knapsack Problem} (dKP)~\cite{kellerer2004knapsack}. Due to its hardness and several pratical applications, the dKP is probably 
the most studied variation of the Knapsack Problem. For an overview on this variation the reader is directed to~\cite{freville2004}. 
Recently, several heuristic approaches have been proposed to solve the dKP. For instance,~\cite{beheshti2013} and~\cite{chih2014} propose two Particle
Swarm Optimization (PSO) algorithms to solve the dKP. The first proposes a binary PSO with accelerated particles and the second presents a binary PSO 
with acceleration varying with time. In~\cite{wang2013} the Fruit Fly algorithm is adapted for binary variables, and the resulting implementation is 
tested using the literature's most used test instances (the OR-Library~\cite{chubeasley1998}). In~\cite{martins2014}, the authors investigate the efficiency
of the application of linkage-learning on an Estimation of Distribution algorithm for the dKP.

All the problems mentioned above are hard to solve optimally, they are known as $\mathcal{NP}$-hard problems~\cite{kellerer2004knapsack} and there 
are no known polynomial time algorithms to solve them, unless $\mathcal{P} = \mathcal{NP}$ \cite{garey1978}.
For the classical knapsack problem there is a FPTAS (\textit{Fully Polynomial Time Approximation Scheme}),
while the other variants mentioned above are hard to aproximate and there are only PTAS (\textit{Polynomial Time Approximation Schemes}) 
to solve them with a certain degree of approximation to the optimal solution, as shown in~\cite{pok2002, puchinger2006core, dawande2000approximation}. 

If we consider a knapsack with the characteristics of every gerenalization presented so far we can define a new generalization
called \textit{Partially-Ordered Multidimensional Multiple Bounded Knapsack Problem} (POMMBKP), which is as far as we know, a novel generalization of the
Knapsack Problem. Looking at the model described at section~\ref{sec:model:model} it is possible to see that it is a generalization of the POMMBKP, 
when we make the following assumptions:

\begin{itemize}
  \item the model's actions are the POMMBKP's items;
  \item the model's years are the POMMBKP's knapsacks;
  \item each resource an action consumes can be seen as a POMMBKP dimension;
  \item in the model, the actions can be executed more than one time, representing the Bounded Knapsack Problem part of the POMMBKP;
  \item the loss reduction plan can be defined for many years, and each year may have several budgets. That may be seen as the Multiple Knapsacks Problem and Multidimensional Knapsack Problem respectively;
  \item actions may depend on other actions being executed previously, like on the Partially-Ordered Knapsack Problem.
\end{itemize}

Even with all the similarities, some differences between the model and the POMMBKP are noted. Firstly, the 
POMMBKP doesn't consider anything similar to the model's internal return rate. Also, only single dependencies
are considered on the POMMBKP, while on the EDCO's problem multiple dependencies may occur, i.e. an action may require
another action to be executed more than one time before. Finally, if we consider the model's goal as a resource, 
an action (item) can consume resources from more than one year (knapsack), something that doesn't happen on the POMMBKP.

Considering the EDCO's problem with an internal return rate equal to zero, only single dependencies and actions recovering energy 
only on the year they were executed, characteristics that can be representend on the model presented in equation~\ref{eq:pmmlpo++}, 
it can be seen as a generalization of the POMMBKP (for a proof the user is directed to~\cite{Moreira2015}). From now on, we refer to 
the EDCO's problem as the \textit{Partially-Ordered Multidimensional Multiple Bounded Knapsack with Multiple Dependencies, 
Spread Weights and Adjustment Rate} (POMMBKPDSA).

In~\cite{Moreira2015} it is shown that the POMMBKPDSA is a generalization of the POMMBKP, so we can tell 
that the first is at least as difficult as the second. Since the POMMBK is a generalization of all the simple 
variations discussed before (BKP, POKP, MKP and dKP), it is also possible to say that the POMMBK is at least as 
difficult as any of them. We can conclude then that the POMMBKPDSA is also at least as difficult as any of those 
variations, so there is not any algorithm able to solve it in polynomial time, considering $\mathcal{P} \ne \mathcal{NP}$. 

\section{Computational Approach}
\label{sec:approach}

To solve the POMMBKPDSA we used three approaches:

\begin{itemize}
  \item An exact approach, using integer programming techniques;
  \item a heuristic based on a greedy strategy, using the truncated solution of the linear relaxation of the problem, called GALP;
  \item a Tabu Search based heuristic, also using the truncated solution of the linear relaxation of the problem, called TSLP.
\end{itemize}

The first approach used, the exact approach, was to solve the formulation of the problem described in 
equation~\ref{eq:pmmlpo++} using the \textit{branch-and-cut} method~\cite{padberg1991branch} implemented on the
CPLEX solver~\cite{Cplex}.

Before explaining the other two approaches, it is important to explain the reason why both heuristics used 
the truncated solution of the LP-relaxed version of the problem as starting points. During our tests, we tried 
to use other alternatives as starting points for the heuristics, like empty solutions or randomly generated 
solutions. Still those alternatives never yielded better results than starting the heuristics from the truncated
solution of the LP-relaxed version of the problem. In fact, we found out that this solution was on the worst case
97.7\% of the solution of the LP-relaxed version of the problem, which is an upper bound, and at 99.65\% on the average case.
Being such a good starting point, and so easily obtainable by modern solvers, it seems a good choice to start the 
heuristic from this solution.

The second applied technique, the GALP, is a local search heuristic using a greedy strategy. Generally, greedy strategies have the disadvantage 
of not being able to escape any local optimum they reach. On the other hand, they usually converge faster than other methods, 
making them suitable alternatives when fast viable solutions are desired. Also, they are usually easy to understand and implement.

Algorithm~\ref{alg:galp} describes the greedy algorithm developed for the POMMBKPDSA. The heuristic receives as initial solution 
the truncated solution of the LP-relaxed version of the problem. It then tries to improve it by iteratively adding items to the 
knapsacks until no more additions are possible. It is divided in three phases: On the first phase, the algorithm generates a list
of all possible allocations which may be made on a solution. Those alocations are all the possible combinations of item vs. knapsack, 
and represent the addition of a certain item to a knapsack. That list is then sorted by decreasing order of each alocation's efficiency,
i.e. how much each alocation contributes to the increase of the objective function. In the last phase, that sorted list of alocations
is used to iterativelly improve the initial solution.

\begin{algorithm}
  \caption{Greedy Algorithm}
  \label{alg:galp}
  \KwIn{Initial Solution: $S_{initial}$}
  \KwOut{Best Solution: $S_{best}$}
  \Begin{

	$S_{best} \gets S_{initial}$ 

	$AlocationList \gets GenerateAlocations()$ \label{alg:galp:gera}
	
	$AlocationList \gets DecreasingOrder(AlocationList)$ \label{alg:galp:ordena}

	$Improved \gets True$ 

	\While {$Improved$}{\label{alg:galp:inicioloop}
	  $Improved \gets False$ 
	  
	  \For{$Alocation \in AlocationList \ \&\& \ not\  Improved$}{\label{alg:galp:inicioloopinterno}
	    \If{$isViable(S_{best}, Alocation)$}{
	    
	      $S_{best} \gets S_{best} + Alocation$ \label{alg:galp:melhoria}
	      
	      $Improved \gets True$ 
	    }
	  }\label{alg:galp:fimloopinterno}
	}\label{alg:galp:fimloop}
	
	retorna $S_{best}$ 
  }
\end{algorithm}

At line~\ref{alg:galp:gera}, GenerateAlocations() function generates a list with all the possible alocations of items in knapsacks. 
For example, for an instance of the problem with two knapsacks and three items, the list generated would be \{(1,1), (1,2), (2,1), (2,2), (3,1), (3,2)\},
where each pair from the list represents adding one copy of an item to a knapsack. On the sequence, on line~\ref{alg:galp:ordena} that list is 
ordered by descending order of efficiency, i.e. how much each item contributes to the increase on the objective function's value. The initial solution is 
then iteratively improved on the loop between lines~\ref{alg:galp:inicioloop} 
and~\ref{alg:galp:fimloop}. At each step, the alocation list is searched for the alocation with the highest efficiency which leads to a viable solution.
Once that alocation is found, the currently best solution is updated with the alocation (line~\ref{alg:galp:melhoria}) and the loop continues, until no
alocation can lead to a viable solution. At that point, the algorithm stops and returns the best solution found.

The last applied heuristic, the TSLP, is an implementation of the metaheuristic Tabu Search~\cite{glover1989tabup1}. This metaheuristic has been heavilly used on the
last decades to tackle Combinatorial Optimization problems, has already been applied on the dKP in~\cite{danmeyer1993} and for some time the Tabu Search presented at~\cite{vasquez2001} 
and improved at~\cite{vasquez2005} was the method which obtained the best results for the dKP intances commonly used in the litarature.

Like the greedy algorithm previously presented, the Tabu Search is also essentially a local search method, so it should also present a tendency to 
become stuck at local optima. To work around this problem, the Tabu Search allows that during the search, when no solution better than the current one is found,
a worse solution may be chosen with the spectation that this choice will lead to better solutions on the long run. Also, in this method, the characteristics
of all the choices made during the search may be taken in consideration when making the next choice. Because of those characteristics, the Tabu Search is usually
able to escape local optima and find better solutions than other simple local searches.

Before explaining the implemented heuristic, it's useful to explain some terms which will be used ahead. The first of them are the 
\textit{moves}. Moves are the operations applied on a solution which takes them to another solution. The set of solutions which can be reached
from solution $x$ by applying one of the possible movements is called x's \textit{Neighborhood}. One characteristic component of the Tabu Search is the
\textit{Tabu List}. This list usually contains moves or characteristics that lead to unwanted solutions, and is used during the search to forbid certain moves
as a way to escape local optima and hopefully guide the search towards better solutions.

The pseudocode presented in algorithm~\ref{alg:tabu} describes the implemented heuristic. It receives the 
initial solution and tries to improve that solution by iteractively searching its neighborhood.

\begin{algorithm}
\caption{Tabu Search}
\label{alg:tabu}
\KwIn{Initial Solution: $S_{initial}$}
\KwOut{Best Solution: $S_{best}$}
\Begin{

  $S_{current} \gets S_{initial}$
  
  $S_{best} \gets S_{initial}$ 
  
  $ListaTabu \gets \emptyset$

  \While{$\neg stoppingCondition$}{\label{alg:tabu:inicioloop}
  
      $moves \gets Neighborhood(S_{current})$\label{alg:tabu:vizinhanca}
      
      $move \gets BestNonTabuMove(moves)$\label{alg:tabu:melhormovimento}
      
      $S_{current} \gets S_{current} + move$\label{alg:tabu:movimento}
      
      $TabuList \gets TabuList + move$\label{alg:tabu:tabu}
      
      \If{$S_{current} \geq S_{best}$}{
	 $S_{best} \gets S_{current}$\label{alg:tabu:melhor}
	 
	 $ResetTabuList()$
      }
  }\label{alg:tabu:fimloop}
  return $S_{best}$
}
\end{algorithm}

The most relevant part of the heuristic happens on the loop between lines~\ref{alg:tabu:inicioloop}
and~\ref{alg:tabu:fimloop}. Firstly, all the moves leading to neighbours of the current solutions are obtained at 
line~\ref{alg:tabu:vizinhanca}. At line~\ref{alg:tabu:melhormovimento} the function \textit{BestMoveNotTabu()} searches
the movement list for the move which will lead to the greatest value for the objective function and is not forbidden
by the Tabu List. The current solution is then updated with the chosen move (line~\ref{alg:tabu:movimento}), 
the Tabu List is also updated with the chosen move (line ~\ref{alg:tabu:tabu}) 
and finally if the current solution is better than the best solution found so far, the best solution is updated 
with the current solution (line~\ref{alg:tabu:melhor}) and the tabu list is reset.

An important aspect to be considered when implementing a Tabu Search is how to implement and manage the Tabu List (TL).
In this work, a dynamic TL management method was implemented based on the Reverse Elimination Method (REM)~\cite{glover1990tabup2}. This method leads to
an exact tabu status, meaning that any move present in the TL leads to a solution that was already visited by the search. The method works
by adding the move made at each iteraction to the Tabu List. When a new move is added to the Tabu List, the list is traced in reverse order
to build an Active Tabu List (ATL), which contains every move that will lead to a solution already visited during the search. Now, in order to check if a move
is forbidden, all we have to do is check if it is contained in the ATL. When a new global optimum is found during the search, the TL is cleared.
For more details on the Reverse Elimination Method the reader is directed to~\cite{glover1990tabup2}.























\section{Experimental Data}
\label{sec:exp_data}

In order to test the hardness of the POMMBKPDSA and assess the quality of the solutions obtained by the methods 
proposed on the previous section, a large number of instances of the problem are needed to make useful 
conclusions. Unfortunatelly, our local EDCO wasn't able to provide that number of instances, so we had 
to develop an instance generator to create a sufficient amount of artificial instances to run the tests. 
The generator creates instances according to the formulation of the POMMBKPDSA, with parameters that can be 
adjusted to build instances of different sizes and characteristics. It receives as parameters:
\begin{itemize}
  \item $Y$, the number of years (knapsacks);
  \item $A$, the number of actions (items);
  \item $R$, the number of budgets or resources (dimensions);
  \item $\alpha$, the level of correlation between each action's cost and profit.
\end{itemize}

The first step on instance generation is the actions generation.

%The algorithm for the instance generator is composed of three parts: the actions generation,
%the budgets and goals generation and the dependencies generation. When the actions are 
%being generated, the first characteristic created is the elecricity value ($v_a$), followed
%by the yearly ($u_{a,y}$) and global ($m_a$) markets. This part also generates the cost of the
%action on each budget ($c_{a,r}$) and the action's electricity recovery curve ($e_{j,k}$). The 
%algorithm's second part covers the creation of the yearly goals ($g_y$) and the yearly budgets
%for each resource ($o_{y,r}$). The last part generates some random dependencies between actions.
%The code for the generator is presented in algorithm~\ref{alg:gen}.

% Explicar \alpha, a variação de anos/ações/recursos,

%Firstly, lines~\ref{alg:gen:line:actionbegin} to~\ref{alg:gen:line:actionend} show the creation of the $A$ actions.
%The electricity value is created at line~\ref{alg:gen:line:actionvalue} as a real value between
%1.0 and 2.0. The yearly market is generated at line~\ref{alg:gen:line:market} as a random integer between 0 e 50 
%and the global market is created at line~\ref{alg:gen:line:uc} as $90\%$ of the yearly markets sum.
%The global market is created a bit below the yearly markets sum because if it is equal or greater than that sum, the 
%global market restriction becomes useless, as it will never be reached. At line~\ref{alg:gen:line:cost}, 
%the action's costs over each resource are created. Lines~\ref{alg:gen:line:vectorbegin} to~\ref{alg:gen:line:vectorend} build a 
%random array of size $Y$ where the sum of all components is 1. This array is used to create the actions loss reduction curve at 
%line~\ref{alg:gen:line:rec}.

%The total electricity loss avoided by the action (the electricity recovered) is created as a function of two components: the sum of the action's
%cost plus a fixed value and a random number. The correlation parameter ($\alpha$) is used to dose how much each component will be used on the curve's 
%creation. When $\alpha$ is zero, only the first component is utilized and the amount of electricity recovered by the action, and consequently its profit,
%is strongly correlated to the action's total cost, in other words, costlier actions recover more electricity. On the other side, when $\alpha$ equals
%one, there is no correlation between an action's cost and it's profit. The total electricity loss is then divided by the number of years to generate the 
%action's loss reduction curve, using the normalized array created before.

% \begin{figure}
\begin{algorithmic}[1]
\REQUIRE $Y, A, R, \alpha$
\ENSURE Generated instance
\STATE $r \gets 0.10$
\FOR{ $a=1$ to $A$}
  \label{alg:gen:line:actionbegin}
  \STATE $v_a \gets RandomReal(1.0, 2.0)$
    \label{alg:gen:line:actionvalue}
  \FOR{ $y=1$ to $Y$}
    \STATE $u_{a,y} \gets RandomInteger(0, 50)$
      \label{alg:gen:line:market}
  \ENDFOR
  \STATE $m_{a} \gets 0.9 * Sum(u_{a})$
    \label{alg:gen:line:uc}
  \FOR{$r=1$ to $R$}
    \STATE $c_{a,r} \gets RandomReal(1.0, 100.0)$
      \label{alg:gen:line:cost}
  \ENDFOR
  \FOR{$y=1$ to $Y$}
    \label{alg:gen:line:vectorbegin}
    \STATE $W_{y} \gets RandomReal(0.0, 1.0)$
  \ENDFOR
  \FOR{$y=1$ to $Y$}
    \STATE $W_{y} \gets W_{y}/Sum(W_y)$
  \ENDFOR
    \label{alg:gen:line:vectorend}
  \FOR{$y=1$ to $Y$}
    \STATE $e_{a,y} \gets ((1 - \alpha) \times ((2 \times Sum(c_{a}) + 10)/v_a) + (\alpha \times RandomReal(1.0, 100.0) \times R)) \times W_{y} \times (1 + r)^y$
      \label{alg:gen:line:rec}
  \ENDFOR
\ENDFOR
  \label{alg:gen:line:actionend}
\FOR{$y=1$ to $Y$}
  \label{alg:gen:line:budgbegin}
  \STATE $aux \gets 0.5 * Sum(e_{y})/Y$
  \STATE $g_{y} \gets RandomReal(0.95*aux, 1.05*aux)$
  \FOR{$r=1$ to $R$}
    \STATE $aux \gets 0.5 * Sum(c_{y,r})/Y$
    \STATE $o_{y,r} \gets RandomReal(0.95*aux, 1.05*aux)$
  \ENDFOR
\ENDFOR
  \label{alg:gen:line:budgend}
\FOR{$i=1$ to $A/10$}
  \STATE GenerateDependency()
    \label{alg:gen:line:deps}
\ENDFOR
\end{algorithmic}
\caption{Instance Generator}
\label{alg:gen}
\end{figure}


%At the second part, between lines~\ref{alg:gen:line:budgbegin} and~\ref{alg:gen:line:budgend}, the loss reduction goals and the budgets for
%each resource are created, for each year. Yearly goals are created as half the sum of all electricity recovered by all actions, divided 
%by the number of years($\pm5\%$). This approach ensures solutions won't have too few actions, and won't be composed of all possible 
%actions allocations. Bugdets are created analogously, but one budget per resource per year is created.

%Line~\ref{alg:gen:line:deps} creates the dependencies between actions. For each execution of the loop, function GenerateDependency() 
%choses two random actions and creates a dependency between them, avoiding cyclical dependencies, as any action involved in a cyclical 
%dependency would never be chosen.

\section{Experimental Results}
\label{sec:exp_results}

In order to evaluate the hardness of the instances created with the generator and the effectiveness of the proposed
solution approaches, computational tests were conducted using both heuristics and the generic solver CPLEX.
%solution approaches, computational tests were conducted using both heuristics and the generic solver CPLEX~\cite{Cplex}.

Firstly, the created instances were solved with the CPLEX solver version 12.5.0, in order to verify their hardness. 
The solver was used in its default configuration. Then, the GALP and TSLP algorithms were executed on the same instances.
While GALP doesn't use any adjustable parameter, the TSLP has the stopping condition, which was 10000 iterations in our
tests. All the tests were executed on Intel Core i5-3570 @ 3.40 GHz machines with 8GB RAM memory and executing Ubuntu 13.04.
Both heuristics were implemented in Java and run on the 1.7 JVM.

To execute the tests a suite of artificial benchmark instances was created using the generator presented at the last session.
The instances were created for all combinations of 3 to 6 years, 25, 50 and 100 actions, 1, 2 and 4 resources and
correlation levels ($\alpha$) of 0.0, 0.1 and 1.0. Those numbers of years, actions and resources were chosen to simulate 
the dimensions of the instances expected to be found in practice. The $\alpha$ values were chosen to verify if this 
generalization of the KP is also sensitive to the correlation level on the items profits and costs, as predicted
by the KP literature~\cite{pisinger2005}, and represent respectivelly strongly, weakly and uncorrelated instances. Therefore, 100 instances 
for each possible parameter combination were created, totaling 10800 instances.

Next session shows a hardness analysis using the results of the CPLEX tests. Then, the results of the tests with
the two implemented heuristics are discussed.

\subsection{Problem Hardness}

The first battery of tests involved solving the generated instances with the CPLEX solver, to check if they were
indeed hard to solve. During preliminary tests, while the CPEX solver was able to fastly solve some instances, it took 
a lot of time to solve some others. So to be able to continue the experiments, an alternative strategy was used: instances
that took more than 20 minutes to be solved were considered prohibitive for the continuity of the experiment, so when an 
instance reached this threshold, it was interrupted and the current solution saved.

Figure~\ref{fig:result10} shows the results of the first tests, 
considering the interruption rate. Each of the three heatmaps represents a level of correlation, while each heatmap 
shows the instances with certain number of resources. The colors on the heatmaps represent the ratio of interrupted 
instances during the CPLEX execution, varying from the combinations of actions/years where no instance was interrupted 
(paler squares) to the combinations where all instances were interrupted (darker squares).

\figpar
\begin{figure}[H]
  \centering
  \resizebox{\columnwidth}{!}{%
    \subfloat[1 resource]{% GNUPLOT: LaTeX picture using PSTRICKS macros
% Define new PST objects, if not already defined
\ifx\PSTloaded\undefined
\def\PSTloaded{t}

\catcode`@=11

\newpsobject{PST@Border}{psline}{linewidth=.0015,linestyle=solid}

\catcode`@=12

\fi
\psset{unit=5.0in,xunit=5.0in,yunit=3.0in}
\pspicture(0.000000,0.000000)(0.31, 0.35)
\ifx\nofigs\undefined
\catcode`@=11

\newrgbcolor{PST@COLOR0}{1 1 1}
\newrgbcolor{PST@COLOR1}{0.992 0.992 0.992}
\newrgbcolor{PST@COLOR3}{0.976 0.976 0.976}
\newrgbcolor{PST@COLOR6}{0.952 0.952 0.952}


\def\polypmIIId#1{\pspolygon[linestyle=none,fillstyle=solid,fillcolor=PST@COLOR#1]}

\polypmIIId{1}(0.1432,0.19)(0.0864,0.19)(0.0864,0.1078)(0.1432,0.1078)
\polypmIIId{0}(0.1432,0.272)(0.0864,0.272)(0.0864,0.19)(0.1432,0.19)
\polypmIIId{0}(0.1432,0.3542)(0.0864,0.3542)(0.0864,0.272)(0.1432,0.272)

\polypmIIId{1}(0.2,0.19)(0.1432,0.19)(0.1432,0.1078)(0.2,0.1078)
\polypmIIId{0}(0.2,0.272)(0.1432,0.272)(0.1432,0.19)(0.2,0.19)
\polypmIIId{0}(0.2,0.3542)(0.1432,0.3542)(0.1432,0.272)(0.2,0.272)

\polypmIIId{3}(0.2568,0.19)(0.2,0.19)(0.2,0.1078)(0.2568,0.1078)
\polypmIIId{0}(0.2568,0.272)(0.2,0.272)(0.2,0.19)(0.2568,0.19)
\polypmIIId{0}(0.2568,0.3542)(0.2,0.3542)(0.2,0.272)(0.2568,0.272)

\polypmIIId{6}(0.3136,0.19)(0.2568,0.19)(0.2568,0.1078)(0.3136,0.1078)
\polypmIIId{0}(0.3136,0.272)(0.2568,0.272)(0.2568,0.19)(0.3136,0.19)
\polypmIIId{0}(0.3136,0.3542)(0.2568,0.3542)(0.2568,0.272)(0.3136,0.272)

\rput(0.1148,0.07){3}
\rput(0.1716,0.07){4}
\rput(0.2284,0.07){5}
\rput(0.2852,0.07){6}
\rput(0.2000,0.0070){years}

\rput[r](0.0806,0.1489){25}
\rput[r](0.0806,0.2310){50}
\rput[r](0.0806,0.3131){100}
\rput{L}(0.0096,0.2310){actions}

\PST@Border(0.0864,0.3542)
(0.0864,0.1078)
(0.3136,0.1078)
(0.3136,0.3542)
(0.0864,0.3542)

\catcode`@=12
\fi
\endpspicture
}
    \subfloat[2 resources]{% GNUPLOT: LaTeX picture using PSTRICKS macros
% Define new PST objects, if not already defined
\ifx\PSTloaded\undefined
\def\PSTloaded{t}

\catcode`@=11

\newpsobject{PST@Border}{psline}{linewidth=.0015,linestyle=solid}

\catcode`@=12

\fi
\psset{unit=5.0in,xunit=5.0in,yunit=3.0in}
\pspicture(0.000000,0.000000)(0.225000,0.35)
\ifx\nofigs\undefined
\catcode`@=11

\newrgbcolor{PST@COLOR0}{1 1 1}
\newrgbcolor{PST@COLOR2}{0.984 0.984 0.984}
\newrgbcolor{PST@COLOR8}{0.937 0.937 0.937}
\newrgbcolor{PST@COLOR10}{0.921 0.921 0.921}
\newrgbcolor{PST@COLOR55}{0.566 0.566 0.566}
\newrgbcolor{PST@COLOR96}{0.244 0.244 0.244}
\newrgbcolor{PST@COLOR117}{0.078 0.078 0.078}

\def\polypmIIId#1{\pspolygon[linestyle=none,fillstyle=solid,fillcolor=PST@COLOR#1]}

\polypmIIId{55} (0.0568,0.19)  (0.0,0.19)  (0.0,0.1078)(0.0568,0.1078)
\polypmIIId{0}  (0.0568,0.272) (0.0,0.272) (0.0,0.19)  (0.0568,0.19)
\polypmIIId{0}  (0.0568,0.3542)(0.0,0.3542)(0.0,0.272) (0.0568,0.272)

\polypmIIId{96} (0.1136,   0.19)  (0.0568,0.19)  (0.0568,0.1078)(0.1136,0.1078)
\polypmIIId{2}  (0.1136,   0.272) (0.0568,0.272) (0.0568,0.19)  (0.1136,0.19)
\polypmIIId{0}  (0.1136,   0.3542)(0.0568,0.3542)(0.0568,0.272) (0.1136,0.272)

\polypmIIId{117}(0.1704,0.19)  (0.1136,   0.19)  (0.1136,   0.1078)(0.1704,0.1078)
\polypmIIId{10} (0.1704,0.272) (0.1136,   0.272) (0.1136,   0.19)  (0.1704,0.19)
\polypmIIId{0}  (0.1704,0.3542)(0.1136,   0.3542)(0.1136,   0.272) (0.1704,0.272)

\polypmIIId{117}(0.2272,0.19)  (0.1704,0.19)  (0.1704,0.1078)(0.2272,0.1078)
\polypmIIId{8}  (0.2272,0.272) (0.1704,0.272) (0.1704,0.19)  (0.2272,0.19)
\polypmIIId{0}  (0.2272,0.3542)(0.1704,0.3542)(0.1704,0.272) (0.2272,0.272)

\rput(0.0284,0.07){3}
\rput(0.0852,0.07){4}
\rput(0.1420,0.07){5}
\rput(0.1988,0.07){6}
\rput(0.1136,0.0070){years}


\PST@Border(0.0,0.3542)
(0.0,0.1078)
(0.2272,0.1078)
(0.2272,0.3542)
(0.0,0.3542)

\catcode`@=12
\fi
\endpspicture
}
    \subfloat[4 resources]{% GNUPLOT: LaTeX picture using PSTRICKS macros
% Define new PST objects, if not already defined
\ifx\PSTloaded\undefined
\def\PSTloaded{t}

\catcode`@=11

\newpsobject{PST@Border}{psline}{linewidth=.0015,linestyle=solid}

\catcode`@=12

\fi
\psset{unit=5.0in,xunit=5.0in,yunit=3.0in}
\pspicture(0.000000,0.000000)(0.3136,0.35)
\ifx\nofigs\undefined
\catcode`@=11

\newrgbcolor{PST@COLOR0}{1 1 1}
\newrgbcolor{PST@COLOR1}{0.992 0.992 0.992}
\newrgbcolor{PST@COLOR2}{0.984 0.984 0.984}
\newrgbcolor{PST@COLOR3}{0.976 0.976 0.976}
\newrgbcolor{PST@COLOR4}{0.968 0.968 0.968}
\newrgbcolor{PST@COLOR5}{0.96 0.96 0.96}
\newrgbcolor{PST@COLOR6}{0.952 0.952 0.952}
\newrgbcolor{PST@COLOR7}{0.944 0.944 0.944}
\newrgbcolor{PST@COLOR8}{0.937 0.937 0.937}
\newrgbcolor{PST@COLOR9}{0.929 0.929 0.929}
\newrgbcolor{PST@COLOR10}{0.921 0.921 0.921}
\newrgbcolor{PST@COLOR11}{0.913 0.913 0.913}
\newrgbcolor{PST@COLOR12}{0.905 0.905 0.905}
\newrgbcolor{PST@COLOR13}{0.897 0.897 0.897}
\newrgbcolor{PST@COLOR14}{0.889 0.889 0.889}
\newrgbcolor{PST@COLOR15}{0.881 0.881 0.881}
\newrgbcolor{PST@COLOR16}{0.874 0.874 0.874}
\newrgbcolor{PST@COLOR17}{0.866 0.866 0.866}
\newrgbcolor{PST@COLOR18}{0.858 0.858 0.858}
\newrgbcolor{PST@COLOR19}{0.85 0.85 0.85}
\newrgbcolor{PST@COLOR20}{0.842 0.842 0.842}
\newrgbcolor{PST@COLOR21}{0.834 0.834 0.834}
\newrgbcolor{PST@COLOR22}{0.826 0.826 0.826}
\newrgbcolor{PST@COLOR23}{0.818 0.818 0.818}
\newrgbcolor{PST@COLOR24}{0.811 0.811 0.811}
\newrgbcolor{PST@COLOR25}{0.803 0.803 0.803}
\newrgbcolor{PST@COLOR26}{0.795 0.795 0.795}
\newrgbcolor{PST@COLOR27}{0.787 0.787 0.787}
\newrgbcolor{PST@COLOR28}{0.779 0.779 0.779}
\newrgbcolor{PST@COLOR29}{0.771 0.771 0.771}
\newrgbcolor{PST@COLOR30}{0.763 0.763 0.763}
\newrgbcolor{PST@COLOR31}{0.755 0.755 0.755}
\newrgbcolor{PST@COLOR32}{0.748 0.748 0.748}
\newrgbcolor{PST@COLOR33}{0.74 0.74 0.74}
\newrgbcolor{PST@COLOR34}{0.732 0.732 0.732}
\newrgbcolor{PST@COLOR35}{0.724 0.724 0.724}
\newrgbcolor{PST@COLOR36}{0.716 0.716 0.716}
\newrgbcolor{PST@COLOR37}{0.708 0.708 0.708}
\newrgbcolor{PST@COLOR38}{0.7 0.7 0.7}
\newrgbcolor{PST@COLOR39}{0.692 0.692 0.692}
\newrgbcolor{PST@COLOR40}{0.685 0.685 0.685}
\newrgbcolor{PST@COLOR41}{0.677 0.677 0.677}
\newrgbcolor{PST@COLOR42}{0.669 0.669 0.669}
\newrgbcolor{PST@COLOR43}{0.661 0.661 0.661}
\newrgbcolor{PST@COLOR44}{0.653 0.653 0.653}
\newrgbcolor{PST@COLOR45}{0.645 0.645 0.645}
\newrgbcolor{PST@COLOR46}{0.637 0.637 0.637}
\newrgbcolor{PST@COLOR47}{0.629 0.629 0.629}
\newrgbcolor{PST@COLOR48}{0.622 0.622 0.622}
\newrgbcolor{PST@COLOR49}{0.614 0.614 0.614}
\newrgbcolor{PST@COLOR50}{0.606 0.606 0.606}
\newrgbcolor{PST@COLOR51}{0.598 0.598 0.598}
\newrgbcolor{PST@COLOR52}{0.59 0.59 0.59}
\newrgbcolor{PST@COLOR53}{0.582 0.582 0.582}
\newrgbcolor{PST@COLOR54}{0.574 0.574 0.574}
\newrgbcolor{PST@COLOR55}{0.566 0.566 0.566}
\newrgbcolor{PST@COLOR56}{0.559 0.559 0.559}
\newrgbcolor{PST@COLOR57}{0.551 0.551 0.551}
\newrgbcolor{PST@COLOR58}{0.543 0.543 0.543}
\newrgbcolor{PST@COLOR59}{0.535 0.535 0.535}
\newrgbcolor{PST@COLOR60}{0.527 0.527 0.527}
\newrgbcolor{PST@COLOR61}{0.519 0.519 0.519}
\newrgbcolor{PST@COLOR62}{0.511 0.511 0.511}
\newrgbcolor{PST@COLOR63}{0.503 0.503 0.503}
\newrgbcolor{PST@COLOR64}{0.496 0.496 0.496}
\newrgbcolor{PST@COLOR65}{0.488 0.488 0.488}
\newrgbcolor{PST@COLOR66}{0.48 0.48 0.48}
\newrgbcolor{PST@COLOR67}{0.472 0.472 0.472}
\newrgbcolor{PST@COLOR68}{0.464 0.464 0.464}
\newrgbcolor{PST@COLOR69}{0.456 0.456 0.456}
\newrgbcolor{PST@COLOR70}{0.448 0.448 0.448}
\newrgbcolor{PST@COLOR71}{0.44 0.44 0.44}
\newrgbcolor{PST@COLOR72}{0.433 0.433 0.433}
\newrgbcolor{PST@COLOR73}{0.425 0.425 0.425}
\newrgbcolor{PST@COLOR74}{0.417 0.417 0.417}
\newrgbcolor{PST@COLOR75}{0.409 0.409 0.409}
\newrgbcolor{PST@COLOR76}{0.401 0.401 0.401}
\newrgbcolor{PST@COLOR77}{0.393 0.393 0.393}
\newrgbcolor{PST@COLOR78}{0.385 0.385 0.385}
\newrgbcolor{PST@COLOR79}{0.377 0.377 0.377}
\newrgbcolor{PST@COLOR80}{0.37 0.37 0.37}
\newrgbcolor{PST@COLOR81}{0.362 0.362 0.362}
\newrgbcolor{PST@COLOR82}{0.354 0.354 0.354}
\newrgbcolor{PST@COLOR83}{0.346 0.346 0.346}
\newrgbcolor{PST@COLOR84}{0.338 0.338 0.338}
\newrgbcolor{PST@COLOR85}{0.33 0.33 0.33}
\newrgbcolor{PST@COLOR86}{0.322 0.322 0.322}
\newrgbcolor{PST@COLOR87}{0.314 0.314 0.314}
\newrgbcolor{PST@COLOR88}{0.307 0.307 0.307}
\newrgbcolor{PST@COLOR89}{0.299 0.299 0.299}
\newrgbcolor{PST@COLOR90}{0.291 0.291 0.291}
\newrgbcolor{PST@COLOR91}{0.283 0.283 0.283}
\newrgbcolor{PST@COLOR92}{0.275 0.275 0.275}
\newrgbcolor{PST@COLOR93}{0.267 0.267 0.267}
\newrgbcolor{PST@COLOR94}{0.259 0.259 0.259}
\newrgbcolor{PST@COLOR95}{0.251 0.251 0.251}
\newrgbcolor{PST@COLOR96}{0.244 0.244 0.244}
\newrgbcolor{PST@COLOR97}{0.236 0.236 0.236}
\newrgbcolor{PST@COLOR98}{0.228 0.228 0.228}
\newrgbcolor{PST@COLOR99}{0.22 0.22 0.22}
\newrgbcolor{PST@COLOR100}{0.212 0.212 0.212}
\newrgbcolor{PST@COLOR101}{0.204 0.204 0.204}
\newrgbcolor{PST@COLOR102}{0.196 0.196 0.196}
\newrgbcolor{PST@COLOR103}{0.188 0.188 0.188}
\newrgbcolor{PST@COLOR104}{0.181 0.181 0.181}
\newrgbcolor{PST@COLOR105}{0.173 0.173 0.173}
\newrgbcolor{PST@COLOR106}{0.165 0.165 0.165}
\newrgbcolor{PST@COLOR107}{0.157 0.157 0.157}
\newrgbcolor{PST@COLOR108}{0.149 0.149 0.149}
\newrgbcolor{PST@COLOR109}{0.141 0.141 0.141}
\newrgbcolor{PST@COLOR110}{0.133 0.133 0.133}
\newrgbcolor{PST@COLOR111}{0.125 0.125 0.125}
\newrgbcolor{PST@COLOR112}{0.118 0.118 0.118}
\newrgbcolor{PST@COLOR113}{0.11 0.11 0.11}
\newrgbcolor{PST@COLOR114}{0.102 0.102 0.102}
\newrgbcolor{PST@COLOR115}{0.094 0.094 0.094}
\newrgbcolor{PST@COLOR116}{0.086 0.086 0.086}
\newrgbcolor{PST@COLOR117}{0.078 0.078 0.078}
\newrgbcolor{PST@COLOR118}{0.07 0.07 0.07}
\newrgbcolor{PST@COLOR119}{0.062 0.062 0.062}
\newrgbcolor{PST@COLOR120}{0.055 0.055 0.055}
\newrgbcolor{PST@COLOR121}{0.047 0.047 0.047}
\newrgbcolor{PST@COLOR122}{0.039 0.039 0.039}
\newrgbcolor{PST@COLOR123}{0.031 0.031 0.031}
\newrgbcolor{PST@COLOR124}{0.023 0.023 0.023}
\newrgbcolor{PST@COLOR125}{0.015 0.015 0.015}
\newrgbcolor{PST@COLOR126}{0.007 0.007 0.007}
\newrgbcolor{PST@COLOR127}{0 0 0}

\def\polypmIIId#1{\pspolygon[linestyle=none,fillstyle=solid,fillcolor=PST@COLOR#1]}

\polypmIIId{127} (0.0568,0.19)  (0.0,0.19)  (0.0,0.1078)(0.0568,0.1078)
\polypmIIId{124}  (0.0568,0.272) (0.0,0.272) (0.0,0.19)  (0.0568,0.19)
\polypmIIId{0}  (0.0568,0.3542)(0.0,0.3542)(0.0,0.272) (0.0568,0.272)

\polypmIIId{127} (0.1136,   0.19)  (0.0568,0.19)  (0.0568,0.1078)(0.1136,0.1078)
\polypmIIId{127}  (0.1136,   0.272) (0.0568,0.272) (0.0568,0.19)  (0.1136,0.19)
\polypmIIId{3}  (0.1136,   0.3542)(0.0568,0.3542)(0.0568,0.272) (0.1136,0.272)

\polypmIIId{127}(0.1704,0.19)  (0.1136,   0.19)  (0.1136,   0.1078)(0.1704,0.1078)
\polypmIIId{127} (0.1704,0.272) (0.1136,   0.272) (0.1136,   0.19)  (0.1704,0.19)
\polypmIIId{10}  (0.1704,0.3542)(0.1136,   0.3542)(0.1136,   0.272) (0.1704,0.272)

\polypmIIId{127}(0.2272,0.19)  (0.1704,0.19)  (0.1704,0.1078)(0.2272,0.1078)
\polypmIIId{127}  (0.2272,0.272) (0.1704,0.272) (0.1704,0.19)  (0.2272,0.19)
\polypmIIId{15}  (0.2272,0.3542)(0.1704,0.3542)(0.1704,0.272) (0.2272,0.272)

\rput(0.0284,0.07){3}
\rput(0.0852,0.07){4}
\rput(0.1420,0.07){5}
\rput(0.1988,0.07){6}
\rput(0.1136,0.0070){years}

\PST@Border(0.0,0.3542)
(0.0,0.1078)
(0.2272,0.1078)
(0.2272,0.3542)
(0.0,0.3542)

\polypmIIId{0}(0.2329,0.1078)(0.2442,0.1078)(0.2442,0.1098)(0.2329,0.1098)
\polypmIIId{1}(0.2329,0.1097)(0.2442,0.1097)(0.2442,0.1117)(0.2329,0.1117)
\polypmIIId{2}(0.2329,0.1116)(0.2442,0.1116)(0.2442,0.1136)(0.2329,0.1136)
\polypmIIId{3}(0.2329,0.1135)(0.2442,0.1135)(0.2442,0.1156)(0.2329,0.1156)
\polypmIIId{4}(0.2329,0.1155)(0.2442,0.1155)(0.2442,0.1175)(0.2329,0.1175)
\polypmIIId{5}(0.2329,0.1174)(0.2442,0.1174)(0.2442,0.1194)(0.2329,0.1194)
\polypmIIId{6}(0.2329,0.1193)(0.2442,0.1193)(0.2442,0.1213)(0.2329,0.1213)
\polypmIIId{7}(0.2329,0.1212)(0.2442,0.1212)(0.2442,0.1233)(0.2329,0.1233)
\polypmIIId{8}(0.2329,0.1232)(0.2442,0.1232)(0.2442,0.1252)(0.2329,0.1252)
\polypmIIId{9}(0.2329,0.1251)(0.2442,0.1251)(0.2442,0.1271)(0.2329,0.1271)
\polypmIIId{10}(0.2329,0.127)(0.2442,0.127)(0.2442,0.129)(0.2329,0.129)
\polypmIIId{11}(0.2329,0.1289)(0.2442,0.1289)(0.2442,0.131)(0.2329,0.131)
\polypmIIId{12}(0.2329,0.1309)(0.2442,0.1309)(0.2442,0.1329)(0.2329,0.1329)
\polypmIIId{13}(0.2329,0.1328)(0.2442,0.1328)(0.2442,0.1348)(0.2329,0.1348)
\polypmIIId{14}(0.2329,0.1347)(0.2442,0.1347)(0.2442,0.1367)(0.2329,0.1367)
\polypmIIId{15}(0.2329,0.1366)(0.2442,0.1366)(0.2442,0.1387)(0.2329,0.1387)
\polypmIIId{16}(0.2329,0.1386)(0.2442,0.1386)(0.2442,0.1406)(0.2329,0.1406)
\polypmIIId{17}(0.2329,0.1405)(0.2442,0.1405)(0.2442,0.1425)(0.2329,0.1425)
\polypmIIId{18}(0.2329,0.1424)(0.2442,0.1424)(0.2442,0.1444)(0.2329,0.1444)
\polypmIIId{19}(0.2329,0.1443)(0.2442,0.1443)(0.2442,0.1464)(0.2329,0.1464)
\polypmIIId{20}(0.2329,0.1463)(0.2442,0.1463)(0.2442,0.1483)(0.2329,0.1483)
\polypmIIId{21}(0.2329,0.1482)(0.2442,0.1482)(0.2442,0.1502)(0.2329,0.1502)
\polypmIIId{22}(0.2329,0.1501)(0.2442,0.1501)(0.2442,0.1521)(0.2329,0.1521)
\polypmIIId{23}(0.2329,0.152)(0.2442,0.152)(0.2442,0.1541)(0.2329,0.1541)
\polypmIIId{24}(0.2329,0.154)(0.2442,0.154)(0.2442,0.156)(0.2329,0.156)
\polypmIIId{25}(0.2329,0.1559)(0.2442,0.1559)(0.2442,0.1579)(0.2329,0.1579)
\polypmIIId{26}(0.2329,0.1578)(0.2442,0.1578)(0.2442,0.1598)(0.2329,0.1598)
\polypmIIId{27}(0.2329,0.1597)(0.2442,0.1597)(0.2442,0.1618)(0.2329,0.1618)
\polypmIIId{28}(0.2329,0.1617)(0.2442,0.1617)(0.2442,0.1637)(0.2329,0.1637)
\polypmIIId{29}(0.2329,0.1636)(0.2442,0.1636)(0.2442,0.1656)(0.2329,0.1656)
\polypmIIId{30}(0.2329,0.1655)(0.2442,0.1655)(0.2442,0.1675)(0.2329,0.1675)
\polypmIIId{31}(0.2329,0.1674)(0.2442,0.1674)(0.2442,0.1695)(0.2329,0.1695)
\polypmIIId{32}(0.2329,0.1694)(0.2442,0.1694)(0.2442,0.1714)(0.2329,0.1714)
\polypmIIId{33}(0.2329,0.1713)(0.2442,0.1713)(0.2442,0.1733)(0.2329,0.1733)
\polypmIIId{34}(0.2329,0.1732)(0.2442,0.1732)(0.2442,0.1752)(0.2329,0.1752)
\polypmIIId{35}(0.2329,0.1751)(0.2442,0.1751)(0.2442,0.1772)(0.2329,0.1772)
\polypmIIId{36}(0.2329,0.1771)(0.2442,0.1771)(0.2442,0.1791)(0.2329,0.1791)
\polypmIIId{37}(0.2329,0.179)(0.2442,0.179)(0.2442,0.181)(0.2329,0.181)
\polypmIIId{38}(0.2329,0.1809)(0.2442,0.1809)(0.2442,0.1829)(0.2329,0.1829)
\polypmIIId{39}(0.2329,0.1828)(0.2442,0.1828)(0.2442,0.1849)(0.2329,0.1849)
\polypmIIId{40}(0.2329,0.1848)(0.2442,0.1848)(0.2442,0.1868)(0.2329,0.1868)
\polypmIIId{41}(0.2329,0.1867)(0.2442,0.1867)(0.2442,0.1887)(0.2329,0.1887)
\polypmIIId{42}(0.2329,0.1886)(0.2442,0.1886)(0.2442,0.1906)(0.2329,0.1906)
\polypmIIId{43}(0.2329,0.1905)(0.2442,0.1905)(0.2442,0.1926)(0.2329,0.1926)
\polypmIIId{44}(0.2329,0.1925)(0.2442,0.1925)(0.2442,0.1945)(0.2329,0.1945)
\polypmIIId{45}(0.2329,0.1944)(0.2442,0.1944)(0.2442,0.1964)(0.2329,0.1964)
\polypmIIId{46}(0.2329,0.1963)(0.2442,0.1963)(0.2442,0.1983)(0.2329,0.1983)
\polypmIIId{47}(0.2329,0.1982)(0.2442,0.1982)(0.2442,0.2003)(0.2329,0.2003)
\polypmIIId{48}(0.2329,0.2002)(0.2442,0.2002)(0.2442,0.2022)(0.2329,0.2022)
\polypmIIId{49}(0.2329,0.2021)(0.2442,0.2021)(0.2442,0.2041)(0.2329,0.2041)
\polypmIIId{50}(0.2329,0.204)(0.2442,0.204)(0.2442,0.206)(0.2329,0.206)
\polypmIIId{51}(0.2329,0.2059)(0.2442,0.2059)(0.2442,0.208)(0.2329,0.208)
\polypmIIId{52}(0.2329,0.2079)(0.2442,0.2079)(0.2442,0.2099)(0.2329,0.2099)
\polypmIIId{53}(0.2329,0.2098)(0.2442,0.2098)(0.2442,0.2118)(0.2329,0.2118)
\polypmIIId{54}(0.2329,0.2117)(0.2442,0.2117)(0.2442,0.2137)(0.2329,0.2137)
\polypmIIId{55}(0.2329,0.2136)(0.2442,0.2136)(0.2442,0.2157)(0.2329,0.2157)
\polypmIIId{56}(0.2329,0.2156)(0.2442,0.2156)(0.2442,0.2176)(0.2329,0.2176)
\polypmIIId{57}(0.2329,0.2175)(0.2442,0.2175)(0.2442,0.2195)(0.2329,0.2195)
\polypmIIId{58}(0.2329,0.2194)(0.2442,0.2194)(0.2442,0.2214)(0.2329,0.2214)
\polypmIIId{59}(0.2329,0.2213)(0.2442,0.2213)(0.2442,0.2234)(0.2329,0.2234)
\polypmIIId{60}(0.2329,0.2233)(0.2442,0.2233)(0.2442,0.2253)(0.2329,0.2253)
\polypmIIId{61}(0.2329,0.2252)(0.2442,0.2252)(0.2442,0.2272)(0.2329,0.2272)
\polypmIIId{62}(0.2329,0.2271)(0.2442,0.2271)(0.2442,0.2291)(0.2329,0.2291)
\polypmIIId{63}(0.2329,0.229)(0.2442,0.229)(0.2442,0.2311)(0.2329,0.2311)
\polypmIIId{64}(0.2329,0.231)(0.2442,0.231)(0.2442,0.233)(0.2329,0.233)
\polypmIIId{65}(0.2329,0.2329)(0.2442,0.2329)(0.2442,0.2349)(0.2329,0.2349)
\polypmIIId{66}(0.2329,0.2348)(0.2442,0.2348)(0.2442,0.2368)(0.2329,0.2368)
\polypmIIId{67}(0.2329,0.2367)(0.2442,0.2367)(0.2442,0.2388)(0.2329,0.2388)
\polypmIIId{68}(0.2329,0.2387)(0.2442,0.2387)(0.2442,0.2407)(0.2329,0.2407)
\polypmIIId{69}(0.2329,0.2406)(0.2442,0.2406)(0.2442,0.2426)(0.2329,0.2426)
\polypmIIId{70}(0.2329,0.2425)(0.2442,0.2425)(0.2442,0.2445)(0.2329,0.2445)
\polypmIIId{71}(0.2329,0.2444)(0.2442,0.2444)(0.2442,0.2465)(0.2329,0.2465)
\polypmIIId{72}(0.2329,0.2464)(0.2442,0.2464)(0.2442,0.2484)(0.2329,0.2484)
\polypmIIId{73}(0.2329,0.2483)(0.2442,0.2483)(0.2442,0.2503)(0.2329,0.2503)
\polypmIIId{74}(0.2329,0.2502)(0.2442,0.2502)(0.2442,0.2522)(0.2329,0.2522)
\polypmIIId{75}(0.2329,0.2521)(0.2442,0.2521)(0.2442,0.2542)(0.2329,0.2542)
\polypmIIId{76}(0.2329,0.2541)(0.2442,0.2541)(0.2442,0.2561)(0.2329,0.2561)
\polypmIIId{77}(0.2329,0.256)(0.2442,0.256)(0.2442,0.258)(0.2329,0.258)
\polypmIIId{78}(0.2329,0.2579)(0.2442,0.2579)(0.2442,0.2599)(0.2329,0.2599)
\polypmIIId{79}(0.2329,0.2598)(0.2442,0.2598)(0.2442,0.2619)(0.2329,0.2619)
\polypmIIId{80}(0.2329,0.2618)(0.2442,0.2618)(0.2442,0.2638)(0.2329,0.2638)
\polypmIIId{81}(0.2329,0.2637)(0.2442,0.2637)(0.2442,0.2657)(0.2329,0.2657)
\polypmIIId{82}(0.2329,0.2656)(0.2442,0.2656)(0.2442,0.2676)(0.2329,0.2676)
\polypmIIId{83}(0.2329,0.2675)(0.2442,0.2675)(0.2442,0.2696)(0.2329,0.2696)
\polypmIIId{84}(0.2329,0.2695)(0.2442,0.2695)(0.2442,0.2715)(0.2329,0.2715)
\polypmIIId{85}(0.2329,0.2714)(0.2442,0.2714)(0.2442,0.2734)(0.2329,0.2734)
\polypmIIId{86}(0.2329,0.2733)(0.2442,0.2733)(0.2442,0.2753)(0.2329,0.2753)
\polypmIIId{87}(0.2329,0.2752)(0.2442,0.2752)(0.2442,0.2773)(0.2329,0.2773)
\polypmIIId{88}(0.2329,0.2772)(0.2442,0.2772)(0.2442,0.2792)(0.2329,0.2792)
\polypmIIId{89}(0.2329,0.2791)(0.2442,0.2791)(0.2442,0.2811)(0.2329,0.2811)
\polypmIIId{90}(0.2329,0.281)(0.2442,0.281)(0.2442,0.283)(0.2329,0.283)
\polypmIIId{91}(0.2329,0.2829)(0.2442,0.2829)(0.2442,0.285)(0.2329,0.285)
\polypmIIId{92}(0.2329,0.2849)(0.2442,0.2849)(0.2442,0.2869)(0.2329,0.2869)
\polypmIIId{93}(0.2329,0.2868)(0.2442,0.2868)(0.2442,0.2888)(0.2329,0.2888)
\polypmIIId{94}(0.2329,0.2887)(0.2442,0.2887)(0.2442,0.2907)(0.2329,0.2907)
\polypmIIId{95}(0.2329,0.2906)(0.2442,0.2906)(0.2442,0.2927)(0.2329,0.2927)
\polypmIIId{96}(0.2329,0.2926)(0.2442,0.2926)(0.2442,0.2946)(0.2329,0.2946)
\polypmIIId{97}(0.2329,0.2945)(0.2442,0.2945)(0.2442,0.2965)(0.2329,0.2965)
\polypmIIId{98}(0.2329,0.2964)(0.2442,0.2964)(0.2442,0.2984)(0.2329,0.2984)
\polypmIIId{99}(0.2329,0.2983)(0.2442,0.2983)(0.2442,0.3004)(0.2329,0.3004)
\polypmIIId{100}(0.2329,0.3003)(0.2442,0.3003)(0.2442,0.3023)(0.2329,0.3023)
\polypmIIId{101}(0.2329,0.3022)(0.2442,0.3022)(0.2442,0.3042)(0.2329,0.3042)
\polypmIIId{102}(0.2329,0.3041)(0.2442,0.3041)(0.2442,0.3061)(0.2329,0.3061)
\polypmIIId{103}(0.2329,0.306)(0.2442,0.306)(0.2442,0.3081)(0.2329,0.3081)
\polypmIIId{104}(0.2329,0.308)(0.2442,0.308)(0.2442,0.31)(0.2329,0.31)
\polypmIIId{105}(0.2329,0.3099)(0.2442,0.3099)(0.2442,0.3119)(0.2329,0.3119)
\polypmIIId{106}(0.2329,0.3118)(0.2442,0.3118)(0.2442,0.3138)(0.2329,0.3138)
\polypmIIId{107}(0.2329,0.3137)(0.2442,0.3137)(0.2442,0.3158)(0.2329,0.3158)
\polypmIIId{108}(0.2329,0.3157)(0.2442,0.3157)(0.2442,0.3177)(0.2329,0.3177)
\polypmIIId{109}(0.2329,0.3176)(0.2442,0.3176)(0.2442,0.3196)(0.2329,0.3196)
\polypmIIId{110}(0.2329,0.3195)(0.2442,0.3195)(0.2442,0.3215)(0.2329,0.3215)
\polypmIIId{111}(0.2329,0.3214)(0.2442,0.3214)(0.2442,0.3235)(0.2329,0.3235)
\polypmIIId{112}(0.2329,0.3234)(0.2442,0.3234)(0.2442,0.3254)(0.2329,0.3254)
\polypmIIId{113}(0.2329,0.3253)(0.2442,0.3253)(0.2442,0.3273)(0.2329,0.3273)
\polypmIIId{114}(0.2329,0.3272)(0.2442,0.3272)(0.2442,0.3292)(0.2329,0.3292)
\polypmIIId{115}(0.2329,0.3291)(0.2442,0.3291)(0.2442,0.3312)(0.2329,0.3312)
\polypmIIId{116}(0.2329,0.3311)(0.2442,0.3311)(0.2442,0.3331)(0.2329,0.3331)
\polypmIIId{117}(0.2329,0.333)(0.2442,0.333)(0.2442,0.335)(0.2329,0.335)
\polypmIIId{118}(0.2329,0.3349)(0.2442,0.3349)(0.2442,0.3369)(0.2329,0.3369)
\polypmIIId{119}(0.2329,0.3368)(0.2442,0.3368)(0.2442,0.3389)(0.2329,0.3389)
\polypmIIId{120}(0.2329,0.3388)(0.2442,0.3388)(0.2442,0.3408)(0.2329,0.3408)
\polypmIIId{121}(0.2329,0.3407)(0.2442,0.3407)(0.2442,0.3427)(0.2329,0.3427)
\polypmIIId{122}(0.2329,0.3426)(0.2442,0.3426)(0.2442,0.3446)(0.2329,0.3446)
\polypmIIId{123}(0.2329,0.3445)(0.2442,0.3445)(0.2442,0.3466)(0.2329,0.3466)
\polypmIIId{124}(0.2329,0.3465)(0.2442,0.3465)(0.2442,0.3485)(0.2329,0.3485)
\polypmIIId{125}(0.2329,0.3484)(0.2442,0.3484)(0.2442,0.3504)(0.2329,0.3504)
\polypmIIId{126}(0.2329,0.3503)(0.2442,0.3503)(0.2442,0.3523)(0.2329,0.3523)
\polypmIIId{127}(0.2329,0.3522)(0.2442,0.3522)(0.2442,0.3542)(0.2329,0.3542)

\PST@Border(0.2329,0.1078)
(0.2442,0.1078)
(0.2442,0.3542)
(0.2329,0.3542)
(0.2329,0.1078)

\rput[l](0.2502,0.1078){0}
\rput[l](0.2502,0.1570){0.2}
\rput[l](0.2502,0.2063){0.4}
\rput[l](0.2502,0.2556){0.6}
\rput[l](0.2502,0.3049){0.8}
\rput[l](0.2502,0.3542){1}

\catcode`@=12
\fi
\endpspicture
}
  }
  $\alpha = 0.0$
  %\label{fig:result00}
\end{figure}

\figspaces
\begin{figure}[H]
  \centering
  \resizebox{\columnwidth}{!}{%
    \subfloat[1 resource]{% GNUPLOT: LaTeX picture using PSTRICKS macros
% Define new PST objects, if not already defined
\ifx\PSTloaded\undefined
\def\PSTloaded{t}

\catcode`@=11

\newpsobject{PST@Border}{psline}{linewidth=.0015,linestyle=solid}

\catcode`@=12

\fi
\psset{unit=5.0in,xunit=5.0in,yunit=3.0in}
\pspicture(0.000000,0.000000)(0.31, 0.35)
\ifx\nofigs\undefined
\catcode`@=11

\newrgbcolor{PST@COLOR0}{1 1 1}
\newrgbcolor{PST@COLOR1}{0.992 0.992 0.992}
\newrgbcolor{PST@COLOR3}{0.976 0.976 0.976}
\newrgbcolor{PST@COLOR6}{0.952 0.952 0.952}


\def\polypmIIId#1{\pspolygon[linestyle=none,fillstyle=solid,fillcolor=PST@COLOR#1]}

\polypmIIId{0}(0.1432,0.19)(0.0864,0.19)(0.0864,0.1078)(0.1432,0.1078)
\polypmIIId{0}(0.1432,0.272)(0.0864,0.272)(0.0864,0.19)(0.1432,0.19)
\polypmIIId{0}(0.1432,0.3542)(0.0864,0.3542)(0.0864,0.272)(0.1432,0.272)

\polypmIIId{0}(0.2,0.19)(0.1432,0.19)(0.1432,0.1078)(0.2,0.1078)
\polypmIIId{0}(0.2,0.272)(0.1432,0.272)(0.1432,0.19)(0.2,0.19)
\polypmIIId{0}(0.2,0.3542)(0.1432,0.3542)(0.1432,0.272)(0.2,0.272)

\polypmIIId{1}(0.2568,0.19)(0.2,0.19)(0.2,0.1078)(0.2568,0.1078)
\polypmIIId{0}(0.2568,0.272)(0.2,0.272)(0.2,0.19)(0.2568,0.19)
\polypmIIId{0}(0.2568,0.3542)(0.2,0.3542)(0.2,0.272)(0.2568,0.272)

\polypmIIId{0}(0.3136,0.19)(0.2568,0.19)(0.2568,0.1078)(0.3136,0.1078)
\polypmIIId{0}(0.3136,0.272)(0.2568,0.272)(0.2568,0.19)(0.3136,0.19)
\polypmIIId{0}(0.3136,0.3542)(0.2568,0.3542)(0.2568,0.272)(0.3136,0.272)

\rput(0.1148,0.07){3}
\rput(0.1716,0.07){4}
\rput(0.2284,0.07){5}
\rput(0.2852,0.07){6}
\rput(0.2000,0.0070){years}

\rput[r](0.0806,0.1489){25}
\rput[r](0.0806,0.2310){50}
\rput[r](0.0806,0.3131){100}
\rput{L}(0.0096,0.2310){actions}

\PST@Border(0.0864,0.3542)
(0.0864,0.1078)
(0.3136,0.1078)
(0.3136,0.3542)
(0.0864,0.3542)

\catcode`@=12
\fi
\endpspicture}
    \subfloat[2 resources]{% GNUPLOT: LaTeX picture using PSTRICKS macros
% Define new PST objects, if not already defined
\ifx\PSTloaded\undefined
\def\PSTloaded{t}

\catcode`@=11

\newpsobject{PST@Border}{psline}{linewidth=.0015,linestyle=solid}

\catcode`@=12

\fi
\psset{unit=5.0in,xunit=5.0in,yunit=3.0in}
\pspicture(0.000000,0.000000)(0.225000,0.35)
\ifx\nofigs\undefined
\catcode`@=11

\newrgbcolor{PST@COLOR0}{1 1 1}
\newrgbcolor{PST@COLOR2}{0.984 0.984 0.984}
\newrgbcolor{PST@COLOR8}{0.937 0.937 0.937}
\newrgbcolor{PST@COLOR10}{0.921 0.921 0.921}
\newrgbcolor{PST@COLOR55}{0.566 0.566 0.566}
\newrgbcolor{PST@COLOR96}{0.244 0.244 0.244}
\newrgbcolor{PST@COLOR117}{0.078 0.078 0.078}

\def\polypmIIId#1{\pspolygon[linestyle=none,fillstyle=solid,fillcolor=PST@COLOR#1]}

\polypmIIId{32} (0.0568,0.19)  (0.0,0.19)  (0.0,0.1078)(0.0568,0.1078)
\polypmIIId{6}  (0.0568,0.272) (0.0,0.272) (0.0,0.19)  (0.0568,0.19)
\polypmIIId{0}  (0.0568,0.3942)(0.0,0.3942)(0.0,0.272) (0.0568,0.272)

\polypmIIId{83} (0.1136,   0.19)  (0.0568,0.19)  (0.0568,0.1078)(0.1136,0.1078)
\polypmIIId{25}  (0.1136,   0.272) (0.0568,0.272) (0.0568,0.19)  (0.1136,0.19)
\polypmIIId{0}  (0.1136,   0.3942)(0.0568,0.3942)(0.0568,0.272) (0.1136,0.272)

\polypmIIId{108}(0.1704,0.19)  (0.1136,   0.19)  (0.1136,   0.1078)(0.1704,0.1078)
\polypmIIId{46} (0.1704,0.272) (0.1136,   0.272) (0.1136,   0.19)  (0.1704,0.19)
\polypmIIId{0}  (0.1704,0.3942)(0.1136,   0.3942)(0.1136,   0.272) (0.1704,0.272)

\polypmIIId{124}(0.2272,0.19)  (0.1704,0.19)  (0.1704,0.1078)(0.2272,0.1078)
\polypmIIId{65}  (0.2272,0.272) (0.1704,0.272) (0.1704,0.19)  (0.2272,0.19)
\polypmIIId{0}  (0.2272,0.3942)(0.1704,0.3942)(0.1704,0.272) (0.2272,0.272)


\rput(0.0284,0.07){3}
\rput(0.0852,0.07){4}
\rput(0.1420,0.07){5}
\rput(0.1988,0.07){6}
\rput(0.1136,0.0070){years}


\PST@Border(0.0,0.3542)
(0.0,0.1078)
(0.2272,0.1078)
(0.2272,0.3542)
(0.0,0.3542)

\catcode`@=12
\fi
\endpspicture}
    \subfloat[4 resources]{% GNUPLOT: LaTeX picture using PSTRICKS macros
% Define new PST objects, if not already defined
\ifx\PSTloaded\undefined
\def\PSTloaded{t}

\catcode`@=11

\newpsobject{PST@Border}{psline}{linewidth=.0015,linestyle=solid}

\catcode`@=12

\fi
\psset{unit=5.0in,xunit=5.0in,yunit=3.0in}
\pspicture(0.000000,0.000000)(0.3136,0.35)
\ifx\nofigs\undefined
\catcode`@=11

\newrgbcolor{PST@COLOR0}{1 1 1}
\newrgbcolor{PST@COLOR1}{0.992 0.992 0.992}
\newrgbcolor{PST@COLOR2}{0.984 0.984 0.984}
\newrgbcolor{PST@COLOR3}{0.976 0.976 0.976}
\newrgbcolor{PST@COLOR4}{0.968 0.968 0.968}
\newrgbcolor{PST@COLOR5}{0.96 0.96 0.96}
\newrgbcolor{PST@COLOR6}{0.952 0.952 0.952}
\newrgbcolor{PST@COLOR7}{0.944 0.944 0.944}
\newrgbcolor{PST@COLOR8}{0.937 0.937 0.937}
\newrgbcolor{PST@COLOR9}{0.929 0.929 0.929}
\newrgbcolor{PST@COLOR10}{0.921 0.921 0.921}
\newrgbcolor{PST@COLOR11}{0.913 0.913 0.913}
\newrgbcolor{PST@COLOR12}{0.905 0.905 0.905}
\newrgbcolor{PST@COLOR13}{0.897 0.897 0.897}
\newrgbcolor{PST@COLOR14}{0.889 0.889 0.889}
\newrgbcolor{PST@COLOR15}{0.881 0.881 0.881}
\newrgbcolor{PST@COLOR16}{0.874 0.874 0.874}
\newrgbcolor{PST@COLOR17}{0.866 0.866 0.866}
\newrgbcolor{PST@COLOR18}{0.858 0.858 0.858}
\newrgbcolor{PST@COLOR19}{0.85 0.85 0.85}
\newrgbcolor{PST@COLOR20}{0.842 0.842 0.842}
\newrgbcolor{PST@COLOR21}{0.834 0.834 0.834}
\newrgbcolor{PST@COLOR22}{0.826 0.826 0.826}
\newrgbcolor{PST@COLOR23}{0.818 0.818 0.818}
\newrgbcolor{PST@COLOR24}{0.811 0.811 0.811}
\newrgbcolor{PST@COLOR25}{0.803 0.803 0.803}
\newrgbcolor{PST@COLOR26}{0.795 0.795 0.795}
\newrgbcolor{PST@COLOR27}{0.787 0.787 0.787}
\newrgbcolor{PST@COLOR28}{0.779 0.779 0.779}
\newrgbcolor{PST@COLOR29}{0.771 0.771 0.771}
\newrgbcolor{PST@COLOR30}{0.763 0.763 0.763}
\newrgbcolor{PST@COLOR31}{0.755 0.755 0.755}
\newrgbcolor{PST@COLOR32}{0.748 0.748 0.748}
\newrgbcolor{PST@COLOR33}{0.74 0.74 0.74}
\newrgbcolor{PST@COLOR34}{0.732 0.732 0.732}
\newrgbcolor{PST@COLOR35}{0.724 0.724 0.724}
\newrgbcolor{PST@COLOR36}{0.716 0.716 0.716}
\newrgbcolor{PST@COLOR37}{0.708 0.708 0.708}
\newrgbcolor{PST@COLOR38}{0.7 0.7 0.7}
\newrgbcolor{PST@COLOR39}{0.692 0.692 0.692}
\newrgbcolor{PST@COLOR40}{0.685 0.685 0.685}
\newrgbcolor{PST@COLOR41}{0.677 0.677 0.677}
\newrgbcolor{PST@COLOR42}{0.669 0.669 0.669}
\newrgbcolor{PST@COLOR43}{0.661 0.661 0.661}
\newrgbcolor{PST@COLOR44}{0.653 0.653 0.653}
\newrgbcolor{PST@COLOR45}{0.645 0.645 0.645}
\newrgbcolor{PST@COLOR46}{0.637 0.637 0.637}
\newrgbcolor{PST@COLOR47}{0.629 0.629 0.629}
\newrgbcolor{PST@COLOR48}{0.622 0.622 0.622}
\newrgbcolor{PST@COLOR49}{0.614 0.614 0.614}
\newrgbcolor{PST@COLOR50}{0.606 0.606 0.606}
\newrgbcolor{PST@COLOR51}{0.598 0.598 0.598}
\newrgbcolor{PST@COLOR52}{0.59 0.59 0.59}
\newrgbcolor{PST@COLOR53}{0.582 0.582 0.582}
\newrgbcolor{PST@COLOR54}{0.574 0.574 0.574}
\newrgbcolor{PST@COLOR55}{0.566 0.566 0.566}
\newrgbcolor{PST@COLOR56}{0.559 0.559 0.559}
\newrgbcolor{PST@COLOR57}{0.551 0.551 0.551}
\newrgbcolor{PST@COLOR58}{0.543 0.543 0.543}
\newrgbcolor{PST@COLOR59}{0.535 0.535 0.535}
\newrgbcolor{PST@COLOR60}{0.527 0.527 0.527}
\newrgbcolor{PST@COLOR61}{0.519 0.519 0.519}
\newrgbcolor{PST@COLOR62}{0.511 0.511 0.511}
\newrgbcolor{PST@COLOR63}{0.503 0.503 0.503}
\newrgbcolor{PST@COLOR64}{0.496 0.496 0.496}
\newrgbcolor{PST@COLOR65}{0.488 0.488 0.488}
\newrgbcolor{PST@COLOR66}{0.48 0.48 0.48}
\newrgbcolor{PST@COLOR67}{0.472 0.472 0.472}
\newrgbcolor{PST@COLOR68}{0.464 0.464 0.464}
\newrgbcolor{PST@COLOR69}{0.456 0.456 0.456}
\newrgbcolor{PST@COLOR70}{0.448 0.448 0.448}
\newrgbcolor{PST@COLOR71}{0.44 0.44 0.44}
\newrgbcolor{PST@COLOR72}{0.433 0.433 0.433}
\newrgbcolor{PST@COLOR73}{0.425 0.425 0.425}
\newrgbcolor{PST@COLOR74}{0.417 0.417 0.417}
\newrgbcolor{PST@COLOR75}{0.409 0.409 0.409}
\newrgbcolor{PST@COLOR76}{0.401 0.401 0.401}
\newrgbcolor{PST@COLOR77}{0.393 0.393 0.393}
\newrgbcolor{PST@COLOR78}{0.385 0.385 0.385}
\newrgbcolor{PST@COLOR79}{0.377 0.377 0.377}
\newrgbcolor{PST@COLOR80}{0.37 0.37 0.37}
\newrgbcolor{PST@COLOR81}{0.362 0.362 0.362}
\newrgbcolor{PST@COLOR82}{0.354 0.354 0.354}
\newrgbcolor{PST@COLOR83}{0.346 0.346 0.346}
\newrgbcolor{PST@COLOR84}{0.338 0.338 0.338}
\newrgbcolor{PST@COLOR85}{0.33 0.33 0.33}
\newrgbcolor{PST@COLOR86}{0.322 0.322 0.322}
\newrgbcolor{PST@COLOR87}{0.314 0.314 0.314}
\newrgbcolor{PST@COLOR88}{0.307 0.307 0.307}
\newrgbcolor{PST@COLOR89}{0.299 0.299 0.299}
\newrgbcolor{PST@COLOR90}{0.291 0.291 0.291}
\newrgbcolor{PST@COLOR91}{0.283 0.283 0.283}
\newrgbcolor{PST@COLOR92}{0.275 0.275 0.275}
\newrgbcolor{PST@COLOR93}{0.267 0.267 0.267}
\newrgbcolor{PST@COLOR94}{0.259 0.259 0.259}
\newrgbcolor{PST@COLOR95}{0.251 0.251 0.251}
\newrgbcolor{PST@COLOR96}{0.244 0.244 0.244}
\newrgbcolor{PST@COLOR97}{0.236 0.236 0.236}
\newrgbcolor{PST@COLOR98}{0.228 0.228 0.228}
\newrgbcolor{PST@COLOR99}{0.22 0.22 0.22}
\newrgbcolor{PST@COLOR100}{0.212 0.212 0.212}
\newrgbcolor{PST@COLOR101}{0.204 0.204 0.204}
\newrgbcolor{PST@COLOR102}{0.196 0.196 0.196}
\newrgbcolor{PST@COLOR103}{0.188 0.188 0.188}
\newrgbcolor{PST@COLOR104}{0.181 0.181 0.181}
\newrgbcolor{PST@COLOR105}{0.173 0.173 0.173}
\newrgbcolor{PST@COLOR106}{0.165 0.165 0.165}
\newrgbcolor{PST@COLOR107}{0.157 0.157 0.157}
\newrgbcolor{PST@COLOR108}{0.149 0.149 0.149}
\newrgbcolor{PST@COLOR109}{0.141 0.141 0.141}
\newrgbcolor{PST@COLOR110}{0.133 0.133 0.133}
\newrgbcolor{PST@COLOR111}{0.125 0.125 0.125}
\newrgbcolor{PST@COLOR112}{0.118 0.118 0.118}
\newrgbcolor{PST@COLOR113}{0.11 0.11 0.11}
\newrgbcolor{PST@COLOR114}{0.102 0.102 0.102}
\newrgbcolor{PST@COLOR115}{0.094 0.094 0.094}
\newrgbcolor{PST@COLOR116}{0.086 0.086 0.086}
\newrgbcolor{PST@COLOR117}{0.078 0.078 0.078}
\newrgbcolor{PST@COLOR118}{0.07 0.07 0.07}
\newrgbcolor{PST@COLOR119}{0.062 0.062 0.062}
\newrgbcolor{PST@COLOR120}{0.055 0.055 0.055}
\newrgbcolor{PST@COLOR121}{0.047 0.047 0.047}
\newrgbcolor{PST@COLOR122}{0.039 0.039 0.039}
\newrgbcolor{PST@COLOR123}{0.031 0.031 0.031}
\newrgbcolor{PST@COLOR124}{0.023 0.023 0.023}
\newrgbcolor{PST@COLOR125}{0.015 0.015 0.015}
\newrgbcolor{PST@COLOR126}{0.007 0.007 0.007}
\newrgbcolor{PST@COLOR127}{0 0 0}

\def\polypmIIId#1{\pspolygon[linestyle=none,fillstyle=solid,fillcolor=PST@COLOR#1]}

\polypmIIId{125} (0.0568,0.19)  (0.0,0.19)  (0.0,0.1078)(0.0568,0.1078)
\polypmIIId{127}  (0.0568,0.272) (0.0,0.272) (0.0,0.19)  (0.0568,0.19)
\polypmIIId{74}  (0.0568,0.3542)(0.0,0.3542)(0.0,0.272) (0.0568,0.272)

\polypmIIId{127} (0.1136,   0.19)  (0.0568,0.19)  (0.0568,0.1078)(0.1136,0.1078)
\polypmIIId{127}  (0.1136,   0.272) (0.0568,0.272) (0.0568,0.19)  (0.1136,0.19)
\polypmIIId{74}  (0.1136,   0.3542)(0.0568,0.3542)(0.0568,0.272) (0.1136,0.272)

\polypmIIId{127}(0.1704,0.19)  (0.1136,   0.19)  (0.1136,   0.1078)(0.1704,0.1078)
\polypmIIId{127} (0.1704,0.272) (0.1136,   0.272) (0.1136,   0.19)  (0.1704,0.19)
\polypmIIId{125}  (0.1704,0.3542)(0.1136,   0.3542)(0.1136,   0.272) (0.1704,0.272)

\polypmIIId{127}(0.2272,0.19)  (0.1704,0.19)  (0.1704,0.1078)(0.2272,0.1078)
\polypmIIId{127}  (0.2272,0.272) (0.1704,0.272) (0.1704,0.19)  (0.2272,0.19)
\polypmIIId{127}  (0.2272,0.3542)(0.1704,0.3542)(0.1704,0.272) (0.2272,0.272)

\rput(0.0284,0.07){3}
\rput(0.0852,0.07){4}
\rput(0.1420,0.07){5}
\rput(0.1988,0.07){6}
\rput(0.1136,0.0070){years}

\PST@Border(0.0,0.3542)
(0.0,0.1078)
(0.2272,0.1078)
(0.2272,0.3542)
(0.0,0.3542)

\polypmIIId{0}(0.2329,0.1078)(0.2442,0.1078)(0.2442,0.1098)(0.2329,0.1098)
\polypmIIId{1}(0.2329,0.1097)(0.2442,0.1097)(0.2442,0.1117)(0.2329,0.1117)
\polypmIIId{2}(0.2329,0.1116)(0.2442,0.1116)(0.2442,0.1136)(0.2329,0.1136)
\polypmIIId{3}(0.2329,0.1135)(0.2442,0.1135)(0.2442,0.1156)(0.2329,0.1156)
\polypmIIId{4}(0.2329,0.1155)(0.2442,0.1155)(0.2442,0.1175)(0.2329,0.1175)
\polypmIIId{5}(0.2329,0.1174)(0.2442,0.1174)(0.2442,0.1194)(0.2329,0.1194)
\polypmIIId{6}(0.2329,0.1193)(0.2442,0.1193)(0.2442,0.1213)(0.2329,0.1213)
\polypmIIId{7}(0.2329,0.1212)(0.2442,0.1212)(0.2442,0.1233)(0.2329,0.1233)
\polypmIIId{8}(0.2329,0.1232)(0.2442,0.1232)(0.2442,0.1252)(0.2329,0.1252)
\polypmIIId{9}(0.2329,0.1251)(0.2442,0.1251)(0.2442,0.1271)(0.2329,0.1271)
\polypmIIId{10}(0.2329,0.127)(0.2442,0.127)(0.2442,0.129)(0.2329,0.129)
\polypmIIId{11}(0.2329,0.1289)(0.2442,0.1289)(0.2442,0.131)(0.2329,0.131)
\polypmIIId{12}(0.2329,0.1309)(0.2442,0.1309)(0.2442,0.1329)(0.2329,0.1329)
\polypmIIId{13}(0.2329,0.1328)(0.2442,0.1328)(0.2442,0.1348)(0.2329,0.1348)
\polypmIIId{14}(0.2329,0.1347)(0.2442,0.1347)(0.2442,0.1367)(0.2329,0.1367)
\polypmIIId{15}(0.2329,0.1366)(0.2442,0.1366)(0.2442,0.1387)(0.2329,0.1387)
\polypmIIId{16}(0.2329,0.1386)(0.2442,0.1386)(0.2442,0.1406)(0.2329,0.1406)
\polypmIIId{17}(0.2329,0.1405)(0.2442,0.1405)(0.2442,0.1425)(0.2329,0.1425)
\polypmIIId{18}(0.2329,0.1424)(0.2442,0.1424)(0.2442,0.1444)(0.2329,0.1444)
\polypmIIId{19}(0.2329,0.1443)(0.2442,0.1443)(0.2442,0.1464)(0.2329,0.1464)
\polypmIIId{20}(0.2329,0.1463)(0.2442,0.1463)(0.2442,0.1483)(0.2329,0.1483)
\polypmIIId{21}(0.2329,0.1482)(0.2442,0.1482)(0.2442,0.1502)(0.2329,0.1502)
\polypmIIId{22}(0.2329,0.1501)(0.2442,0.1501)(0.2442,0.1521)(0.2329,0.1521)
\polypmIIId{23}(0.2329,0.152)(0.2442,0.152)(0.2442,0.1541)(0.2329,0.1541)
\polypmIIId{24}(0.2329,0.154)(0.2442,0.154)(0.2442,0.156)(0.2329,0.156)
\polypmIIId{25}(0.2329,0.1559)(0.2442,0.1559)(0.2442,0.1579)(0.2329,0.1579)
\polypmIIId{26}(0.2329,0.1578)(0.2442,0.1578)(0.2442,0.1598)(0.2329,0.1598)
\polypmIIId{27}(0.2329,0.1597)(0.2442,0.1597)(0.2442,0.1618)(0.2329,0.1618)
\polypmIIId{28}(0.2329,0.1617)(0.2442,0.1617)(0.2442,0.1637)(0.2329,0.1637)
\polypmIIId{29}(0.2329,0.1636)(0.2442,0.1636)(0.2442,0.1656)(0.2329,0.1656)
\polypmIIId{30}(0.2329,0.1655)(0.2442,0.1655)(0.2442,0.1675)(0.2329,0.1675)
\polypmIIId{31}(0.2329,0.1674)(0.2442,0.1674)(0.2442,0.1695)(0.2329,0.1695)
\polypmIIId{32}(0.2329,0.1694)(0.2442,0.1694)(0.2442,0.1714)(0.2329,0.1714)
\polypmIIId{33}(0.2329,0.1713)(0.2442,0.1713)(0.2442,0.1733)(0.2329,0.1733)
\polypmIIId{34}(0.2329,0.1732)(0.2442,0.1732)(0.2442,0.1752)(0.2329,0.1752)
\polypmIIId{35}(0.2329,0.1751)(0.2442,0.1751)(0.2442,0.1772)(0.2329,0.1772)
\polypmIIId{36}(0.2329,0.1771)(0.2442,0.1771)(0.2442,0.1791)(0.2329,0.1791)
\polypmIIId{37}(0.2329,0.179)(0.2442,0.179)(0.2442,0.181)(0.2329,0.181)
\polypmIIId{38}(0.2329,0.1809)(0.2442,0.1809)(0.2442,0.1829)(0.2329,0.1829)
\polypmIIId{39}(0.2329,0.1828)(0.2442,0.1828)(0.2442,0.1849)(0.2329,0.1849)
\polypmIIId{40}(0.2329,0.1848)(0.2442,0.1848)(0.2442,0.1868)(0.2329,0.1868)
\polypmIIId{41}(0.2329,0.1867)(0.2442,0.1867)(0.2442,0.1887)(0.2329,0.1887)
\polypmIIId{42}(0.2329,0.1886)(0.2442,0.1886)(0.2442,0.1906)(0.2329,0.1906)
\polypmIIId{43}(0.2329,0.1905)(0.2442,0.1905)(0.2442,0.1926)(0.2329,0.1926)
\polypmIIId{44}(0.2329,0.1925)(0.2442,0.1925)(0.2442,0.1945)(0.2329,0.1945)
\polypmIIId{45}(0.2329,0.1944)(0.2442,0.1944)(0.2442,0.1964)(0.2329,0.1964)
\polypmIIId{46}(0.2329,0.1963)(0.2442,0.1963)(0.2442,0.1983)(0.2329,0.1983)
\polypmIIId{47}(0.2329,0.1982)(0.2442,0.1982)(0.2442,0.2003)(0.2329,0.2003)
\polypmIIId{48}(0.2329,0.2002)(0.2442,0.2002)(0.2442,0.2022)(0.2329,0.2022)
\polypmIIId{49}(0.2329,0.2021)(0.2442,0.2021)(0.2442,0.2041)(0.2329,0.2041)
\polypmIIId{50}(0.2329,0.204)(0.2442,0.204)(0.2442,0.206)(0.2329,0.206)
\polypmIIId{51}(0.2329,0.2059)(0.2442,0.2059)(0.2442,0.208)(0.2329,0.208)
\polypmIIId{52}(0.2329,0.2079)(0.2442,0.2079)(0.2442,0.2099)(0.2329,0.2099)
\polypmIIId{53}(0.2329,0.2098)(0.2442,0.2098)(0.2442,0.2118)(0.2329,0.2118)
\polypmIIId{54}(0.2329,0.2117)(0.2442,0.2117)(0.2442,0.2137)(0.2329,0.2137)
\polypmIIId{55}(0.2329,0.2136)(0.2442,0.2136)(0.2442,0.2157)(0.2329,0.2157)
\polypmIIId{56}(0.2329,0.2156)(0.2442,0.2156)(0.2442,0.2176)(0.2329,0.2176)
\polypmIIId{57}(0.2329,0.2175)(0.2442,0.2175)(0.2442,0.2195)(0.2329,0.2195)
\polypmIIId{58}(0.2329,0.2194)(0.2442,0.2194)(0.2442,0.2214)(0.2329,0.2214)
\polypmIIId{59}(0.2329,0.2213)(0.2442,0.2213)(0.2442,0.2234)(0.2329,0.2234)
\polypmIIId{60}(0.2329,0.2233)(0.2442,0.2233)(0.2442,0.2253)(0.2329,0.2253)
\polypmIIId{61}(0.2329,0.2252)(0.2442,0.2252)(0.2442,0.2272)(0.2329,0.2272)
\polypmIIId{62}(0.2329,0.2271)(0.2442,0.2271)(0.2442,0.2291)(0.2329,0.2291)
\polypmIIId{63}(0.2329,0.229)(0.2442,0.229)(0.2442,0.2311)(0.2329,0.2311)
\polypmIIId{64}(0.2329,0.231)(0.2442,0.231)(0.2442,0.233)(0.2329,0.233)
\polypmIIId{65}(0.2329,0.2329)(0.2442,0.2329)(0.2442,0.2349)(0.2329,0.2349)
\polypmIIId{66}(0.2329,0.2348)(0.2442,0.2348)(0.2442,0.2368)(0.2329,0.2368)
\polypmIIId{67}(0.2329,0.2367)(0.2442,0.2367)(0.2442,0.2388)(0.2329,0.2388)
\polypmIIId{68}(0.2329,0.2387)(0.2442,0.2387)(0.2442,0.2407)(0.2329,0.2407)
\polypmIIId{69}(0.2329,0.2406)(0.2442,0.2406)(0.2442,0.2426)(0.2329,0.2426)
\polypmIIId{70}(0.2329,0.2425)(0.2442,0.2425)(0.2442,0.2445)(0.2329,0.2445)
\polypmIIId{71}(0.2329,0.2444)(0.2442,0.2444)(0.2442,0.2465)(0.2329,0.2465)
\polypmIIId{72}(0.2329,0.2464)(0.2442,0.2464)(0.2442,0.2484)(0.2329,0.2484)
\polypmIIId{73}(0.2329,0.2483)(0.2442,0.2483)(0.2442,0.2503)(0.2329,0.2503)
\polypmIIId{74}(0.2329,0.2502)(0.2442,0.2502)(0.2442,0.2522)(0.2329,0.2522)
\polypmIIId{75}(0.2329,0.2521)(0.2442,0.2521)(0.2442,0.2542)(0.2329,0.2542)
\polypmIIId{76}(0.2329,0.2541)(0.2442,0.2541)(0.2442,0.2561)(0.2329,0.2561)
\polypmIIId{77}(0.2329,0.256)(0.2442,0.256)(0.2442,0.258)(0.2329,0.258)
\polypmIIId{78}(0.2329,0.2579)(0.2442,0.2579)(0.2442,0.2599)(0.2329,0.2599)
\polypmIIId{79}(0.2329,0.2598)(0.2442,0.2598)(0.2442,0.2619)(0.2329,0.2619)
\polypmIIId{80}(0.2329,0.2618)(0.2442,0.2618)(0.2442,0.2638)(0.2329,0.2638)
\polypmIIId{81}(0.2329,0.2637)(0.2442,0.2637)(0.2442,0.2657)(0.2329,0.2657)
\polypmIIId{82}(0.2329,0.2656)(0.2442,0.2656)(0.2442,0.2676)(0.2329,0.2676)
\polypmIIId{83}(0.2329,0.2675)(0.2442,0.2675)(0.2442,0.2696)(0.2329,0.2696)
\polypmIIId{84}(0.2329,0.2695)(0.2442,0.2695)(0.2442,0.2715)(0.2329,0.2715)
\polypmIIId{85}(0.2329,0.2714)(0.2442,0.2714)(0.2442,0.2734)(0.2329,0.2734)
\polypmIIId{86}(0.2329,0.2733)(0.2442,0.2733)(0.2442,0.2753)(0.2329,0.2753)
\polypmIIId{87}(0.2329,0.2752)(0.2442,0.2752)(0.2442,0.2773)(0.2329,0.2773)
\polypmIIId{88}(0.2329,0.2772)(0.2442,0.2772)(0.2442,0.2792)(0.2329,0.2792)
\polypmIIId{89}(0.2329,0.2791)(0.2442,0.2791)(0.2442,0.2811)(0.2329,0.2811)
\polypmIIId{90}(0.2329,0.281)(0.2442,0.281)(0.2442,0.283)(0.2329,0.283)
\polypmIIId{91}(0.2329,0.2829)(0.2442,0.2829)(0.2442,0.285)(0.2329,0.285)
\polypmIIId{92}(0.2329,0.2849)(0.2442,0.2849)(0.2442,0.2869)(0.2329,0.2869)
\polypmIIId{93}(0.2329,0.2868)(0.2442,0.2868)(0.2442,0.2888)(0.2329,0.2888)
\polypmIIId{94}(0.2329,0.2887)(0.2442,0.2887)(0.2442,0.2907)(0.2329,0.2907)
\polypmIIId{95}(0.2329,0.2906)(0.2442,0.2906)(0.2442,0.2927)(0.2329,0.2927)
\polypmIIId{96}(0.2329,0.2926)(0.2442,0.2926)(0.2442,0.2946)(0.2329,0.2946)
\polypmIIId{97}(0.2329,0.2945)(0.2442,0.2945)(0.2442,0.2965)(0.2329,0.2965)
\polypmIIId{98}(0.2329,0.2964)(0.2442,0.2964)(0.2442,0.2984)(0.2329,0.2984)
\polypmIIId{99}(0.2329,0.2983)(0.2442,0.2983)(0.2442,0.3004)(0.2329,0.3004)
\polypmIIId{100}(0.2329,0.3003)(0.2442,0.3003)(0.2442,0.3023)(0.2329,0.3023)
\polypmIIId{101}(0.2329,0.3022)(0.2442,0.3022)(0.2442,0.3042)(0.2329,0.3042)
\polypmIIId{102}(0.2329,0.3041)(0.2442,0.3041)(0.2442,0.3061)(0.2329,0.3061)
\polypmIIId{103}(0.2329,0.306)(0.2442,0.306)(0.2442,0.3081)(0.2329,0.3081)
\polypmIIId{104}(0.2329,0.308)(0.2442,0.308)(0.2442,0.31)(0.2329,0.31)
\polypmIIId{105}(0.2329,0.3099)(0.2442,0.3099)(0.2442,0.3119)(0.2329,0.3119)
\polypmIIId{106}(0.2329,0.3118)(0.2442,0.3118)(0.2442,0.3138)(0.2329,0.3138)
\polypmIIId{107}(0.2329,0.3137)(0.2442,0.3137)(0.2442,0.3158)(0.2329,0.3158)
\polypmIIId{108}(0.2329,0.3157)(0.2442,0.3157)(0.2442,0.3177)(0.2329,0.3177)
\polypmIIId{109}(0.2329,0.3176)(0.2442,0.3176)(0.2442,0.3196)(0.2329,0.3196)
\polypmIIId{110}(0.2329,0.3195)(0.2442,0.3195)(0.2442,0.3215)(0.2329,0.3215)
\polypmIIId{111}(0.2329,0.3214)(0.2442,0.3214)(0.2442,0.3235)(0.2329,0.3235)
\polypmIIId{112}(0.2329,0.3234)(0.2442,0.3234)(0.2442,0.3254)(0.2329,0.3254)
\polypmIIId{113}(0.2329,0.3253)(0.2442,0.3253)(0.2442,0.3273)(0.2329,0.3273)
\polypmIIId{114}(0.2329,0.3272)(0.2442,0.3272)(0.2442,0.3292)(0.2329,0.3292)
\polypmIIId{115}(0.2329,0.3291)(0.2442,0.3291)(0.2442,0.3312)(0.2329,0.3312)
\polypmIIId{116}(0.2329,0.3311)(0.2442,0.3311)(0.2442,0.3331)(0.2329,0.3331)
\polypmIIId{117}(0.2329,0.333)(0.2442,0.333)(0.2442,0.335)(0.2329,0.335)
\polypmIIId{118}(0.2329,0.3349)(0.2442,0.3349)(0.2442,0.3369)(0.2329,0.3369)
\polypmIIId{119}(0.2329,0.3368)(0.2442,0.3368)(0.2442,0.3389)(0.2329,0.3389)
\polypmIIId{120}(0.2329,0.3388)(0.2442,0.3388)(0.2442,0.3408)(0.2329,0.3408)
\polypmIIId{121}(0.2329,0.3407)(0.2442,0.3407)(0.2442,0.3427)(0.2329,0.3427)
\polypmIIId{122}(0.2329,0.3426)(0.2442,0.3426)(0.2442,0.3446)(0.2329,0.3446)
\polypmIIId{123}(0.2329,0.3445)(0.2442,0.3445)(0.2442,0.3466)(0.2329,0.3466)
\polypmIIId{124}(0.2329,0.3465)(0.2442,0.3465)(0.2442,0.3485)(0.2329,0.3485)
\polypmIIId{125}(0.2329,0.3484)(0.2442,0.3484)(0.2442,0.3504)(0.2329,0.3504)
\polypmIIId{126}(0.2329,0.3503)(0.2442,0.3503)(0.2442,0.3523)(0.2329,0.3523)
\polypmIIId{127}(0.2329,0.3522)(0.2442,0.3522)(0.2442,0.3542)(0.2329,0.3542)

\PST@Border(0.2329,0.1078)
(0.2442,0.1078)
(0.2442,0.3542)
(0.2329,0.3542)
(0.2329,0.1078)

\rput[l](0.2502,0.1078){0}
\rput[l](0.2502,0.1570){0.2}
\rput[l](0.2502,0.2063){0.4}
\rput[l](0.2502,0.2556){0.6}
\rput[l](0.2502,0.3049){0.8}
\rput[l](0.2502,0.3542){1}

\catcode`@=12
\fi
\endpspicture
}
  }
  $\alpha = 0.1$
  %\label{fig:result01}
\end{figure}

\figspaces
\begin{figure}[H]
  \centering
  \resizebox{\columnwidth}{!}{%
    \subfloat[1 resource]{% GNUPLOT: LaTeX picture using PSTRICKS macros
% Define new PST objects, if not already defined
\ifx\PSTloaded\undefined
\def\PSTloaded{t}

\catcode`@=11

\newpsobject{PST@Border}{psline}{linewidth=.0015,linestyle=solid}

\catcode`@=12

\fi
\psset{unit=5.0in,xunit=5.0in,yunit=3.0in}
\pspicture(0.000000,0.000000)(0.31, 0.35)
\ifx\nofigs\undefined
\catcode`@=11

\newrgbcolor{PST@COLOR0}{1 1 1}
\newrgbcolor{PST@COLOR1}{0.992 0.992 0.992}
\newrgbcolor{PST@COLOR3}{0.976 0.976 0.976}
\newrgbcolor{PST@COLOR6}{0.952 0.952 0.952}


\def\polypmIIId#1{\pspolygon[linestyle=none,fillstyle=solid,fillcolor=PST@COLOR#1]}

\polypmIIId{0}(0.1432,0.19)(0.0864,0.19)(0.0864,0.1078)(0.1432,0.1078)
\polypmIIId{0}(0.1432,0.272)(0.0864,0.272)(0.0864,0.19)(0.1432,0.19)
\polypmIIId{0}(0.1432,0.3542)(0.0864,0.3542)(0.0864,0.272)(0.1432,0.272)

\polypmIIId{0}(0.2,0.19)(0.1432,0.19)(0.1432,0.1078)(0.2,0.1078)
\polypmIIId{0}(0.2,0.272)(0.1432,0.272)(0.1432,0.19)(0.2,0.19)
\polypmIIId{0}(0.2,0.3542)(0.1432,0.3542)(0.1432,0.272)(0.2,0.272)

\polypmIIId{0}(0.2568,0.19)(0.2,0.19)(0.2,0.1078)(0.2568,0.1078)
\polypmIIId{0}(0.2568,0.272)(0.2,0.272)(0.2,0.19)(0.2568,0.19)
\polypmIIId{0}(0.2568,0.3542)(0.2,0.3542)(0.2,0.272)(0.2568,0.272)

\polypmIIId{0}(0.3136,0.19)(0.2568,0.19)(0.2568,0.1078)(0.3136,0.1078)
\polypmIIId{0}(0.3136,0.272)(0.2568,0.272)(0.2568,0.19)(0.3136,0.19)
\polypmIIId{0}(0.3136,0.3542)(0.2568,0.3542)(0.2568,0.272)(0.3136,0.272)

\rput(0.1148,0.07){3}
\rput(0.1716,0.07){4}
\rput(0.2284,0.07){5}
\rput(0.2852,0.07){6}
\rput(0.2000,0.0070){years}

\rput[r](0.0806,0.1489){25}
\rput[r](0.0806,0.2310){50}
\rput[r](0.0806,0.3131){100}
\rput{L}(0.0096,0.2310){actions}

\PST@Border(0.0864,0.3542)
(0.0864,0.1078)
(0.3136,0.1078)
(0.3136,0.3542)
(0.0864,0.3542)

\catcode`@=12
\fi
\endpspicture}
    \subfloat[2 resources]{% GNUPLOT: LaTeX picture using PSTRICKS macros
% Define new PST objects, if not already defined
\ifx\PSTloaded\undefined
\def\PSTloaded{t}

\catcode`@=11

\newpsobject{PST@Border}{psline}{linewidth=.0015,linestyle=solid}

\catcode`@=12

\fi
\psset{unit=5.0in,xunit=5.0in,yunit=3.0in}
\pspicture(0.000000,0.000000)(0.225000,0.35)
\ifx\nofigs\undefined
\catcode`@=11

\newrgbcolor{PST@COLOR0}{1 1 1}
\newrgbcolor{PST@COLOR2}{0.984 0.984 0.984}
\newrgbcolor{PST@COLOR8}{0.937 0.937 0.937}
\newrgbcolor{PST@COLOR10}{0.921 0.921 0.921}
\newrgbcolor{PST@COLOR55}{0.566 0.566 0.566}
\newrgbcolor{PST@COLOR96}{0.244 0.244 0.244}
\newrgbcolor{PST@COLOR117}{0.078 0.078 0.078}

\def\polypmIIId#1{\pspolygon[linestyle=none,fillstyle=solid,fillcolor=PST@COLOR#1]}

\polypmIIId{0} (0.0568,0.19)  (0.0,0.19)  (0.0,0.1078)(0.0568,0.1078)
\polypmIIId{0}  (0.0568,0.272) (0.0,0.272) (0.0,0.19)  (0.0568,0.19)
\polypmIIId{0}  (0.0568,0.3942)(0.0,0.3942)(0.0,0.272) (0.0568,0.272)

\polypmIIId{0} (0.1136,   0.19)  (0.0568,0.19)  (0.0568,0.1078)(0.1136,0.1078)
\polypmIIId{0}  (0.1136,   0.272) (0.0568,0.272) (0.0568,0.19)  (0.1136,0.19)
\polypmIIId{0}  (0.1136,   0.3942)(0.0568,0.3942)(0.0568,0.272) (0.1136,0.272)

\polypmIIId{0}(0.1704,0.19)  (0.1136,   0.19)  (0.1136,   0.1078)(0.1704,0.1078)
\polypmIIId{0} (0.1704,0.272) (0.1136,   0.272) (0.1136,   0.19)  (0.1704,0.19)
\polypmIIId{0}  (0.1704,0.3942)(0.1136,   0.3942)(0.1136,   0.272) (0.1704,0.272)

\polypmIIId{1}(0.2272,0.19)  (0.1704,0.19)  (0.1704,0.1078)(0.2272,0.1078)
\polypmIIId{0}  (0.2272,0.272) (0.1704,0.272) (0.1704,0.19)  (0.2272,0.19)
\polypmIIId{0}  (0.2272,0.3942)(0.1704,0.3942)(0.1704,0.272) (0.2272,0.272)

\rput(0.0284,0.07){3}
\rput(0.0852,0.07){4}
\rput(0.1420,0.07){5}
\rput(0.1988,0.07){6}
\rput(0.1136,0.0070){years}


\PST@Border(0.0,0.3542)
(0.0,0.1078)
(0.2272,0.1078)
(0.2272,0.3542)
(0.0,0.3542)

\catcode`@=12
\fi
\endpspicture}
    \subfloat[4 resources]{% GNUPLOT: LaTeX picture using PSTRICKS macros
% Define new PST objects, if not already defined
\ifx\PSTloaded\undefined
\def\PSTloaded{t}

\catcode`@=11

\newpsobject{PST@Border}{psline}{linewidth=.0015,linestyle=solid}

\catcode`@=12

\fi
\psset{unit=5.0in,xunit=5.0in,yunit=3.0in}
\pspicture(0.000000,0.000000)(0.3136,0.35)
\ifx\nofigs\undefined
\catcode`@=11

\newrgbcolor{PST@COLOR0}{1 1 1}
\newrgbcolor{PST@COLOR1}{0.992 0.992 0.992}
\newrgbcolor{PST@COLOR2}{0.984 0.984 0.984}
\newrgbcolor{PST@COLOR3}{0.976 0.976 0.976}
\newrgbcolor{PST@COLOR4}{0.968 0.968 0.968}
\newrgbcolor{PST@COLOR5}{0.96 0.96 0.96}
\newrgbcolor{PST@COLOR6}{0.952 0.952 0.952}
\newrgbcolor{PST@COLOR7}{0.944 0.944 0.944}
\newrgbcolor{PST@COLOR8}{0.937 0.937 0.937}
\newrgbcolor{PST@COLOR9}{0.929 0.929 0.929}
\newrgbcolor{PST@COLOR10}{0.921 0.921 0.921}
\newrgbcolor{PST@COLOR11}{0.913 0.913 0.913}
\newrgbcolor{PST@COLOR12}{0.905 0.905 0.905}
\newrgbcolor{PST@COLOR13}{0.897 0.897 0.897}
\newrgbcolor{PST@COLOR14}{0.889 0.889 0.889}
\newrgbcolor{PST@COLOR15}{0.881 0.881 0.881}
\newrgbcolor{PST@COLOR16}{0.874 0.874 0.874}
\newrgbcolor{PST@COLOR17}{0.866 0.866 0.866}
\newrgbcolor{PST@COLOR18}{0.858 0.858 0.858}
\newrgbcolor{PST@COLOR19}{0.85 0.85 0.85}
\newrgbcolor{PST@COLOR20}{0.842 0.842 0.842}
\newrgbcolor{PST@COLOR21}{0.834 0.834 0.834}
\newrgbcolor{PST@COLOR22}{0.826 0.826 0.826}
\newrgbcolor{PST@COLOR23}{0.818 0.818 0.818}
\newrgbcolor{PST@COLOR24}{0.811 0.811 0.811}
\newrgbcolor{PST@COLOR25}{0.803 0.803 0.803}
\newrgbcolor{PST@COLOR26}{0.795 0.795 0.795}
\newrgbcolor{PST@COLOR27}{0.787 0.787 0.787}
\newrgbcolor{PST@COLOR28}{0.779 0.779 0.779}
\newrgbcolor{PST@COLOR29}{0.771 0.771 0.771}
\newrgbcolor{PST@COLOR30}{0.763 0.763 0.763}
\newrgbcolor{PST@COLOR31}{0.755 0.755 0.755}
\newrgbcolor{PST@COLOR32}{0.748 0.748 0.748}
\newrgbcolor{PST@COLOR33}{0.74 0.74 0.74}
\newrgbcolor{PST@COLOR34}{0.732 0.732 0.732}
\newrgbcolor{PST@COLOR35}{0.724 0.724 0.724}
\newrgbcolor{PST@COLOR36}{0.716 0.716 0.716}
\newrgbcolor{PST@COLOR37}{0.708 0.708 0.708}
\newrgbcolor{PST@COLOR38}{0.7 0.7 0.7}
\newrgbcolor{PST@COLOR39}{0.692 0.692 0.692}
\newrgbcolor{PST@COLOR40}{0.685 0.685 0.685}
\newrgbcolor{PST@COLOR41}{0.677 0.677 0.677}
\newrgbcolor{PST@COLOR42}{0.669 0.669 0.669}
\newrgbcolor{PST@COLOR43}{0.661 0.661 0.661}
\newrgbcolor{PST@COLOR44}{0.653 0.653 0.653}
\newrgbcolor{PST@COLOR45}{0.645 0.645 0.645}
\newrgbcolor{PST@COLOR46}{0.637 0.637 0.637}
\newrgbcolor{PST@COLOR47}{0.629 0.629 0.629}
\newrgbcolor{PST@COLOR48}{0.622 0.622 0.622}
\newrgbcolor{PST@COLOR49}{0.614 0.614 0.614}
\newrgbcolor{PST@COLOR50}{0.606 0.606 0.606}
\newrgbcolor{PST@COLOR51}{0.598 0.598 0.598}
\newrgbcolor{PST@COLOR52}{0.59 0.59 0.59}
\newrgbcolor{PST@COLOR53}{0.582 0.582 0.582}
\newrgbcolor{PST@COLOR54}{0.574 0.574 0.574}
\newrgbcolor{PST@COLOR55}{0.566 0.566 0.566}
\newrgbcolor{PST@COLOR56}{0.559 0.559 0.559}
\newrgbcolor{PST@COLOR57}{0.551 0.551 0.551}
\newrgbcolor{PST@COLOR58}{0.543 0.543 0.543}
\newrgbcolor{PST@COLOR59}{0.535 0.535 0.535}
\newrgbcolor{PST@COLOR60}{0.527 0.527 0.527}
\newrgbcolor{PST@COLOR61}{0.519 0.519 0.519}
\newrgbcolor{PST@COLOR62}{0.511 0.511 0.511}
\newrgbcolor{PST@COLOR63}{0.503 0.503 0.503}
\newrgbcolor{PST@COLOR64}{0.496 0.496 0.496}
\newrgbcolor{PST@COLOR65}{0.488 0.488 0.488}
\newrgbcolor{PST@COLOR66}{0.48 0.48 0.48}
\newrgbcolor{PST@COLOR67}{0.472 0.472 0.472}
\newrgbcolor{PST@COLOR68}{0.464 0.464 0.464}
\newrgbcolor{PST@COLOR69}{0.456 0.456 0.456}
\newrgbcolor{PST@COLOR70}{0.448 0.448 0.448}
\newrgbcolor{PST@COLOR71}{0.44 0.44 0.44}
\newrgbcolor{PST@COLOR72}{0.433 0.433 0.433}
\newrgbcolor{PST@COLOR73}{0.425 0.425 0.425}
\newrgbcolor{PST@COLOR74}{0.417 0.417 0.417}
\newrgbcolor{PST@COLOR75}{0.409 0.409 0.409}
\newrgbcolor{PST@COLOR76}{0.401 0.401 0.401}
\newrgbcolor{PST@COLOR77}{0.393 0.393 0.393}
\newrgbcolor{PST@COLOR78}{0.385 0.385 0.385}
\newrgbcolor{PST@COLOR79}{0.377 0.377 0.377}
\newrgbcolor{PST@COLOR80}{0.37 0.37 0.37}
\newrgbcolor{PST@COLOR81}{0.362 0.362 0.362}
\newrgbcolor{PST@COLOR82}{0.354 0.354 0.354}
\newrgbcolor{PST@COLOR83}{0.346 0.346 0.346}
\newrgbcolor{PST@COLOR84}{0.338 0.338 0.338}
\newrgbcolor{PST@COLOR85}{0.33 0.33 0.33}
\newrgbcolor{PST@COLOR86}{0.322 0.322 0.322}
\newrgbcolor{PST@COLOR87}{0.314 0.314 0.314}
\newrgbcolor{PST@COLOR88}{0.307 0.307 0.307}
\newrgbcolor{PST@COLOR89}{0.299 0.299 0.299}
\newrgbcolor{PST@COLOR90}{0.291 0.291 0.291}
\newrgbcolor{PST@COLOR91}{0.283 0.283 0.283}
\newrgbcolor{PST@COLOR92}{0.275 0.275 0.275}
\newrgbcolor{PST@COLOR93}{0.267 0.267 0.267}
\newrgbcolor{PST@COLOR94}{0.259 0.259 0.259}
\newrgbcolor{PST@COLOR95}{0.251 0.251 0.251}
\newrgbcolor{PST@COLOR96}{0.244 0.244 0.244}
\newrgbcolor{PST@COLOR97}{0.236 0.236 0.236}
\newrgbcolor{PST@COLOR98}{0.228 0.228 0.228}
\newrgbcolor{PST@COLOR99}{0.22 0.22 0.22}
\newrgbcolor{PST@COLOR100}{0.212 0.212 0.212}
\newrgbcolor{PST@COLOR101}{0.204 0.204 0.204}
\newrgbcolor{PST@COLOR102}{0.196 0.196 0.196}
\newrgbcolor{PST@COLOR103}{0.188 0.188 0.188}
\newrgbcolor{PST@COLOR104}{0.181 0.181 0.181}
\newrgbcolor{PST@COLOR105}{0.173 0.173 0.173}
\newrgbcolor{PST@COLOR106}{0.165 0.165 0.165}
\newrgbcolor{PST@COLOR107}{0.157 0.157 0.157}
\newrgbcolor{PST@COLOR108}{0.149 0.149 0.149}
\newrgbcolor{PST@COLOR109}{0.141 0.141 0.141}
\newrgbcolor{PST@COLOR110}{0.133 0.133 0.133}
\newrgbcolor{PST@COLOR111}{0.125 0.125 0.125}
\newrgbcolor{PST@COLOR112}{0.118 0.118 0.118}
\newrgbcolor{PST@COLOR113}{0.11 0.11 0.11}
\newrgbcolor{PST@COLOR114}{0.102 0.102 0.102}
\newrgbcolor{PST@COLOR115}{0.094 0.094 0.094}
\newrgbcolor{PST@COLOR116}{0.086 0.086 0.086}
\newrgbcolor{PST@COLOR117}{0.078 0.078 0.078}
\newrgbcolor{PST@COLOR118}{0.07 0.07 0.07}
\newrgbcolor{PST@COLOR119}{0.062 0.062 0.062}
\newrgbcolor{PST@COLOR120}{0.055 0.055 0.055}
\newrgbcolor{PST@COLOR121}{0.047 0.047 0.047}
\newrgbcolor{PST@COLOR122}{0.039 0.039 0.039}
\newrgbcolor{PST@COLOR123}{0.031 0.031 0.031}
\newrgbcolor{PST@COLOR124}{0.023 0.023 0.023}
\newrgbcolor{PST@COLOR125}{0.015 0.015 0.015}
\newrgbcolor{PST@COLOR126}{0.007 0.007 0.007}
\newrgbcolor{PST@COLOR127}{0 0 0}

\def\polypmIIId#1{\pspolygon[linestyle=none,fillstyle=solid,fillcolor=PST@COLOR#1]}

\polypmIIId{0} (0.0568,0.19)  (0.0,0.19)  (0.0,0.1078)(0.0568,0.1078)
\polypmIIId{0}  (0.0568,0.272) (0.0,0.272) (0.0,0.19)  (0.0568,0.19)
\polypmIIId{0}  (0.0568,0.3542)(0.0,0.3542)(0.0,0.272) (0.0568,0.272)

\polypmIIId{0} (0.1136,   0.19)  (0.0568,0.19)  (0.0568,0.1078)(0.1136,0.1078)
\polypmIIId{0}  (0.1136,   0.272) (0.0568,0.272) (0.0568,0.19)  (0.1136,0.19)
\polypmIIId{0}  (0.1136,   0.3542)(0.0568,0.3542)(0.0568,0.272) (0.1136,0.272)

\polypmIIId{3}(0.1704,0.19)  (0.1136,   0.19)  (0.1136,   0.1078)(0.1704,0.1078)
\polypmIIId{2} (0.1704,0.272) (0.1136,   0.272) (0.1136,   0.19)  (0.1704,0.19)
\polypmIIId{1}  (0.1704,0.3542)(0.1136,   0.3542)(0.1136,   0.272) (0.1704,0.272)

\polypmIIId{7}(0.2272,0.19)  (0.1704,0.19)  (0.1704,0.1078)(0.2272,0.1078)
\polypmIIId{7}  (0.2272,0.272) (0.1704,0.272) (0.1704,0.19)  (0.2272,0.19)
\polypmIIId{0}  (0.2272,0.3542)(0.1704,0.3542)(0.1704,0.272) (0.2272,0.272)

\rput(0.0284,0.07){3}
\rput(0.0852,0.07){4}
\rput(0.1420,0.07){5}
\rput(0.1988,0.07){6}
\rput(0.1136,0.0070){years}

\PST@Border(0.0,0.3542)
(0.0,0.1078)
(0.2272,0.1078)
(0.2272,0.3542)
(0.0,0.3542)

\polypmIIId{0}(0.2329,0.1078)(0.2442,0.1078)(0.2442,0.1098)(0.2329,0.1098)
\polypmIIId{1}(0.2329,0.1097)(0.2442,0.1097)(0.2442,0.1117)(0.2329,0.1117)
\polypmIIId{2}(0.2329,0.1116)(0.2442,0.1116)(0.2442,0.1136)(0.2329,0.1136)
\polypmIIId{3}(0.2329,0.1135)(0.2442,0.1135)(0.2442,0.1156)(0.2329,0.1156)
\polypmIIId{4}(0.2329,0.1155)(0.2442,0.1155)(0.2442,0.1175)(0.2329,0.1175)
\polypmIIId{5}(0.2329,0.1174)(0.2442,0.1174)(0.2442,0.1194)(0.2329,0.1194)
\polypmIIId{6}(0.2329,0.1193)(0.2442,0.1193)(0.2442,0.1213)(0.2329,0.1213)
\polypmIIId{7}(0.2329,0.1212)(0.2442,0.1212)(0.2442,0.1233)(0.2329,0.1233)
\polypmIIId{8}(0.2329,0.1232)(0.2442,0.1232)(0.2442,0.1252)(0.2329,0.1252)
\polypmIIId{9}(0.2329,0.1251)(0.2442,0.1251)(0.2442,0.1271)(0.2329,0.1271)
\polypmIIId{10}(0.2329,0.127)(0.2442,0.127)(0.2442,0.129)(0.2329,0.129)
\polypmIIId{11}(0.2329,0.1289)(0.2442,0.1289)(0.2442,0.131)(0.2329,0.131)
\polypmIIId{12}(0.2329,0.1309)(0.2442,0.1309)(0.2442,0.1329)(0.2329,0.1329)
\polypmIIId{13}(0.2329,0.1328)(0.2442,0.1328)(0.2442,0.1348)(0.2329,0.1348)
\polypmIIId{14}(0.2329,0.1347)(0.2442,0.1347)(0.2442,0.1367)(0.2329,0.1367)
\polypmIIId{15}(0.2329,0.1366)(0.2442,0.1366)(0.2442,0.1387)(0.2329,0.1387)
\polypmIIId{16}(0.2329,0.1386)(0.2442,0.1386)(0.2442,0.1406)(0.2329,0.1406)
\polypmIIId{17}(0.2329,0.1405)(0.2442,0.1405)(0.2442,0.1425)(0.2329,0.1425)
\polypmIIId{18}(0.2329,0.1424)(0.2442,0.1424)(0.2442,0.1444)(0.2329,0.1444)
\polypmIIId{19}(0.2329,0.1443)(0.2442,0.1443)(0.2442,0.1464)(0.2329,0.1464)
\polypmIIId{20}(0.2329,0.1463)(0.2442,0.1463)(0.2442,0.1483)(0.2329,0.1483)
\polypmIIId{21}(0.2329,0.1482)(0.2442,0.1482)(0.2442,0.1502)(0.2329,0.1502)
\polypmIIId{22}(0.2329,0.1501)(0.2442,0.1501)(0.2442,0.1521)(0.2329,0.1521)
\polypmIIId{23}(0.2329,0.152)(0.2442,0.152)(0.2442,0.1541)(0.2329,0.1541)
\polypmIIId{24}(0.2329,0.154)(0.2442,0.154)(0.2442,0.156)(0.2329,0.156)
\polypmIIId{25}(0.2329,0.1559)(0.2442,0.1559)(0.2442,0.1579)(0.2329,0.1579)
\polypmIIId{26}(0.2329,0.1578)(0.2442,0.1578)(0.2442,0.1598)(0.2329,0.1598)
\polypmIIId{27}(0.2329,0.1597)(0.2442,0.1597)(0.2442,0.1618)(0.2329,0.1618)
\polypmIIId{28}(0.2329,0.1617)(0.2442,0.1617)(0.2442,0.1637)(0.2329,0.1637)
\polypmIIId{29}(0.2329,0.1636)(0.2442,0.1636)(0.2442,0.1656)(0.2329,0.1656)
\polypmIIId{30}(0.2329,0.1655)(0.2442,0.1655)(0.2442,0.1675)(0.2329,0.1675)
\polypmIIId{31}(0.2329,0.1674)(0.2442,0.1674)(0.2442,0.1695)(0.2329,0.1695)
\polypmIIId{32}(0.2329,0.1694)(0.2442,0.1694)(0.2442,0.1714)(0.2329,0.1714)
\polypmIIId{33}(0.2329,0.1713)(0.2442,0.1713)(0.2442,0.1733)(0.2329,0.1733)
\polypmIIId{34}(0.2329,0.1732)(0.2442,0.1732)(0.2442,0.1752)(0.2329,0.1752)
\polypmIIId{35}(0.2329,0.1751)(0.2442,0.1751)(0.2442,0.1772)(0.2329,0.1772)
\polypmIIId{36}(0.2329,0.1771)(0.2442,0.1771)(0.2442,0.1791)(0.2329,0.1791)
\polypmIIId{37}(0.2329,0.179)(0.2442,0.179)(0.2442,0.181)(0.2329,0.181)
\polypmIIId{38}(0.2329,0.1809)(0.2442,0.1809)(0.2442,0.1829)(0.2329,0.1829)
\polypmIIId{39}(0.2329,0.1828)(0.2442,0.1828)(0.2442,0.1849)(0.2329,0.1849)
\polypmIIId{40}(0.2329,0.1848)(0.2442,0.1848)(0.2442,0.1868)(0.2329,0.1868)
\polypmIIId{41}(0.2329,0.1867)(0.2442,0.1867)(0.2442,0.1887)(0.2329,0.1887)
\polypmIIId{42}(0.2329,0.1886)(0.2442,0.1886)(0.2442,0.1906)(0.2329,0.1906)
\polypmIIId{43}(0.2329,0.1905)(0.2442,0.1905)(0.2442,0.1926)(0.2329,0.1926)
\polypmIIId{44}(0.2329,0.1925)(0.2442,0.1925)(0.2442,0.1945)(0.2329,0.1945)
\polypmIIId{45}(0.2329,0.1944)(0.2442,0.1944)(0.2442,0.1964)(0.2329,0.1964)
\polypmIIId{46}(0.2329,0.1963)(0.2442,0.1963)(0.2442,0.1983)(0.2329,0.1983)
\polypmIIId{47}(0.2329,0.1982)(0.2442,0.1982)(0.2442,0.2003)(0.2329,0.2003)
\polypmIIId{48}(0.2329,0.2002)(0.2442,0.2002)(0.2442,0.2022)(0.2329,0.2022)
\polypmIIId{49}(0.2329,0.2021)(0.2442,0.2021)(0.2442,0.2041)(0.2329,0.2041)
\polypmIIId{50}(0.2329,0.204)(0.2442,0.204)(0.2442,0.206)(0.2329,0.206)
\polypmIIId{51}(0.2329,0.2059)(0.2442,0.2059)(0.2442,0.208)(0.2329,0.208)
\polypmIIId{52}(0.2329,0.2079)(0.2442,0.2079)(0.2442,0.2099)(0.2329,0.2099)
\polypmIIId{53}(0.2329,0.2098)(0.2442,0.2098)(0.2442,0.2118)(0.2329,0.2118)
\polypmIIId{54}(0.2329,0.2117)(0.2442,0.2117)(0.2442,0.2137)(0.2329,0.2137)
\polypmIIId{55}(0.2329,0.2136)(0.2442,0.2136)(0.2442,0.2157)(0.2329,0.2157)
\polypmIIId{56}(0.2329,0.2156)(0.2442,0.2156)(0.2442,0.2176)(0.2329,0.2176)
\polypmIIId{57}(0.2329,0.2175)(0.2442,0.2175)(0.2442,0.2195)(0.2329,0.2195)
\polypmIIId{58}(0.2329,0.2194)(0.2442,0.2194)(0.2442,0.2214)(0.2329,0.2214)
\polypmIIId{59}(0.2329,0.2213)(0.2442,0.2213)(0.2442,0.2234)(0.2329,0.2234)
\polypmIIId{60}(0.2329,0.2233)(0.2442,0.2233)(0.2442,0.2253)(0.2329,0.2253)
\polypmIIId{61}(0.2329,0.2252)(0.2442,0.2252)(0.2442,0.2272)(0.2329,0.2272)
\polypmIIId{62}(0.2329,0.2271)(0.2442,0.2271)(0.2442,0.2291)(0.2329,0.2291)
\polypmIIId{63}(0.2329,0.229)(0.2442,0.229)(0.2442,0.2311)(0.2329,0.2311)
\polypmIIId{64}(0.2329,0.231)(0.2442,0.231)(0.2442,0.233)(0.2329,0.233)
\polypmIIId{65}(0.2329,0.2329)(0.2442,0.2329)(0.2442,0.2349)(0.2329,0.2349)
\polypmIIId{66}(0.2329,0.2348)(0.2442,0.2348)(0.2442,0.2368)(0.2329,0.2368)
\polypmIIId{67}(0.2329,0.2367)(0.2442,0.2367)(0.2442,0.2388)(0.2329,0.2388)
\polypmIIId{68}(0.2329,0.2387)(0.2442,0.2387)(0.2442,0.2407)(0.2329,0.2407)
\polypmIIId{69}(0.2329,0.2406)(0.2442,0.2406)(0.2442,0.2426)(0.2329,0.2426)
\polypmIIId{70}(0.2329,0.2425)(0.2442,0.2425)(0.2442,0.2445)(0.2329,0.2445)
\polypmIIId{71}(0.2329,0.2444)(0.2442,0.2444)(0.2442,0.2465)(0.2329,0.2465)
\polypmIIId{72}(0.2329,0.2464)(0.2442,0.2464)(0.2442,0.2484)(0.2329,0.2484)
\polypmIIId{73}(0.2329,0.2483)(0.2442,0.2483)(0.2442,0.2503)(0.2329,0.2503)
\polypmIIId{74}(0.2329,0.2502)(0.2442,0.2502)(0.2442,0.2522)(0.2329,0.2522)
\polypmIIId{75}(0.2329,0.2521)(0.2442,0.2521)(0.2442,0.2542)(0.2329,0.2542)
\polypmIIId{76}(0.2329,0.2541)(0.2442,0.2541)(0.2442,0.2561)(0.2329,0.2561)
\polypmIIId{77}(0.2329,0.256)(0.2442,0.256)(0.2442,0.258)(0.2329,0.258)
\polypmIIId{78}(0.2329,0.2579)(0.2442,0.2579)(0.2442,0.2599)(0.2329,0.2599)
\polypmIIId{79}(0.2329,0.2598)(0.2442,0.2598)(0.2442,0.2619)(0.2329,0.2619)
\polypmIIId{80}(0.2329,0.2618)(0.2442,0.2618)(0.2442,0.2638)(0.2329,0.2638)
\polypmIIId{81}(0.2329,0.2637)(0.2442,0.2637)(0.2442,0.2657)(0.2329,0.2657)
\polypmIIId{82}(0.2329,0.2656)(0.2442,0.2656)(0.2442,0.2676)(0.2329,0.2676)
\polypmIIId{83}(0.2329,0.2675)(0.2442,0.2675)(0.2442,0.2696)(0.2329,0.2696)
\polypmIIId{84}(0.2329,0.2695)(0.2442,0.2695)(0.2442,0.2715)(0.2329,0.2715)
\polypmIIId{85}(0.2329,0.2714)(0.2442,0.2714)(0.2442,0.2734)(0.2329,0.2734)
\polypmIIId{86}(0.2329,0.2733)(0.2442,0.2733)(0.2442,0.2753)(0.2329,0.2753)
\polypmIIId{87}(0.2329,0.2752)(0.2442,0.2752)(0.2442,0.2773)(0.2329,0.2773)
\polypmIIId{88}(0.2329,0.2772)(0.2442,0.2772)(0.2442,0.2792)(0.2329,0.2792)
\polypmIIId{89}(0.2329,0.2791)(0.2442,0.2791)(0.2442,0.2811)(0.2329,0.2811)
\polypmIIId{90}(0.2329,0.281)(0.2442,0.281)(0.2442,0.283)(0.2329,0.283)
\polypmIIId{91}(0.2329,0.2829)(0.2442,0.2829)(0.2442,0.285)(0.2329,0.285)
\polypmIIId{92}(0.2329,0.2849)(0.2442,0.2849)(0.2442,0.2869)(0.2329,0.2869)
\polypmIIId{93}(0.2329,0.2868)(0.2442,0.2868)(0.2442,0.2888)(0.2329,0.2888)
\polypmIIId{94}(0.2329,0.2887)(0.2442,0.2887)(0.2442,0.2907)(0.2329,0.2907)
\polypmIIId{95}(0.2329,0.2906)(0.2442,0.2906)(0.2442,0.2927)(0.2329,0.2927)
\polypmIIId{96}(0.2329,0.2926)(0.2442,0.2926)(0.2442,0.2946)(0.2329,0.2946)
\polypmIIId{97}(0.2329,0.2945)(0.2442,0.2945)(0.2442,0.2965)(0.2329,0.2965)
\polypmIIId{98}(0.2329,0.2964)(0.2442,0.2964)(0.2442,0.2984)(0.2329,0.2984)
\polypmIIId{99}(0.2329,0.2983)(0.2442,0.2983)(0.2442,0.3004)(0.2329,0.3004)
\polypmIIId{100}(0.2329,0.3003)(0.2442,0.3003)(0.2442,0.3023)(0.2329,0.3023)
\polypmIIId{101}(0.2329,0.3022)(0.2442,0.3022)(0.2442,0.3042)(0.2329,0.3042)
\polypmIIId{102}(0.2329,0.3041)(0.2442,0.3041)(0.2442,0.3061)(0.2329,0.3061)
\polypmIIId{103}(0.2329,0.306)(0.2442,0.306)(0.2442,0.3081)(0.2329,0.3081)
\polypmIIId{104}(0.2329,0.308)(0.2442,0.308)(0.2442,0.31)(0.2329,0.31)
\polypmIIId{105}(0.2329,0.3099)(0.2442,0.3099)(0.2442,0.3119)(0.2329,0.3119)
\polypmIIId{106}(0.2329,0.3118)(0.2442,0.3118)(0.2442,0.3138)(0.2329,0.3138)
\polypmIIId{107}(0.2329,0.3137)(0.2442,0.3137)(0.2442,0.3158)(0.2329,0.3158)
\polypmIIId{108}(0.2329,0.3157)(0.2442,0.3157)(0.2442,0.3177)(0.2329,0.3177)
\polypmIIId{109}(0.2329,0.3176)(0.2442,0.3176)(0.2442,0.3196)(0.2329,0.3196)
\polypmIIId{110}(0.2329,0.3195)(0.2442,0.3195)(0.2442,0.3215)(0.2329,0.3215)
\polypmIIId{111}(0.2329,0.3214)(0.2442,0.3214)(0.2442,0.3235)(0.2329,0.3235)
\polypmIIId{112}(0.2329,0.3234)(0.2442,0.3234)(0.2442,0.3254)(0.2329,0.3254)
\polypmIIId{113}(0.2329,0.3253)(0.2442,0.3253)(0.2442,0.3273)(0.2329,0.3273)
\polypmIIId{114}(0.2329,0.3272)(0.2442,0.3272)(0.2442,0.3292)(0.2329,0.3292)
\polypmIIId{115}(0.2329,0.3291)(0.2442,0.3291)(0.2442,0.3312)(0.2329,0.3312)
\polypmIIId{116}(0.2329,0.3311)(0.2442,0.3311)(0.2442,0.3331)(0.2329,0.3331)
\polypmIIId{117}(0.2329,0.333)(0.2442,0.333)(0.2442,0.335)(0.2329,0.335)
\polypmIIId{118}(0.2329,0.3349)(0.2442,0.3349)(0.2442,0.3369)(0.2329,0.3369)
\polypmIIId{119}(0.2329,0.3368)(0.2442,0.3368)(0.2442,0.3389)(0.2329,0.3389)
\polypmIIId{120}(0.2329,0.3388)(0.2442,0.3388)(0.2442,0.3408)(0.2329,0.3408)
\polypmIIId{121}(0.2329,0.3407)(0.2442,0.3407)(0.2442,0.3427)(0.2329,0.3427)
\polypmIIId{122}(0.2329,0.3426)(0.2442,0.3426)(0.2442,0.3446)(0.2329,0.3446)
\polypmIIId{123}(0.2329,0.3445)(0.2442,0.3445)(0.2442,0.3466)(0.2329,0.3466)
\polypmIIId{124}(0.2329,0.3465)(0.2442,0.3465)(0.2442,0.3485)(0.2329,0.3485)
\polypmIIId{125}(0.2329,0.3484)(0.2442,0.3484)(0.2442,0.3504)(0.2329,0.3504)
\polypmIIId{126}(0.2329,0.3503)(0.2442,0.3503)(0.2442,0.3523)(0.2329,0.3523)
\polypmIIId{127}(0.2329,0.3522)(0.2442,0.3522)(0.2442,0.3542)(0.2329,0.3542)

\PST@Border(0.2329,0.1078)
(0.2442,0.1078)
(0.2442,0.3542)
(0.2329,0.3542)
(0.2329,0.1078)

\rput[l](0.2502,0.1078){0}
\rput[l](0.2502,0.1570){0.2}
\rput[l](0.2502,0.2063){0.4}
\rput[l](0.2502,0.2556){0.6}
\rput[l](0.2502,0.3049){0.8}
\rput[l](0.2502,0.3542){1}

\catcode`@=12
\fi
\endpspicture}
  }
  $\alpha = 1.0$
  \caption{Ratio of uncorrelated instances interrupted.}
  \label{fig:result10}
\end{figure}

Comparing figure~\ref{fig:result10} one can see the first two heatmaps are darker 
than the last, meaning that the correlation level indeed affects the instances hardness. As expected, uncorrelated instances
are easier for the CPLEX to solve, with just a few instances with bigger dimensions being interrupted. However, comparing just
the two first heatmaps, it appears that weakly correlated instances are harder to solve than
strongly correlated instances, as opposed to the expected.

Concerning quantity of years and resources of the instances, the obtained results confirm what was expected, since 
Figure~\ref{fig:result10}
show that the bigger the instances are on this two parameters, the harder it gets for the CPLEX solver to solve them to optimality. 
Still, comparing the figure observing the hardness in relation to the variation on the number of actions, it seems that instances with fewer actions are 
harder to solve to optimallity.

In relation to the time taken to solve the instances, the CPLEX took on average 380 seconds to solve the instances. The average gap was 
0.01\%, meaning that the CPLEX found solutions that were, in average, at least 99.99\% of the optimal ones.

\subsection{Solutions Quality}

To analyse the quality of the solutions obtained by the heuristics, the same instances previously solved with CPLEX were now solved with
the GALP and TSLP, and the solutions found were compared to the ones obtained by CPLEX before reaching the time limit. 
Figure~\ref{fig:tabusolcomp10} shows the results for the TSLP. The heatmaps follow the same configuration 
of the CPLEX tests, but now showing the average quality of the solutions. The color scale represents
the average ratio between the solutions found by the TSLP and the best known solution, in other words, darker tones indicate solutions
with better quality were found by the TSLP. The same representations are used on Figure~\ref{fig:greedysolcomp10} 
to present the results obtained with GALP.

\figpar
\begin{figure}[H]
  \centering
  \resizebox{\columnwidth}{!}{%
    \subfloat[1 resource]{% GNUPLOT: LaTeX picture using PSTRICKS macros
% Define new PST objects, if not already defined
\ifx\PSTloaded\undefined
\def\PSTloaded{t}

\catcode`@=11

\newpsobject{PST@Border}{psline}{linewidth=.0015,linestyle=solid}

\catcode`@=12

\fi
\psset{unit=5.0in,xunit=5.0in,yunit=3.0in}
\pspicture(0.000000,0.000000)(0.31, 0.35)
\ifx\nofigs\undefined
\catcode`@=11

\newrgbcolor{PST@COLOR0}{1 1 1}
\newrgbcolor{PST@COLOR1}{0.992 0.992 0.992}
\newrgbcolor{PST@COLOR2}{0.984 0.984 0.984}
\newrgbcolor{PST@COLOR3}{0.976 0.976 0.976}
\newrgbcolor{PST@COLOR4}{0.968 0.968 0.968}
\newrgbcolor{PST@COLOR5}{0.96 0.96 0.96}
\newrgbcolor{PST@COLOR6}{0.952 0.952 0.952}
\newrgbcolor{PST@COLOR7}{0.944 0.944 0.944}
\newrgbcolor{PST@COLOR8}{0.937 0.937 0.937}
\newrgbcolor{PST@COLOR9}{0.929 0.929 0.929}
\newrgbcolor{PST@COLOR10}{0.921 0.921 0.921}
\newrgbcolor{PST@COLOR11}{0.913 0.913 0.913}
\newrgbcolor{PST@COLOR12}{0.905 0.905 0.905}
\newrgbcolor{PST@COLOR13}{0.897 0.897 0.897}
\newrgbcolor{PST@COLOR14}{0.889 0.889 0.889}
\newrgbcolor{PST@COLOR15}{0.881 0.881 0.881}
\newrgbcolor{PST@COLOR16}{0.874 0.874 0.874}
\newrgbcolor{PST@COLOR17}{0.866 0.866 0.866}
\newrgbcolor{PST@COLOR18}{0.858 0.858 0.858}
\newrgbcolor{PST@COLOR19}{0.85 0.85 0.85}
\newrgbcolor{PST@COLOR20}{0.842 0.842 0.842}
\newrgbcolor{PST@COLOR21}{0.834 0.834 0.834}
\newrgbcolor{PST@COLOR22}{0.826 0.826 0.826}
\newrgbcolor{PST@COLOR23}{0.818 0.818 0.818}
\newrgbcolor{PST@COLOR24}{0.811 0.811 0.811}
\newrgbcolor{PST@COLOR25}{0.803 0.803 0.803}
\newrgbcolor{PST@COLOR26}{0.795 0.795 0.795}
\newrgbcolor{PST@COLOR27}{0.787 0.787 0.787}
\newrgbcolor{PST@COLOR28}{0.779 0.779 0.779}
\newrgbcolor{PST@COLOR29}{0.771 0.771 0.771}
\newrgbcolor{PST@COLOR30}{0.763 0.763 0.763}
\newrgbcolor{PST@COLOR31}{0.755 0.755 0.755}
\newrgbcolor{PST@COLOR32}{0.748 0.748 0.748}
\newrgbcolor{PST@COLOR33}{0.74 0.74 0.74}
\newrgbcolor{PST@COLOR34}{0.732 0.732 0.732}
\newrgbcolor{PST@COLOR35}{0.724 0.724 0.724}
\newrgbcolor{PST@COLOR36}{0.716 0.716 0.716}
\newrgbcolor{PST@COLOR37}{0.708 0.708 0.708}
\newrgbcolor{PST@COLOR38}{0.7 0.7 0.7}
\newrgbcolor{PST@COLOR39}{0.692 0.692 0.692}
\newrgbcolor{PST@COLOR40}{0.685 0.685 0.685}
\newrgbcolor{PST@COLOR41}{0.677 0.677 0.677}
\newrgbcolor{PST@COLOR42}{0.669 0.669 0.669}
\newrgbcolor{PST@COLOR43}{0.661 0.661 0.661}
\newrgbcolor{PST@COLOR44}{0.653 0.653 0.653}
\newrgbcolor{PST@COLOR45}{0.645 0.645 0.645}
\newrgbcolor{PST@COLOR46}{0.637 0.637 0.637}
\newrgbcolor{PST@COLOR47}{0.629 0.629 0.629}
\newrgbcolor{PST@COLOR48}{0.622 0.622 0.622}
\newrgbcolor{PST@COLOR49}{0.614 0.614 0.614}
\newrgbcolor{PST@COLOR50}{0.606 0.606 0.606}
\newrgbcolor{PST@COLOR51}{0.598 0.598 0.598}
\newrgbcolor{PST@COLOR52}{0.59 0.59 0.59}
\newrgbcolor{PST@COLOR53}{0.582 0.582 0.582}
\newrgbcolor{PST@COLOR54}{0.574 0.574 0.574}
\newrgbcolor{PST@COLOR55}{0.566 0.566 0.566}
\newrgbcolor{PST@COLOR56}{0.559 0.559 0.559}
\newrgbcolor{PST@COLOR57}{0.551 0.551 0.551}
\newrgbcolor{PST@COLOR58}{0.543 0.543 0.543}
\newrgbcolor{PST@COLOR59}{0.535 0.535 0.535}
\newrgbcolor{PST@COLOR60}{0.527 0.527 0.527}
\newrgbcolor{PST@COLOR61}{0.519 0.519 0.519}
\newrgbcolor{PST@COLOR62}{0.511 0.511 0.511}
\newrgbcolor{PST@COLOR63}{0.503 0.503 0.503}
\newrgbcolor{PST@COLOR64}{0.496 0.496 0.496}
\newrgbcolor{PST@COLOR65}{0.488 0.488 0.488}
\newrgbcolor{PST@COLOR66}{0.48 0.48 0.48}
\newrgbcolor{PST@COLOR67}{0.472 0.472 0.472}
\newrgbcolor{PST@COLOR68}{0.464 0.464 0.464}
\newrgbcolor{PST@COLOR69}{0.456 0.456 0.456}
\newrgbcolor{PST@COLOR70}{0.448 0.448 0.448}
\newrgbcolor{PST@COLOR71}{0.44 0.44 0.44}
\newrgbcolor{PST@COLOR72}{0.433 0.433 0.433}
\newrgbcolor{PST@COLOR73}{0.425 0.425 0.425}
\newrgbcolor{PST@COLOR74}{0.417 0.417 0.417}
\newrgbcolor{PST@COLOR75}{0.409 0.409 0.409}
\newrgbcolor{PST@COLOR76}{0.401 0.401 0.401}
\newrgbcolor{PST@COLOR77}{0.393 0.393 0.393}
\newrgbcolor{PST@COLOR78}{0.385 0.385 0.385}
\newrgbcolor{PST@COLOR79}{0.377 0.377 0.377}
\newrgbcolor{PST@COLOR80}{0.37 0.37 0.37}
\newrgbcolor{PST@COLOR81}{0.362 0.362 0.362}
\newrgbcolor{PST@COLOR82}{0.354 0.354 0.354}
\newrgbcolor{PST@COLOR83}{0.346 0.346 0.346}
\newrgbcolor{PST@COLOR84}{0.338 0.338 0.338}
\newrgbcolor{PST@COLOR85}{0.33 0.33 0.33}
\newrgbcolor{PST@COLOR86}{0.322 0.322 0.322}
\newrgbcolor{PST@COLOR87}{0.314 0.314 0.314}
\newrgbcolor{PST@COLOR88}{0.307 0.307 0.307}
\newrgbcolor{PST@COLOR89}{0.299 0.299 0.299}
\newrgbcolor{PST@COLOR90}{0.291 0.291 0.291}
\newrgbcolor{PST@COLOR91}{0.283 0.283 0.283}
\newrgbcolor{PST@COLOR92}{0.275 0.275 0.275}
\newrgbcolor{PST@COLOR93}{0.267 0.267 0.267}
\newrgbcolor{PST@COLOR94}{0.259 0.259 0.259}
\newrgbcolor{PST@COLOR95}{0.251 0.251 0.251}
\newrgbcolor{PST@COLOR96}{0.244 0.244 0.244}
\newrgbcolor{PST@COLOR97}{0.236 0.236 0.236}
\newrgbcolor{PST@COLOR98}{0.228 0.228 0.228}
\newrgbcolor{PST@COLOR99}{0.22 0.22 0.22}
\newrgbcolor{PST@COLOR100}{0.212 0.212 0.212}
\newrgbcolor{PST@COLOR101}{0.204 0.204 0.204}
\newrgbcolor{PST@COLOR102}{0.196 0.196 0.196}
\newrgbcolor{PST@COLOR103}{0.188 0.188 0.188}
\newrgbcolor{PST@COLOR104}{0.181 0.181 0.181}
\newrgbcolor{PST@COLOR105}{0.173 0.173 0.173}
\newrgbcolor{PST@COLOR106}{0.165 0.165 0.165}
\newrgbcolor{PST@COLOR107}{0.157 0.157 0.157}
\newrgbcolor{PST@COLOR108}{0.149 0.149 0.149}
\newrgbcolor{PST@COLOR109}{0.141 0.141 0.141}
\newrgbcolor{PST@COLOR110}{0.133 0.133 0.133}
\newrgbcolor{PST@COLOR111}{0.125 0.125 0.125}
\newrgbcolor{PST@COLOR112}{0.118 0.118 0.118}
\newrgbcolor{PST@COLOR113}{0.11 0.11 0.11}
\newrgbcolor{PST@COLOR114}{0.102 0.102 0.102}
\newrgbcolor{PST@COLOR115}{0.094 0.094 0.094}
\newrgbcolor{PST@COLOR116}{0.086 0.086 0.086}
\newrgbcolor{PST@COLOR117}{0.078 0.078 0.078}
\newrgbcolor{PST@COLOR118}{0.07 0.07 0.07}
\newrgbcolor{PST@COLOR119}{0.062 0.062 0.062}
\newrgbcolor{PST@COLOR120}{0.055 0.055 0.055}
\newrgbcolor{PST@COLOR121}{0.047 0.047 0.047}
\newrgbcolor{PST@COLOR122}{0.039 0.039 0.039}
\newrgbcolor{PST@COLOR123}{0.031 0.031 0.031}
\newrgbcolor{PST@COLOR124}{0.023 0.023 0.023}
\newrgbcolor{PST@COLOR125}{0.015 0.015 0.015}
\newrgbcolor{PST@COLOR126}{0.007 0.007 0.007}
\newrgbcolor{PST@COLOR127}{0 0 0}


\def\polypmIIId#1{\pspolygon[linestyle=none,fillstyle=solid,fillcolor=PST@COLOR#1]}

\polypmIIId{116}(0.1432,0.19)(0.0864,0.19)(0.0864,0.1078)(0.1432,0.1078)
\polypmIIId{121}(0.1432,0.272)(0.0864,0.272)(0.0864,0.19)(0.1432,0.19)
\polypmIIId{125}(0.1432,0.3542)(0.0864,0.3542)(0.0864,0.272)(0.1432,0.272)

\polypmIIId{116}(0.2,0.19)(0.1432,0.19)(0.1432,0.1078)(0.2,0.1078)
\polypmIIId{121}(0.2,0.272)(0.1432,0.272)(0.1432,0.19)(0.2,0.19)
\polypmIIId{125}(0.2,0.3542)(0.1432,0.3542)(0.1432,0.272)(0.2,0.272)

\polypmIIId{117}(0.2568,0.19)(0.2,0.19)(0.2,0.1078)(0.2568,0.1078)
\polypmIIId{122}(0.2568,0.272)(0.2,0.272)(0.2,0.19)(0.2568,0.19)
\polypmIIId{125}(0.2568,0.3542)(0.2,0.3542)(0.2,0.272)(0.2568,0.272)

\polypmIIId{116}(0.3136,0.19)(0.2568,0.19)(0.2568,0.1078)(0.3136,0.1078)
\polypmIIId{122}(0.3136,0.272)(0.2568,0.272)(0.2568,0.19)(0.3136,0.19)
\polypmIIId{126}(0.3136,0.3542)(0.2568,0.3542)(0.2568,0.272)(0.3136,0.272)

\rput(0.1148,0.07){3}
\rput(0.1716,0.07){4}
\rput(0.2284,0.07){5}
\rput(0.2852,0.07){6}
\rput(0.2000,0.0070){years}

\rput[r](0.0806,0.1489){25}
\rput[r](0.0806,0.2310){50}
\rput[r](0.0806,0.3131){100}
\rput{L}(0.0096,0.2310){actions}

\PST@Border(0.0864,0.3542)
(0.0864,0.1078)
(0.3136,0.1078)
(0.3136,0.3542)
(0.0864,0.3542)

\catcode`@=12
\fi
\endpspicture}
    \subfloat[2 resources]{% GNUPLOT: LaTeX picture using PSTRICKS macros
% Define new PST objects, if not already defined
\ifx\PSTloaded\undefined
\def\PSTloaded{t}

\catcode`@=11

\newpsobject{PST@Border}{psline}{linewidth=.0015,linestyle=solid}

\catcode`@=12

\fi
\psset{unit=5.0in,xunit=5.0in,yunit=3.0in}
\pspicture(0.000000,0.000000)(0.225000,0.35)
\ifx\nofigs\undefined
\catcode`@=11

\newrgbcolor{PST@COLOR0}{1 1 1}
\newrgbcolor{PST@COLOR1}{0.992 0.992 0.992}
\newrgbcolor{PST@COLOR2}{0.984 0.984 0.984}
\newrgbcolor{PST@COLOR3}{0.976 0.976 0.976}
\newrgbcolor{PST@COLOR4}{0.968 0.968 0.968}
\newrgbcolor{PST@COLOR5}{0.96 0.96 0.96}
\newrgbcolor{PST@COLOR6}{0.952 0.952 0.952}
\newrgbcolor{PST@COLOR7}{0.944 0.944 0.944}
\newrgbcolor{PST@COLOR8}{0.937 0.937 0.937}
\newrgbcolor{PST@COLOR9}{0.929 0.929 0.929}
\newrgbcolor{PST@COLOR10}{0.921 0.921 0.921}
\newrgbcolor{PST@COLOR11}{0.913 0.913 0.913}
\newrgbcolor{PST@COLOR12}{0.905 0.905 0.905}
\newrgbcolor{PST@COLOR13}{0.897 0.897 0.897}
\newrgbcolor{PST@COLOR14}{0.889 0.889 0.889}
\newrgbcolor{PST@COLOR15}{0.881 0.881 0.881}
\newrgbcolor{PST@COLOR16}{0.874 0.874 0.874}
\newrgbcolor{PST@COLOR17}{0.866 0.866 0.866}
\newrgbcolor{PST@COLOR18}{0.858 0.858 0.858}
\newrgbcolor{PST@COLOR19}{0.85 0.85 0.85}
\newrgbcolor{PST@COLOR20}{0.842 0.842 0.842}
\newrgbcolor{PST@COLOR21}{0.834 0.834 0.834}
\newrgbcolor{PST@COLOR22}{0.826 0.826 0.826}
\newrgbcolor{PST@COLOR23}{0.818 0.818 0.818}
\newrgbcolor{PST@COLOR24}{0.811 0.811 0.811}
\newrgbcolor{PST@COLOR25}{0.803 0.803 0.803}
\newrgbcolor{PST@COLOR26}{0.795 0.795 0.795}
\newrgbcolor{PST@COLOR27}{0.787 0.787 0.787}
\newrgbcolor{PST@COLOR28}{0.779 0.779 0.779}
\newrgbcolor{PST@COLOR29}{0.771 0.771 0.771}
\newrgbcolor{PST@COLOR30}{0.763 0.763 0.763}
\newrgbcolor{PST@COLOR31}{0.755 0.755 0.755}
\newrgbcolor{PST@COLOR32}{0.748 0.748 0.748}
\newrgbcolor{PST@COLOR33}{0.74 0.74 0.74}
\newrgbcolor{PST@COLOR34}{0.732 0.732 0.732}
\newrgbcolor{PST@COLOR35}{0.724 0.724 0.724}
\newrgbcolor{PST@COLOR36}{0.716 0.716 0.716}
\newrgbcolor{PST@COLOR37}{0.708 0.708 0.708}
\newrgbcolor{PST@COLOR38}{0.7 0.7 0.7}
\newrgbcolor{PST@COLOR39}{0.692 0.692 0.692}
\newrgbcolor{PST@COLOR40}{0.685 0.685 0.685}
\newrgbcolor{PST@COLOR41}{0.677 0.677 0.677}
\newrgbcolor{PST@COLOR42}{0.669 0.669 0.669}
\newrgbcolor{PST@COLOR43}{0.661 0.661 0.661}
\newrgbcolor{PST@COLOR44}{0.653 0.653 0.653}
\newrgbcolor{PST@COLOR45}{0.645 0.645 0.645}
\newrgbcolor{PST@COLOR46}{0.637 0.637 0.637}
\newrgbcolor{PST@COLOR47}{0.629 0.629 0.629}
\newrgbcolor{PST@COLOR48}{0.622 0.622 0.622}
\newrgbcolor{PST@COLOR49}{0.614 0.614 0.614}
\newrgbcolor{PST@COLOR50}{0.606 0.606 0.606}
\newrgbcolor{PST@COLOR51}{0.598 0.598 0.598}
\newrgbcolor{PST@COLOR52}{0.59 0.59 0.59}
\newrgbcolor{PST@COLOR53}{0.582 0.582 0.582}
\newrgbcolor{PST@COLOR54}{0.574 0.574 0.574}
\newrgbcolor{PST@COLOR55}{0.566 0.566 0.566}
\newrgbcolor{PST@COLOR56}{0.559 0.559 0.559}
\newrgbcolor{PST@COLOR57}{0.551 0.551 0.551}
\newrgbcolor{PST@COLOR58}{0.543 0.543 0.543}
\newrgbcolor{PST@COLOR59}{0.535 0.535 0.535}
\newrgbcolor{PST@COLOR60}{0.527 0.527 0.527}
\newrgbcolor{PST@COLOR61}{0.519 0.519 0.519}
\newrgbcolor{PST@COLOR62}{0.511 0.511 0.511}
\newrgbcolor{PST@COLOR63}{0.503 0.503 0.503}
\newrgbcolor{PST@COLOR64}{0.496 0.496 0.496}
\newrgbcolor{PST@COLOR65}{0.488 0.488 0.488}
\newrgbcolor{PST@COLOR66}{0.48 0.48 0.48}
\newrgbcolor{PST@COLOR67}{0.472 0.472 0.472}
\newrgbcolor{PST@COLOR68}{0.464 0.464 0.464}
\newrgbcolor{PST@COLOR69}{0.456 0.456 0.456}
\newrgbcolor{PST@COLOR70}{0.448 0.448 0.448}
\newrgbcolor{PST@COLOR71}{0.44 0.44 0.44}
\newrgbcolor{PST@COLOR72}{0.433 0.433 0.433}
\newrgbcolor{PST@COLOR73}{0.425 0.425 0.425}
\newrgbcolor{PST@COLOR74}{0.417 0.417 0.417}
\newrgbcolor{PST@COLOR75}{0.409 0.409 0.409}
\newrgbcolor{PST@COLOR76}{0.401 0.401 0.401}
\newrgbcolor{PST@COLOR77}{0.393 0.393 0.393}
\newrgbcolor{PST@COLOR78}{0.385 0.385 0.385}
\newrgbcolor{PST@COLOR79}{0.377 0.377 0.377}
\newrgbcolor{PST@COLOR80}{0.37 0.37 0.37}
\newrgbcolor{PST@COLOR81}{0.362 0.362 0.362}
\newrgbcolor{PST@COLOR82}{0.354 0.354 0.354}
\newrgbcolor{PST@COLOR83}{0.346 0.346 0.346}
\newrgbcolor{PST@COLOR84}{0.338 0.338 0.338}
\newrgbcolor{PST@COLOR85}{0.33 0.33 0.33}
\newrgbcolor{PST@COLOR86}{0.322 0.322 0.322}
\newrgbcolor{PST@COLOR87}{0.314 0.314 0.314}
\newrgbcolor{PST@COLOR88}{0.307 0.307 0.307}
\newrgbcolor{PST@COLOR89}{0.299 0.299 0.299}
\newrgbcolor{PST@COLOR90}{0.291 0.291 0.291}
\newrgbcolor{PST@COLOR91}{0.283 0.283 0.283}
\newrgbcolor{PST@COLOR92}{0.275 0.275 0.275}
\newrgbcolor{PST@COLOR93}{0.267 0.267 0.267}
\newrgbcolor{PST@COLOR94}{0.259 0.259 0.259}
\newrgbcolor{PST@COLOR95}{0.251 0.251 0.251}
\newrgbcolor{PST@COLOR96}{0.244 0.244 0.244}
\newrgbcolor{PST@COLOR97}{0.236 0.236 0.236}
\newrgbcolor{PST@COLOR98}{0.228 0.228 0.228}
\newrgbcolor{PST@COLOR99}{0.22 0.22 0.22}
\newrgbcolor{PST@COLOR100}{0.212 0.212 0.212}
\newrgbcolor{PST@COLOR101}{0.204 0.204 0.204}
\newrgbcolor{PST@COLOR102}{0.196 0.196 0.196}
\newrgbcolor{PST@COLOR103}{0.188 0.188 0.188}
\newrgbcolor{PST@COLOR104}{0.181 0.181 0.181}
\newrgbcolor{PST@COLOR105}{0.173 0.173 0.173}
\newrgbcolor{PST@COLOR106}{0.165 0.165 0.165}
\newrgbcolor{PST@COLOR107}{0.157 0.157 0.157}
\newrgbcolor{PST@COLOR108}{0.149 0.149 0.149}
\newrgbcolor{PST@COLOR109}{0.141 0.141 0.141}
\newrgbcolor{PST@COLOR110}{0.133 0.133 0.133}
\newrgbcolor{PST@COLOR111}{0.125 0.125 0.125}
\newrgbcolor{PST@COLOR112}{0.118 0.118 0.118}
\newrgbcolor{PST@COLOR113}{0.11 0.11 0.11}
\newrgbcolor{PST@COLOR114}{0.102 0.102 0.102}
\newrgbcolor{PST@COLOR115}{0.094 0.094 0.094}
\newrgbcolor{PST@COLOR116}{0.086 0.086 0.086}
\newrgbcolor{PST@COLOR117}{0.078 0.078 0.078}
\newrgbcolor{PST@COLOR118}{0.07 0.07 0.07}
\newrgbcolor{PST@COLOR119}{0.062 0.062 0.062}
\newrgbcolor{PST@COLOR120}{0.055 0.055 0.055}
\newrgbcolor{PST@COLOR121}{0.047 0.047 0.047}
\newrgbcolor{PST@COLOR122}{0.039 0.039 0.039}
\newrgbcolor{PST@COLOR123}{0.031 0.031 0.031}
\newrgbcolor{PST@COLOR124}{0.023 0.023 0.023}
\newrgbcolor{PST@COLOR125}{0.015 0.015 0.015}
\newrgbcolor{PST@COLOR126}{0.007 0.007 0.007}
\newrgbcolor{PST@COLOR127}{0 0 0}

\def\polypmIIId#1{\pspolygon[linestyle=none,fillstyle=solid,fillcolor=PST@COLOR#1]}

\polypmIIId{111} (0.0568,0.19)  (0.0,0.19)  (0.0,0.1078)(0.0568,0.1078)
\polypmIIId{119}  (0.0568,0.272) (0.0,0.272) (0.0,0.19)  (0.0568,0.19)
\polypmIIId{124}  (0.0568,0.3542)(0.0,0.3542)(0.0,0.272) (0.0568,0.272)

\polypmIIId{111} (0.1136,   0.19)  (0.0568,0.19)  (0.0568,0.1078)(0.1136,0.1078)
\polypmIIId{120}  (0.1136,   0.272) (0.0568,0.272) (0.0568,0.19)  (0.1136,0.19)
\polypmIIId{124}  (0.1136,   0.3542)(0.0568,0.3542)(0.0568,0.272) (0.1136,0.272)

\polypmIIId{112}(0.1704,0.19)  (0.1136,   0.19)  (0.1136,   0.1078)(0.1704,0.1078)
\polypmIIId{120} (0.1704,0.272) (0.1136,   0.272) (0.1136,   0.19)  (0.1704,0.19)
\polypmIIId{125}  (0.1704,0.3542)(0.1136,   0.3542)(0.1136,   0.272) (0.1704,0.272)

\polypmIIId{112}(0.2272,0.19)  (0.1704,0.19)  (0.1704,0.1078)(0.2272,0.1078)
\polypmIIId{120}  (0.2272,0.272) (0.1704,0.272) (0.1704,0.19)  (0.2272,0.19)
\polypmIIId{125}  (0.2272,0.3542)(0.1704,0.3542)(0.1704,0.272) (0.2272,0.272)

\rput(0.0284,0.07){3}
\rput(0.0852,0.07){4}
\rput(0.1420,0.07){5}
\rput(0.1988,0.07){6}
\rput(0.1136,0.0070){years}


\PST@Border(0.0,0.3542)
(0.0,0.1078)
(0.2272,0.1078)
(0.2272,0.3542)
(0.0,0.3542)

\catcode`@=12
\fi
\endpspicture}
    \subfloat[4 resources]{% GNUPLOT: LaTeX picture using PSTRICKS macros
% Define new PST objects, if not already defined
\ifx\PSTloaded\undefined
\def\PSTloaded{t}

\catcode`@=11

\newpsobject{PST@Border}{psline}{linewidth=.0015,linestyle=solid}

\catcode`@=12

\fi
\psset{unit=5.0in,xunit=5.0in,yunit=3.0in}
\pspicture(0.000000,0.000000)(0.3136,0.35)
\ifx\nofigs\undefined
\catcode`@=11

\newrgbcolor{PST@COLOR0}{1 1 1}
\newrgbcolor{PST@COLOR1}{0.992 0.992 0.992}
\newrgbcolor{PST@COLOR2}{0.984 0.984 0.984}
\newrgbcolor{PST@COLOR3}{0.976 0.976 0.976}
\newrgbcolor{PST@COLOR4}{0.968 0.968 0.968}
\newrgbcolor{PST@COLOR5}{0.96 0.96 0.96}
\newrgbcolor{PST@COLOR6}{0.952 0.952 0.952}
\newrgbcolor{PST@COLOR7}{0.944 0.944 0.944}
\newrgbcolor{PST@COLOR8}{0.937 0.937 0.937}
\newrgbcolor{PST@COLOR9}{0.929 0.929 0.929}
\newrgbcolor{PST@COLOR10}{0.921 0.921 0.921}
\newrgbcolor{PST@COLOR11}{0.913 0.913 0.913}
\newrgbcolor{PST@COLOR12}{0.905 0.905 0.905}
\newrgbcolor{PST@COLOR13}{0.897 0.897 0.897}
\newrgbcolor{PST@COLOR14}{0.889 0.889 0.889}
\newrgbcolor{PST@COLOR15}{0.881 0.881 0.881}
\newrgbcolor{PST@COLOR16}{0.874 0.874 0.874}
\newrgbcolor{PST@COLOR17}{0.866 0.866 0.866}
\newrgbcolor{PST@COLOR18}{0.858 0.858 0.858}
\newrgbcolor{PST@COLOR19}{0.85 0.85 0.85}
\newrgbcolor{PST@COLOR20}{0.842 0.842 0.842}
\newrgbcolor{PST@COLOR21}{0.834 0.834 0.834}
\newrgbcolor{PST@COLOR22}{0.826 0.826 0.826}
\newrgbcolor{PST@COLOR23}{0.818 0.818 0.818}
\newrgbcolor{PST@COLOR24}{0.811 0.811 0.811}
\newrgbcolor{PST@COLOR25}{0.803 0.803 0.803}
\newrgbcolor{PST@COLOR26}{0.795 0.795 0.795}
\newrgbcolor{PST@COLOR27}{0.787 0.787 0.787}
\newrgbcolor{PST@COLOR28}{0.779 0.779 0.779}
\newrgbcolor{PST@COLOR29}{0.771 0.771 0.771}
\newrgbcolor{PST@COLOR30}{0.763 0.763 0.763}
\newrgbcolor{PST@COLOR31}{0.755 0.755 0.755}
\newrgbcolor{PST@COLOR32}{0.748 0.748 0.748}
\newrgbcolor{PST@COLOR33}{0.74 0.74 0.74}
\newrgbcolor{PST@COLOR34}{0.732 0.732 0.732}
\newrgbcolor{PST@COLOR35}{0.724 0.724 0.724}
\newrgbcolor{PST@COLOR36}{0.716 0.716 0.716}
\newrgbcolor{PST@COLOR37}{0.708 0.708 0.708}
\newrgbcolor{PST@COLOR38}{0.7 0.7 0.7}
\newrgbcolor{PST@COLOR39}{0.692 0.692 0.692}
\newrgbcolor{PST@COLOR40}{0.685 0.685 0.685}
\newrgbcolor{PST@COLOR41}{0.677 0.677 0.677}
\newrgbcolor{PST@COLOR42}{0.669 0.669 0.669}
\newrgbcolor{PST@COLOR43}{0.661 0.661 0.661}
\newrgbcolor{PST@COLOR44}{0.653 0.653 0.653}
\newrgbcolor{PST@COLOR45}{0.645 0.645 0.645}
\newrgbcolor{PST@COLOR46}{0.637 0.637 0.637}
\newrgbcolor{PST@COLOR47}{0.629 0.629 0.629}
\newrgbcolor{PST@COLOR48}{0.622 0.622 0.622}
\newrgbcolor{PST@COLOR49}{0.614 0.614 0.614}
\newrgbcolor{PST@COLOR50}{0.606 0.606 0.606}
\newrgbcolor{PST@COLOR51}{0.598 0.598 0.598}
\newrgbcolor{PST@COLOR52}{0.59 0.59 0.59}
\newrgbcolor{PST@COLOR53}{0.582 0.582 0.582}
\newrgbcolor{PST@COLOR54}{0.574 0.574 0.574}
\newrgbcolor{PST@COLOR55}{0.566 0.566 0.566}
\newrgbcolor{PST@COLOR56}{0.559 0.559 0.559}
\newrgbcolor{PST@COLOR57}{0.551 0.551 0.551}
\newrgbcolor{PST@COLOR58}{0.543 0.543 0.543}
\newrgbcolor{PST@COLOR59}{0.535 0.535 0.535}
\newrgbcolor{PST@COLOR60}{0.527 0.527 0.527}
\newrgbcolor{PST@COLOR61}{0.519 0.519 0.519}
\newrgbcolor{PST@COLOR62}{0.511 0.511 0.511}
\newrgbcolor{PST@COLOR63}{0.503 0.503 0.503}
\newrgbcolor{PST@COLOR64}{0.496 0.496 0.496}
\newrgbcolor{PST@COLOR65}{0.488 0.488 0.488}
\newrgbcolor{PST@COLOR66}{0.48 0.48 0.48}
\newrgbcolor{PST@COLOR67}{0.472 0.472 0.472}
\newrgbcolor{PST@COLOR68}{0.464 0.464 0.464}
\newrgbcolor{PST@COLOR69}{0.456 0.456 0.456}
\newrgbcolor{PST@COLOR70}{0.448 0.448 0.448}
\newrgbcolor{PST@COLOR71}{0.44 0.44 0.44}
\newrgbcolor{PST@COLOR72}{0.433 0.433 0.433}
\newrgbcolor{PST@COLOR73}{0.425 0.425 0.425}
\newrgbcolor{PST@COLOR74}{0.417 0.417 0.417}
\newrgbcolor{PST@COLOR75}{0.409 0.409 0.409}
\newrgbcolor{PST@COLOR76}{0.401 0.401 0.401}
\newrgbcolor{PST@COLOR77}{0.393 0.393 0.393}
\newrgbcolor{PST@COLOR78}{0.385 0.385 0.385}
\newrgbcolor{PST@COLOR79}{0.377 0.377 0.377}
\newrgbcolor{PST@COLOR80}{0.37 0.37 0.37}
\newrgbcolor{PST@COLOR81}{0.362 0.362 0.362}
\newrgbcolor{PST@COLOR82}{0.354 0.354 0.354}
\newrgbcolor{PST@COLOR83}{0.346 0.346 0.346}
\newrgbcolor{PST@COLOR84}{0.338 0.338 0.338}
\newrgbcolor{PST@COLOR85}{0.33 0.33 0.33}
\newrgbcolor{PST@COLOR86}{0.322 0.322 0.322}
\newrgbcolor{PST@COLOR87}{0.314 0.314 0.314}
\newrgbcolor{PST@COLOR88}{0.307 0.307 0.307}
\newrgbcolor{PST@COLOR89}{0.299 0.299 0.299}
\newrgbcolor{PST@COLOR90}{0.291 0.291 0.291}
\newrgbcolor{PST@COLOR91}{0.283 0.283 0.283}
\newrgbcolor{PST@COLOR92}{0.275 0.275 0.275}
\newrgbcolor{PST@COLOR93}{0.267 0.267 0.267}
\newrgbcolor{PST@COLOR94}{0.259 0.259 0.259}
\newrgbcolor{PST@COLOR95}{0.251 0.251 0.251}
\newrgbcolor{PST@COLOR96}{0.244 0.244 0.244}
\newrgbcolor{PST@COLOR97}{0.236 0.236 0.236}
\newrgbcolor{PST@COLOR98}{0.228 0.228 0.228}
\newrgbcolor{PST@COLOR99}{0.22 0.22 0.22}
\newrgbcolor{PST@COLOR100}{0.212 0.212 0.212}
\newrgbcolor{PST@COLOR101}{0.204 0.204 0.204}
\newrgbcolor{PST@COLOR102}{0.196 0.196 0.196}
\newrgbcolor{PST@COLOR103}{0.188 0.188 0.188}
\newrgbcolor{PST@COLOR104}{0.181 0.181 0.181}
\newrgbcolor{PST@COLOR105}{0.173 0.173 0.173}
\newrgbcolor{PST@COLOR106}{0.165 0.165 0.165}
\newrgbcolor{PST@COLOR107}{0.157 0.157 0.157}
\newrgbcolor{PST@COLOR108}{0.149 0.149 0.149}
\newrgbcolor{PST@COLOR109}{0.141 0.141 0.141}
\newrgbcolor{PST@COLOR110}{0.133 0.133 0.133}
\newrgbcolor{PST@COLOR111}{0.125 0.125 0.125}
\newrgbcolor{PST@COLOR112}{0.118 0.118 0.118}
\newrgbcolor{PST@COLOR113}{0.11 0.11 0.11}
\newrgbcolor{PST@COLOR114}{0.102 0.102 0.102}
\newrgbcolor{PST@COLOR115}{0.094 0.094 0.094}
\newrgbcolor{PST@COLOR116}{0.086 0.086 0.086}
\newrgbcolor{PST@COLOR117}{0.078 0.078 0.078}
\newrgbcolor{PST@COLOR118}{0.07 0.07 0.07}
\newrgbcolor{PST@COLOR119}{0.062 0.062 0.062}
\newrgbcolor{PST@COLOR120}{0.055 0.055 0.055}
\newrgbcolor{PST@COLOR121}{0.047 0.047 0.047}
\newrgbcolor{PST@COLOR122}{0.039 0.039 0.039}
\newrgbcolor{PST@COLOR123}{0.031 0.031 0.031}
\newrgbcolor{PST@COLOR124}{0.023 0.023 0.023}
\newrgbcolor{PST@COLOR125}{0.015 0.015 0.015}
\newrgbcolor{PST@COLOR126}{0.007 0.007 0.007}
\newrgbcolor{PST@COLOR127}{0 0 0}

\def\polypmIIId#1{\pspolygon[linestyle=none,fillstyle=solid,fillcolor=PST@COLOR#1]}

\polypmIIId{101} (0.0568,0.19)  (0.0,0.19)  (0.0,0.1078)(0.0568,0.1078)
\polypmIIId{113}  (0.0568,0.272) (0.0,0.272) (0.0,0.19)  (0.0568,0.19)
\polypmIIId{121}  (0.0568,0.3542)(0.0,0.3542)(0.0,0.272) (0.0568,0.272)

\polypmIIId{101} (0.1136,   0.19)  (0.0568,0.19)  (0.0568,0.1078)(0.1136,0.1078)
\polypmIIId{113}  (0.1136,   0.272) (0.0568,0.272) (0.0568,0.19)  (0.1136,0.19)
\polypmIIId{121}  (0.1136,   0.3542)(0.0568,0.3542)(0.0568,0.272) (0.1136,0.272)

\polypmIIId{102}(0.1704,0.19)  (0.1136,   0.19)  (0.1136,   0.1078)(0.1704,0.1078)
\polypmIIId{114} (0.1704,0.272) (0.1136,   0.272) (0.1136,   0.19)  (0.1704,0.19)
\polypmIIId{121}  (0.1704,0.3542)(0.1136,   0.3542)(0.1136,   0.272) (0.1704,0.272)

\polypmIIId{102}(0.2272,0.19)  (0.1704,0.19)  (0.1704,0.1078)(0.2272,0.1078)
\polypmIIId{114}  (0.2272,0.272) (0.1704,0.272) (0.1704,0.19)  (0.2272,0.19)
\polypmIIId{122}  (0.2272,0.3542)(0.1704,0.3542)(0.1704,0.272) (0.2272,0.272)

\rput(0.0284,0.07){3}
\rput(0.0852,0.07){4}
\rput(0.1420,0.07){5}
\rput(0.1988,0.07){6}
\rput(0.1136,0.0070){years}

\PST@Border(0.0,0.3542)
(0.0,0.1078)
(0.2272,0.1078)
(0.2272,0.3542)
(0.0,0.3542)

\polypmIIId{0}(0.2329,0.1078)(0.2442,0.1078)(0.2442,0.1098)(0.2329,0.1098)
\polypmIIId{1}(0.2329,0.1097)(0.2442,0.1097)(0.2442,0.1117)(0.2329,0.1117)
\polypmIIId{2}(0.2329,0.1116)(0.2442,0.1116)(0.2442,0.1136)(0.2329,0.1136)
\polypmIIId{3}(0.2329,0.1135)(0.2442,0.1135)(0.2442,0.1156)(0.2329,0.1156)
\polypmIIId{4}(0.2329,0.1155)(0.2442,0.1155)(0.2442,0.1175)(0.2329,0.1175)
\polypmIIId{5}(0.2329,0.1174)(0.2442,0.1174)(0.2442,0.1194)(0.2329,0.1194)
\polypmIIId{6}(0.2329,0.1193)(0.2442,0.1193)(0.2442,0.1213)(0.2329,0.1213)
\polypmIIId{7}(0.2329,0.1212)(0.2442,0.1212)(0.2442,0.1233)(0.2329,0.1233)
\polypmIIId{8}(0.2329,0.1232)(0.2442,0.1232)(0.2442,0.1252)(0.2329,0.1252)
\polypmIIId{9}(0.2329,0.1251)(0.2442,0.1251)(0.2442,0.1271)(0.2329,0.1271)
\polypmIIId{10}(0.2329,0.127)(0.2442,0.127)(0.2442,0.129)(0.2329,0.129)
\polypmIIId{11}(0.2329,0.1289)(0.2442,0.1289)(0.2442,0.131)(0.2329,0.131)
\polypmIIId{12}(0.2329,0.1309)(0.2442,0.1309)(0.2442,0.1329)(0.2329,0.1329)
\polypmIIId{13}(0.2329,0.1328)(0.2442,0.1328)(0.2442,0.1348)(0.2329,0.1348)
\polypmIIId{14}(0.2329,0.1347)(0.2442,0.1347)(0.2442,0.1367)(0.2329,0.1367)
\polypmIIId{15}(0.2329,0.1366)(0.2442,0.1366)(0.2442,0.1387)(0.2329,0.1387)
\polypmIIId{16}(0.2329,0.1386)(0.2442,0.1386)(0.2442,0.1406)(0.2329,0.1406)
\polypmIIId{17}(0.2329,0.1405)(0.2442,0.1405)(0.2442,0.1425)(0.2329,0.1425)
\polypmIIId{18}(0.2329,0.1424)(0.2442,0.1424)(0.2442,0.1444)(0.2329,0.1444)
\polypmIIId{19}(0.2329,0.1443)(0.2442,0.1443)(0.2442,0.1464)(0.2329,0.1464)
\polypmIIId{20}(0.2329,0.1463)(0.2442,0.1463)(0.2442,0.1483)(0.2329,0.1483)
\polypmIIId{21}(0.2329,0.1482)(0.2442,0.1482)(0.2442,0.1502)(0.2329,0.1502)
\polypmIIId{22}(0.2329,0.1501)(0.2442,0.1501)(0.2442,0.1521)(0.2329,0.1521)
\polypmIIId{23}(0.2329,0.152)(0.2442,0.152)(0.2442,0.1541)(0.2329,0.1541)
\polypmIIId{24}(0.2329,0.154)(0.2442,0.154)(0.2442,0.156)(0.2329,0.156)
\polypmIIId{25}(0.2329,0.1559)(0.2442,0.1559)(0.2442,0.1579)(0.2329,0.1579)
\polypmIIId{26}(0.2329,0.1578)(0.2442,0.1578)(0.2442,0.1598)(0.2329,0.1598)
\polypmIIId{27}(0.2329,0.1597)(0.2442,0.1597)(0.2442,0.1618)(0.2329,0.1618)
\polypmIIId{28}(0.2329,0.1617)(0.2442,0.1617)(0.2442,0.1637)(0.2329,0.1637)
\polypmIIId{29}(0.2329,0.1636)(0.2442,0.1636)(0.2442,0.1656)(0.2329,0.1656)
\polypmIIId{30}(0.2329,0.1655)(0.2442,0.1655)(0.2442,0.1675)(0.2329,0.1675)
\polypmIIId{31}(0.2329,0.1674)(0.2442,0.1674)(0.2442,0.1695)(0.2329,0.1695)
\polypmIIId{32}(0.2329,0.1694)(0.2442,0.1694)(0.2442,0.1714)(0.2329,0.1714)
\polypmIIId{33}(0.2329,0.1713)(0.2442,0.1713)(0.2442,0.1733)(0.2329,0.1733)
\polypmIIId{34}(0.2329,0.1732)(0.2442,0.1732)(0.2442,0.1752)(0.2329,0.1752)
\polypmIIId{35}(0.2329,0.1751)(0.2442,0.1751)(0.2442,0.1772)(0.2329,0.1772)
\polypmIIId{36}(0.2329,0.1771)(0.2442,0.1771)(0.2442,0.1791)(0.2329,0.1791)
\polypmIIId{37}(0.2329,0.179)(0.2442,0.179)(0.2442,0.181)(0.2329,0.181)
\polypmIIId{38}(0.2329,0.1809)(0.2442,0.1809)(0.2442,0.1829)(0.2329,0.1829)
\polypmIIId{39}(0.2329,0.1828)(0.2442,0.1828)(0.2442,0.1849)(0.2329,0.1849)
\polypmIIId{40}(0.2329,0.1848)(0.2442,0.1848)(0.2442,0.1868)(0.2329,0.1868)
\polypmIIId{41}(0.2329,0.1867)(0.2442,0.1867)(0.2442,0.1887)(0.2329,0.1887)
\polypmIIId{42}(0.2329,0.1886)(0.2442,0.1886)(0.2442,0.1906)(0.2329,0.1906)
\polypmIIId{43}(0.2329,0.1905)(0.2442,0.1905)(0.2442,0.1926)(0.2329,0.1926)
\polypmIIId{44}(0.2329,0.1925)(0.2442,0.1925)(0.2442,0.1945)(0.2329,0.1945)
\polypmIIId{45}(0.2329,0.1944)(0.2442,0.1944)(0.2442,0.1964)(0.2329,0.1964)
\polypmIIId{46}(0.2329,0.1963)(0.2442,0.1963)(0.2442,0.1983)(0.2329,0.1983)
\polypmIIId{47}(0.2329,0.1982)(0.2442,0.1982)(0.2442,0.2003)(0.2329,0.2003)
\polypmIIId{48}(0.2329,0.2002)(0.2442,0.2002)(0.2442,0.2022)(0.2329,0.2022)
\polypmIIId{49}(0.2329,0.2021)(0.2442,0.2021)(0.2442,0.2041)(0.2329,0.2041)
\polypmIIId{50}(0.2329,0.204)(0.2442,0.204)(0.2442,0.206)(0.2329,0.206)
\polypmIIId{51}(0.2329,0.2059)(0.2442,0.2059)(0.2442,0.208)(0.2329,0.208)
\polypmIIId{52}(0.2329,0.2079)(0.2442,0.2079)(0.2442,0.2099)(0.2329,0.2099)
\polypmIIId{53}(0.2329,0.2098)(0.2442,0.2098)(0.2442,0.2118)(0.2329,0.2118)
\polypmIIId{54}(0.2329,0.2117)(0.2442,0.2117)(0.2442,0.2137)(0.2329,0.2137)
\polypmIIId{55}(0.2329,0.2136)(0.2442,0.2136)(0.2442,0.2157)(0.2329,0.2157)
\polypmIIId{56}(0.2329,0.2156)(0.2442,0.2156)(0.2442,0.2176)(0.2329,0.2176)
\polypmIIId{57}(0.2329,0.2175)(0.2442,0.2175)(0.2442,0.2195)(0.2329,0.2195)
\polypmIIId{58}(0.2329,0.2194)(0.2442,0.2194)(0.2442,0.2214)(0.2329,0.2214)
\polypmIIId{59}(0.2329,0.2213)(0.2442,0.2213)(0.2442,0.2234)(0.2329,0.2234)
\polypmIIId{60}(0.2329,0.2233)(0.2442,0.2233)(0.2442,0.2253)(0.2329,0.2253)
\polypmIIId{61}(0.2329,0.2252)(0.2442,0.2252)(0.2442,0.2272)(0.2329,0.2272)
\polypmIIId{62}(0.2329,0.2271)(0.2442,0.2271)(0.2442,0.2291)(0.2329,0.2291)
\polypmIIId{63}(0.2329,0.229)(0.2442,0.229)(0.2442,0.2311)(0.2329,0.2311)
\polypmIIId{64}(0.2329,0.231)(0.2442,0.231)(0.2442,0.233)(0.2329,0.233)
\polypmIIId{65}(0.2329,0.2329)(0.2442,0.2329)(0.2442,0.2349)(0.2329,0.2349)
\polypmIIId{66}(0.2329,0.2348)(0.2442,0.2348)(0.2442,0.2368)(0.2329,0.2368)
\polypmIIId{67}(0.2329,0.2367)(0.2442,0.2367)(0.2442,0.2388)(0.2329,0.2388)
\polypmIIId{68}(0.2329,0.2387)(0.2442,0.2387)(0.2442,0.2407)(0.2329,0.2407)
\polypmIIId{69}(0.2329,0.2406)(0.2442,0.2406)(0.2442,0.2426)(0.2329,0.2426)
\polypmIIId{70}(0.2329,0.2425)(0.2442,0.2425)(0.2442,0.2445)(0.2329,0.2445)
\polypmIIId{71}(0.2329,0.2444)(0.2442,0.2444)(0.2442,0.2465)(0.2329,0.2465)
\polypmIIId{72}(0.2329,0.2464)(0.2442,0.2464)(0.2442,0.2484)(0.2329,0.2484)
\polypmIIId{73}(0.2329,0.2483)(0.2442,0.2483)(0.2442,0.2503)(0.2329,0.2503)
\polypmIIId{74}(0.2329,0.2502)(0.2442,0.2502)(0.2442,0.2522)(0.2329,0.2522)
\polypmIIId{75}(0.2329,0.2521)(0.2442,0.2521)(0.2442,0.2542)(0.2329,0.2542)
\polypmIIId{76}(0.2329,0.2541)(0.2442,0.2541)(0.2442,0.2561)(0.2329,0.2561)
\polypmIIId{77}(0.2329,0.256)(0.2442,0.256)(0.2442,0.258)(0.2329,0.258)
\polypmIIId{78}(0.2329,0.2579)(0.2442,0.2579)(0.2442,0.2599)(0.2329,0.2599)
\polypmIIId{79}(0.2329,0.2598)(0.2442,0.2598)(0.2442,0.2619)(0.2329,0.2619)
\polypmIIId{80}(0.2329,0.2618)(0.2442,0.2618)(0.2442,0.2638)(0.2329,0.2638)
\polypmIIId{81}(0.2329,0.2637)(0.2442,0.2637)(0.2442,0.2657)(0.2329,0.2657)
\polypmIIId{82}(0.2329,0.2656)(0.2442,0.2656)(0.2442,0.2676)(0.2329,0.2676)
\polypmIIId{83}(0.2329,0.2675)(0.2442,0.2675)(0.2442,0.2696)(0.2329,0.2696)
\polypmIIId{84}(0.2329,0.2695)(0.2442,0.2695)(0.2442,0.2715)(0.2329,0.2715)
\polypmIIId{85}(0.2329,0.2714)(0.2442,0.2714)(0.2442,0.2734)(0.2329,0.2734)
\polypmIIId{86}(0.2329,0.2733)(0.2442,0.2733)(0.2442,0.2753)(0.2329,0.2753)
\polypmIIId{87}(0.2329,0.2752)(0.2442,0.2752)(0.2442,0.2773)(0.2329,0.2773)
\polypmIIId{88}(0.2329,0.2772)(0.2442,0.2772)(0.2442,0.2792)(0.2329,0.2792)
\polypmIIId{89}(0.2329,0.2791)(0.2442,0.2791)(0.2442,0.2811)(0.2329,0.2811)
\polypmIIId{90}(0.2329,0.281)(0.2442,0.281)(0.2442,0.283)(0.2329,0.283)
\polypmIIId{91}(0.2329,0.2829)(0.2442,0.2829)(0.2442,0.285)(0.2329,0.285)
\polypmIIId{92}(0.2329,0.2849)(0.2442,0.2849)(0.2442,0.2869)(0.2329,0.2869)
\polypmIIId{93}(0.2329,0.2868)(0.2442,0.2868)(0.2442,0.2888)(0.2329,0.2888)
\polypmIIId{94}(0.2329,0.2887)(0.2442,0.2887)(0.2442,0.2907)(0.2329,0.2907)
\polypmIIId{95}(0.2329,0.2906)(0.2442,0.2906)(0.2442,0.2927)(0.2329,0.2927)
\polypmIIId{96}(0.2329,0.2926)(0.2442,0.2926)(0.2442,0.2946)(0.2329,0.2946)
\polypmIIId{97}(0.2329,0.2945)(0.2442,0.2945)(0.2442,0.2965)(0.2329,0.2965)
\polypmIIId{98}(0.2329,0.2964)(0.2442,0.2964)(0.2442,0.2984)(0.2329,0.2984)
\polypmIIId{99}(0.2329,0.2983)(0.2442,0.2983)(0.2442,0.3004)(0.2329,0.3004)
\polypmIIId{100}(0.2329,0.3003)(0.2442,0.3003)(0.2442,0.3023)(0.2329,0.3023)
\polypmIIId{101}(0.2329,0.3022)(0.2442,0.3022)(0.2442,0.3042)(0.2329,0.3042)
\polypmIIId{102}(0.2329,0.3041)(0.2442,0.3041)(0.2442,0.3061)(0.2329,0.3061)
\polypmIIId{103}(0.2329,0.306)(0.2442,0.306)(0.2442,0.3081)(0.2329,0.3081)
\polypmIIId{104}(0.2329,0.308)(0.2442,0.308)(0.2442,0.31)(0.2329,0.31)
\polypmIIId{105}(0.2329,0.3099)(0.2442,0.3099)(0.2442,0.3119)(0.2329,0.3119)
\polypmIIId{106}(0.2329,0.3118)(0.2442,0.3118)(0.2442,0.3138)(0.2329,0.3138)
\polypmIIId{107}(0.2329,0.3137)(0.2442,0.3137)(0.2442,0.3158)(0.2329,0.3158)
\polypmIIId{108}(0.2329,0.3157)(0.2442,0.3157)(0.2442,0.3177)(0.2329,0.3177)
\polypmIIId{109}(0.2329,0.3176)(0.2442,0.3176)(0.2442,0.3196)(0.2329,0.3196)
\polypmIIId{110}(0.2329,0.3195)(0.2442,0.3195)(0.2442,0.3215)(0.2329,0.3215)
\polypmIIId{111}(0.2329,0.3214)(0.2442,0.3214)(0.2442,0.3235)(0.2329,0.3235)
\polypmIIId{112}(0.2329,0.3234)(0.2442,0.3234)(0.2442,0.3254)(0.2329,0.3254)
\polypmIIId{113}(0.2329,0.3253)(0.2442,0.3253)(0.2442,0.3273)(0.2329,0.3273)
\polypmIIId{114}(0.2329,0.3272)(0.2442,0.3272)(0.2442,0.3292)(0.2329,0.3292)
\polypmIIId{115}(0.2329,0.3291)(0.2442,0.3291)(0.2442,0.3312)(0.2329,0.3312)
\polypmIIId{116}(0.2329,0.3311)(0.2442,0.3311)(0.2442,0.3331)(0.2329,0.3331)
\polypmIIId{117}(0.2329,0.333)(0.2442,0.333)(0.2442,0.335)(0.2329,0.335)
\polypmIIId{118}(0.2329,0.3349)(0.2442,0.3349)(0.2442,0.3369)(0.2329,0.3369)
\polypmIIId{119}(0.2329,0.3368)(0.2442,0.3368)(0.2442,0.3389)(0.2329,0.3389)
\polypmIIId{120}(0.2329,0.3388)(0.2442,0.3388)(0.2442,0.3408)(0.2329,0.3408)
\polypmIIId{121}(0.2329,0.3407)(0.2442,0.3407)(0.2442,0.3427)(0.2329,0.3427)
\polypmIIId{122}(0.2329,0.3426)(0.2442,0.3426)(0.2442,0.3446)(0.2329,0.3446)
\polypmIIId{123}(0.2329,0.3445)(0.2442,0.3445)(0.2442,0.3466)(0.2329,0.3466)
\polypmIIId{124}(0.2329,0.3465)(0.2442,0.3465)(0.2442,0.3485)(0.2329,0.3485)
\polypmIIId{125}(0.2329,0.3484)(0.2442,0.3484)(0.2442,0.3504)(0.2329,0.3504)
\polypmIIId{126}(0.2329,0.3503)(0.2442,0.3503)(0.2442,0.3523)(0.2329,0.3523)
\polypmIIId{127}(0.2329,0.3522)(0.2442,0.3522)(0.2442,0.3542)(0.2329,0.3542)

\PST@Border(0.2329,0.1078)
(0.2442,0.1078)
(0.2442,0.3542)
(0.2329,0.3542)
(0.2329,0.1078)


\rput[l](0.2502,0.1301){0.997}
\rput[l](0.2502,0.2048){0.998}
\rput[l](0.2502,0.2795){0.999}
\rput[l](0.2502,0.3542){1}

\catcode`@=12
\fi
\endpspicture}
  }
  $\alpha = 0.0$
  %\label{fig:tabusolcomp00}
\end{figure}

\figspaces
\begin{figure}[H]
  \centering
  \resizebox{\columnwidth}{!}{%
    \subfloat[1 resource]{% GNUPLOT: LaTeX picture using PSTRICKS macros
% Define new PST objects, if not already defined
\ifx\PSTloaded\undefined
\def\PSTloaded{t}

\catcode`@=11

\newpsobject{PST@Border}{psline}{linewidth=.0015,linestyle=solid}

\catcode`@=12

\fi
\psset{unit=5.0in,xunit=5.0in,yunit=3.0in}
\pspicture(0.000000,0.000000)(0.31, 0.35)
\ifx\nofigs\undefined
\catcode`@=11

\newrgbcolor{PST@COLOR0}{1 1 1}
\newrgbcolor{PST@COLOR1}{0.992 0.992 0.992}
\newrgbcolor{PST@COLOR2}{0.984 0.984 0.984}
\newrgbcolor{PST@COLOR3}{0.976 0.976 0.976}
\newrgbcolor{PST@COLOR4}{0.968 0.968 0.968}
\newrgbcolor{PST@COLOR5}{0.96 0.96 0.96}
\newrgbcolor{PST@COLOR6}{0.952 0.952 0.952}
\newrgbcolor{PST@COLOR7}{0.944 0.944 0.944}
\newrgbcolor{PST@COLOR8}{0.937 0.937 0.937}
\newrgbcolor{PST@COLOR9}{0.929 0.929 0.929}
\newrgbcolor{PST@COLOR10}{0.921 0.921 0.921}
\newrgbcolor{PST@COLOR11}{0.913 0.913 0.913}
\newrgbcolor{PST@COLOR12}{0.905 0.905 0.905}
\newrgbcolor{PST@COLOR13}{0.897 0.897 0.897}
\newrgbcolor{PST@COLOR14}{0.889 0.889 0.889}
\newrgbcolor{PST@COLOR15}{0.881 0.881 0.881}
\newrgbcolor{PST@COLOR16}{0.874 0.874 0.874}
\newrgbcolor{PST@COLOR17}{0.866 0.866 0.866}
\newrgbcolor{PST@COLOR18}{0.858 0.858 0.858}
\newrgbcolor{PST@COLOR19}{0.85 0.85 0.85}
\newrgbcolor{PST@COLOR20}{0.842 0.842 0.842}
\newrgbcolor{PST@COLOR21}{0.834 0.834 0.834}
\newrgbcolor{PST@COLOR22}{0.826 0.826 0.826}
\newrgbcolor{PST@COLOR23}{0.818 0.818 0.818}
\newrgbcolor{PST@COLOR24}{0.811 0.811 0.811}
\newrgbcolor{PST@COLOR25}{0.803 0.803 0.803}
\newrgbcolor{PST@COLOR26}{0.795 0.795 0.795}
\newrgbcolor{PST@COLOR27}{0.787 0.787 0.787}
\newrgbcolor{PST@COLOR28}{0.779 0.779 0.779}
\newrgbcolor{PST@COLOR29}{0.771 0.771 0.771}
\newrgbcolor{PST@COLOR30}{0.763 0.763 0.763}
\newrgbcolor{PST@COLOR31}{0.755 0.755 0.755}
\newrgbcolor{PST@COLOR32}{0.748 0.748 0.748}
\newrgbcolor{PST@COLOR33}{0.74 0.74 0.74}
\newrgbcolor{PST@COLOR34}{0.732 0.732 0.732}
\newrgbcolor{PST@COLOR35}{0.724 0.724 0.724}
\newrgbcolor{PST@COLOR36}{0.716 0.716 0.716}
\newrgbcolor{PST@COLOR37}{0.708 0.708 0.708}
\newrgbcolor{PST@COLOR38}{0.7 0.7 0.7}
\newrgbcolor{PST@COLOR39}{0.692 0.692 0.692}
\newrgbcolor{PST@COLOR40}{0.685 0.685 0.685}
\newrgbcolor{PST@COLOR41}{0.677 0.677 0.677}
\newrgbcolor{PST@COLOR42}{0.669 0.669 0.669}
\newrgbcolor{PST@COLOR43}{0.661 0.661 0.661}
\newrgbcolor{PST@COLOR44}{0.653 0.653 0.653}
\newrgbcolor{PST@COLOR45}{0.645 0.645 0.645}
\newrgbcolor{PST@COLOR46}{0.637 0.637 0.637}
\newrgbcolor{PST@COLOR47}{0.629 0.629 0.629}
\newrgbcolor{PST@COLOR48}{0.622 0.622 0.622}
\newrgbcolor{PST@COLOR49}{0.614 0.614 0.614}
\newrgbcolor{PST@COLOR50}{0.606 0.606 0.606}
\newrgbcolor{PST@COLOR51}{0.598 0.598 0.598}
\newrgbcolor{PST@COLOR52}{0.59 0.59 0.59}
\newrgbcolor{PST@COLOR53}{0.582 0.582 0.582}
\newrgbcolor{PST@COLOR54}{0.574 0.574 0.574}
\newrgbcolor{PST@COLOR55}{0.566 0.566 0.566}
\newrgbcolor{PST@COLOR56}{0.559 0.559 0.559}
\newrgbcolor{PST@COLOR57}{0.551 0.551 0.551}
\newrgbcolor{PST@COLOR58}{0.543 0.543 0.543}
\newrgbcolor{PST@COLOR59}{0.535 0.535 0.535}
\newrgbcolor{PST@COLOR60}{0.527 0.527 0.527}
\newrgbcolor{PST@COLOR61}{0.519 0.519 0.519}
\newrgbcolor{PST@COLOR62}{0.511 0.511 0.511}
\newrgbcolor{PST@COLOR63}{0.503 0.503 0.503}
\newrgbcolor{PST@COLOR64}{0.496 0.496 0.496}
\newrgbcolor{PST@COLOR65}{0.488 0.488 0.488}
\newrgbcolor{PST@COLOR66}{0.48 0.48 0.48}
\newrgbcolor{PST@COLOR67}{0.472 0.472 0.472}
\newrgbcolor{PST@COLOR68}{0.464 0.464 0.464}
\newrgbcolor{PST@COLOR69}{0.456 0.456 0.456}
\newrgbcolor{PST@COLOR70}{0.448 0.448 0.448}
\newrgbcolor{PST@COLOR71}{0.44 0.44 0.44}
\newrgbcolor{PST@COLOR72}{0.433 0.433 0.433}
\newrgbcolor{PST@COLOR73}{0.425 0.425 0.425}
\newrgbcolor{PST@COLOR74}{0.417 0.417 0.417}
\newrgbcolor{PST@COLOR75}{0.409 0.409 0.409}
\newrgbcolor{PST@COLOR76}{0.401 0.401 0.401}
\newrgbcolor{PST@COLOR77}{0.393 0.393 0.393}
\newrgbcolor{PST@COLOR78}{0.385 0.385 0.385}
\newrgbcolor{PST@COLOR79}{0.377 0.377 0.377}
\newrgbcolor{PST@COLOR80}{0.37 0.37 0.37}
\newrgbcolor{PST@COLOR81}{0.362 0.362 0.362}
\newrgbcolor{PST@COLOR82}{0.354 0.354 0.354}
\newrgbcolor{PST@COLOR83}{0.346 0.346 0.346}
\newrgbcolor{PST@COLOR84}{0.338 0.338 0.338}
\newrgbcolor{PST@COLOR85}{0.33 0.33 0.33}
\newrgbcolor{PST@COLOR86}{0.322 0.322 0.322}
\newrgbcolor{PST@COLOR87}{0.314 0.314 0.314}
\newrgbcolor{PST@COLOR88}{0.307 0.307 0.307}
\newrgbcolor{PST@COLOR89}{0.299 0.299 0.299}
\newrgbcolor{PST@COLOR90}{0.291 0.291 0.291}
\newrgbcolor{PST@COLOR91}{0.283 0.283 0.283}
\newrgbcolor{PST@COLOR92}{0.275 0.275 0.275}
\newrgbcolor{PST@COLOR93}{0.267 0.267 0.267}
\newrgbcolor{PST@COLOR94}{0.259 0.259 0.259}
\newrgbcolor{PST@COLOR95}{0.251 0.251 0.251}
\newrgbcolor{PST@COLOR96}{0.244 0.244 0.244}
\newrgbcolor{PST@COLOR97}{0.236 0.236 0.236}
\newrgbcolor{PST@COLOR98}{0.228 0.228 0.228}
\newrgbcolor{PST@COLOR99}{0.22 0.22 0.22}
\newrgbcolor{PST@COLOR100}{0.212 0.212 0.212}
\newrgbcolor{PST@COLOR101}{0.204 0.204 0.204}
\newrgbcolor{PST@COLOR102}{0.196 0.196 0.196}
\newrgbcolor{PST@COLOR103}{0.188 0.188 0.188}
\newrgbcolor{PST@COLOR104}{0.181 0.181 0.181}
\newrgbcolor{PST@COLOR105}{0.173 0.173 0.173}
\newrgbcolor{PST@COLOR106}{0.165 0.165 0.165}
\newrgbcolor{PST@COLOR107}{0.157 0.157 0.157}
\newrgbcolor{PST@COLOR108}{0.149 0.149 0.149}
\newrgbcolor{PST@COLOR109}{0.141 0.141 0.141}
\newrgbcolor{PST@COLOR110}{0.133 0.133 0.133}
\newrgbcolor{PST@COLOR111}{0.125 0.125 0.125}
\newrgbcolor{PST@COLOR112}{0.118 0.118 0.118}
\newrgbcolor{PST@COLOR113}{0.11 0.11 0.11}
\newrgbcolor{PST@COLOR114}{0.102 0.102 0.102}
\newrgbcolor{PST@COLOR115}{0.094 0.094 0.094}
\newrgbcolor{PST@COLOR116}{0.086 0.086 0.086}
\newrgbcolor{PST@COLOR117}{0.078 0.078 0.078}
\newrgbcolor{PST@COLOR118}{0.07 0.07 0.07}
\newrgbcolor{PST@COLOR119}{0.062 0.062 0.062}
\newrgbcolor{PST@COLOR120}{0.055 0.055 0.055}
\newrgbcolor{PST@COLOR121}{0.047 0.047 0.047}
\newrgbcolor{PST@COLOR122}{0.039 0.039 0.039}
\newrgbcolor{PST@COLOR123}{0.031 0.031 0.031}
\newrgbcolor{PST@COLOR124}{0.023 0.023 0.023}
\newrgbcolor{PST@COLOR125}{0.015 0.015 0.015}
\newrgbcolor{PST@COLOR126}{0.007 0.007 0.007}
\newrgbcolor{PST@COLOR127}{0 0 0}


\def\polypmIIId#1{\pspolygon[linestyle=none,fillstyle=solid,fillcolor=PST@COLOR#1]}

\polypmIIId{110}(0.1432,0.19)(0.0864,0.19)(0.0864,0.1078)(0.1432,0.1078)
\polypmIIId{117}(0.1432,0.272)(0.0864,0.272)(0.0864,0.19)(0.1432,0.19)
\polypmIIId{122}(0.1432,0.3542)(0.0864,0.3542)(0.0864,0.272)(0.1432,0.272)

\polypmIIId{107}(0.2,0.19)(0.1432,0.19)(0.1432,0.1078)(0.2,0.1078)
\polypmIIId{117}(0.2,0.272)(0.1432,0.272)(0.1432,0.19)(0.2,0.19)
\polypmIIId{122}(0.2,0.3542)(0.1432,0.3542)(0.1432,0.272)(0.2,0.272)

\polypmIIId{108}(0.2568,0.19)(0.2,0.19)(0.2,0.1078)(0.2568,0.1078)
\polypmIIId{116}(0.2568,0.272)(0.2,0.272)(0.2,0.19)(0.2568,0.19)
\polypmIIId{123}(0.2568,0.3542)(0.2,0.3542)(0.2,0.272)(0.2568,0.272)

\polypmIIId{107}(0.3136,0.19)(0.2568,0.19)(0.2568,0.1078)(0.3136,0.1078)
\polypmIIId{117}(0.3136,0.272)(0.2568,0.272)(0.2568,0.19)(0.3136,0.19)
\polypmIIId{123}(0.3136,0.3542)(0.2568,0.3542)(0.2568,0.272)(0.3136,0.272)

\rput(0.1148,0.07){3}
\rput(0.1716,0.07){4}
\rput(0.2284,0.07){5}
\rput(0.2852,0.07){6}
\rput(0.2000,0.0070){years}

\rput[r](0.0806,0.1489){25}
\rput[r](0.0806,0.2310){50}
\rput[r](0.0806,0.3131){100}
\rput{L}(0.0096,0.2310){actions}

\PST@Border(0.0864,0.3542)
(0.0864,0.1078)
(0.3136,0.1078)
(0.3136,0.3542)
(0.0864,0.3542)

\catcode`@=12
\fi
\endpspicture}
    \subfloat[2 resources]{% GNUPLOT: LaTeX picture using PSTRICKS macros
% Define new PST objects, if not already defined
\ifx\PSTloaded\undefined
\def\PSTloaded{t}

\catcode`@=11

\newpsobject{PST@Border}{psline}{linewidth=.0015,linestyle=solid}

\catcode`@=12

\fi
\psset{unit=5.0in,xunit=5.0in,yunit=3.0in}
\pspicture(0.000000,0.000000)(0.225000,0.35)
\ifx\nofigs\undefined
\catcode`@=11

\newrgbcolor{PST@COLOR0}{1 1 1}
\newrgbcolor{PST@COLOR1}{0.992 0.992 0.992}
\newrgbcolor{PST@COLOR2}{0.984 0.984 0.984}
\newrgbcolor{PST@COLOR3}{0.976 0.976 0.976}
\newrgbcolor{PST@COLOR4}{0.968 0.968 0.968}
\newrgbcolor{PST@COLOR5}{0.96 0.96 0.96}
\newrgbcolor{PST@COLOR6}{0.952 0.952 0.952}
\newrgbcolor{PST@COLOR7}{0.944 0.944 0.944}
\newrgbcolor{PST@COLOR8}{0.937 0.937 0.937}
\newrgbcolor{PST@COLOR9}{0.929 0.929 0.929}
\newrgbcolor{PST@COLOR10}{0.921 0.921 0.921}
\newrgbcolor{PST@COLOR11}{0.913 0.913 0.913}
\newrgbcolor{PST@COLOR12}{0.905 0.905 0.905}
\newrgbcolor{PST@COLOR13}{0.897 0.897 0.897}
\newrgbcolor{PST@COLOR14}{0.889 0.889 0.889}
\newrgbcolor{PST@COLOR15}{0.881 0.881 0.881}
\newrgbcolor{PST@COLOR16}{0.874 0.874 0.874}
\newrgbcolor{PST@COLOR17}{0.866 0.866 0.866}
\newrgbcolor{PST@COLOR18}{0.858 0.858 0.858}
\newrgbcolor{PST@COLOR19}{0.85 0.85 0.85}
\newrgbcolor{PST@COLOR20}{0.842 0.842 0.842}
\newrgbcolor{PST@COLOR21}{0.834 0.834 0.834}
\newrgbcolor{PST@COLOR22}{0.826 0.826 0.826}
\newrgbcolor{PST@COLOR23}{0.818 0.818 0.818}
\newrgbcolor{PST@COLOR24}{0.811 0.811 0.811}
\newrgbcolor{PST@COLOR25}{0.803 0.803 0.803}
\newrgbcolor{PST@COLOR26}{0.795 0.795 0.795}
\newrgbcolor{PST@COLOR27}{0.787 0.787 0.787}
\newrgbcolor{PST@COLOR28}{0.779 0.779 0.779}
\newrgbcolor{PST@COLOR29}{0.771 0.771 0.771}
\newrgbcolor{PST@COLOR30}{0.763 0.763 0.763}
\newrgbcolor{PST@COLOR31}{0.755 0.755 0.755}
\newrgbcolor{PST@COLOR32}{0.748 0.748 0.748}
\newrgbcolor{PST@COLOR33}{0.74 0.74 0.74}
\newrgbcolor{PST@COLOR34}{0.732 0.732 0.732}
\newrgbcolor{PST@COLOR35}{0.724 0.724 0.724}
\newrgbcolor{PST@COLOR36}{0.716 0.716 0.716}
\newrgbcolor{PST@COLOR37}{0.708 0.708 0.708}
\newrgbcolor{PST@COLOR38}{0.7 0.7 0.7}
\newrgbcolor{PST@COLOR39}{0.692 0.692 0.692}
\newrgbcolor{PST@COLOR40}{0.685 0.685 0.685}
\newrgbcolor{PST@COLOR41}{0.677 0.677 0.677}
\newrgbcolor{PST@COLOR42}{0.669 0.669 0.669}
\newrgbcolor{PST@COLOR43}{0.661 0.661 0.661}
\newrgbcolor{PST@COLOR44}{0.653 0.653 0.653}
\newrgbcolor{PST@COLOR45}{0.645 0.645 0.645}
\newrgbcolor{PST@COLOR46}{0.637 0.637 0.637}
\newrgbcolor{PST@COLOR47}{0.629 0.629 0.629}
\newrgbcolor{PST@COLOR48}{0.622 0.622 0.622}
\newrgbcolor{PST@COLOR49}{0.614 0.614 0.614}
\newrgbcolor{PST@COLOR50}{0.606 0.606 0.606}
\newrgbcolor{PST@COLOR51}{0.598 0.598 0.598}
\newrgbcolor{PST@COLOR52}{0.59 0.59 0.59}
\newrgbcolor{PST@COLOR53}{0.582 0.582 0.582}
\newrgbcolor{PST@COLOR54}{0.574 0.574 0.574}
\newrgbcolor{PST@COLOR55}{0.566 0.566 0.566}
\newrgbcolor{PST@COLOR56}{0.559 0.559 0.559}
\newrgbcolor{PST@COLOR57}{0.551 0.551 0.551}
\newrgbcolor{PST@COLOR58}{0.543 0.543 0.543}
\newrgbcolor{PST@COLOR59}{0.535 0.535 0.535}
\newrgbcolor{PST@COLOR60}{0.527 0.527 0.527}
\newrgbcolor{PST@COLOR61}{0.519 0.519 0.519}
\newrgbcolor{PST@COLOR62}{0.511 0.511 0.511}
\newrgbcolor{PST@COLOR63}{0.503 0.503 0.503}
\newrgbcolor{PST@COLOR64}{0.496 0.496 0.496}
\newrgbcolor{PST@COLOR65}{0.488 0.488 0.488}
\newrgbcolor{PST@COLOR66}{0.48 0.48 0.48}
\newrgbcolor{PST@COLOR67}{0.472 0.472 0.472}
\newrgbcolor{PST@COLOR68}{0.464 0.464 0.464}
\newrgbcolor{PST@COLOR69}{0.456 0.456 0.456}
\newrgbcolor{PST@COLOR70}{0.448 0.448 0.448}
\newrgbcolor{PST@COLOR71}{0.44 0.44 0.44}
\newrgbcolor{PST@COLOR72}{0.433 0.433 0.433}
\newrgbcolor{PST@COLOR73}{0.425 0.425 0.425}
\newrgbcolor{PST@COLOR74}{0.417 0.417 0.417}
\newrgbcolor{PST@COLOR75}{0.409 0.409 0.409}
\newrgbcolor{PST@COLOR76}{0.401 0.401 0.401}
\newrgbcolor{PST@COLOR77}{0.393 0.393 0.393}
\newrgbcolor{PST@COLOR78}{0.385 0.385 0.385}
\newrgbcolor{PST@COLOR79}{0.377 0.377 0.377}
\newrgbcolor{PST@COLOR80}{0.37 0.37 0.37}
\newrgbcolor{PST@COLOR81}{0.362 0.362 0.362}
\newrgbcolor{PST@COLOR82}{0.354 0.354 0.354}
\newrgbcolor{PST@COLOR83}{0.346 0.346 0.346}
\newrgbcolor{PST@COLOR84}{0.338 0.338 0.338}
\newrgbcolor{PST@COLOR85}{0.33 0.33 0.33}
\newrgbcolor{PST@COLOR86}{0.322 0.322 0.322}
\newrgbcolor{PST@COLOR87}{0.314 0.314 0.314}
\newrgbcolor{PST@COLOR88}{0.307 0.307 0.307}
\newrgbcolor{PST@COLOR89}{0.299 0.299 0.299}
\newrgbcolor{PST@COLOR90}{0.291 0.291 0.291}
\newrgbcolor{PST@COLOR91}{0.283 0.283 0.283}
\newrgbcolor{PST@COLOR92}{0.275 0.275 0.275}
\newrgbcolor{PST@COLOR93}{0.267 0.267 0.267}
\newrgbcolor{PST@COLOR94}{0.259 0.259 0.259}
\newrgbcolor{PST@COLOR95}{0.251 0.251 0.251}
\newrgbcolor{PST@COLOR96}{0.244 0.244 0.244}
\newrgbcolor{PST@COLOR97}{0.236 0.236 0.236}
\newrgbcolor{PST@COLOR98}{0.228 0.228 0.228}
\newrgbcolor{PST@COLOR99}{0.22 0.22 0.22}
\newrgbcolor{PST@COLOR100}{0.212 0.212 0.212}
\newrgbcolor{PST@COLOR101}{0.204 0.204 0.204}
\newrgbcolor{PST@COLOR102}{0.196 0.196 0.196}
\newrgbcolor{PST@COLOR103}{0.188 0.188 0.188}
\newrgbcolor{PST@COLOR104}{0.181 0.181 0.181}
\newrgbcolor{PST@COLOR105}{0.173 0.173 0.173}
\newrgbcolor{PST@COLOR106}{0.165 0.165 0.165}
\newrgbcolor{PST@COLOR107}{0.157 0.157 0.157}
\newrgbcolor{PST@COLOR108}{0.149 0.149 0.149}
\newrgbcolor{PST@COLOR109}{0.141 0.141 0.141}
\newrgbcolor{PST@COLOR110}{0.133 0.133 0.133}
\newrgbcolor{PST@COLOR111}{0.125 0.125 0.125}
\newrgbcolor{PST@COLOR112}{0.118 0.118 0.118}
\newrgbcolor{PST@COLOR113}{0.11 0.11 0.11}
\newrgbcolor{PST@COLOR114}{0.102 0.102 0.102}
\newrgbcolor{PST@COLOR115}{0.094 0.094 0.094}
\newrgbcolor{PST@COLOR116}{0.086 0.086 0.086}
\newrgbcolor{PST@COLOR117}{0.078 0.078 0.078}
\newrgbcolor{PST@COLOR118}{0.07 0.07 0.07}
\newrgbcolor{PST@COLOR119}{0.062 0.062 0.062}
\newrgbcolor{PST@COLOR120}{0.055 0.055 0.055}
\newrgbcolor{PST@COLOR121}{0.047 0.047 0.047}
\newrgbcolor{PST@COLOR122}{0.039 0.039 0.039}
\newrgbcolor{PST@COLOR123}{0.031 0.031 0.031}
\newrgbcolor{PST@COLOR124}{0.023 0.023 0.023}
\newrgbcolor{PST@COLOR125}{0.015 0.015 0.015}
\newrgbcolor{PST@COLOR126}{0.007 0.007 0.007}
\newrgbcolor{PST@COLOR127}{0 0 0}

\def\polypmIIId#1{\pspolygon[linestyle=none,fillstyle=solid,fillcolor=PST@COLOR#1]}

\polypmIIId{94} (0.0568,0.19)  (0.0,0.19)  (0.0,0.1078)(0.0568,0.1078)
\polypmIIId{106}  (0.0568,0.272) (0.0,0.272) (0.0,0.19)  (0.0568,0.19)
\polypmIIId{117}  (0.0568,0.3542)(0.0,0.3542)(0.0,0.272) (0.0568,0.272)

\polypmIIId{93} (0.1136,   0.19)  (0.0568,0.19)  (0.0568,0.1078)(0.1136,0.1078)
\polypmIIId{106}  (0.1136,   0.272) (0.0568,0.272) (0.0568,0.19)  (0.1136,0.19)
\polypmIIId{118}  (0.1136,   0.3542)(0.0568,0.3542)(0.0568,0.272) (0.1136,0.272)

\polypmIIId{93}(0.1704,0.19)  (0.1136,   0.19)  (0.1136,   0.1078)(0.1704,0.1078)
\polypmIIId{108} (0.1704,0.272) (0.1136,   0.272) (0.1136,   0.19)  (0.1704,0.19)
\polypmIIId{118}  (0.1704,0.3542)(0.1136,   0.3542)(0.1136,   0.272) (0.1704,0.272)

\polypmIIId{95}(0.2272,0.19)  (0.1704,0.19)  (0.1704,0.1078)(0.2272,0.1078)
\polypmIIId{109}  (0.2272,0.272) (0.1704,0.272) (0.1704,0.19)  (0.2272,0.19)
\polypmIIId{119}  (0.2272,0.3542)(0.1704,0.3542)(0.1704,0.272) (0.2272,0.272)

\rput(0.0284,0.07){3}
\rput(0.0852,0.07){4}
\rput(0.1420,0.07){5}
\rput(0.1988,0.07){6}
\rput(0.1136,0.0070){years}


\PST@Border(0.0,0.3542)
(0.0,0.1078)
(0.2272,0.1078)
(0.2272,0.3542)
(0.0,0.3542)

\catcode`@=12
\fi
\endpspicture}
    \subfloat[4 resources]{% GNUPLOT: LaTeX picture using PSTRICKS macros
% Define new PST objects, if not already defined
\ifx\PSTloaded\undefined
\def\PSTloaded{t}

\catcode`@=11

\newpsobject{PST@Border}{psline}{linewidth=.0015,linestyle=solid}

\catcode`@=12

\fi
\psset{unit=5.0in,xunit=5.0in,yunit=3.0in}
\pspicture(0.000000,0.000000)(0.3136,0.35)
\ifx\nofigs\undefined
\catcode`@=11

\newrgbcolor{PST@COLOR0}{1 1 1}
\newrgbcolor{PST@COLOR1}{0.992 0.992 0.992}
\newrgbcolor{PST@COLOR2}{0.984 0.984 0.984}
\newrgbcolor{PST@COLOR3}{0.976 0.976 0.976}
\newrgbcolor{PST@COLOR4}{0.968 0.968 0.968}
\newrgbcolor{PST@COLOR5}{0.96 0.96 0.96}
\newrgbcolor{PST@COLOR6}{0.952 0.952 0.952}
\newrgbcolor{PST@COLOR7}{0.944 0.944 0.944}
\newrgbcolor{PST@COLOR8}{0.937 0.937 0.937}
\newrgbcolor{PST@COLOR9}{0.929 0.929 0.929}
\newrgbcolor{PST@COLOR10}{0.921 0.921 0.921}
\newrgbcolor{PST@COLOR11}{0.913 0.913 0.913}
\newrgbcolor{PST@COLOR12}{0.905 0.905 0.905}
\newrgbcolor{PST@COLOR13}{0.897 0.897 0.897}
\newrgbcolor{PST@COLOR14}{0.889 0.889 0.889}
\newrgbcolor{PST@COLOR15}{0.881 0.881 0.881}
\newrgbcolor{PST@COLOR16}{0.874 0.874 0.874}
\newrgbcolor{PST@COLOR17}{0.866 0.866 0.866}
\newrgbcolor{PST@COLOR18}{0.858 0.858 0.858}
\newrgbcolor{PST@COLOR19}{0.85 0.85 0.85}
\newrgbcolor{PST@COLOR20}{0.842 0.842 0.842}
\newrgbcolor{PST@COLOR21}{0.834 0.834 0.834}
\newrgbcolor{PST@COLOR22}{0.826 0.826 0.826}
\newrgbcolor{PST@COLOR23}{0.818 0.818 0.818}
\newrgbcolor{PST@COLOR24}{0.811 0.811 0.811}
\newrgbcolor{PST@COLOR25}{0.803 0.803 0.803}
\newrgbcolor{PST@COLOR26}{0.795 0.795 0.795}
\newrgbcolor{PST@COLOR27}{0.787 0.787 0.787}
\newrgbcolor{PST@COLOR28}{0.779 0.779 0.779}
\newrgbcolor{PST@COLOR29}{0.771 0.771 0.771}
\newrgbcolor{PST@COLOR30}{0.763 0.763 0.763}
\newrgbcolor{PST@COLOR31}{0.755 0.755 0.755}
\newrgbcolor{PST@COLOR32}{0.748 0.748 0.748}
\newrgbcolor{PST@COLOR33}{0.74 0.74 0.74}
\newrgbcolor{PST@COLOR34}{0.732 0.732 0.732}
\newrgbcolor{PST@COLOR35}{0.724 0.724 0.724}
\newrgbcolor{PST@COLOR36}{0.716 0.716 0.716}
\newrgbcolor{PST@COLOR37}{0.708 0.708 0.708}
\newrgbcolor{PST@COLOR38}{0.7 0.7 0.7}
\newrgbcolor{PST@COLOR39}{0.692 0.692 0.692}
\newrgbcolor{PST@COLOR40}{0.685 0.685 0.685}
\newrgbcolor{PST@COLOR41}{0.677 0.677 0.677}
\newrgbcolor{PST@COLOR42}{0.669 0.669 0.669}
\newrgbcolor{PST@COLOR43}{0.661 0.661 0.661}
\newrgbcolor{PST@COLOR44}{0.653 0.653 0.653}
\newrgbcolor{PST@COLOR45}{0.645 0.645 0.645}
\newrgbcolor{PST@COLOR46}{0.637 0.637 0.637}
\newrgbcolor{PST@COLOR47}{0.629 0.629 0.629}
\newrgbcolor{PST@COLOR48}{0.622 0.622 0.622}
\newrgbcolor{PST@COLOR49}{0.614 0.614 0.614}
\newrgbcolor{PST@COLOR50}{0.606 0.606 0.606}
\newrgbcolor{PST@COLOR51}{0.598 0.598 0.598}
\newrgbcolor{PST@COLOR52}{0.59 0.59 0.59}
\newrgbcolor{PST@COLOR53}{0.582 0.582 0.582}
\newrgbcolor{PST@COLOR54}{0.574 0.574 0.574}
\newrgbcolor{PST@COLOR55}{0.566 0.566 0.566}
\newrgbcolor{PST@COLOR56}{0.559 0.559 0.559}
\newrgbcolor{PST@COLOR57}{0.551 0.551 0.551}
\newrgbcolor{PST@COLOR58}{0.543 0.543 0.543}
\newrgbcolor{PST@COLOR59}{0.535 0.535 0.535}
\newrgbcolor{PST@COLOR60}{0.527 0.527 0.527}
\newrgbcolor{PST@COLOR61}{0.519 0.519 0.519}
\newrgbcolor{PST@COLOR62}{0.511 0.511 0.511}
\newrgbcolor{PST@COLOR63}{0.503 0.503 0.503}
\newrgbcolor{PST@COLOR64}{0.496 0.496 0.496}
\newrgbcolor{PST@COLOR65}{0.488 0.488 0.488}
\newrgbcolor{PST@COLOR66}{0.48 0.48 0.48}
\newrgbcolor{PST@COLOR67}{0.472 0.472 0.472}
\newrgbcolor{PST@COLOR68}{0.464 0.464 0.464}
\newrgbcolor{PST@COLOR69}{0.456 0.456 0.456}
\newrgbcolor{PST@COLOR70}{0.448 0.448 0.448}
\newrgbcolor{PST@COLOR71}{0.44 0.44 0.44}
\newrgbcolor{PST@COLOR72}{0.433 0.433 0.433}
\newrgbcolor{PST@COLOR73}{0.425 0.425 0.425}
\newrgbcolor{PST@COLOR74}{0.417 0.417 0.417}
\newrgbcolor{PST@COLOR75}{0.409 0.409 0.409}
\newrgbcolor{PST@COLOR76}{0.401 0.401 0.401}
\newrgbcolor{PST@COLOR77}{0.393 0.393 0.393}
\newrgbcolor{PST@COLOR78}{0.385 0.385 0.385}
\newrgbcolor{PST@COLOR79}{0.377 0.377 0.377}
\newrgbcolor{PST@COLOR80}{0.37 0.37 0.37}
\newrgbcolor{PST@COLOR81}{0.362 0.362 0.362}
\newrgbcolor{PST@COLOR82}{0.354 0.354 0.354}
\newrgbcolor{PST@COLOR83}{0.346 0.346 0.346}
\newrgbcolor{PST@COLOR84}{0.338 0.338 0.338}
\newrgbcolor{PST@COLOR85}{0.33 0.33 0.33}
\newrgbcolor{PST@COLOR86}{0.322 0.322 0.322}
\newrgbcolor{PST@COLOR87}{0.314 0.314 0.314}
\newrgbcolor{PST@COLOR88}{0.307 0.307 0.307}
\newrgbcolor{PST@COLOR89}{0.299 0.299 0.299}
\newrgbcolor{PST@COLOR90}{0.291 0.291 0.291}
\newrgbcolor{PST@COLOR91}{0.283 0.283 0.283}
\newrgbcolor{PST@COLOR92}{0.275 0.275 0.275}
\newrgbcolor{PST@COLOR93}{0.267 0.267 0.267}
\newrgbcolor{PST@COLOR94}{0.259 0.259 0.259}
\newrgbcolor{PST@COLOR95}{0.251 0.251 0.251}
\newrgbcolor{PST@COLOR96}{0.244 0.244 0.244}
\newrgbcolor{PST@COLOR97}{0.236 0.236 0.236}
\newrgbcolor{PST@COLOR98}{0.228 0.228 0.228}
\newrgbcolor{PST@COLOR99}{0.22 0.22 0.22}
\newrgbcolor{PST@COLOR100}{0.212 0.212 0.212}
\newrgbcolor{PST@COLOR101}{0.204 0.204 0.204}
\newrgbcolor{PST@COLOR102}{0.196 0.196 0.196}
\newrgbcolor{PST@COLOR103}{0.188 0.188 0.188}
\newrgbcolor{PST@COLOR104}{0.181 0.181 0.181}
\newrgbcolor{PST@COLOR105}{0.173 0.173 0.173}
\newrgbcolor{PST@COLOR106}{0.165 0.165 0.165}
\newrgbcolor{PST@COLOR107}{0.157 0.157 0.157}
\newrgbcolor{PST@COLOR108}{0.149 0.149 0.149}
\newrgbcolor{PST@COLOR109}{0.141 0.141 0.141}
\newrgbcolor{PST@COLOR110}{0.133 0.133 0.133}
\newrgbcolor{PST@COLOR111}{0.125 0.125 0.125}
\newrgbcolor{PST@COLOR112}{0.118 0.118 0.118}
\newrgbcolor{PST@COLOR113}{0.11 0.11 0.11}
\newrgbcolor{PST@COLOR114}{0.102 0.102 0.102}
\newrgbcolor{PST@COLOR115}{0.094 0.094 0.094}
\newrgbcolor{PST@COLOR116}{0.086 0.086 0.086}
\newrgbcolor{PST@COLOR117}{0.078 0.078 0.078}
\newrgbcolor{PST@COLOR118}{0.07 0.07 0.07}
\newrgbcolor{PST@COLOR119}{0.062 0.062 0.062}
\newrgbcolor{PST@COLOR120}{0.055 0.055 0.055}
\newrgbcolor{PST@COLOR121}{0.047 0.047 0.047}
\newrgbcolor{PST@COLOR122}{0.039 0.039 0.039}
\newrgbcolor{PST@COLOR123}{0.031 0.031 0.031}
\newrgbcolor{PST@COLOR124}{0.023 0.023 0.023}
\newrgbcolor{PST@COLOR125}{0.015 0.015 0.015}
\newrgbcolor{PST@COLOR126}{0.007 0.007 0.007}
\newrgbcolor{PST@COLOR127}{0 0 0}

\def\polypmIIId#1{\pspolygon[linestyle=none,fillstyle=solid,fillcolor=PST@COLOR#1]}

\polypmIIId{74} (0.0568,0.19)  (0.0,0.19)  (0.0,0.1078)(0.0568,0.1078)
\polypmIIId{93}  (0.0568,0.272) (0.0,0.272) (0.0,0.19)  (0.0568,0.19)
\polypmIIId{106}  (0.0568,0.3542)(0.0,0.3542)(0.0,0.272) (0.0568,0.272)

\polypmIIId{75} (0.1136,   0.19)  (0.0568,0.19)  (0.0568,0.1078)(0.1136,0.1078)
\polypmIIId{92}  (0.1136,   0.272) (0.0568,0.272) (0.0568,0.19)  (0.1136,0.19)
\polypmIIId{108}  (0.1136,   0.3542)(0.0568,0.3542)(0.0568,0.272) (0.1136,0.272)

\polypmIIId{71}(0.1704,0.19)  (0.1136,   0.19)  (0.1136,   0.1078)(0.1704,0.1078)
\polypmIIId{95} (0.1704,0.272) (0.1136,   0.272) (0.1136,   0.19)  (0.1704,0.19)
\polypmIIId{108}  (0.1704,0.3542)(0.1136,   0.3542)(0.1136,   0.272) (0.1704,0.272)

\polypmIIId{71}(0.2272,0.19)  (0.1704,0.19)  (0.1704,0.1078)(0.2272,0.1078)
\polypmIIId{97}  (0.2272,0.272) (0.1704,0.272) (0.1704,0.19)  (0.2272,0.19)
\polypmIIId{109}  (0.2272,0.3542)(0.1704,0.3542)(0.1704,0.272) (0.2272,0.272)

\rput(0.0284,0.07){3}
\rput(0.0852,0.07){4}
\rput(0.1420,0.07){5}
\rput(0.1988,0.07){6}
\rput(0.1136,0.0070){years}

\PST@Border(0.0,0.3542)
(0.0,0.1078)
(0.2272,0.1078)
(0.2272,0.3542)
(0.0,0.3542)

\polypmIIId{0}(0.2329,0.1078)(0.2442,0.1078)(0.2442,0.1098)(0.2329,0.1098)
\polypmIIId{1}(0.2329,0.1097)(0.2442,0.1097)(0.2442,0.1117)(0.2329,0.1117)
\polypmIIId{2}(0.2329,0.1116)(0.2442,0.1116)(0.2442,0.1136)(0.2329,0.1136)
\polypmIIId{3}(0.2329,0.1135)(0.2442,0.1135)(0.2442,0.1156)(0.2329,0.1156)
\polypmIIId{4}(0.2329,0.1155)(0.2442,0.1155)(0.2442,0.1175)(0.2329,0.1175)
\polypmIIId{5}(0.2329,0.1174)(0.2442,0.1174)(0.2442,0.1194)(0.2329,0.1194)
\polypmIIId{6}(0.2329,0.1193)(0.2442,0.1193)(0.2442,0.1213)(0.2329,0.1213)
\polypmIIId{7}(0.2329,0.1212)(0.2442,0.1212)(0.2442,0.1233)(0.2329,0.1233)
\polypmIIId{8}(0.2329,0.1232)(0.2442,0.1232)(0.2442,0.1252)(0.2329,0.1252)
\polypmIIId{9}(0.2329,0.1251)(0.2442,0.1251)(0.2442,0.1271)(0.2329,0.1271)
\polypmIIId{10}(0.2329,0.127)(0.2442,0.127)(0.2442,0.129)(0.2329,0.129)
\polypmIIId{11}(0.2329,0.1289)(0.2442,0.1289)(0.2442,0.131)(0.2329,0.131)
\polypmIIId{12}(0.2329,0.1309)(0.2442,0.1309)(0.2442,0.1329)(0.2329,0.1329)
\polypmIIId{13}(0.2329,0.1328)(0.2442,0.1328)(0.2442,0.1348)(0.2329,0.1348)
\polypmIIId{14}(0.2329,0.1347)(0.2442,0.1347)(0.2442,0.1367)(0.2329,0.1367)
\polypmIIId{15}(0.2329,0.1366)(0.2442,0.1366)(0.2442,0.1387)(0.2329,0.1387)
\polypmIIId{16}(0.2329,0.1386)(0.2442,0.1386)(0.2442,0.1406)(0.2329,0.1406)
\polypmIIId{17}(0.2329,0.1405)(0.2442,0.1405)(0.2442,0.1425)(0.2329,0.1425)
\polypmIIId{18}(0.2329,0.1424)(0.2442,0.1424)(0.2442,0.1444)(0.2329,0.1444)
\polypmIIId{19}(0.2329,0.1443)(0.2442,0.1443)(0.2442,0.1464)(0.2329,0.1464)
\polypmIIId{20}(0.2329,0.1463)(0.2442,0.1463)(0.2442,0.1483)(0.2329,0.1483)
\polypmIIId{21}(0.2329,0.1482)(0.2442,0.1482)(0.2442,0.1502)(0.2329,0.1502)
\polypmIIId{22}(0.2329,0.1501)(0.2442,0.1501)(0.2442,0.1521)(0.2329,0.1521)
\polypmIIId{23}(0.2329,0.152)(0.2442,0.152)(0.2442,0.1541)(0.2329,0.1541)
\polypmIIId{24}(0.2329,0.154)(0.2442,0.154)(0.2442,0.156)(0.2329,0.156)
\polypmIIId{25}(0.2329,0.1559)(0.2442,0.1559)(0.2442,0.1579)(0.2329,0.1579)
\polypmIIId{26}(0.2329,0.1578)(0.2442,0.1578)(0.2442,0.1598)(0.2329,0.1598)
\polypmIIId{27}(0.2329,0.1597)(0.2442,0.1597)(0.2442,0.1618)(0.2329,0.1618)
\polypmIIId{28}(0.2329,0.1617)(0.2442,0.1617)(0.2442,0.1637)(0.2329,0.1637)
\polypmIIId{29}(0.2329,0.1636)(0.2442,0.1636)(0.2442,0.1656)(0.2329,0.1656)
\polypmIIId{30}(0.2329,0.1655)(0.2442,0.1655)(0.2442,0.1675)(0.2329,0.1675)
\polypmIIId{31}(0.2329,0.1674)(0.2442,0.1674)(0.2442,0.1695)(0.2329,0.1695)
\polypmIIId{32}(0.2329,0.1694)(0.2442,0.1694)(0.2442,0.1714)(0.2329,0.1714)
\polypmIIId{33}(0.2329,0.1713)(0.2442,0.1713)(0.2442,0.1733)(0.2329,0.1733)
\polypmIIId{34}(0.2329,0.1732)(0.2442,0.1732)(0.2442,0.1752)(0.2329,0.1752)
\polypmIIId{35}(0.2329,0.1751)(0.2442,0.1751)(0.2442,0.1772)(0.2329,0.1772)
\polypmIIId{36}(0.2329,0.1771)(0.2442,0.1771)(0.2442,0.1791)(0.2329,0.1791)
\polypmIIId{37}(0.2329,0.179)(0.2442,0.179)(0.2442,0.181)(0.2329,0.181)
\polypmIIId{38}(0.2329,0.1809)(0.2442,0.1809)(0.2442,0.1829)(0.2329,0.1829)
\polypmIIId{39}(0.2329,0.1828)(0.2442,0.1828)(0.2442,0.1849)(0.2329,0.1849)
\polypmIIId{40}(0.2329,0.1848)(0.2442,0.1848)(0.2442,0.1868)(0.2329,0.1868)
\polypmIIId{41}(0.2329,0.1867)(0.2442,0.1867)(0.2442,0.1887)(0.2329,0.1887)
\polypmIIId{42}(0.2329,0.1886)(0.2442,0.1886)(0.2442,0.1906)(0.2329,0.1906)
\polypmIIId{43}(0.2329,0.1905)(0.2442,0.1905)(0.2442,0.1926)(0.2329,0.1926)
\polypmIIId{44}(0.2329,0.1925)(0.2442,0.1925)(0.2442,0.1945)(0.2329,0.1945)
\polypmIIId{45}(0.2329,0.1944)(0.2442,0.1944)(0.2442,0.1964)(0.2329,0.1964)
\polypmIIId{46}(0.2329,0.1963)(0.2442,0.1963)(0.2442,0.1983)(0.2329,0.1983)
\polypmIIId{47}(0.2329,0.1982)(0.2442,0.1982)(0.2442,0.2003)(0.2329,0.2003)
\polypmIIId{48}(0.2329,0.2002)(0.2442,0.2002)(0.2442,0.2022)(0.2329,0.2022)
\polypmIIId{49}(0.2329,0.2021)(0.2442,0.2021)(0.2442,0.2041)(0.2329,0.2041)
\polypmIIId{50}(0.2329,0.204)(0.2442,0.204)(0.2442,0.206)(0.2329,0.206)
\polypmIIId{51}(0.2329,0.2059)(0.2442,0.2059)(0.2442,0.208)(0.2329,0.208)
\polypmIIId{52}(0.2329,0.2079)(0.2442,0.2079)(0.2442,0.2099)(0.2329,0.2099)
\polypmIIId{53}(0.2329,0.2098)(0.2442,0.2098)(0.2442,0.2118)(0.2329,0.2118)
\polypmIIId{54}(0.2329,0.2117)(0.2442,0.2117)(0.2442,0.2137)(0.2329,0.2137)
\polypmIIId{55}(0.2329,0.2136)(0.2442,0.2136)(0.2442,0.2157)(0.2329,0.2157)
\polypmIIId{56}(0.2329,0.2156)(0.2442,0.2156)(0.2442,0.2176)(0.2329,0.2176)
\polypmIIId{57}(0.2329,0.2175)(0.2442,0.2175)(0.2442,0.2195)(0.2329,0.2195)
\polypmIIId{58}(0.2329,0.2194)(0.2442,0.2194)(0.2442,0.2214)(0.2329,0.2214)
\polypmIIId{59}(0.2329,0.2213)(0.2442,0.2213)(0.2442,0.2234)(0.2329,0.2234)
\polypmIIId{60}(0.2329,0.2233)(0.2442,0.2233)(0.2442,0.2253)(0.2329,0.2253)
\polypmIIId{61}(0.2329,0.2252)(0.2442,0.2252)(0.2442,0.2272)(0.2329,0.2272)
\polypmIIId{62}(0.2329,0.2271)(0.2442,0.2271)(0.2442,0.2291)(0.2329,0.2291)
\polypmIIId{63}(0.2329,0.229)(0.2442,0.229)(0.2442,0.2311)(0.2329,0.2311)
\polypmIIId{64}(0.2329,0.231)(0.2442,0.231)(0.2442,0.233)(0.2329,0.233)
\polypmIIId{65}(0.2329,0.2329)(0.2442,0.2329)(0.2442,0.2349)(0.2329,0.2349)
\polypmIIId{66}(0.2329,0.2348)(0.2442,0.2348)(0.2442,0.2368)(0.2329,0.2368)
\polypmIIId{67}(0.2329,0.2367)(0.2442,0.2367)(0.2442,0.2388)(0.2329,0.2388)
\polypmIIId{68}(0.2329,0.2387)(0.2442,0.2387)(0.2442,0.2407)(0.2329,0.2407)
\polypmIIId{69}(0.2329,0.2406)(0.2442,0.2406)(0.2442,0.2426)(0.2329,0.2426)
\polypmIIId{70}(0.2329,0.2425)(0.2442,0.2425)(0.2442,0.2445)(0.2329,0.2445)
\polypmIIId{71}(0.2329,0.2444)(0.2442,0.2444)(0.2442,0.2465)(0.2329,0.2465)
\polypmIIId{72}(0.2329,0.2464)(0.2442,0.2464)(0.2442,0.2484)(0.2329,0.2484)
\polypmIIId{73}(0.2329,0.2483)(0.2442,0.2483)(0.2442,0.2503)(0.2329,0.2503)
\polypmIIId{74}(0.2329,0.2502)(0.2442,0.2502)(0.2442,0.2522)(0.2329,0.2522)
\polypmIIId{75}(0.2329,0.2521)(0.2442,0.2521)(0.2442,0.2542)(0.2329,0.2542)
\polypmIIId{76}(0.2329,0.2541)(0.2442,0.2541)(0.2442,0.2561)(0.2329,0.2561)
\polypmIIId{77}(0.2329,0.256)(0.2442,0.256)(0.2442,0.258)(0.2329,0.258)
\polypmIIId{78}(0.2329,0.2579)(0.2442,0.2579)(0.2442,0.2599)(0.2329,0.2599)
\polypmIIId{79}(0.2329,0.2598)(0.2442,0.2598)(0.2442,0.2619)(0.2329,0.2619)
\polypmIIId{80}(0.2329,0.2618)(0.2442,0.2618)(0.2442,0.2638)(0.2329,0.2638)
\polypmIIId{81}(0.2329,0.2637)(0.2442,0.2637)(0.2442,0.2657)(0.2329,0.2657)
\polypmIIId{82}(0.2329,0.2656)(0.2442,0.2656)(0.2442,0.2676)(0.2329,0.2676)
\polypmIIId{83}(0.2329,0.2675)(0.2442,0.2675)(0.2442,0.2696)(0.2329,0.2696)
\polypmIIId{84}(0.2329,0.2695)(0.2442,0.2695)(0.2442,0.2715)(0.2329,0.2715)
\polypmIIId{85}(0.2329,0.2714)(0.2442,0.2714)(0.2442,0.2734)(0.2329,0.2734)
\polypmIIId{86}(0.2329,0.2733)(0.2442,0.2733)(0.2442,0.2753)(0.2329,0.2753)
\polypmIIId{87}(0.2329,0.2752)(0.2442,0.2752)(0.2442,0.2773)(0.2329,0.2773)
\polypmIIId{88}(0.2329,0.2772)(0.2442,0.2772)(0.2442,0.2792)(0.2329,0.2792)
\polypmIIId{89}(0.2329,0.2791)(0.2442,0.2791)(0.2442,0.2811)(0.2329,0.2811)
\polypmIIId{90}(0.2329,0.281)(0.2442,0.281)(0.2442,0.283)(0.2329,0.283)
\polypmIIId{91}(0.2329,0.2829)(0.2442,0.2829)(0.2442,0.285)(0.2329,0.285)
\polypmIIId{92}(0.2329,0.2849)(0.2442,0.2849)(0.2442,0.2869)(0.2329,0.2869)
\polypmIIId{93}(0.2329,0.2868)(0.2442,0.2868)(0.2442,0.2888)(0.2329,0.2888)
\polypmIIId{94}(0.2329,0.2887)(0.2442,0.2887)(0.2442,0.2907)(0.2329,0.2907)
\polypmIIId{95}(0.2329,0.2906)(0.2442,0.2906)(0.2442,0.2927)(0.2329,0.2927)
\polypmIIId{96}(0.2329,0.2926)(0.2442,0.2926)(0.2442,0.2946)(0.2329,0.2946)
\polypmIIId{97}(0.2329,0.2945)(0.2442,0.2945)(0.2442,0.2965)(0.2329,0.2965)
\polypmIIId{98}(0.2329,0.2964)(0.2442,0.2964)(0.2442,0.2984)(0.2329,0.2984)
\polypmIIId{99}(0.2329,0.2983)(0.2442,0.2983)(0.2442,0.3004)(0.2329,0.3004)
\polypmIIId{100}(0.2329,0.3003)(0.2442,0.3003)(0.2442,0.3023)(0.2329,0.3023)
\polypmIIId{101}(0.2329,0.3022)(0.2442,0.3022)(0.2442,0.3042)(0.2329,0.3042)
\polypmIIId{102}(0.2329,0.3041)(0.2442,0.3041)(0.2442,0.3061)(0.2329,0.3061)
\polypmIIId{103}(0.2329,0.306)(0.2442,0.306)(0.2442,0.3081)(0.2329,0.3081)
\polypmIIId{104}(0.2329,0.308)(0.2442,0.308)(0.2442,0.31)(0.2329,0.31)
\polypmIIId{105}(0.2329,0.3099)(0.2442,0.3099)(0.2442,0.3119)(0.2329,0.3119)
\polypmIIId{106}(0.2329,0.3118)(0.2442,0.3118)(0.2442,0.3138)(0.2329,0.3138)
\polypmIIId{107}(0.2329,0.3137)(0.2442,0.3137)(0.2442,0.3158)(0.2329,0.3158)
\polypmIIId{108}(0.2329,0.3157)(0.2442,0.3157)(0.2442,0.3177)(0.2329,0.3177)
\polypmIIId{109}(0.2329,0.3176)(0.2442,0.3176)(0.2442,0.3196)(0.2329,0.3196)
\polypmIIId{110}(0.2329,0.3195)(0.2442,0.3195)(0.2442,0.3215)(0.2329,0.3215)
\polypmIIId{111}(0.2329,0.3214)(0.2442,0.3214)(0.2442,0.3235)(0.2329,0.3235)
\polypmIIId{112}(0.2329,0.3234)(0.2442,0.3234)(0.2442,0.3254)(0.2329,0.3254)
\polypmIIId{113}(0.2329,0.3253)(0.2442,0.3253)(0.2442,0.3273)(0.2329,0.3273)
\polypmIIId{114}(0.2329,0.3272)(0.2442,0.3272)(0.2442,0.3292)(0.2329,0.3292)
\polypmIIId{115}(0.2329,0.3291)(0.2442,0.3291)(0.2442,0.3312)(0.2329,0.3312)
\polypmIIId{116}(0.2329,0.3311)(0.2442,0.3311)(0.2442,0.3331)(0.2329,0.3331)
\polypmIIId{117}(0.2329,0.333)(0.2442,0.333)(0.2442,0.335)(0.2329,0.335)
\polypmIIId{118}(0.2329,0.3349)(0.2442,0.3349)(0.2442,0.3369)(0.2329,0.3369)
\polypmIIId{119}(0.2329,0.3368)(0.2442,0.3368)(0.2442,0.3389)(0.2329,0.3389)
\polypmIIId{120}(0.2329,0.3388)(0.2442,0.3388)(0.2442,0.3408)(0.2329,0.3408)
\polypmIIId{121}(0.2329,0.3407)(0.2442,0.3407)(0.2442,0.3427)(0.2329,0.3427)
\polypmIIId{122}(0.2329,0.3426)(0.2442,0.3426)(0.2442,0.3446)(0.2329,0.3446)
\polypmIIId{123}(0.2329,0.3445)(0.2442,0.3445)(0.2442,0.3466)(0.2329,0.3466)
\polypmIIId{124}(0.2329,0.3465)(0.2442,0.3465)(0.2442,0.3485)(0.2329,0.3485)
\polypmIIId{125}(0.2329,0.3484)(0.2442,0.3484)(0.2442,0.3504)(0.2329,0.3504)
\polypmIIId{126}(0.2329,0.3503)(0.2442,0.3503)(0.2442,0.3523)(0.2329,0.3523)
\polypmIIId{127}(0.2329,0.3522)(0.2442,0.3522)(0.2442,0.3542)(0.2329,0.3542)

\PST@Border(0.2329,0.1078)
(0.2442,0.1078)
(0.2442,0.3542)
(0.2329,0.3542)
(0.2329,0.1078)


\rput[l](0.2502,0.1301){0.997}
\rput[l](0.2502,0.2048){0.998}
\rput[l](0.2502,0.2795){0.999}
\rput[l](0.2502,0.3542){1}

\catcode`@=12
\fi
\endpspicture}
  }
  $\alpha = 0.1$
  %\label{fig:tabusolcomp01}
\end{figure}

\figspaces
\begin{figure}[H]
  \centering
  \resizebox{\columnwidth}{!}{%
    \subfloat[1 resource]{% GNUPLOT: LaTeX picture using PSTRICKS macros
% Define new PST objects, if not already defined
\ifx\PSTloaded\undefined
\def\PSTloaded{t}

\catcode`@=11

\newpsobject{PST@Border}{psline}{linewidth=.0015,linestyle=solid}

\catcode`@=12

\fi
\psset{unit=5.0in,xunit=5.0in,yunit=3.0in}
\pspicture(0.000000,0.000000)(0.31, 0.35)
\ifx\nofigs\undefined
\catcode`@=11

\newrgbcolor{PST@COLOR0}{1 1 1}
\newrgbcolor{PST@COLOR1}{0.992 0.992 0.992}
\newrgbcolor{PST@COLOR2}{0.984 0.984 0.984}
\newrgbcolor{PST@COLOR3}{0.976 0.976 0.976}
\newrgbcolor{PST@COLOR4}{0.968 0.968 0.968}
\newrgbcolor{PST@COLOR5}{0.96 0.96 0.96}
\newrgbcolor{PST@COLOR6}{0.952 0.952 0.952}
\newrgbcolor{PST@COLOR7}{0.944 0.944 0.944}
\newrgbcolor{PST@COLOR8}{0.937 0.937 0.937}
\newrgbcolor{PST@COLOR9}{0.929 0.929 0.929}
\newrgbcolor{PST@COLOR10}{0.921 0.921 0.921}
\newrgbcolor{PST@COLOR11}{0.913 0.913 0.913}
\newrgbcolor{PST@COLOR12}{0.905 0.905 0.905}
\newrgbcolor{PST@COLOR13}{0.897 0.897 0.897}
\newrgbcolor{PST@COLOR14}{0.889 0.889 0.889}
\newrgbcolor{PST@COLOR15}{0.881 0.881 0.881}
\newrgbcolor{PST@COLOR16}{0.874 0.874 0.874}
\newrgbcolor{PST@COLOR17}{0.866 0.866 0.866}
\newrgbcolor{PST@COLOR18}{0.858 0.858 0.858}
\newrgbcolor{PST@COLOR19}{0.85 0.85 0.85}
\newrgbcolor{PST@COLOR20}{0.842 0.842 0.842}
\newrgbcolor{PST@COLOR21}{0.834 0.834 0.834}
\newrgbcolor{PST@COLOR22}{0.826 0.826 0.826}
\newrgbcolor{PST@COLOR23}{0.818 0.818 0.818}
\newrgbcolor{PST@COLOR24}{0.811 0.811 0.811}
\newrgbcolor{PST@COLOR25}{0.803 0.803 0.803}
\newrgbcolor{PST@COLOR26}{0.795 0.795 0.795}
\newrgbcolor{PST@COLOR27}{0.787 0.787 0.787}
\newrgbcolor{PST@COLOR28}{0.779 0.779 0.779}
\newrgbcolor{PST@COLOR29}{0.771 0.771 0.771}
\newrgbcolor{PST@COLOR30}{0.763 0.763 0.763}
\newrgbcolor{PST@COLOR31}{0.755 0.755 0.755}
\newrgbcolor{PST@COLOR32}{0.748 0.748 0.748}
\newrgbcolor{PST@COLOR33}{0.74 0.74 0.74}
\newrgbcolor{PST@COLOR34}{0.732 0.732 0.732}
\newrgbcolor{PST@COLOR35}{0.724 0.724 0.724}
\newrgbcolor{PST@COLOR36}{0.716 0.716 0.716}
\newrgbcolor{PST@COLOR37}{0.708 0.708 0.708}
\newrgbcolor{PST@COLOR38}{0.7 0.7 0.7}
\newrgbcolor{PST@COLOR39}{0.692 0.692 0.692}
\newrgbcolor{PST@COLOR40}{0.685 0.685 0.685}
\newrgbcolor{PST@COLOR41}{0.677 0.677 0.677}
\newrgbcolor{PST@COLOR42}{0.669 0.669 0.669}
\newrgbcolor{PST@COLOR43}{0.661 0.661 0.661}
\newrgbcolor{PST@COLOR44}{0.653 0.653 0.653}
\newrgbcolor{PST@COLOR45}{0.645 0.645 0.645}
\newrgbcolor{PST@COLOR46}{0.637 0.637 0.637}
\newrgbcolor{PST@COLOR47}{0.629 0.629 0.629}
\newrgbcolor{PST@COLOR48}{0.622 0.622 0.622}
\newrgbcolor{PST@COLOR49}{0.614 0.614 0.614}
\newrgbcolor{PST@COLOR50}{0.606 0.606 0.606}
\newrgbcolor{PST@COLOR51}{0.598 0.598 0.598}
\newrgbcolor{PST@COLOR52}{0.59 0.59 0.59}
\newrgbcolor{PST@COLOR53}{0.582 0.582 0.582}
\newrgbcolor{PST@COLOR54}{0.574 0.574 0.574}
\newrgbcolor{PST@COLOR55}{0.566 0.566 0.566}
\newrgbcolor{PST@COLOR56}{0.559 0.559 0.559}
\newrgbcolor{PST@COLOR57}{0.551 0.551 0.551}
\newrgbcolor{PST@COLOR58}{0.543 0.543 0.543}
\newrgbcolor{PST@COLOR59}{0.535 0.535 0.535}
\newrgbcolor{PST@COLOR60}{0.527 0.527 0.527}
\newrgbcolor{PST@COLOR61}{0.519 0.519 0.519}
\newrgbcolor{PST@COLOR62}{0.511 0.511 0.511}
\newrgbcolor{PST@COLOR63}{0.503 0.503 0.503}
\newrgbcolor{PST@COLOR64}{0.496 0.496 0.496}
\newrgbcolor{PST@COLOR65}{0.488 0.488 0.488}
\newrgbcolor{PST@COLOR66}{0.48 0.48 0.48}
\newrgbcolor{PST@COLOR67}{0.472 0.472 0.472}
\newrgbcolor{PST@COLOR68}{0.464 0.464 0.464}
\newrgbcolor{PST@COLOR69}{0.456 0.456 0.456}
\newrgbcolor{PST@COLOR70}{0.448 0.448 0.448}
\newrgbcolor{PST@COLOR71}{0.44 0.44 0.44}
\newrgbcolor{PST@COLOR72}{0.433 0.433 0.433}
\newrgbcolor{PST@COLOR73}{0.425 0.425 0.425}
\newrgbcolor{PST@COLOR74}{0.417 0.417 0.417}
\newrgbcolor{PST@COLOR75}{0.409 0.409 0.409}
\newrgbcolor{PST@COLOR76}{0.401 0.401 0.401}
\newrgbcolor{PST@COLOR77}{0.393 0.393 0.393}
\newrgbcolor{PST@COLOR78}{0.385 0.385 0.385}
\newrgbcolor{PST@COLOR79}{0.377 0.377 0.377}
\newrgbcolor{PST@COLOR80}{0.37 0.37 0.37}
\newrgbcolor{PST@COLOR81}{0.362 0.362 0.362}
\newrgbcolor{PST@COLOR82}{0.354 0.354 0.354}
\newrgbcolor{PST@COLOR83}{0.346 0.346 0.346}
\newrgbcolor{PST@COLOR84}{0.338 0.338 0.338}
\newrgbcolor{PST@COLOR85}{0.33 0.33 0.33}
\newrgbcolor{PST@COLOR86}{0.322 0.322 0.322}
\newrgbcolor{PST@COLOR87}{0.314 0.314 0.314}
\newrgbcolor{PST@COLOR88}{0.307 0.307 0.307}
\newrgbcolor{PST@COLOR89}{0.299 0.299 0.299}
\newrgbcolor{PST@COLOR90}{0.291 0.291 0.291}
\newrgbcolor{PST@COLOR91}{0.283 0.283 0.283}
\newrgbcolor{PST@COLOR92}{0.275 0.275 0.275}
\newrgbcolor{PST@COLOR93}{0.267 0.267 0.267}
\newrgbcolor{PST@COLOR94}{0.259 0.259 0.259}
\newrgbcolor{PST@COLOR95}{0.251 0.251 0.251}
\newrgbcolor{PST@COLOR96}{0.244 0.244 0.244}
\newrgbcolor{PST@COLOR97}{0.236 0.236 0.236}
\newrgbcolor{PST@COLOR98}{0.228 0.228 0.228}
\newrgbcolor{PST@COLOR99}{0.22 0.22 0.22}
\newrgbcolor{PST@COLOR100}{0.212 0.212 0.212}
\newrgbcolor{PST@COLOR101}{0.204 0.204 0.204}
\newrgbcolor{PST@COLOR102}{0.196 0.196 0.196}
\newrgbcolor{PST@COLOR103}{0.188 0.188 0.188}
\newrgbcolor{PST@COLOR104}{0.181 0.181 0.181}
\newrgbcolor{PST@COLOR105}{0.173 0.173 0.173}
\newrgbcolor{PST@COLOR106}{0.165 0.165 0.165}
\newrgbcolor{PST@COLOR107}{0.157 0.157 0.157}
\newrgbcolor{PST@COLOR108}{0.149 0.149 0.149}
\newrgbcolor{PST@COLOR109}{0.141 0.141 0.141}
\newrgbcolor{PST@COLOR110}{0.133 0.133 0.133}
\newrgbcolor{PST@COLOR111}{0.125 0.125 0.125}
\newrgbcolor{PST@COLOR112}{0.118 0.118 0.118}
\newrgbcolor{PST@COLOR113}{0.11 0.11 0.11}
\newrgbcolor{PST@COLOR114}{0.102 0.102 0.102}
\newrgbcolor{PST@COLOR115}{0.094 0.094 0.094}
\newrgbcolor{PST@COLOR116}{0.086 0.086 0.086}
\newrgbcolor{PST@COLOR117}{0.078 0.078 0.078}
\newrgbcolor{PST@COLOR118}{0.07 0.07 0.07}
\newrgbcolor{PST@COLOR119}{0.062 0.062 0.062}
\newrgbcolor{PST@COLOR120}{0.055 0.055 0.055}
\newrgbcolor{PST@COLOR121}{0.047 0.047 0.047}
\newrgbcolor{PST@COLOR122}{0.039 0.039 0.039}
\newrgbcolor{PST@COLOR123}{0.031 0.031 0.031}
\newrgbcolor{PST@COLOR124}{0.023 0.023 0.023}
\newrgbcolor{PST@COLOR125}{0.015 0.015 0.015}
\newrgbcolor{PST@COLOR126}{0.007 0.007 0.007}
\newrgbcolor{PST@COLOR127}{0 0 0}


\def\polypmIIId#1{\pspolygon[linestyle=none,fillstyle=solid,fillcolor=PST@COLOR#1]}

\polypmIIId{124}(0.1432,0.19)(0.0864,0.19)(0.0864,0.1078)(0.1432,0.1078)
\polypmIIId{126}(0.1432,0.272)(0.0864,0.272)(0.0864,0.19)(0.1432,0.19)
\polypmIIId{127}(0.1432,0.3542)(0.0864,0.3542)(0.0864,0.272)(0.1432,0.272)

\polypmIIId{125}(0.2,0.19)(0.1432,0.19)(0.1432,0.1078)(0.2,0.1078)
\polypmIIId{126}(0.2,0.272)(0.1432,0.272)(0.1432,0.19)(0.2,0.19)
\polypmIIId{127}(0.2,0.3542)(0.1432,0.3542)(0.1432,0.272)(0.2,0.272)

\polypmIIId{125}(0.2568,0.19)(0.2,0.19)(0.2,0.1078)(0.2568,0.1078)
\polypmIIId{126}(0.2568,0.272)(0.2,0.272)(0.2,0.19)(0.2568,0.19)
\polypmIIId{127}(0.2568,0.3542)(0.2,0.3542)(0.2,0.272)(0.2568,0.272)

\polypmIIId{124}(0.3136,0.19)(0.2568,0.19)(0.2568,0.1078)(0.3136,0.1078)
\polypmIIId{126}(0.3136,0.272)(0.2568,0.272)(0.2568,0.19)(0.3136,0.19)
\polypmIIId{127}(0.3136,0.3542)(0.2568,0.3542)(0.2568,0.272)(0.3136,0.272)

\rput(0.1148,0.07){3}
\rput(0.1716,0.07){4}
\rput(0.2284,0.07){5}
\rput(0.2852,0.07){6}
\rput(0.2000,0.0070){years}

\rput[r](0.0806,0.1489){25}
\rput[r](0.0806,0.2310){50}
\rput[r](0.0806,0.3131){100}
\rput{L}(0.0096,0.2310){actions}

\PST@Border(0.0864,0.3542)
(0.0864,0.1078)
(0.3136,0.1078)
(0.3136,0.3542)
(0.0864,0.3542)

\catcode`@=12
\fi
\endpspicture} 
    \subfloat[2 resources]{% GNUPLOT: LaTeX picture using PSTRICKS macros
% Define new PST objects, if not already defined
\ifx\PSTloaded\undefined
\def\PSTloaded{t}

\catcode`@=11

\newpsobject{PST@Border}{psline}{linewidth=.0015,linestyle=solid}

\catcode`@=12

\fi
\psset{unit=5.0in,xunit=5.0in,yunit=3.0in}
\pspicture(0.000000,0.000000)(0.225000,0.35)
\ifx\nofigs\undefined
\catcode`@=11

\newrgbcolor{PST@COLOR0}{1 1 1}
\newrgbcolor{PST@COLOR1}{0.992 0.992 0.992}
\newrgbcolor{PST@COLOR2}{0.984 0.984 0.984}
\newrgbcolor{PST@COLOR3}{0.976 0.976 0.976}
\newrgbcolor{PST@COLOR4}{0.968 0.968 0.968}
\newrgbcolor{PST@COLOR5}{0.96 0.96 0.96}
\newrgbcolor{PST@COLOR6}{0.952 0.952 0.952}
\newrgbcolor{PST@COLOR7}{0.944 0.944 0.944}
\newrgbcolor{PST@COLOR8}{0.937 0.937 0.937}
\newrgbcolor{PST@COLOR9}{0.929 0.929 0.929}
\newrgbcolor{PST@COLOR10}{0.921 0.921 0.921}
\newrgbcolor{PST@COLOR11}{0.913 0.913 0.913}
\newrgbcolor{PST@COLOR12}{0.905 0.905 0.905}
\newrgbcolor{PST@COLOR13}{0.897 0.897 0.897}
\newrgbcolor{PST@COLOR14}{0.889 0.889 0.889}
\newrgbcolor{PST@COLOR15}{0.881 0.881 0.881}
\newrgbcolor{PST@COLOR16}{0.874 0.874 0.874}
\newrgbcolor{PST@COLOR17}{0.866 0.866 0.866}
\newrgbcolor{PST@COLOR18}{0.858 0.858 0.858}
\newrgbcolor{PST@COLOR19}{0.85 0.85 0.85}
\newrgbcolor{PST@COLOR20}{0.842 0.842 0.842}
\newrgbcolor{PST@COLOR21}{0.834 0.834 0.834}
\newrgbcolor{PST@COLOR22}{0.826 0.826 0.826}
\newrgbcolor{PST@COLOR23}{0.818 0.818 0.818}
\newrgbcolor{PST@COLOR24}{0.811 0.811 0.811}
\newrgbcolor{PST@COLOR25}{0.803 0.803 0.803}
\newrgbcolor{PST@COLOR26}{0.795 0.795 0.795}
\newrgbcolor{PST@COLOR27}{0.787 0.787 0.787}
\newrgbcolor{PST@COLOR28}{0.779 0.779 0.779}
\newrgbcolor{PST@COLOR29}{0.771 0.771 0.771}
\newrgbcolor{PST@COLOR30}{0.763 0.763 0.763}
\newrgbcolor{PST@COLOR31}{0.755 0.755 0.755}
\newrgbcolor{PST@COLOR32}{0.748 0.748 0.748}
\newrgbcolor{PST@COLOR33}{0.74 0.74 0.74}
\newrgbcolor{PST@COLOR34}{0.732 0.732 0.732}
\newrgbcolor{PST@COLOR35}{0.724 0.724 0.724}
\newrgbcolor{PST@COLOR36}{0.716 0.716 0.716}
\newrgbcolor{PST@COLOR37}{0.708 0.708 0.708}
\newrgbcolor{PST@COLOR38}{0.7 0.7 0.7}
\newrgbcolor{PST@COLOR39}{0.692 0.692 0.692}
\newrgbcolor{PST@COLOR40}{0.685 0.685 0.685}
\newrgbcolor{PST@COLOR41}{0.677 0.677 0.677}
\newrgbcolor{PST@COLOR42}{0.669 0.669 0.669}
\newrgbcolor{PST@COLOR43}{0.661 0.661 0.661}
\newrgbcolor{PST@COLOR44}{0.653 0.653 0.653}
\newrgbcolor{PST@COLOR45}{0.645 0.645 0.645}
\newrgbcolor{PST@COLOR46}{0.637 0.637 0.637}
\newrgbcolor{PST@COLOR47}{0.629 0.629 0.629}
\newrgbcolor{PST@COLOR48}{0.622 0.622 0.622}
\newrgbcolor{PST@COLOR49}{0.614 0.614 0.614}
\newrgbcolor{PST@COLOR50}{0.606 0.606 0.606}
\newrgbcolor{PST@COLOR51}{0.598 0.598 0.598}
\newrgbcolor{PST@COLOR52}{0.59 0.59 0.59}
\newrgbcolor{PST@COLOR53}{0.582 0.582 0.582}
\newrgbcolor{PST@COLOR54}{0.574 0.574 0.574}
\newrgbcolor{PST@COLOR55}{0.566 0.566 0.566}
\newrgbcolor{PST@COLOR56}{0.559 0.559 0.559}
\newrgbcolor{PST@COLOR57}{0.551 0.551 0.551}
\newrgbcolor{PST@COLOR58}{0.543 0.543 0.543}
\newrgbcolor{PST@COLOR59}{0.535 0.535 0.535}
\newrgbcolor{PST@COLOR60}{0.527 0.527 0.527}
\newrgbcolor{PST@COLOR61}{0.519 0.519 0.519}
\newrgbcolor{PST@COLOR62}{0.511 0.511 0.511}
\newrgbcolor{PST@COLOR63}{0.503 0.503 0.503}
\newrgbcolor{PST@COLOR64}{0.496 0.496 0.496}
\newrgbcolor{PST@COLOR65}{0.488 0.488 0.488}
\newrgbcolor{PST@COLOR66}{0.48 0.48 0.48}
\newrgbcolor{PST@COLOR67}{0.472 0.472 0.472}
\newrgbcolor{PST@COLOR68}{0.464 0.464 0.464}
\newrgbcolor{PST@COLOR69}{0.456 0.456 0.456}
\newrgbcolor{PST@COLOR70}{0.448 0.448 0.448}
\newrgbcolor{PST@COLOR71}{0.44 0.44 0.44}
\newrgbcolor{PST@COLOR72}{0.433 0.433 0.433}
\newrgbcolor{PST@COLOR73}{0.425 0.425 0.425}
\newrgbcolor{PST@COLOR74}{0.417 0.417 0.417}
\newrgbcolor{PST@COLOR75}{0.409 0.409 0.409}
\newrgbcolor{PST@COLOR76}{0.401 0.401 0.401}
\newrgbcolor{PST@COLOR77}{0.393 0.393 0.393}
\newrgbcolor{PST@COLOR78}{0.385 0.385 0.385}
\newrgbcolor{PST@COLOR79}{0.377 0.377 0.377}
\newrgbcolor{PST@COLOR80}{0.37 0.37 0.37}
\newrgbcolor{PST@COLOR81}{0.362 0.362 0.362}
\newrgbcolor{PST@COLOR82}{0.354 0.354 0.354}
\newrgbcolor{PST@COLOR83}{0.346 0.346 0.346}
\newrgbcolor{PST@COLOR84}{0.338 0.338 0.338}
\newrgbcolor{PST@COLOR85}{0.33 0.33 0.33}
\newrgbcolor{PST@COLOR86}{0.322 0.322 0.322}
\newrgbcolor{PST@COLOR87}{0.314 0.314 0.314}
\newrgbcolor{PST@COLOR88}{0.307 0.307 0.307}
\newrgbcolor{PST@COLOR89}{0.299 0.299 0.299}
\newrgbcolor{PST@COLOR90}{0.291 0.291 0.291}
\newrgbcolor{PST@COLOR91}{0.283 0.283 0.283}
\newrgbcolor{PST@COLOR92}{0.275 0.275 0.275}
\newrgbcolor{PST@COLOR93}{0.267 0.267 0.267}
\newrgbcolor{PST@COLOR94}{0.259 0.259 0.259}
\newrgbcolor{PST@COLOR95}{0.251 0.251 0.251}
\newrgbcolor{PST@COLOR96}{0.244 0.244 0.244}
\newrgbcolor{PST@COLOR97}{0.236 0.236 0.236}
\newrgbcolor{PST@COLOR98}{0.228 0.228 0.228}
\newrgbcolor{PST@COLOR99}{0.22 0.22 0.22}
\newrgbcolor{PST@COLOR100}{0.212 0.212 0.212}
\newrgbcolor{PST@COLOR101}{0.204 0.204 0.204}
\newrgbcolor{PST@COLOR102}{0.196 0.196 0.196}
\newrgbcolor{PST@COLOR103}{0.188 0.188 0.188}
\newrgbcolor{PST@COLOR104}{0.181 0.181 0.181}
\newrgbcolor{PST@COLOR105}{0.173 0.173 0.173}
\newrgbcolor{PST@COLOR106}{0.165 0.165 0.165}
\newrgbcolor{PST@COLOR107}{0.157 0.157 0.157}
\newrgbcolor{PST@COLOR108}{0.149 0.149 0.149}
\newrgbcolor{PST@COLOR109}{0.141 0.141 0.141}
\newrgbcolor{PST@COLOR110}{0.133 0.133 0.133}
\newrgbcolor{PST@COLOR111}{0.125 0.125 0.125}
\newrgbcolor{PST@COLOR112}{0.118 0.118 0.118}
\newrgbcolor{PST@COLOR113}{0.11 0.11 0.11}
\newrgbcolor{PST@COLOR114}{0.102 0.102 0.102}
\newrgbcolor{PST@COLOR115}{0.094 0.094 0.094}
\newrgbcolor{PST@COLOR116}{0.086 0.086 0.086}
\newrgbcolor{PST@COLOR117}{0.078 0.078 0.078}
\newrgbcolor{PST@COLOR118}{0.07 0.07 0.07}
\newrgbcolor{PST@COLOR119}{0.062 0.062 0.062}
\newrgbcolor{PST@COLOR120}{0.055 0.055 0.055}
\newrgbcolor{PST@COLOR121}{0.047 0.047 0.047}
\newrgbcolor{PST@COLOR122}{0.039 0.039 0.039}
\newrgbcolor{PST@COLOR123}{0.031 0.031 0.031}
\newrgbcolor{PST@COLOR124}{0.023 0.023 0.023}
\newrgbcolor{PST@COLOR125}{0.015 0.015 0.015}
\newrgbcolor{PST@COLOR126}{0.007 0.007 0.007}
\newrgbcolor{PST@COLOR127}{0 0 0}

\def\polypmIIId#1{\pspolygon[linestyle=none,fillstyle=solid,fillcolor=PST@COLOR#1]}

\polypmIIId{122} (0.0568,0.19)  (0.0,0.19)  (0.0,0.1078)(0.0568,0.1078)
\polypmIIId{125}  (0.0568,0.272) (0.0,0.272) (0.0,0.19)  (0.0568,0.19)
\polypmIIId{126}  (0.0568,0.3542)(0.0,0.3542)(0.0,0.272) (0.0568,0.272)

\polypmIIId{121} (0.1136,   0.19)  (0.0568,0.19)  (0.0568,0.1078)(0.1136,0.1078)
\polypmIIId{124}  (0.1136,   0.272) (0.0568,0.272) (0.0568,0.19)  (0.1136,0.19)
\polypmIIId{126}  (0.1136,   0.3542)(0.0568,0.3542)(0.0568,0.272) (0.1136,0.272)

\polypmIIId{120}(0.1704,0.19)  (0.1136,   0.19)  (0.1136,   0.1078)(0.1704,0.1078)
\polypmIIId{124} (0.1704,0.272) (0.1136,   0.272) (0.1136,   0.19)  (0.1704,0.19)
\polypmIIId{125}  (0.1704,0.3542)(0.1136,   0.3542)(0.1136,   0.272) (0.1704,0.272)

\polypmIIId{119}(0.2272,0.19)  (0.1704,0.19)  (0.1704,0.1078)(0.2272,0.1078)
\polypmIIId{123}  (0.2272,0.272) (0.1704,0.272) (0.1704,0.19)  (0.2272,0.19)
\polypmIIId{125}  (0.2272,0.3542)(0.1704,0.3542)(0.1704,0.272) (0.2272,0.272)

\rput(0.0284,0.07){3}
\rput(0.0852,0.07){4}
\rput(0.1420,0.07){5}
\rput(0.1988,0.07){6}
\rput(0.1136,0.0070){years}


\PST@Border(0.0,0.3542)
(0.0,0.1078)
(0.2272,0.1078)
(0.2272,0.3542)
(0.0,0.3542)

\catcode`@=12
\fi
\endpspicture}
    \subfloat[4 resources]{% GNUPLOT: LaTeX picture using PSTRICKS macros
% Define new PST objects, if not already defined
\ifx\PSTloaded\undefined
\def\PSTloaded{t}

\catcode`@=11

\newpsobject{PST@Border}{psline}{linewidth=.0015,linestyle=solid}

\catcode`@=12

\fi
\psset{unit=5.0in,xunit=5.0in,yunit=3.0in}
\pspicture(0.000000,0.000000)(0.3136,0.35)
\ifx\nofigs\undefined
\catcode`@=11

\newrgbcolor{PST@COLOR0}{1 1 1}
\newrgbcolor{PST@COLOR1}{0.992 0.992 0.992}
\newrgbcolor{PST@COLOR2}{0.984 0.984 0.984}
\newrgbcolor{PST@COLOR3}{0.976 0.976 0.976}
\newrgbcolor{PST@COLOR4}{0.968 0.968 0.968}
\newrgbcolor{PST@COLOR5}{0.96 0.96 0.96}
\newrgbcolor{PST@COLOR6}{0.952 0.952 0.952}
\newrgbcolor{PST@COLOR7}{0.944 0.944 0.944}
\newrgbcolor{PST@COLOR8}{0.937 0.937 0.937}
\newrgbcolor{PST@COLOR9}{0.929 0.929 0.929}
\newrgbcolor{PST@COLOR10}{0.921 0.921 0.921}
\newrgbcolor{PST@COLOR11}{0.913 0.913 0.913}
\newrgbcolor{PST@COLOR12}{0.905 0.905 0.905}
\newrgbcolor{PST@COLOR13}{0.897 0.897 0.897}
\newrgbcolor{PST@COLOR14}{0.889 0.889 0.889}
\newrgbcolor{PST@COLOR15}{0.881 0.881 0.881}
\newrgbcolor{PST@COLOR16}{0.874 0.874 0.874}
\newrgbcolor{PST@COLOR17}{0.866 0.866 0.866}
\newrgbcolor{PST@COLOR18}{0.858 0.858 0.858}
\newrgbcolor{PST@COLOR19}{0.85 0.85 0.85}
\newrgbcolor{PST@COLOR20}{0.842 0.842 0.842}
\newrgbcolor{PST@COLOR21}{0.834 0.834 0.834}
\newrgbcolor{PST@COLOR22}{0.826 0.826 0.826}
\newrgbcolor{PST@COLOR23}{0.818 0.818 0.818}
\newrgbcolor{PST@COLOR24}{0.811 0.811 0.811}
\newrgbcolor{PST@COLOR25}{0.803 0.803 0.803}
\newrgbcolor{PST@COLOR26}{0.795 0.795 0.795}
\newrgbcolor{PST@COLOR27}{0.787 0.787 0.787}
\newrgbcolor{PST@COLOR28}{0.779 0.779 0.779}
\newrgbcolor{PST@COLOR29}{0.771 0.771 0.771}
\newrgbcolor{PST@COLOR30}{0.763 0.763 0.763}
\newrgbcolor{PST@COLOR31}{0.755 0.755 0.755}
\newrgbcolor{PST@COLOR32}{0.748 0.748 0.748}
\newrgbcolor{PST@COLOR33}{0.74 0.74 0.74}
\newrgbcolor{PST@COLOR34}{0.732 0.732 0.732}
\newrgbcolor{PST@COLOR35}{0.724 0.724 0.724}
\newrgbcolor{PST@COLOR36}{0.716 0.716 0.716}
\newrgbcolor{PST@COLOR37}{0.708 0.708 0.708}
\newrgbcolor{PST@COLOR38}{0.7 0.7 0.7}
\newrgbcolor{PST@COLOR39}{0.692 0.692 0.692}
\newrgbcolor{PST@COLOR40}{0.685 0.685 0.685}
\newrgbcolor{PST@COLOR41}{0.677 0.677 0.677}
\newrgbcolor{PST@COLOR42}{0.669 0.669 0.669}
\newrgbcolor{PST@COLOR43}{0.661 0.661 0.661}
\newrgbcolor{PST@COLOR44}{0.653 0.653 0.653}
\newrgbcolor{PST@COLOR45}{0.645 0.645 0.645}
\newrgbcolor{PST@COLOR46}{0.637 0.637 0.637}
\newrgbcolor{PST@COLOR47}{0.629 0.629 0.629}
\newrgbcolor{PST@COLOR48}{0.622 0.622 0.622}
\newrgbcolor{PST@COLOR49}{0.614 0.614 0.614}
\newrgbcolor{PST@COLOR50}{0.606 0.606 0.606}
\newrgbcolor{PST@COLOR51}{0.598 0.598 0.598}
\newrgbcolor{PST@COLOR52}{0.59 0.59 0.59}
\newrgbcolor{PST@COLOR53}{0.582 0.582 0.582}
\newrgbcolor{PST@COLOR54}{0.574 0.574 0.574}
\newrgbcolor{PST@COLOR55}{0.566 0.566 0.566}
\newrgbcolor{PST@COLOR56}{0.559 0.559 0.559}
\newrgbcolor{PST@COLOR57}{0.551 0.551 0.551}
\newrgbcolor{PST@COLOR58}{0.543 0.543 0.543}
\newrgbcolor{PST@COLOR59}{0.535 0.535 0.535}
\newrgbcolor{PST@COLOR60}{0.527 0.527 0.527}
\newrgbcolor{PST@COLOR61}{0.519 0.519 0.519}
\newrgbcolor{PST@COLOR62}{0.511 0.511 0.511}
\newrgbcolor{PST@COLOR63}{0.503 0.503 0.503}
\newrgbcolor{PST@COLOR64}{0.496 0.496 0.496}
\newrgbcolor{PST@COLOR65}{0.488 0.488 0.488}
\newrgbcolor{PST@COLOR66}{0.48 0.48 0.48}
\newrgbcolor{PST@COLOR67}{0.472 0.472 0.472}
\newrgbcolor{PST@COLOR68}{0.464 0.464 0.464}
\newrgbcolor{PST@COLOR69}{0.456 0.456 0.456}
\newrgbcolor{PST@COLOR70}{0.448 0.448 0.448}
\newrgbcolor{PST@COLOR71}{0.44 0.44 0.44}
\newrgbcolor{PST@COLOR72}{0.433 0.433 0.433}
\newrgbcolor{PST@COLOR73}{0.425 0.425 0.425}
\newrgbcolor{PST@COLOR74}{0.417 0.417 0.417}
\newrgbcolor{PST@COLOR75}{0.409 0.409 0.409}
\newrgbcolor{PST@COLOR76}{0.401 0.401 0.401}
\newrgbcolor{PST@COLOR77}{0.393 0.393 0.393}
\newrgbcolor{PST@COLOR78}{0.385 0.385 0.385}
\newrgbcolor{PST@COLOR79}{0.377 0.377 0.377}
\newrgbcolor{PST@COLOR80}{0.37 0.37 0.37}
\newrgbcolor{PST@COLOR81}{0.362 0.362 0.362}
\newrgbcolor{PST@COLOR82}{0.354 0.354 0.354}
\newrgbcolor{PST@COLOR83}{0.346 0.346 0.346}
\newrgbcolor{PST@COLOR84}{0.338 0.338 0.338}
\newrgbcolor{PST@COLOR85}{0.33 0.33 0.33}
\newrgbcolor{PST@COLOR86}{0.322 0.322 0.322}
\newrgbcolor{PST@COLOR87}{0.314 0.314 0.314}
\newrgbcolor{PST@COLOR88}{0.307 0.307 0.307}
\newrgbcolor{PST@COLOR89}{0.299 0.299 0.299}
\newrgbcolor{PST@COLOR90}{0.291 0.291 0.291}
\newrgbcolor{PST@COLOR91}{0.283 0.283 0.283}
\newrgbcolor{PST@COLOR92}{0.275 0.275 0.275}
\newrgbcolor{PST@COLOR93}{0.267 0.267 0.267}
\newrgbcolor{PST@COLOR94}{0.259 0.259 0.259}
\newrgbcolor{PST@COLOR95}{0.251 0.251 0.251}
\newrgbcolor{PST@COLOR96}{0.244 0.244 0.244}
\newrgbcolor{PST@COLOR97}{0.236 0.236 0.236}
\newrgbcolor{PST@COLOR98}{0.228 0.228 0.228}
\newrgbcolor{PST@COLOR99}{0.22 0.22 0.22}
\newrgbcolor{PST@COLOR100}{0.212 0.212 0.212}
\newrgbcolor{PST@COLOR101}{0.204 0.204 0.204}
\newrgbcolor{PST@COLOR102}{0.196 0.196 0.196}
\newrgbcolor{PST@COLOR103}{0.188 0.188 0.188}
\newrgbcolor{PST@COLOR104}{0.181 0.181 0.181}
\newrgbcolor{PST@COLOR105}{0.173 0.173 0.173}
\newrgbcolor{PST@COLOR106}{0.165 0.165 0.165}
\newrgbcolor{PST@COLOR107}{0.157 0.157 0.157}
\newrgbcolor{PST@COLOR108}{0.149 0.149 0.149}
\newrgbcolor{PST@COLOR109}{0.141 0.141 0.141}
\newrgbcolor{PST@COLOR110}{0.133 0.133 0.133}
\newrgbcolor{PST@COLOR111}{0.125 0.125 0.125}
\newrgbcolor{PST@COLOR112}{0.118 0.118 0.118}
\newrgbcolor{PST@COLOR113}{0.11 0.11 0.11}
\newrgbcolor{PST@COLOR114}{0.102 0.102 0.102}
\newrgbcolor{PST@COLOR115}{0.094 0.094 0.094}
\newrgbcolor{PST@COLOR116}{0.086 0.086 0.086}
\newrgbcolor{PST@COLOR117}{0.078 0.078 0.078}
\newrgbcolor{PST@COLOR118}{0.07 0.07 0.07}
\newrgbcolor{PST@COLOR119}{0.062 0.062 0.062}
\newrgbcolor{PST@COLOR120}{0.055 0.055 0.055}
\newrgbcolor{PST@COLOR121}{0.047 0.047 0.047}
\newrgbcolor{PST@COLOR122}{0.039 0.039 0.039}
\newrgbcolor{PST@COLOR123}{0.031 0.031 0.031}
\newrgbcolor{PST@COLOR124}{0.023 0.023 0.023}
\newrgbcolor{PST@COLOR125}{0.015 0.015 0.015}
\newrgbcolor{PST@COLOR126}{0.007 0.007 0.007}
\newrgbcolor{PST@COLOR127}{0 0 0}

\def\polypmIIId#1{\pspolygon[linestyle=none,fillstyle=solid,fillcolor=PST@COLOR#1]}

\polypmIIId{117} (0.0568,0.19)  (0.0,0.19)  (0.0,0.1078)(0.0568,0.1078)
\polypmIIId{120}  (0.0568,0.272) (0.0,0.272) (0.0,0.19)  (0.0568,0.19)
\polypmIIId{124}  (0.0568,0.3542)(0.0,0.3542)(0.0,0.272) (0.0568,0.272)

\polypmIIId{115} (0.1136,   0.19)  (0.0568,0.19)  (0.0568,0.1078)(0.1136,0.1078)
\polypmIIId{120}  (0.1136,   0.272) (0.0568,0.272) (0.0568,0.19)  (0.1136,0.19)
\polypmIIId{123}  (0.1136,   0.3542)(0.0568,0.3542)(0.0568,0.272) (0.1136,0.272)

\polypmIIId{113}(0.1704,0.19)  (0.1136,   0.19)  (0.1136,   0.1078)(0.1704,0.1078)
\polypmIIId{118} (0.1704,0.272) (0.1136,   0.272) (0.1136,   0.19)  (0.1704,0.19)
\polypmIIId{122}  (0.1704,0.3542)(0.1136,   0.3542)(0.1136,   0.272) (0.1704,0.272)

\polypmIIId{112}(0.2272,0.19)  (0.1704,0.19)  (0.1704,0.1078)(0.2272,0.1078)
\polypmIIId{118}  (0.2272,0.272) (0.1704,0.272) (0.1704,0.19)  (0.2272,0.19)
\polypmIIId{122}  (0.2272,0.3542)(0.1704,0.3542)(0.1704,0.272) (0.2272,0.272)

\rput(0.0284,0.07){3}
\rput(0.0852,0.07){4}
\rput(0.1420,0.07){5}
\rput(0.1988,0.07){6}
\rput(0.1136,0.0070){years}

\PST@Border(0.0,0.3542)
(0.0,0.1078)
(0.2272,0.1078)
(0.2272,0.3542)
(0.0,0.3542)

\polypmIIId{0}(0.2329,0.1078)(0.2442,0.1078)(0.2442,0.1098)(0.2329,0.1098)
\polypmIIId{1}(0.2329,0.1097)(0.2442,0.1097)(0.2442,0.1117)(0.2329,0.1117)
\polypmIIId{2}(0.2329,0.1116)(0.2442,0.1116)(0.2442,0.1136)(0.2329,0.1136)
\polypmIIId{3}(0.2329,0.1135)(0.2442,0.1135)(0.2442,0.1156)(0.2329,0.1156)
\polypmIIId{4}(0.2329,0.1155)(0.2442,0.1155)(0.2442,0.1175)(0.2329,0.1175)
\polypmIIId{5}(0.2329,0.1174)(0.2442,0.1174)(0.2442,0.1194)(0.2329,0.1194)
\polypmIIId{6}(0.2329,0.1193)(0.2442,0.1193)(0.2442,0.1213)(0.2329,0.1213)
\polypmIIId{7}(0.2329,0.1212)(0.2442,0.1212)(0.2442,0.1233)(0.2329,0.1233)
\polypmIIId{8}(0.2329,0.1232)(0.2442,0.1232)(0.2442,0.1252)(0.2329,0.1252)
\polypmIIId{9}(0.2329,0.1251)(0.2442,0.1251)(0.2442,0.1271)(0.2329,0.1271)
\polypmIIId{10}(0.2329,0.127)(0.2442,0.127)(0.2442,0.129)(0.2329,0.129)
\polypmIIId{11}(0.2329,0.1289)(0.2442,0.1289)(0.2442,0.131)(0.2329,0.131)
\polypmIIId{12}(0.2329,0.1309)(0.2442,0.1309)(0.2442,0.1329)(0.2329,0.1329)
\polypmIIId{13}(0.2329,0.1328)(0.2442,0.1328)(0.2442,0.1348)(0.2329,0.1348)
\polypmIIId{14}(0.2329,0.1347)(0.2442,0.1347)(0.2442,0.1367)(0.2329,0.1367)
\polypmIIId{15}(0.2329,0.1366)(0.2442,0.1366)(0.2442,0.1387)(0.2329,0.1387)
\polypmIIId{16}(0.2329,0.1386)(0.2442,0.1386)(0.2442,0.1406)(0.2329,0.1406)
\polypmIIId{17}(0.2329,0.1405)(0.2442,0.1405)(0.2442,0.1425)(0.2329,0.1425)
\polypmIIId{18}(0.2329,0.1424)(0.2442,0.1424)(0.2442,0.1444)(0.2329,0.1444)
\polypmIIId{19}(0.2329,0.1443)(0.2442,0.1443)(0.2442,0.1464)(0.2329,0.1464)
\polypmIIId{20}(0.2329,0.1463)(0.2442,0.1463)(0.2442,0.1483)(0.2329,0.1483)
\polypmIIId{21}(0.2329,0.1482)(0.2442,0.1482)(0.2442,0.1502)(0.2329,0.1502)
\polypmIIId{22}(0.2329,0.1501)(0.2442,0.1501)(0.2442,0.1521)(0.2329,0.1521)
\polypmIIId{23}(0.2329,0.152)(0.2442,0.152)(0.2442,0.1541)(0.2329,0.1541)
\polypmIIId{24}(0.2329,0.154)(0.2442,0.154)(0.2442,0.156)(0.2329,0.156)
\polypmIIId{25}(0.2329,0.1559)(0.2442,0.1559)(0.2442,0.1579)(0.2329,0.1579)
\polypmIIId{26}(0.2329,0.1578)(0.2442,0.1578)(0.2442,0.1598)(0.2329,0.1598)
\polypmIIId{27}(0.2329,0.1597)(0.2442,0.1597)(0.2442,0.1618)(0.2329,0.1618)
\polypmIIId{28}(0.2329,0.1617)(0.2442,0.1617)(0.2442,0.1637)(0.2329,0.1637)
\polypmIIId{29}(0.2329,0.1636)(0.2442,0.1636)(0.2442,0.1656)(0.2329,0.1656)
\polypmIIId{30}(0.2329,0.1655)(0.2442,0.1655)(0.2442,0.1675)(0.2329,0.1675)
\polypmIIId{31}(0.2329,0.1674)(0.2442,0.1674)(0.2442,0.1695)(0.2329,0.1695)
\polypmIIId{32}(0.2329,0.1694)(0.2442,0.1694)(0.2442,0.1714)(0.2329,0.1714)
\polypmIIId{33}(0.2329,0.1713)(0.2442,0.1713)(0.2442,0.1733)(0.2329,0.1733)
\polypmIIId{34}(0.2329,0.1732)(0.2442,0.1732)(0.2442,0.1752)(0.2329,0.1752)
\polypmIIId{35}(0.2329,0.1751)(0.2442,0.1751)(0.2442,0.1772)(0.2329,0.1772)
\polypmIIId{36}(0.2329,0.1771)(0.2442,0.1771)(0.2442,0.1791)(0.2329,0.1791)
\polypmIIId{37}(0.2329,0.179)(0.2442,0.179)(0.2442,0.181)(0.2329,0.181)
\polypmIIId{38}(0.2329,0.1809)(0.2442,0.1809)(0.2442,0.1829)(0.2329,0.1829)
\polypmIIId{39}(0.2329,0.1828)(0.2442,0.1828)(0.2442,0.1849)(0.2329,0.1849)
\polypmIIId{40}(0.2329,0.1848)(0.2442,0.1848)(0.2442,0.1868)(0.2329,0.1868)
\polypmIIId{41}(0.2329,0.1867)(0.2442,0.1867)(0.2442,0.1887)(0.2329,0.1887)
\polypmIIId{42}(0.2329,0.1886)(0.2442,0.1886)(0.2442,0.1906)(0.2329,0.1906)
\polypmIIId{43}(0.2329,0.1905)(0.2442,0.1905)(0.2442,0.1926)(0.2329,0.1926)
\polypmIIId{44}(0.2329,0.1925)(0.2442,0.1925)(0.2442,0.1945)(0.2329,0.1945)
\polypmIIId{45}(0.2329,0.1944)(0.2442,0.1944)(0.2442,0.1964)(0.2329,0.1964)
\polypmIIId{46}(0.2329,0.1963)(0.2442,0.1963)(0.2442,0.1983)(0.2329,0.1983)
\polypmIIId{47}(0.2329,0.1982)(0.2442,0.1982)(0.2442,0.2003)(0.2329,0.2003)
\polypmIIId{48}(0.2329,0.2002)(0.2442,0.2002)(0.2442,0.2022)(0.2329,0.2022)
\polypmIIId{49}(0.2329,0.2021)(0.2442,0.2021)(0.2442,0.2041)(0.2329,0.2041)
\polypmIIId{50}(0.2329,0.204)(0.2442,0.204)(0.2442,0.206)(0.2329,0.206)
\polypmIIId{51}(0.2329,0.2059)(0.2442,0.2059)(0.2442,0.208)(0.2329,0.208)
\polypmIIId{52}(0.2329,0.2079)(0.2442,0.2079)(0.2442,0.2099)(0.2329,0.2099)
\polypmIIId{53}(0.2329,0.2098)(0.2442,0.2098)(0.2442,0.2118)(0.2329,0.2118)
\polypmIIId{54}(0.2329,0.2117)(0.2442,0.2117)(0.2442,0.2137)(0.2329,0.2137)
\polypmIIId{55}(0.2329,0.2136)(0.2442,0.2136)(0.2442,0.2157)(0.2329,0.2157)
\polypmIIId{56}(0.2329,0.2156)(0.2442,0.2156)(0.2442,0.2176)(0.2329,0.2176)
\polypmIIId{57}(0.2329,0.2175)(0.2442,0.2175)(0.2442,0.2195)(0.2329,0.2195)
\polypmIIId{58}(0.2329,0.2194)(0.2442,0.2194)(0.2442,0.2214)(0.2329,0.2214)
\polypmIIId{59}(0.2329,0.2213)(0.2442,0.2213)(0.2442,0.2234)(0.2329,0.2234)
\polypmIIId{60}(0.2329,0.2233)(0.2442,0.2233)(0.2442,0.2253)(0.2329,0.2253)
\polypmIIId{61}(0.2329,0.2252)(0.2442,0.2252)(0.2442,0.2272)(0.2329,0.2272)
\polypmIIId{62}(0.2329,0.2271)(0.2442,0.2271)(0.2442,0.2291)(0.2329,0.2291)
\polypmIIId{63}(0.2329,0.229)(0.2442,0.229)(0.2442,0.2311)(0.2329,0.2311)
\polypmIIId{64}(0.2329,0.231)(0.2442,0.231)(0.2442,0.233)(0.2329,0.233)
\polypmIIId{65}(0.2329,0.2329)(0.2442,0.2329)(0.2442,0.2349)(0.2329,0.2349)
\polypmIIId{66}(0.2329,0.2348)(0.2442,0.2348)(0.2442,0.2368)(0.2329,0.2368)
\polypmIIId{67}(0.2329,0.2367)(0.2442,0.2367)(0.2442,0.2388)(0.2329,0.2388)
\polypmIIId{68}(0.2329,0.2387)(0.2442,0.2387)(0.2442,0.2407)(0.2329,0.2407)
\polypmIIId{69}(0.2329,0.2406)(0.2442,0.2406)(0.2442,0.2426)(0.2329,0.2426)
\polypmIIId{70}(0.2329,0.2425)(0.2442,0.2425)(0.2442,0.2445)(0.2329,0.2445)
\polypmIIId{71}(0.2329,0.2444)(0.2442,0.2444)(0.2442,0.2465)(0.2329,0.2465)
\polypmIIId{72}(0.2329,0.2464)(0.2442,0.2464)(0.2442,0.2484)(0.2329,0.2484)
\polypmIIId{73}(0.2329,0.2483)(0.2442,0.2483)(0.2442,0.2503)(0.2329,0.2503)
\polypmIIId{74}(0.2329,0.2502)(0.2442,0.2502)(0.2442,0.2522)(0.2329,0.2522)
\polypmIIId{75}(0.2329,0.2521)(0.2442,0.2521)(0.2442,0.2542)(0.2329,0.2542)
\polypmIIId{76}(0.2329,0.2541)(0.2442,0.2541)(0.2442,0.2561)(0.2329,0.2561)
\polypmIIId{77}(0.2329,0.256)(0.2442,0.256)(0.2442,0.258)(0.2329,0.258)
\polypmIIId{78}(0.2329,0.2579)(0.2442,0.2579)(0.2442,0.2599)(0.2329,0.2599)
\polypmIIId{79}(0.2329,0.2598)(0.2442,0.2598)(0.2442,0.2619)(0.2329,0.2619)
\polypmIIId{80}(0.2329,0.2618)(0.2442,0.2618)(0.2442,0.2638)(0.2329,0.2638)
\polypmIIId{81}(0.2329,0.2637)(0.2442,0.2637)(0.2442,0.2657)(0.2329,0.2657)
\polypmIIId{82}(0.2329,0.2656)(0.2442,0.2656)(0.2442,0.2676)(0.2329,0.2676)
\polypmIIId{83}(0.2329,0.2675)(0.2442,0.2675)(0.2442,0.2696)(0.2329,0.2696)
\polypmIIId{84}(0.2329,0.2695)(0.2442,0.2695)(0.2442,0.2715)(0.2329,0.2715)
\polypmIIId{85}(0.2329,0.2714)(0.2442,0.2714)(0.2442,0.2734)(0.2329,0.2734)
\polypmIIId{86}(0.2329,0.2733)(0.2442,0.2733)(0.2442,0.2753)(0.2329,0.2753)
\polypmIIId{87}(0.2329,0.2752)(0.2442,0.2752)(0.2442,0.2773)(0.2329,0.2773)
\polypmIIId{88}(0.2329,0.2772)(0.2442,0.2772)(0.2442,0.2792)(0.2329,0.2792)
\polypmIIId{89}(0.2329,0.2791)(0.2442,0.2791)(0.2442,0.2811)(0.2329,0.2811)
\polypmIIId{90}(0.2329,0.281)(0.2442,0.281)(0.2442,0.283)(0.2329,0.283)
\polypmIIId{91}(0.2329,0.2829)(0.2442,0.2829)(0.2442,0.285)(0.2329,0.285)
\polypmIIId{92}(0.2329,0.2849)(0.2442,0.2849)(0.2442,0.2869)(0.2329,0.2869)
\polypmIIId{93}(0.2329,0.2868)(0.2442,0.2868)(0.2442,0.2888)(0.2329,0.2888)
\polypmIIId{94}(0.2329,0.2887)(0.2442,0.2887)(0.2442,0.2907)(0.2329,0.2907)
\polypmIIId{95}(0.2329,0.2906)(0.2442,0.2906)(0.2442,0.2927)(0.2329,0.2927)
\polypmIIId{96}(0.2329,0.2926)(0.2442,0.2926)(0.2442,0.2946)(0.2329,0.2946)
\polypmIIId{97}(0.2329,0.2945)(0.2442,0.2945)(0.2442,0.2965)(0.2329,0.2965)
\polypmIIId{98}(0.2329,0.2964)(0.2442,0.2964)(0.2442,0.2984)(0.2329,0.2984)
\polypmIIId{99}(0.2329,0.2983)(0.2442,0.2983)(0.2442,0.3004)(0.2329,0.3004)
\polypmIIId{100}(0.2329,0.3003)(0.2442,0.3003)(0.2442,0.3023)(0.2329,0.3023)
\polypmIIId{101}(0.2329,0.3022)(0.2442,0.3022)(0.2442,0.3042)(0.2329,0.3042)
\polypmIIId{102}(0.2329,0.3041)(0.2442,0.3041)(0.2442,0.3061)(0.2329,0.3061)
\polypmIIId{103}(0.2329,0.306)(0.2442,0.306)(0.2442,0.3081)(0.2329,0.3081)
\polypmIIId{104}(0.2329,0.308)(0.2442,0.308)(0.2442,0.31)(0.2329,0.31)
\polypmIIId{105}(0.2329,0.3099)(0.2442,0.3099)(0.2442,0.3119)(0.2329,0.3119)
\polypmIIId{106}(0.2329,0.3118)(0.2442,0.3118)(0.2442,0.3138)(0.2329,0.3138)
\polypmIIId{107}(0.2329,0.3137)(0.2442,0.3137)(0.2442,0.3158)(0.2329,0.3158)
\polypmIIId{108}(0.2329,0.3157)(0.2442,0.3157)(0.2442,0.3177)(0.2329,0.3177)
\polypmIIId{109}(0.2329,0.3176)(0.2442,0.3176)(0.2442,0.3196)(0.2329,0.3196)
\polypmIIId{110}(0.2329,0.3195)(0.2442,0.3195)(0.2442,0.3215)(0.2329,0.3215)
\polypmIIId{111}(0.2329,0.3214)(0.2442,0.3214)(0.2442,0.3235)(0.2329,0.3235)
\polypmIIId{112}(0.2329,0.3234)(0.2442,0.3234)(0.2442,0.3254)(0.2329,0.3254)
\polypmIIId{113}(0.2329,0.3253)(0.2442,0.3253)(0.2442,0.3273)(0.2329,0.3273)
\polypmIIId{114}(0.2329,0.3272)(0.2442,0.3272)(0.2442,0.3292)(0.2329,0.3292)
\polypmIIId{115}(0.2329,0.3291)(0.2442,0.3291)(0.2442,0.3312)(0.2329,0.3312)
\polypmIIId{116}(0.2329,0.3311)(0.2442,0.3311)(0.2442,0.3331)(0.2329,0.3331)
\polypmIIId{117}(0.2329,0.333)(0.2442,0.333)(0.2442,0.335)(0.2329,0.335)
\polypmIIId{118}(0.2329,0.3349)(0.2442,0.3349)(0.2442,0.3369)(0.2329,0.3369)
\polypmIIId{119}(0.2329,0.3368)(0.2442,0.3368)(0.2442,0.3389)(0.2329,0.3389)
\polypmIIId{120}(0.2329,0.3388)(0.2442,0.3388)(0.2442,0.3408)(0.2329,0.3408)
\polypmIIId{121}(0.2329,0.3407)(0.2442,0.3407)(0.2442,0.3427)(0.2329,0.3427)
\polypmIIId{122}(0.2329,0.3426)(0.2442,0.3426)(0.2442,0.3446)(0.2329,0.3446)
\polypmIIId{123}(0.2329,0.3445)(0.2442,0.3445)(0.2442,0.3466)(0.2329,0.3466)
\polypmIIId{124}(0.2329,0.3465)(0.2442,0.3465)(0.2442,0.3485)(0.2329,0.3485)
\polypmIIId{125}(0.2329,0.3484)(0.2442,0.3484)(0.2442,0.3504)(0.2329,0.3504)
\polypmIIId{126}(0.2329,0.3503)(0.2442,0.3503)(0.2442,0.3523)(0.2329,0.3523)
\polypmIIId{127}(0.2329,0.3522)(0.2442,0.3522)(0.2442,0.3542)(0.2329,0.3542)

\PST@Border(0.2329,0.1078)
(0.2442,0.1078)
(0.2442,0.3542)
(0.2329,0.3542)
(0.2329,0.1078)


\rput[l](0.2502,0.1301){0.997}
\rput[l](0.2502,0.2048){0.998}
\rput[l](0.2502,0.2795){0.999}
\rput[l](0.2502,0.3542){1}

\catcode`@=12
\fi
\endpspicture}
  }
  $\alpha = 1.0$
  \caption{TSLP solution quality on uncorrelated instances.}
  \label{fig:tabusolcomp10}
\end{figure}

Looking at figure~\ref{fig:tabusolcomp10}, which shows the results of the TSLP tests, some impact of
the correlation level on the quality of the obtained solutions can again be seen, as the heatmaps for the instances with some level of correlation
are paler than the ones for the uncorrelated instances. Specially on the instances with weak correlation (the second heatmap), the algorithm seems
to obtain the worst solutions.

Comparing the results in respect to the number of resources, it can be seen that instances with more resources are harder to solve. That influence
can be seen mainly on the second heatmap. Once again, an observation of the quality of the solutions related to the amount of actions on the instance
shows that the TSLP also obtained worse solutions on instances with fewer actions. The influence of the number of years in those tests was too weak to be 
considered significant.

Figure~\ref{fig:greedysolcomp10} presents the results of the GALP heuristic.
Once again, it is possible to see a reduction on the quality of the solutions found when the quantity of resources or level of correlation on
the instances are increased. Besides, increasing the number of actions on the instances also enabled GALP to find better solutions, and
the influence of the number of years was also not significant, as observed on the TSLP tests.

\figpar
\begin{figure}[H]
  \centering
  \resizebox{\columnwidth}{!}{%
    \subfloat[1 resource]{% GNUPLOT: LaTeX picture using PSTRICKS macros
% Define new PST objects, if not already defined
\ifx\PSTloaded\undefined
\def\PSTloaded{t}

\catcode`@=11

\newpsobject{PST@Border}{psline}{linewidth=.0015,linestyle=solid}

\catcode`@=12

\fi
\psset{unit=5.0in,xunit=5.0in,yunit=3.0in}
\pspicture(0.000000,0.000000)(0.31, 0.35)
\ifx\nofigs\undefined
\catcode`@=11

\newrgbcolor{PST@COLOR0}{1 1 1}
\newrgbcolor{PST@COLOR1}{0.992 0.992 0.992}
\newrgbcolor{PST@COLOR2}{0.984 0.984 0.984}
\newrgbcolor{PST@COLOR3}{0.976 0.976 0.976}
\newrgbcolor{PST@COLOR4}{0.968 0.968 0.968}
\newrgbcolor{PST@COLOR5}{0.96 0.96 0.96}
\newrgbcolor{PST@COLOR6}{0.952 0.952 0.952}
\newrgbcolor{PST@COLOR7}{0.944 0.944 0.944}
\newrgbcolor{PST@COLOR8}{0.937 0.937 0.937}
\newrgbcolor{PST@COLOR9}{0.929 0.929 0.929}
\newrgbcolor{PST@COLOR10}{0.921 0.921 0.921}
\newrgbcolor{PST@COLOR11}{0.913 0.913 0.913}
\newrgbcolor{PST@COLOR12}{0.905 0.905 0.905}
\newrgbcolor{PST@COLOR13}{0.897 0.897 0.897}
\newrgbcolor{PST@COLOR14}{0.889 0.889 0.889}
\newrgbcolor{PST@COLOR15}{0.881 0.881 0.881}
\newrgbcolor{PST@COLOR16}{0.874 0.874 0.874}
\newrgbcolor{PST@COLOR17}{0.866 0.866 0.866}
\newrgbcolor{PST@COLOR18}{0.858 0.858 0.858}
\newrgbcolor{PST@COLOR19}{0.85 0.85 0.85}
\newrgbcolor{PST@COLOR20}{0.842 0.842 0.842}
\newrgbcolor{PST@COLOR21}{0.834 0.834 0.834}
\newrgbcolor{PST@COLOR22}{0.826 0.826 0.826}
\newrgbcolor{PST@COLOR23}{0.818 0.818 0.818}
\newrgbcolor{PST@COLOR24}{0.811 0.811 0.811}
\newrgbcolor{PST@COLOR25}{0.803 0.803 0.803}
\newrgbcolor{PST@COLOR26}{0.795 0.795 0.795}
\newrgbcolor{PST@COLOR27}{0.787 0.787 0.787}
\newrgbcolor{PST@COLOR28}{0.779 0.779 0.779}
\newrgbcolor{PST@COLOR29}{0.771 0.771 0.771}
\newrgbcolor{PST@COLOR30}{0.763 0.763 0.763}
\newrgbcolor{PST@COLOR31}{0.755 0.755 0.755}
\newrgbcolor{PST@COLOR32}{0.748 0.748 0.748}
\newrgbcolor{PST@COLOR33}{0.74 0.74 0.74}
\newrgbcolor{PST@COLOR34}{0.732 0.732 0.732}
\newrgbcolor{PST@COLOR35}{0.724 0.724 0.724}
\newrgbcolor{PST@COLOR36}{0.716 0.716 0.716}
\newrgbcolor{PST@COLOR37}{0.708 0.708 0.708}
\newrgbcolor{PST@COLOR38}{0.7 0.7 0.7}
\newrgbcolor{PST@COLOR39}{0.692 0.692 0.692}
\newrgbcolor{PST@COLOR40}{0.685 0.685 0.685}
\newrgbcolor{PST@COLOR41}{0.677 0.677 0.677}
\newrgbcolor{PST@COLOR42}{0.669 0.669 0.669}
\newrgbcolor{PST@COLOR43}{0.661 0.661 0.661}
\newrgbcolor{PST@COLOR44}{0.653 0.653 0.653}
\newrgbcolor{PST@COLOR45}{0.645 0.645 0.645}
\newrgbcolor{PST@COLOR46}{0.637 0.637 0.637}
\newrgbcolor{PST@COLOR47}{0.629 0.629 0.629}
\newrgbcolor{PST@COLOR48}{0.622 0.622 0.622}
\newrgbcolor{PST@COLOR49}{0.614 0.614 0.614}
\newrgbcolor{PST@COLOR50}{0.606 0.606 0.606}
\newrgbcolor{PST@COLOR51}{0.598 0.598 0.598}
\newrgbcolor{PST@COLOR52}{0.59 0.59 0.59}
\newrgbcolor{PST@COLOR53}{0.582 0.582 0.582}
\newrgbcolor{PST@COLOR54}{0.574 0.574 0.574}
\newrgbcolor{PST@COLOR55}{0.566 0.566 0.566}
\newrgbcolor{PST@COLOR56}{0.559 0.559 0.559}
\newrgbcolor{PST@COLOR57}{0.551 0.551 0.551}
\newrgbcolor{PST@COLOR58}{0.543 0.543 0.543}
\newrgbcolor{PST@COLOR59}{0.535 0.535 0.535}
\newrgbcolor{PST@COLOR60}{0.527 0.527 0.527}
\newrgbcolor{PST@COLOR61}{0.519 0.519 0.519}
\newrgbcolor{PST@COLOR62}{0.511 0.511 0.511}
\newrgbcolor{PST@COLOR63}{0.503 0.503 0.503}
\newrgbcolor{PST@COLOR64}{0.496 0.496 0.496}
\newrgbcolor{PST@COLOR65}{0.488 0.488 0.488}
\newrgbcolor{PST@COLOR66}{0.48 0.48 0.48}
\newrgbcolor{PST@COLOR67}{0.472 0.472 0.472}
\newrgbcolor{PST@COLOR68}{0.464 0.464 0.464}
\newrgbcolor{PST@COLOR69}{0.456 0.456 0.456}
\newrgbcolor{PST@COLOR70}{0.448 0.448 0.448}
\newrgbcolor{PST@COLOR71}{0.44 0.44 0.44}
\newrgbcolor{PST@COLOR72}{0.433 0.433 0.433}
\newrgbcolor{PST@COLOR73}{0.425 0.425 0.425}
\newrgbcolor{PST@COLOR74}{0.417 0.417 0.417}
\newrgbcolor{PST@COLOR75}{0.409 0.409 0.409}
\newrgbcolor{PST@COLOR76}{0.401 0.401 0.401}
\newrgbcolor{PST@COLOR77}{0.393 0.393 0.393}
\newrgbcolor{PST@COLOR78}{0.385 0.385 0.385}
\newrgbcolor{PST@COLOR79}{0.377 0.377 0.377}
\newrgbcolor{PST@COLOR80}{0.37 0.37 0.37}
\newrgbcolor{PST@COLOR81}{0.362 0.362 0.362}
\newrgbcolor{PST@COLOR82}{0.354 0.354 0.354}
\newrgbcolor{PST@COLOR83}{0.346 0.346 0.346}
\newrgbcolor{PST@COLOR84}{0.338 0.338 0.338}
\newrgbcolor{PST@COLOR85}{0.33 0.33 0.33}
\newrgbcolor{PST@COLOR86}{0.322 0.322 0.322}
\newrgbcolor{PST@COLOR87}{0.314 0.314 0.314}
\newrgbcolor{PST@COLOR88}{0.307 0.307 0.307}
\newrgbcolor{PST@COLOR89}{0.299 0.299 0.299}
\newrgbcolor{PST@COLOR90}{0.291 0.291 0.291}
\newrgbcolor{PST@COLOR91}{0.283 0.283 0.283}
\newrgbcolor{PST@COLOR92}{0.275 0.275 0.275}
\newrgbcolor{PST@COLOR93}{0.267 0.267 0.267}
\newrgbcolor{PST@COLOR94}{0.259 0.259 0.259}
\newrgbcolor{PST@COLOR95}{0.251 0.251 0.251}
\newrgbcolor{PST@COLOR96}{0.244 0.244 0.244}
\newrgbcolor{PST@COLOR97}{0.236 0.236 0.236}
\newrgbcolor{PST@COLOR98}{0.228 0.228 0.228}
\newrgbcolor{PST@COLOR99}{0.22 0.22 0.22}
\newrgbcolor{PST@COLOR100}{0.212 0.212 0.212}
\newrgbcolor{PST@COLOR101}{0.204 0.204 0.204}
\newrgbcolor{PST@COLOR102}{0.196 0.196 0.196}
\newrgbcolor{PST@COLOR103}{0.188 0.188 0.188}
\newrgbcolor{PST@COLOR104}{0.181 0.181 0.181}
\newrgbcolor{PST@COLOR105}{0.173 0.173 0.173}
\newrgbcolor{PST@COLOR106}{0.165 0.165 0.165}
\newrgbcolor{PST@COLOR107}{0.157 0.157 0.157}
\newrgbcolor{PST@COLOR108}{0.149 0.149 0.149}
\newrgbcolor{PST@COLOR109}{0.141 0.141 0.141}
\newrgbcolor{PST@COLOR110}{0.133 0.133 0.133}
\newrgbcolor{PST@COLOR111}{0.125 0.125 0.125}
\newrgbcolor{PST@COLOR112}{0.118 0.118 0.118}
\newrgbcolor{PST@COLOR113}{0.11 0.11 0.11}
\newrgbcolor{PST@COLOR114}{0.102 0.102 0.102}
\newrgbcolor{PST@COLOR115}{0.094 0.094 0.094}
\newrgbcolor{PST@COLOR116}{0.086 0.086 0.086}
\newrgbcolor{PST@COLOR117}{0.078 0.078 0.078}
\newrgbcolor{PST@COLOR118}{0.07 0.07 0.07}
\newrgbcolor{PST@COLOR119}{0.062 0.062 0.062}
\newrgbcolor{PST@COLOR120}{0.055 0.055 0.055}
\newrgbcolor{PST@COLOR121}{0.047 0.047 0.047}
\newrgbcolor{PST@COLOR122}{0.039 0.039 0.039}
\newrgbcolor{PST@COLOR123}{0.031 0.031 0.031}
\newrgbcolor{PST@COLOR124}{0.023 0.023 0.023}
\newrgbcolor{PST@COLOR125}{0.015 0.015 0.015}
\newrgbcolor{PST@COLOR126}{0.007 0.007 0.007}
\newrgbcolor{PST@COLOR127}{0 0 0}


\def\polypmIIId#1{\pspolygon[linestyle=none,fillstyle=solid,fillcolor=PST@COLOR#1]}

\polypmIIId{63}(0.1432,0.19)(0.0864,0.19)(0.0864,0.1078)(0.1432,0.1078)
\polypmIIId{97}(0.1432,0.272)(0.0864,0.272)(0.0864,0.19)(0.1432,0.19)
\polypmIIId{114}(0.1432,0.3542)(0.0864,0.3542)(0.0864,0.272)(0.1432,0.272)

\polypmIIId{68}(0.2,0.19)(0.1432,0.19)(0.1432,0.1078)(0.2,0.1078)
\polypmIIId{97}(0.2,0.272)(0.1432,0.272)(0.1432,0.19)(0.2,0.19)
\polypmIIId{115}(0.2,0.3542)(0.1432,0.3542)(0.1432,0.272)(0.2,0.272)

\polypmIIId{69}(0.2568,0.19)(0.2,0.19)(0.2,0.1078)(0.2568,0.1078)
\polypmIIId{99}(0.2568,0.272)(0.2,0.272)(0.2,0.19)(0.2568,0.19)
\polypmIIId{116}(0.2568,0.3542)(0.2,0.3542)(0.2,0.272)(0.2568,0.272)

\polypmIIId{65}(0.3136,0.19)(0.2568,0.19)(0.2568,0.1078)(0.3136,0.1078)
\polypmIIId{102}(0.3136,0.272)(0.2568,0.272)(0.2568,0.19)(0.3136,0.19)
\polypmIIId{117}(0.3136,0.3542)(0.2568,0.3542)(0.2568,0.272)(0.3136,0.272)

\rput(0.1148,0.07){3}
\rput(0.1716,0.07){4}
\rput(0.2284,0.07){5}
\rput(0.2852,0.07){6}
\rput(0.2000,0.0070){years}

\rput[r](0.0806,0.1489){25}
\rput[r](0.0806,0.2310){50}
\rput[r](0.0806,0.3131){100}
\rput{L}(0.0096,0.2310){actions}

\PST@Border(0.0864,0.3542)
(0.0864,0.1078)
(0.3136,0.1078)
(0.3136,0.3542)
(0.0864,0.3542)

\catcode`@=12
\fi
\endpspicture} 
    \subfloat[2 resources]{% GNUPLOT: LaTeX picture using PSTRICKS macros
% Define new PST objects, if not already defined
\ifx\PSTloaded\undefined
\def\PSTloaded{t}

\catcode`@=11

\newpsobject{PST@Border}{psline}{linewidth=.0015,linestyle=solid}

\catcode`@=12

\fi
\psset{unit=5.0in,xunit=5.0in,yunit=3.0in}
\pspicture(0.000000,0.000000)(0.225000,0.35)
\ifx\nofigs\undefined
\catcode`@=11

\newrgbcolor{PST@COLOR0}{1 1 1}
\newrgbcolor{PST@COLOR1}{0.992 0.992 0.992}
\newrgbcolor{PST@COLOR2}{0.984 0.984 0.984}
\newrgbcolor{PST@COLOR3}{0.976 0.976 0.976}
\newrgbcolor{PST@COLOR4}{0.968 0.968 0.968}
\newrgbcolor{PST@COLOR5}{0.96 0.96 0.96}
\newrgbcolor{PST@COLOR6}{0.952 0.952 0.952}
\newrgbcolor{PST@COLOR7}{0.944 0.944 0.944}
\newrgbcolor{PST@COLOR8}{0.937 0.937 0.937}
\newrgbcolor{PST@COLOR9}{0.929 0.929 0.929}
\newrgbcolor{PST@COLOR10}{0.921 0.921 0.921}
\newrgbcolor{PST@COLOR11}{0.913 0.913 0.913}
\newrgbcolor{PST@COLOR12}{0.905 0.905 0.905}
\newrgbcolor{PST@COLOR13}{0.897 0.897 0.897}
\newrgbcolor{PST@COLOR14}{0.889 0.889 0.889}
\newrgbcolor{PST@COLOR15}{0.881 0.881 0.881}
\newrgbcolor{PST@COLOR16}{0.874 0.874 0.874}
\newrgbcolor{PST@COLOR17}{0.866 0.866 0.866}
\newrgbcolor{PST@COLOR18}{0.858 0.858 0.858}
\newrgbcolor{PST@COLOR19}{0.85 0.85 0.85}
\newrgbcolor{PST@COLOR20}{0.842 0.842 0.842}
\newrgbcolor{PST@COLOR21}{0.834 0.834 0.834}
\newrgbcolor{PST@COLOR22}{0.826 0.826 0.826}
\newrgbcolor{PST@COLOR23}{0.818 0.818 0.818}
\newrgbcolor{PST@COLOR24}{0.811 0.811 0.811}
\newrgbcolor{PST@COLOR25}{0.803 0.803 0.803}
\newrgbcolor{PST@COLOR26}{0.795 0.795 0.795}
\newrgbcolor{PST@COLOR27}{0.787 0.787 0.787}
\newrgbcolor{PST@COLOR28}{0.779 0.779 0.779}
\newrgbcolor{PST@COLOR29}{0.771 0.771 0.771}
\newrgbcolor{PST@COLOR30}{0.763 0.763 0.763}
\newrgbcolor{PST@COLOR31}{0.755 0.755 0.755}
\newrgbcolor{PST@COLOR32}{0.748 0.748 0.748}
\newrgbcolor{PST@COLOR33}{0.74 0.74 0.74}
\newrgbcolor{PST@COLOR34}{0.732 0.732 0.732}
\newrgbcolor{PST@COLOR35}{0.724 0.724 0.724}
\newrgbcolor{PST@COLOR36}{0.716 0.716 0.716}
\newrgbcolor{PST@COLOR37}{0.708 0.708 0.708}
\newrgbcolor{PST@COLOR38}{0.7 0.7 0.7}
\newrgbcolor{PST@COLOR39}{0.692 0.692 0.692}
\newrgbcolor{PST@COLOR40}{0.685 0.685 0.685}
\newrgbcolor{PST@COLOR41}{0.677 0.677 0.677}
\newrgbcolor{PST@COLOR42}{0.669 0.669 0.669}
\newrgbcolor{PST@COLOR43}{0.661 0.661 0.661}
\newrgbcolor{PST@COLOR44}{0.653 0.653 0.653}
\newrgbcolor{PST@COLOR45}{0.645 0.645 0.645}
\newrgbcolor{PST@COLOR46}{0.637 0.637 0.637}
\newrgbcolor{PST@COLOR47}{0.629 0.629 0.629}
\newrgbcolor{PST@COLOR48}{0.622 0.622 0.622}
\newrgbcolor{PST@COLOR49}{0.614 0.614 0.614}
\newrgbcolor{PST@COLOR50}{0.606 0.606 0.606}
\newrgbcolor{PST@COLOR51}{0.598 0.598 0.598}
\newrgbcolor{PST@COLOR52}{0.59 0.59 0.59}
\newrgbcolor{PST@COLOR53}{0.582 0.582 0.582}
\newrgbcolor{PST@COLOR54}{0.574 0.574 0.574}
\newrgbcolor{PST@COLOR55}{0.566 0.566 0.566}
\newrgbcolor{PST@COLOR56}{0.559 0.559 0.559}
\newrgbcolor{PST@COLOR57}{0.551 0.551 0.551}
\newrgbcolor{PST@COLOR58}{0.543 0.543 0.543}
\newrgbcolor{PST@COLOR59}{0.535 0.535 0.535}
\newrgbcolor{PST@COLOR60}{0.527 0.527 0.527}
\newrgbcolor{PST@COLOR61}{0.519 0.519 0.519}
\newrgbcolor{PST@COLOR62}{0.511 0.511 0.511}
\newrgbcolor{PST@COLOR63}{0.503 0.503 0.503}
\newrgbcolor{PST@COLOR64}{0.496 0.496 0.496}
\newrgbcolor{PST@COLOR65}{0.488 0.488 0.488}
\newrgbcolor{PST@COLOR66}{0.48 0.48 0.48}
\newrgbcolor{PST@COLOR67}{0.472 0.472 0.472}
\newrgbcolor{PST@COLOR68}{0.464 0.464 0.464}
\newrgbcolor{PST@COLOR69}{0.456 0.456 0.456}
\newrgbcolor{PST@COLOR70}{0.448 0.448 0.448}
\newrgbcolor{PST@COLOR71}{0.44 0.44 0.44}
\newrgbcolor{PST@COLOR72}{0.433 0.433 0.433}
\newrgbcolor{PST@COLOR73}{0.425 0.425 0.425}
\newrgbcolor{PST@COLOR74}{0.417 0.417 0.417}
\newrgbcolor{PST@COLOR75}{0.409 0.409 0.409}
\newrgbcolor{PST@COLOR76}{0.401 0.401 0.401}
\newrgbcolor{PST@COLOR77}{0.393 0.393 0.393}
\newrgbcolor{PST@COLOR78}{0.385 0.385 0.385}
\newrgbcolor{PST@COLOR79}{0.377 0.377 0.377}
\newrgbcolor{PST@COLOR80}{0.37 0.37 0.37}
\newrgbcolor{PST@COLOR81}{0.362 0.362 0.362}
\newrgbcolor{PST@COLOR82}{0.354 0.354 0.354}
\newrgbcolor{PST@COLOR83}{0.346 0.346 0.346}
\newrgbcolor{PST@COLOR84}{0.338 0.338 0.338}
\newrgbcolor{PST@COLOR85}{0.33 0.33 0.33}
\newrgbcolor{PST@COLOR86}{0.322 0.322 0.322}
\newrgbcolor{PST@COLOR87}{0.314 0.314 0.314}
\newrgbcolor{PST@COLOR88}{0.307 0.307 0.307}
\newrgbcolor{PST@COLOR89}{0.299 0.299 0.299}
\newrgbcolor{PST@COLOR90}{0.291 0.291 0.291}
\newrgbcolor{PST@COLOR91}{0.283 0.283 0.283}
\newrgbcolor{PST@COLOR92}{0.275 0.275 0.275}
\newrgbcolor{PST@COLOR93}{0.267 0.267 0.267}
\newrgbcolor{PST@COLOR94}{0.259 0.259 0.259}
\newrgbcolor{PST@COLOR95}{0.251 0.251 0.251}
\newrgbcolor{PST@COLOR96}{0.244 0.244 0.244}
\newrgbcolor{PST@COLOR97}{0.236 0.236 0.236}
\newrgbcolor{PST@COLOR98}{0.228 0.228 0.228}
\newrgbcolor{PST@COLOR99}{0.22 0.22 0.22}
\newrgbcolor{PST@COLOR100}{0.212 0.212 0.212}
\newrgbcolor{PST@COLOR101}{0.204 0.204 0.204}
\newrgbcolor{PST@COLOR102}{0.196 0.196 0.196}
\newrgbcolor{PST@COLOR103}{0.188 0.188 0.188}
\newrgbcolor{PST@COLOR104}{0.181 0.181 0.181}
\newrgbcolor{PST@COLOR105}{0.173 0.173 0.173}
\newrgbcolor{PST@COLOR106}{0.165 0.165 0.165}
\newrgbcolor{PST@COLOR107}{0.157 0.157 0.157}
\newrgbcolor{PST@COLOR108}{0.149 0.149 0.149}
\newrgbcolor{PST@COLOR109}{0.141 0.141 0.141}
\newrgbcolor{PST@COLOR110}{0.133 0.133 0.133}
\newrgbcolor{PST@COLOR111}{0.125 0.125 0.125}
\newrgbcolor{PST@COLOR112}{0.118 0.118 0.118}
\newrgbcolor{PST@COLOR113}{0.11 0.11 0.11}
\newrgbcolor{PST@COLOR114}{0.102 0.102 0.102}
\newrgbcolor{PST@COLOR115}{0.094 0.094 0.094}
\newrgbcolor{PST@COLOR116}{0.086 0.086 0.086}
\newrgbcolor{PST@COLOR117}{0.078 0.078 0.078}
\newrgbcolor{PST@COLOR118}{0.07 0.07 0.07}
\newrgbcolor{PST@COLOR119}{0.062 0.062 0.062}
\newrgbcolor{PST@COLOR120}{0.055 0.055 0.055}
\newrgbcolor{PST@COLOR121}{0.047 0.047 0.047}
\newrgbcolor{PST@COLOR122}{0.039 0.039 0.039}
\newrgbcolor{PST@COLOR123}{0.031 0.031 0.031}
\newrgbcolor{PST@COLOR124}{0.023 0.023 0.023}
\newrgbcolor{PST@COLOR125}{0.015 0.015 0.015}
\newrgbcolor{PST@COLOR126}{0.007 0.007 0.007}
\newrgbcolor{PST@COLOR127}{0 0 0}

\def\polypmIIId#1{\pspolygon[linestyle=none,fillstyle=solid,fillcolor=PST@COLOR#1]}

\polypmIIId{48} (0.0568,0.19)  (0.0,0.19)  (0.0,0.1078)(0.0568,0.1078)
\polypmIIId{86}  (0.0568,0.272) (0.0,0.272) (0.0,0.19)  (0.0568,0.19)
\polypmIIId{111}  (0.0568,0.3542)(0.0,0.3542)(0.0,0.272) (0.0568,0.272)

\polypmIIId{53} (0.1136,   0.19)  (0.0568,0.19)  (0.0568,0.1078)(0.1136,0.1078)
\polypmIIId{93}  (0.1136,   0.272) (0.0568,0.272) (0.0568,0.19)  (0.1136,0.19)
\polypmIIId{111}  (0.1136,   0.3542)(0.0568,0.3542)(0.0568,0.272) (0.1136,0.272)

\polypmIIId{52}(0.1704,0.19)  (0.1136,   0.19)  (0.1136,   0.1078)(0.1704,0.1078)
\polypmIIId{92} (0.1704,0.272) (0.1136,   0.272) (0.1136,   0.19)  (0.1704,0.19)
\polypmIIId{112}  (0.1704,0.3542)(0.1136,   0.3542)(0.1136,   0.272) (0.1704,0.272)

\polypmIIId{50}(0.2272,0.19)  (0.1704,0.19)  (0.1704,0.1078)(0.2272,0.1078)
\polypmIIId{95}  (0.2272,0.272) (0.1704,0.272) (0.1704,0.19)  (0.2272,0.19)
\polypmIIId{112}  (0.2272,0.3542)(0.1704,0.3542)(0.1704,0.272) (0.2272,0.272)

\rput(0.0284,0.07){3}
\rput(0.0852,0.07){4}
\rput(0.1420,0.07){5}
\rput(0.1988,0.07){6}
\rput(0.1136,0.0070){years}


\PST@Border(0.0,0.3542)
(0.0,0.1078)
(0.2272,0.1078)
(0.2272,0.3542)
(0.0,0.3542)

\catcode`@=12
\fi
\endpspicture}
    \subfloat[4 resources]{% GNUPLOT: LaTeX picture using PSTRICKS macros
% Define new PST objects, if not already defined
\ifx\PSTloaded\undefined
\def\PSTloaded{t}

\catcode`@=11

\newpsobject{PST@Border}{psline}{linewidth=.0015,linestyle=solid}

\catcode`@=12

\fi
\psset{unit=5.0in,xunit=5.0in,yunit=3.0in}
\pspicture(0.000000,0.000000)(0.3136,0.35)
\ifx\nofigs\undefined
\catcode`@=11

\newrgbcolor{PST@COLOR0}{1 1 1}
\newrgbcolor{PST@COLOR1}{0.992 0.992 0.992}
\newrgbcolor{PST@COLOR2}{0.984 0.984 0.984}
\newrgbcolor{PST@COLOR3}{0.976 0.976 0.976}
\newrgbcolor{PST@COLOR4}{0.968 0.968 0.968}
\newrgbcolor{PST@COLOR5}{0.96 0.96 0.96}
\newrgbcolor{PST@COLOR6}{0.952 0.952 0.952}
\newrgbcolor{PST@COLOR7}{0.944 0.944 0.944}
\newrgbcolor{PST@COLOR8}{0.937 0.937 0.937}
\newrgbcolor{PST@COLOR9}{0.929 0.929 0.929}
\newrgbcolor{PST@COLOR10}{0.921 0.921 0.921}
\newrgbcolor{PST@COLOR11}{0.913 0.913 0.913}
\newrgbcolor{PST@COLOR12}{0.905 0.905 0.905}
\newrgbcolor{PST@COLOR13}{0.897 0.897 0.897}
\newrgbcolor{PST@COLOR14}{0.889 0.889 0.889}
\newrgbcolor{PST@COLOR15}{0.881 0.881 0.881}
\newrgbcolor{PST@COLOR16}{0.874 0.874 0.874}
\newrgbcolor{PST@COLOR17}{0.866 0.866 0.866}
\newrgbcolor{PST@COLOR18}{0.858 0.858 0.858}
\newrgbcolor{PST@COLOR19}{0.85 0.85 0.85}
\newrgbcolor{PST@COLOR20}{0.842 0.842 0.842}
\newrgbcolor{PST@COLOR21}{0.834 0.834 0.834}
\newrgbcolor{PST@COLOR22}{0.826 0.826 0.826}
\newrgbcolor{PST@COLOR23}{0.818 0.818 0.818}
\newrgbcolor{PST@COLOR24}{0.811 0.811 0.811}
\newrgbcolor{PST@COLOR25}{0.803 0.803 0.803}
\newrgbcolor{PST@COLOR26}{0.795 0.795 0.795}
\newrgbcolor{PST@COLOR27}{0.787 0.787 0.787}
\newrgbcolor{PST@COLOR28}{0.779 0.779 0.779}
\newrgbcolor{PST@COLOR29}{0.771 0.771 0.771}
\newrgbcolor{PST@COLOR30}{0.763 0.763 0.763}
\newrgbcolor{PST@COLOR31}{0.755 0.755 0.755}
\newrgbcolor{PST@COLOR32}{0.748 0.748 0.748}
\newrgbcolor{PST@COLOR33}{0.74 0.74 0.74}
\newrgbcolor{PST@COLOR34}{0.732 0.732 0.732}
\newrgbcolor{PST@COLOR35}{0.724 0.724 0.724}
\newrgbcolor{PST@COLOR36}{0.716 0.716 0.716}
\newrgbcolor{PST@COLOR37}{0.708 0.708 0.708}
\newrgbcolor{PST@COLOR38}{0.7 0.7 0.7}
\newrgbcolor{PST@COLOR39}{0.692 0.692 0.692}
\newrgbcolor{PST@COLOR40}{0.685 0.685 0.685}
\newrgbcolor{PST@COLOR41}{0.677 0.677 0.677}
\newrgbcolor{PST@COLOR42}{0.669 0.669 0.669}
\newrgbcolor{PST@COLOR43}{0.661 0.661 0.661}
\newrgbcolor{PST@COLOR44}{0.653 0.653 0.653}
\newrgbcolor{PST@COLOR45}{0.645 0.645 0.645}
\newrgbcolor{PST@COLOR46}{0.637 0.637 0.637}
\newrgbcolor{PST@COLOR47}{0.629 0.629 0.629}
\newrgbcolor{PST@COLOR48}{0.622 0.622 0.622}
\newrgbcolor{PST@COLOR49}{0.614 0.614 0.614}
\newrgbcolor{PST@COLOR50}{0.606 0.606 0.606}
\newrgbcolor{PST@COLOR51}{0.598 0.598 0.598}
\newrgbcolor{PST@COLOR52}{0.59 0.59 0.59}
\newrgbcolor{PST@COLOR53}{0.582 0.582 0.582}
\newrgbcolor{PST@COLOR54}{0.574 0.574 0.574}
\newrgbcolor{PST@COLOR55}{0.566 0.566 0.566}
\newrgbcolor{PST@COLOR56}{0.559 0.559 0.559}
\newrgbcolor{PST@COLOR57}{0.551 0.551 0.551}
\newrgbcolor{PST@COLOR58}{0.543 0.543 0.543}
\newrgbcolor{PST@COLOR59}{0.535 0.535 0.535}
\newrgbcolor{PST@COLOR60}{0.527 0.527 0.527}
\newrgbcolor{PST@COLOR61}{0.519 0.519 0.519}
\newrgbcolor{PST@COLOR62}{0.511 0.511 0.511}
\newrgbcolor{PST@COLOR63}{0.503 0.503 0.503}
\newrgbcolor{PST@COLOR64}{0.496 0.496 0.496}
\newrgbcolor{PST@COLOR65}{0.488 0.488 0.488}
\newrgbcolor{PST@COLOR66}{0.48 0.48 0.48}
\newrgbcolor{PST@COLOR67}{0.472 0.472 0.472}
\newrgbcolor{PST@COLOR68}{0.464 0.464 0.464}
\newrgbcolor{PST@COLOR69}{0.456 0.456 0.456}
\newrgbcolor{PST@COLOR70}{0.448 0.448 0.448}
\newrgbcolor{PST@COLOR71}{0.44 0.44 0.44}
\newrgbcolor{PST@COLOR72}{0.433 0.433 0.433}
\newrgbcolor{PST@COLOR73}{0.425 0.425 0.425}
\newrgbcolor{PST@COLOR74}{0.417 0.417 0.417}
\newrgbcolor{PST@COLOR75}{0.409 0.409 0.409}
\newrgbcolor{PST@COLOR76}{0.401 0.401 0.401}
\newrgbcolor{PST@COLOR77}{0.393 0.393 0.393}
\newrgbcolor{PST@COLOR78}{0.385 0.385 0.385}
\newrgbcolor{PST@COLOR79}{0.377 0.377 0.377}
\newrgbcolor{PST@COLOR80}{0.37 0.37 0.37}
\newrgbcolor{PST@COLOR81}{0.362 0.362 0.362}
\newrgbcolor{PST@COLOR82}{0.354 0.354 0.354}
\newrgbcolor{PST@COLOR83}{0.346 0.346 0.346}
\newrgbcolor{PST@COLOR84}{0.338 0.338 0.338}
\newrgbcolor{PST@COLOR85}{0.33 0.33 0.33}
\newrgbcolor{PST@COLOR86}{0.322 0.322 0.322}
\newrgbcolor{PST@COLOR87}{0.314 0.314 0.314}
\newrgbcolor{PST@COLOR88}{0.307 0.307 0.307}
\newrgbcolor{PST@COLOR89}{0.299 0.299 0.299}
\newrgbcolor{PST@COLOR90}{0.291 0.291 0.291}
\newrgbcolor{PST@COLOR91}{0.283 0.283 0.283}
\newrgbcolor{PST@COLOR92}{0.275 0.275 0.275}
\newrgbcolor{PST@COLOR93}{0.267 0.267 0.267}
\newrgbcolor{PST@COLOR94}{0.259 0.259 0.259}
\newrgbcolor{PST@COLOR95}{0.251 0.251 0.251}
\newrgbcolor{PST@COLOR96}{0.244 0.244 0.244}
\newrgbcolor{PST@COLOR97}{0.236 0.236 0.236}
\newrgbcolor{PST@COLOR98}{0.228 0.228 0.228}
\newrgbcolor{PST@COLOR99}{0.22 0.22 0.22}
\newrgbcolor{PST@COLOR100}{0.212 0.212 0.212}
\newrgbcolor{PST@COLOR101}{0.204 0.204 0.204}
\newrgbcolor{PST@COLOR102}{0.196 0.196 0.196}
\newrgbcolor{PST@COLOR103}{0.188 0.188 0.188}
\newrgbcolor{PST@COLOR104}{0.181 0.181 0.181}
\newrgbcolor{PST@COLOR105}{0.173 0.173 0.173}
\newrgbcolor{PST@COLOR106}{0.165 0.165 0.165}
\newrgbcolor{PST@COLOR107}{0.157 0.157 0.157}
\newrgbcolor{PST@COLOR108}{0.149 0.149 0.149}
\newrgbcolor{PST@COLOR109}{0.141 0.141 0.141}
\newrgbcolor{PST@COLOR110}{0.133 0.133 0.133}
\newrgbcolor{PST@COLOR111}{0.125 0.125 0.125}
\newrgbcolor{PST@COLOR112}{0.118 0.118 0.118}
\newrgbcolor{PST@COLOR113}{0.11 0.11 0.11}
\newrgbcolor{PST@COLOR114}{0.102 0.102 0.102}
\newrgbcolor{PST@COLOR115}{0.094 0.094 0.094}
\newrgbcolor{PST@COLOR116}{0.086 0.086 0.086}
\newrgbcolor{PST@COLOR117}{0.078 0.078 0.078}
\newrgbcolor{PST@COLOR118}{0.07 0.07 0.07}
\newrgbcolor{PST@COLOR119}{0.062 0.062 0.062}
\newrgbcolor{PST@COLOR120}{0.055 0.055 0.055}
\newrgbcolor{PST@COLOR121}{0.047 0.047 0.047}
\newrgbcolor{PST@COLOR122}{0.039 0.039 0.039}
\newrgbcolor{PST@COLOR123}{0.031 0.031 0.031}
\newrgbcolor{PST@COLOR124}{0.023 0.023 0.023}
\newrgbcolor{PST@COLOR125}{0.015 0.015 0.015}
\newrgbcolor{PST@COLOR126}{0.007 0.007 0.007}
\newrgbcolor{PST@COLOR127}{0 0 0}

\def\polypmIIId#1{\pspolygon[linestyle=none,fillstyle=solid,fillcolor=PST@COLOR#1]}

\polypmIIId{2} (0.0568,0.19)  (0.0,0.19)  (0.0,0.1078)(0.0568,0.1078)
\polypmIIId{70}  (0.0568,0.272) (0.0,0.272) (0.0,0.19)  (0.0568,0.19)
\polypmIIId{100}  (0.0568,0.3542)(0.0,0.3542)(0.0,0.272) (0.0568,0.272)

\polypmIIId{7} (0.1136,   0.19)  (0.0568,0.19)  (0.0568,0.1078)(0.1136,0.1078)
\polypmIIId{68}  (0.1136,   0.272) (0.0568,0.272) (0.0568,0.19)  (0.1136,0.19)
\polypmIIId{101}  (0.1136,   0.3542)(0.0568,0.3542)(0.0568,0.272) (0.1136,0.272)

\polypmIIId{9}(0.1704,0.19)  (0.1136,   0.19)  (0.1136,   0.1078)(0.1704,0.1078)
\polypmIIId{71} (0.1704,0.272) (0.1136,   0.272) (0.1136,   0.19)  (0.1704,0.19)
\polypmIIId{101}  (0.1704,0.3542)(0.1136,   0.3542)(0.1136,   0.272) (0.1704,0.272)

\polypmIIId{9}(0.2272,0.19)  (0.1704,0.19)  (0.1704,0.1078)(0.2272,0.1078)
\polypmIIId{73}  (0.2272,0.272) (0.1704,0.272) (0.1704,0.19)  (0.2272,0.19)
\polypmIIId{101}  (0.2272,0.3542)(0.1704,0.3542)(0.1704,0.272) (0.2272,0.272)

\rput(0.0284,0.07){3}
\rput(0.0852,0.07){4}
\rput(0.1420,0.07){5}
\rput(0.1988,0.07){6}
\rput(0.1136,0.0070){years}

\PST@Border(0.0,0.3542)
(0.0,0.1078)
(0.2272,0.1078)
(0.2272,0.3542)
(0.0,0.3542)

\polypmIIId{0}(0.2329,0.1078)(0.2442,0.1078)(0.2442,0.1098)(0.2329,0.1098)
\polypmIIId{1}(0.2329,0.1097)(0.2442,0.1097)(0.2442,0.1117)(0.2329,0.1117)
\polypmIIId{2}(0.2329,0.1116)(0.2442,0.1116)(0.2442,0.1136)(0.2329,0.1136)
\polypmIIId{3}(0.2329,0.1135)(0.2442,0.1135)(0.2442,0.1156)(0.2329,0.1156)
\polypmIIId{4}(0.2329,0.1155)(0.2442,0.1155)(0.2442,0.1175)(0.2329,0.1175)
\polypmIIId{5}(0.2329,0.1174)(0.2442,0.1174)(0.2442,0.1194)(0.2329,0.1194)
\polypmIIId{6}(0.2329,0.1193)(0.2442,0.1193)(0.2442,0.1213)(0.2329,0.1213)
\polypmIIId{7}(0.2329,0.1212)(0.2442,0.1212)(0.2442,0.1233)(0.2329,0.1233)
\polypmIIId{8}(0.2329,0.1232)(0.2442,0.1232)(0.2442,0.1252)(0.2329,0.1252)
\polypmIIId{9}(0.2329,0.1251)(0.2442,0.1251)(0.2442,0.1271)(0.2329,0.1271)
\polypmIIId{10}(0.2329,0.127)(0.2442,0.127)(0.2442,0.129)(0.2329,0.129)
\polypmIIId{11}(0.2329,0.1289)(0.2442,0.1289)(0.2442,0.131)(0.2329,0.131)
\polypmIIId{12}(0.2329,0.1309)(0.2442,0.1309)(0.2442,0.1329)(0.2329,0.1329)
\polypmIIId{13}(0.2329,0.1328)(0.2442,0.1328)(0.2442,0.1348)(0.2329,0.1348)
\polypmIIId{14}(0.2329,0.1347)(0.2442,0.1347)(0.2442,0.1367)(0.2329,0.1367)
\polypmIIId{15}(0.2329,0.1366)(0.2442,0.1366)(0.2442,0.1387)(0.2329,0.1387)
\polypmIIId{16}(0.2329,0.1386)(0.2442,0.1386)(0.2442,0.1406)(0.2329,0.1406)
\polypmIIId{17}(0.2329,0.1405)(0.2442,0.1405)(0.2442,0.1425)(0.2329,0.1425)
\polypmIIId{18}(0.2329,0.1424)(0.2442,0.1424)(0.2442,0.1444)(0.2329,0.1444)
\polypmIIId{19}(0.2329,0.1443)(0.2442,0.1443)(0.2442,0.1464)(0.2329,0.1464)
\polypmIIId{20}(0.2329,0.1463)(0.2442,0.1463)(0.2442,0.1483)(0.2329,0.1483)
\polypmIIId{21}(0.2329,0.1482)(0.2442,0.1482)(0.2442,0.1502)(0.2329,0.1502)
\polypmIIId{22}(0.2329,0.1501)(0.2442,0.1501)(0.2442,0.1521)(0.2329,0.1521)
\polypmIIId{23}(0.2329,0.152)(0.2442,0.152)(0.2442,0.1541)(0.2329,0.1541)
\polypmIIId{24}(0.2329,0.154)(0.2442,0.154)(0.2442,0.156)(0.2329,0.156)
\polypmIIId{25}(0.2329,0.1559)(0.2442,0.1559)(0.2442,0.1579)(0.2329,0.1579)
\polypmIIId{26}(0.2329,0.1578)(0.2442,0.1578)(0.2442,0.1598)(0.2329,0.1598)
\polypmIIId{27}(0.2329,0.1597)(0.2442,0.1597)(0.2442,0.1618)(0.2329,0.1618)
\polypmIIId{28}(0.2329,0.1617)(0.2442,0.1617)(0.2442,0.1637)(0.2329,0.1637)
\polypmIIId{29}(0.2329,0.1636)(0.2442,0.1636)(0.2442,0.1656)(0.2329,0.1656)
\polypmIIId{30}(0.2329,0.1655)(0.2442,0.1655)(0.2442,0.1675)(0.2329,0.1675)
\polypmIIId{31}(0.2329,0.1674)(0.2442,0.1674)(0.2442,0.1695)(0.2329,0.1695)
\polypmIIId{32}(0.2329,0.1694)(0.2442,0.1694)(0.2442,0.1714)(0.2329,0.1714)
\polypmIIId{33}(0.2329,0.1713)(0.2442,0.1713)(0.2442,0.1733)(0.2329,0.1733)
\polypmIIId{34}(0.2329,0.1732)(0.2442,0.1732)(0.2442,0.1752)(0.2329,0.1752)
\polypmIIId{35}(0.2329,0.1751)(0.2442,0.1751)(0.2442,0.1772)(0.2329,0.1772)
\polypmIIId{36}(0.2329,0.1771)(0.2442,0.1771)(0.2442,0.1791)(0.2329,0.1791)
\polypmIIId{37}(0.2329,0.179)(0.2442,0.179)(0.2442,0.181)(0.2329,0.181)
\polypmIIId{38}(0.2329,0.1809)(0.2442,0.1809)(0.2442,0.1829)(0.2329,0.1829)
\polypmIIId{39}(0.2329,0.1828)(0.2442,0.1828)(0.2442,0.1849)(0.2329,0.1849)
\polypmIIId{40}(0.2329,0.1848)(0.2442,0.1848)(0.2442,0.1868)(0.2329,0.1868)
\polypmIIId{41}(0.2329,0.1867)(0.2442,0.1867)(0.2442,0.1887)(0.2329,0.1887)
\polypmIIId{42}(0.2329,0.1886)(0.2442,0.1886)(0.2442,0.1906)(0.2329,0.1906)
\polypmIIId{43}(0.2329,0.1905)(0.2442,0.1905)(0.2442,0.1926)(0.2329,0.1926)
\polypmIIId{44}(0.2329,0.1925)(0.2442,0.1925)(0.2442,0.1945)(0.2329,0.1945)
\polypmIIId{45}(0.2329,0.1944)(0.2442,0.1944)(0.2442,0.1964)(0.2329,0.1964)
\polypmIIId{46}(0.2329,0.1963)(0.2442,0.1963)(0.2442,0.1983)(0.2329,0.1983)
\polypmIIId{47}(0.2329,0.1982)(0.2442,0.1982)(0.2442,0.2003)(0.2329,0.2003)
\polypmIIId{48}(0.2329,0.2002)(0.2442,0.2002)(0.2442,0.2022)(0.2329,0.2022)
\polypmIIId{49}(0.2329,0.2021)(0.2442,0.2021)(0.2442,0.2041)(0.2329,0.2041)
\polypmIIId{50}(0.2329,0.204)(0.2442,0.204)(0.2442,0.206)(0.2329,0.206)
\polypmIIId{51}(0.2329,0.2059)(0.2442,0.2059)(0.2442,0.208)(0.2329,0.208)
\polypmIIId{52}(0.2329,0.2079)(0.2442,0.2079)(0.2442,0.2099)(0.2329,0.2099)
\polypmIIId{53}(0.2329,0.2098)(0.2442,0.2098)(0.2442,0.2118)(0.2329,0.2118)
\polypmIIId{54}(0.2329,0.2117)(0.2442,0.2117)(0.2442,0.2137)(0.2329,0.2137)
\polypmIIId{55}(0.2329,0.2136)(0.2442,0.2136)(0.2442,0.2157)(0.2329,0.2157)
\polypmIIId{56}(0.2329,0.2156)(0.2442,0.2156)(0.2442,0.2176)(0.2329,0.2176)
\polypmIIId{57}(0.2329,0.2175)(0.2442,0.2175)(0.2442,0.2195)(0.2329,0.2195)
\polypmIIId{58}(0.2329,0.2194)(0.2442,0.2194)(0.2442,0.2214)(0.2329,0.2214)
\polypmIIId{59}(0.2329,0.2213)(0.2442,0.2213)(0.2442,0.2234)(0.2329,0.2234)
\polypmIIId{60}(0.2329,0.2233)(0.2442,0.2233)(0.2442,0.2253)(0.2329,0.2253)
\polypmIIId{61}(0.2329,0.2252)(0.2442,0.2252)(0.2442,0.2272)(0.2329,0.2272)
\polypmIIId{62}(0.2329,0.2271)(0.2442,0.2271)(0.2442,0.2291)(0.2329,0.2291)
\polypmIIId{63}(0.2329,0.229)(0.2442,0.229)(0.2442,0.2311)(0.2329,0.2311)
\polypmIIId{64}(0.2329,0.231)(0.2442,0.231)(0.2442,0.233)(0.2329,0.233)
\polypmIIId{65}(0.2329,0.2329)(0.2442,0.2329)(0.2442,0.2349)(0.2329,0.2349)
\polypmIIId{66}(0.2329,0.2348)(0.2442,0.2348)(0.2442,0.2368)(0.2329,0.2368)
\polypmIIId{67}(0.2329,0.2367)(0.2442,0.2367)(0.2442,0.2388)(0.2329,0.2388)
\polypmIIId{68}(0.2329,0.2387)(0.2442,0.2387)(0.2442,0.2407)(0.2329,0.2407)
\polypmIIId{69}(0.2329,0.2406)(0.2442,0.2406)(0.2442,0.2426)(0.2329,0.2426)
\polypmIIId{70}(0.2329,0.2425)(0.2442,0.2425)(0.2442,0.2445)(0.2329,0.2445)
\polypmIIId{71}(0.2329,0.2444)(0.2442,0.2444)(0.2442,0.2465)(0.2329,0.2465)
\polypmIIId{72}(0.2329,0.2464)(0.2442,0.2464)(0.2442,0.2484)(0.2329,0.2484)
\polypmIIId{73}(0.2329,0.2483)(0.2442,0.2483)(0.2442,0.2503)(0.2329,0.2503)
\polypmIIId{74}(0.2329,0.2502)(0.2442,0.2502)(0.2442,0.2522)(0.2329,0.2522)
\polypmIIId{75}(0.2329,0.2521)(0.2442,0.2521)(0.2442,0.2542)(0.2329,0.2542)
\polypmIIId{76}(0.2329,0.2541)(0.2442,0.2541)(0.2442,0.2561)(0.2329,0.2561)
\polypmIIId{77}(0.2329,0.256)(0.2442,0.256)(0.2442,0.258)(0.2329,0.258)
\polypmIIId{78}(0.2329,0.2579)(0.2442,0.2579)(0.2442,0.2599)(0.2329,0.2599)
\polypmIIId{79}(0.2329,0.2598)(0.2442,0.2598)(0.2442,0.2619)(0.2329,0.2619)
\polypmIIId{80}(0.2329,0.2618)(0.2442,0.2618)(0.2442,0.2638)(0.2329,0.2638)
\polypmIIId{81}(0.2329,0.2637)(0.2442,0.2637)(0.2442,0.2657)(0.2329,0.2657)
\polypmIIId{82}(0.2329,0.2656)(0.2442,0.2656)(0.2442,0.2676)(0.2329,0.2676)
\polypmIIId{83}(0.2329,0.2675)(0.2442,0.2675)(0.2442,0.2696)(0.2329,0.2696)
\polypmIIId{84}(0.2329,0.2695)(0.2442,0.2695)(0.2442,0.2715)(0.2329,0.2715)
\polypmIIId{85}(0.2329,0.2714)(0.2442,0.2714)(0.2442,0.2734)(0.2329,0.2734)
\polypmIIId{86}(0.2329,0.2733)(0.2442,0.2733)(0.2442,0.2753)(0.2329,0.2753)
\polypmIIId{87}(0.2329,0.2752)(0.2442,0.2752)(0.2442,0.2773)(0.2329,0.2773)
\polypmIIId{88}(0.2329,0.2772)(0.2442,0.2772)(0.2442,0.2792)(0.2329,0.2792)
\polypmIIId{89}(0.2329,0.2791)(0.2442,0.2791)(0.2442,0.2811)(0.2329,0.2811)
\polypmIIId{90}(0.2329,0.281)(0.2442,0.281)(0.2442,0.283)(0.2329,0.283)
\polypmIIId{91}(0.2329,0.2829)(0.2442,0.2829)(0.2442,0.285)(0.2329,0.285)
\polypmIIId{92}(0.2329,0.2849)(0.2442,0.2849)(0.2442,0.2869)(0.2329,0.2869)
\polypmIIId{93}(0.2329,0.2868)(0.2442,0.2868)(0.2442,0.2888)(0.2329,0.2888)
\polypmIIId{94}(0.2329,0.2887)(0.2442,0.2887)(0.2442,0.2907)(0.2329,0.2907)
\polypmIIId{95}(0.2329,0.2906)(0.2442,0.2906)(0.2442,0.2927)(0.2329,0.2927)
\polypmIIId{96}(0.2329,0.2926)(0.2442,0.2926)(0.2442,0.2946)(0.2329,0.2946)
\polypmIIId{97}(0.2329,0.2945)(0.2442,0.2945)(0.2442,0.2965)(0.2329,0.2965)
\polypmIIId{98}(0.2329,0.2964)(0.2442,0.2964)(0.2442,0.2984)(0.2329,0.2984)
\polypmIIId{99}(0.2329,0.2983)(0.2442,0.2983)(0.2442,0.3004)(0.2329,0.3004)
\polypmIIId{100}(0.2329,0.3003)(0.2442,0.3003)(0.2442,0.3023)(0.2329,0.3023)
\polypmIIId{101}(0.2329,0.3022)(0.2442,0.3022)(0.2442,0.3042)(0.2329,0.3042)
\polypmIIId{102}(0.2329,0.3041)(0.2442,0.3041)(0.2442,0.3061)(0.2329,0.3061)
\polypmIIId{103}(0.2329,0.306)(0.2442,0.306)(0.2442,0.3081)(0.2329,0.3081)
\polypmIIId{104}(0.2329,0.308)(0.2442,0.308)(0.2442,0.31)(0.2329,0.31)
\polypmIIId{105}(0.2329,0.3099)(0.2442,0.3099)(0.2442,0.3119)(0.2329,0.3119)
\polypmIIId{106}(0.2329,0.3118)(0.2442,0.3118)(0.2442,0.3138)(0.2329,0.3138)
\polypmIIId{107}(0.2329,0.3137)(0.2442,0.3137)(0.2442,0.3158)(0.2329,0.3158)
\polypmIIId{108}(0.2329,0.3157)(0.2442,0.3157)(0.2442,0.3177)(0.2329,0.3177)
\polypmIIId{109}(0.2329,0.3176)(0.2442,0.3176)(0.2442,0.3196)(0.2329,0.3196)
\polypmIIId{110}(0.2329,0.3195)(0.2442,0.3195)(0.2442,0.3215)(0.2329,0.3215)
\polypmIIId{111}(0.2329,0.3214)(0.2442,0.3214)(0.2442,0.3235)(0.2329,0.3235)
\polypmIIId{112}(0.2329,0.3234)(0.2442,0.3234)(0.2442,0.3254)(0.2329,0.3254)
\polypmIIId{113}(0.2329,0.3253)(0.2442,0.3253)(0.2442,0.3273)(0.2329,0.3273)
\polypmIIId{114}(0.2329,0.3272)(0.2442,0.3272)(0.2442,0.3292)(0.2329,0.3292)
\polypmIIId{115}(0.2329,0.3291)(0.2442,0.3291)(0.2442,0.3312)(0.2329,0.3312)
\polypmIIId{116}(0.2329,0.3311)(0.2442,0.3311)(0.2442,0.3331)(0.2329,0.3331)
\polypmIIId{117}(0.2329,0.333)(0.2442,0.333)(0.2442,0.335)(0.2329,0.335)
\polypmIIId{118}(0.2329,0.3349)(0.2442,0.3349)(0.2442,0.3369)(0.2329,0.3369)
\polypmIIId{119}(0.2329,0.3368)(0.2442,0.3368)(0.2442,0.3389)(0.2329,0.3389)
\polypmIIId{120}(0.2329,0.3388)(0.2442,0.3388)(0.2442,0.3408)(0.2329,0.3408)
\polypmIIId{121}(0.2329,0.3407)(0.2442,0.3407)(0.2442,0.3427)(0.2329,0.3427)
\polypmIIId{122}(0.2329,0.3426)(0.2442,0.3426)(0.2442,0.3446)(0.2329,0.3446)
\polypmIIId{123}(0.2329,0.3445)(0.2442,0.3445)(0.2442,0.3466)(0.2329,0.3466)
\polypmIIId{124}(0.2329,0.3465)(0.2442,0.3465)(0.2442,0.3485)(0.2329,0.3485)
\polypmIIId{125}(0.2329,0.3484)(0.2442,0.3484)(0.2442,0.3504)(0.2329,0.3504)
\polypmIIId{126}(0.2329,0.3503)(0.2442,0.3503)(0.2442,0.3523)(0.2329,0.3523)
\polypmIIId{127}(0.2329,0.3522)(0.2442,0.3522)(0.2442,0.3542)(0.2329,0.3542)

\PST@Border(0.2329,0.1078)
(0.2442,0.1078)
(0.2442,0.3542)
(0.2329,0.3542)
(0.2329,0.1078)


\rput[l](0.2502,0.1301){0.997}
\rput[l](0.2502,0.2048){0.998}
\rput[l](0.2502,0.2795){0.999}
\rput[l](0.2502,0.3542){1}

\catcode`@=12
\fi
\endpspicture} 
  }
  $\alpha = 0.0$
  %\label{fig:greedysolcomp00}
\end{figure}

\figspaces
\begin{figure}[H]
  \centering
  \resizebox{\columnwidth}{!}{%
    \subfloat[1 resource]{% GNUPLOT: LaTeX picture using PSTRICKS macros
% Define new PST objects, if not already defined
\ifx\PSTloaded\undefined
\def\PSTloaded{t}

\catcode`@=11

\newpsobject{PST@Border}{psline}{linewidth=.0015,linestyle=solid}

\catcode`@=12

\fi
\psset{unit=5.0in,xunit=5.0in,yunit=3.0in}
\pspicture(0.000000,0.000000)(0.31, 0.35)
\ifx\nofigs\undefined
\catcode`@=11

\newrgbcolor{PST@COLOR0}{1 1 1}
\newrgbcolor{PST@COLOR1}{0.992 0.992 0.992}
\newrgbcolor{PST@COLOR2}{0.984 0.984 0.984}
\newrgbcolor{PST@COLOR3}{0.976 0.976 0.976}
\newrgbcolor{PST@COLOR4}{0.968 0.968 0.968}
\newrgbcolor{PST@COLOR5}{0.96 0.96 0.96}
\newrgbcolor{PST@COLOR6}{0.952 0.952 0.952}
\newrgbcolor{PST@COLOR7}{0.944 0.944 0.944}
\newrgbcolor{PST@COLOR8}{0.937 0.937 0.937}
\newrgbcolor{PST@COLOR9}{0.929 0.929 0.929}
\newrgbcolor{PST@COLOR10}{0.921 0.921 0.921}
\newrgbcolor{PST@COLOR11}{0.913 0.913 0.913}
\newrgbcolor{PST@COLOR12}{0.905 0.905 0.905}
\newrgbcolor{PST@COLOR13}{0.897 0.897 0.897}
\newrgbcolor{PST@COLOR14}{0.889 0.889 0.889}
\newrgbcolor{PST@COLOR15}{0.881 0.881 0.881}
\newrgbcolor{PST@COLOR16}{0.874 0.874 0.874}
\newrgbcolor{PST@COLOR17}{0.866 0.866 0.866}
\newrgbcolor{PST@COLOR18}{0.858 0.858 0.858}
\newrgbcolor{PST@COLOR19}{0.85 0.85 0.85}
\newrgbcolor{PST@COLOR20}{0.842 0.842 0.842}
\newrgbcolor{PST@COLOR21}{0.834 0.834 0.834}
\newrgbcolor{PST@COLOR22}{0.826 0.826 0.826}
\newrgbcolor{PST@COLOR23}{0.818 0.818 0.818}
\newrgbcolor{PST@COLOR24}{0.811 0.811 0.811}
\newrgbcolor{PST@COLOR25}{0.803 0.803 0.803}
\newrgbcolor{PST@COLOR26}{0.795 0.795 0.795}
\newrgbcolor{PST@COLOR27}{0.787 0.787 0.787}
\newrgbcolor{PST@COLOR28}{0.779 0.779 0.779}
\newrgbcolor{PST@COLOR29}{0.771 0.771 0.771}
\newrgbcolor{PST@COLOR30}{0.763 0.763 0.763}
\newrgbcolor{PST@COLOR31}{0.755 0.755 0.755}
\newrgbcolor{PST@COLOR32}{0.748 0.748 0.748}
\newrgbcolor{PST@COLOR33}{0.74 0.74 0.74}
\newrgbcolor{PST@COLOR34}{0.732 0.732 0.732}
\newrgbcolor{PST@COLOR35}{0.724 0.724 0.724}
\newrgbcolor{PST@COLOR36}{0.716 0.716 0.716}
\newrgbcolor{PST@COLOR37}{0.708 0.708 0.708}
\newrgbcolor{PST@COLOR38}{0.7 0.7 0.7}
\newrgbcolor{PST@COLOR39}{0.692 0.692 0.692}
\newrgbcolor{PST@COLOR40}{0.685 0.685 0.685}
\newrgbcolor{PST@COLOR41}{0.677 0.677 0.677}
\newrgbcolor{PST@COLOR42}{0.669 0.669 0.669}
\newrgbcolor{PST@COLOR43}{0.661 0.661 0.661}
\newrgbcolor{PST@COLOR44}{0.653 0.653 0.653}
\newrgbcolor{PST@COLOR45}{0.645 0.645 0.645}
\newrgbcolor{PST@COLOR46}{0.637 0.637 0.637}
\newrgbcolor{PST@COLOR47}{0.629 0.629 0.629}
\newrgbcolor{PST@COLOR48}{0.622 0.622 0.622}
\newrgbcolor{PST@COLOR49}{0.614 0.614 0.614}
\newrgbcolor{PST@COLOR50}{0.606 0.606 0.606}
\newrgbcolor{PST@COLOR51}{0.598 0.598 0.598}
\newrgbcolor{PST@COLOR52}{0.59 0.59 0.59}
\newrgbcolor{PST@COLOR53}{0.582 0.582 0.582}
\newrgbcolor{PST@COLOR54}{0.574 0.574 0.574}
\newrgbcolor{PST@COLOR55}{0.566 0.566 0.566}
\newrgbcolor{PST@COLOR56}{0.559 0.559 0.559}
\newrgbcolor{PST@COLOR57}{0.551 0.551 0.551}
\newrgbcolor{PST@COLOR58}{0.543 0.543 0.543}
\newrgbcolor{PST@COLOR59}{0.535 0.535 0.535}
\newrgbcolor{PST@COLOR60}{0.527 0.527 0.527}
\newrgbcolor{PST@COLOR61}{0.519 0.519 0.519}
\newrgbcolor{PST@COLOR62}{0.511 0.511 0.511}
\newrgbcolor{PST@COLOR63}{0.503 0.503 0.503}
\newrgbcolor{PST@COLOR64}{0.496 0.496 0.496}
\newrgbcolor{PST@COLOR65}{0.488 0.488 0.488}
\newrgbcolor{PST@COLOR66}{0.48 0.48 0.48}
\newrgbcolor{PST@COLOR67}{0.472 0.472 0.472}
\newrgbcolor{PST@COLOR68}{0.464 0.464 0.464}
\newrgbcolor{PST@COLOR69}{0.456 0.456 0.456}
\newrgbcolor{PST@COLOR70}{0.448 0.448 0.448}
\newrgbcolor{PST@COLOR71}{0.44 0.44 0.44}
\newrgbcolor{PST@COLOR72}{0.433 0.433 0.433}
\newrgbcolor{PST@COLOR73}{0.425 0.425 0.425}
\newrgbcolor{PST@COLOR74}{0.417 0.417 0.417}
\newrgbcolor{PST@COLOR75}{0.409 0.409 0.409}
\newrgbcolor{PST@COLOR76}{0.401 0.401 0.401}
\newrgbcolor{PST@COLOR77}{0.393 0.393 0.393}
\newrgbcolor{PST@COLOR78}{0.385 0.385 0.385}
\newrgbcolor{PST@COLOR79}{0.377 0.377 0.377}
\newrgbcolor{PST@COLOR80}{0.37 0.37 0.37}
\newrgbcolor{PST@COLOR81}{0.362 0.362 0.362}
\newrgbcolor{PST@COLOR82}{0.354 0.354 0.354}
\newrgbcolor{PST@COLOR83}{0.346 0.346 0.346}
\newrgbcolor{PST@COLOR84}{0.338 0.338 0.338}
\newrgbcolor{PST@COLOR85}{0.33 0.33 0.33}
\newrgbcolor{PST@COLOR86}{0.322 0.322 0.322}
\newrgbcolor{PST@COLOR87}{0.314 0.314 0.314}
\newrgbcolor{PST@COLOR88}{0.307 0.307 0.307}
\newrgbcolor{PST@COLOR89}{0.299 0.299 0.299}
\newrgbcolor{PST@COLOR90}{0.291 0.291 0.291}
\newrgbcolor{PST@COLOR91}{0.283 0.283 0.283}
\newrgbcolor{PST@COLOR92}{0.275 0.275 0.275}
\newrgbcolor{PST@COLOR93}{0.267 0.267 0.267}
\newrgbcolor{PST@COLOR94}{0.259 0.259 0.259}
\newrgbcolor{PST@COLOR95}{0.251 0.251 0.251}
\newrgbcolor{PST@COLOR96}{0.244 0.244 0.244}
\newrgbcolor{PST@COLOR97}{0.236 0.236 0.236}
\newrgbcolor{PST@COLOR98}{0.228 0.228 0.228}
\newrgbcolor{PST@COLOR99}{0.22 0.22 0.22}
\newrgbcolor{PST@COLOR100}{0.212 0.212 0.212}
\newrgbcolor{PST@COLOR101}{0.204 0.204 0.204}
\newrgbcolor{PST@COLOR102}{0.196 0.196 0.196}
\newrgbcolor{PST@COLOR103}{0.188 0.188 0.188}
\newrgbcolor{PST@COLOR104}{0.181 0.181 0.181}
\newrgbcolor{PST@COLOR105}{0.173 0.173 0.173}
\newrgbcolor{PST@COLOR106}{0.165 0.165 0.165}
\newrgbcolor{PST@COLOR107}{0.157 0.157 0.157}
\newrgbcolor{PST@COLOR108}{0.149 0.149 0.149}
\newrgbcolor{PST@COLOR109}{0.141 0.141 0.141}
\newrgbcolor{PST@COLOR110}{0.133 0.133 0.133}
\newrgbcolor{PST@COLOR111}{0.125 0.125 0.125}
\newrgbcolor{PST@COLOR112}{0.118 0.118 0.118}
\newrgbcolor{PST@COLOR113}{0.11 0.11 0.11}
\newrgbcolor{PST@COLOR114}{0.102 0.102 0.102}
\newrgbcolor{PST@COLOR115}{0.094 0.094 0.094}
\newrgbcolor{PST@COLOR116}{0.086 0.086 0.086}
\newrgbcolor{PST@COLOR117}{0.078 0.078 0.078}
\newrgbcolor{PST@COLOR118}{0.07 0.07 0.07}
\newrgbcolor{PST@COLOR119}{0.062 0.062 0.062}
\newrgbcolor{PST@COLOR120}{0.055 0.055 0.055}
\newrgbcolor{PST@COLOR121}{0.047 0.047 0.047}
\newrgbcolor{PST@COLOR122}{0.039 0.039 0.039}
\newrgbcolor{PST@COLOR123}{0.031 0.031 0.031}
\newrgbcolor{PST@COLOR124}{0.023 0.023 0.023}
\newrgbcolor{PST@COLOR125}{0.015 0.015 0.015}
\newrgbcolor{PST@COLOR126}{0.007 0.007 0.007}
\newrgbcolor{PST@COLOR127}{0 0 0}


\def\polypmIIId#1{\pspolygon[linestyle=none,fillstyle=solid,fillcolor=PST@COLOR#1]}

\polypmIIId{74}(0.1432,0.19)(0.0864,0.19)(0.0864,0.1078)(0.1432,0.1078)
\polypmIIId{103}(0.1432,0.272)(0.0864,0.272)(0.0864,0.19)(0.1432,0.19)
\polypmIIId{117}(0.1432,0.3542)(0.0864,0.3542)(0.0864,0.272)(0.1432,0.272)

\polypmIIId{71}(0.2,0.19)(0.1432,0.19)(0.1432,0.1078)(0.2,0.1078)
\polypmIIId{102}(0.2,0.272)(0.1432,0.272)(0.1432,0.19)(0.2,0.19)
\polypmIIId{118}(0.2,0.3542)(0.1432,0.3542)(0.1432,0.272)(0.2,0.272)

\polypmIIId{70}(0.2568,0.19)(0.2,0.19)(0.2,0.1078)(0.2568,0.1078)
\polypmIIId{104}(0.2568,0.272)(0.2,0.272)(0.2,0.19)(0.2568,0.19)
\polypmIIId{118}(0.2568,0.3542)(0.2,0.3542)(0.2,0.272)(0.2568,0.272)

\polypmIIId{73}(0.3136,0.19)(0.2568,0.19)(0.2568,0.1078)(0.3136,0.1078)
\polypmIIId{105}(0.3136,0.272)(0.2568,0.272)(0.2568,0.19)(0.3136,0.19)
\polypmIIId{118}(0.3136,0.3542)(0.2568,0.3542)(0.2568,0.272)(0.3136,0.272)


\rput(0.1148,0.07){3}
\rput(0.1716,0.07){4}
\rput(0.2284,0.07){5}
\rput(0.2852,0.07){6}
\rput(0.2000,0.0070){years}

\rput[r](0.0806,0.1489){25}
\rput[r](0.0806,0.2310){50}
\rput[r](0.0806,0.3131){100}
\rput{L}(0.0096,0.2310){actions}

\PST@Border(0.0864,0.3542)
(0.0864,0.1078)
(0.3136,0.1078)
(0.3136,0.3542)
(0.0864,0.3542)

\catcode`@=12
\fi
\endpspicture} 
    \subfloat[2 resources]{% GNUPLOT: LaTeX picture using PSTRICKS macros
% Define new PST objects, if not already defined
\ifx\PSTloaded\undefined
\def\PSTloaded{t}

\catcode`@=11

\newpsobject{PST@Border}{psline}{linewidth=.0015,linestyle=solid}

\catcode`@=12

\fi
\psset{unit=5.0in,xunit=5.0in,yunit=3.0in}
\pspicture(0.000000,0.000000)(0.225000,0.35)
\ifx\nofigs\undefined
\catcode`@=11

\newrgbcolor{PST@COLOR0}{1 1 1}
\newrgbcolor{PST@COLOR1}{0.992 0.992 0.992}
\newrgbcolor{PST@COLOR2}{0.984 0.984 0.984}
\newrgbcolor{PST@COLOR3}{0.976 0.976 0.976}
\newrgbcolor{PST@COLOR4}{0.968 0.968 0.968}
\newrgbcolor{PST@COLOR5}{0.96 0.96 0.96}
\newrgbcolor{PST@COLOR6}{0.952 0.952 0.952}
\newrgbcolor{PST@COLOR7}{0.944 0.944 0.944}
\newrgbcolor{PST@COLOR8}{0.937 0.937 0.937}
\newrgbcolor{PST@COLOR9}{0.929 0.929 0.929}
\newrgbcolor{PST@COLOR10}{0.921 0.921 0.921}
\newrgbcolor{PST@COLOR11}{0.913 0.913 0.913}
\newrgbcolor{PST@COLOR12}{0.905 0.905 0.905}
\newrgbcolor{PST@COLOR13}{0.897 0.897 0.897}
\newrgbcolor{PST@COLOR14}{0.889 0.889 0.889}
\newrgbcolor{PST@COLOR15}{0.881 0.881 0.881}
\newrgbcolor{PST@COLOR16}{0.874 0.874 0.874}
\newrgbcolor{PST@COLOR17}{0.866 0.866 0.866}
\newrgbcolor{PST@COLOR18}{0.858 0.858 0.858}
\newrgbcolor{PST@COLOR19}{0.85 0.85 0.85}
\newrgbcolor{PST@COLOR20}{0.842 0.842 0.842}
\newrgbcolor{PST@COLOR21}{0.834 0.834 0.834}
\newrgbcolor{PST@COLOR22}{0.826 0.826 0.826}
\newrgbcolor{PST@COLOR23}{0.818 0.818 0.818}
\newrgbcolor{PST@COLOR24}{0.811 0.811 0.811}
\newrgbcolor{PST@COLOR25}{0.803 0.803 0.803}
\newrgbcolor{PST@COLOR26}{0.795 0.795 0.795}
\newrgbcolor{PST@COLOR27}{0.787 0.787 0.787}
\newrgbcolor{PST@COLOR28}{0.779 0.779 0.779}
\newrgbcolor{PST@COLOR29}{0.771 0.771 0.771}
\newrgbcolor{PST@COLOR30}{0.763 0.763 0.763}
\newrgbcolor{PST@COLOR31}{0.755 0.755 0.755}
\newrgbcolor{PST@COLOR32}{0.748 0.748 0.748}
\newrgbcolor{PST@COLOR33}{0.74 0.74 0.74}
\newrgbcolor{PST@COLOR34}{0.732 0.732 0.732}
\newrgbcolor{PST@COLOR35}{0.724 0.724 0.724}
\newrgbcolor{PST@COLOR36}{0.716 0.716 0.716}
\newrgbcolor{PST@COLOR37}{0.708 0.708 0.708}
\newrgbcolor{PST@COLOR38}{0.7 0.7 0.7}
\newrgbcolor{PST@COLOR39}{0.692 0.692 0.692}
\newrgbcolor{PST@COLOR40}{0.685 0.685 0.685}
\newrgbcolor{PST@COLOR41}{0.677 0.677 0.677}
\newrgbcolor{PST@COLOR42}{0.669 0.669 0.669}
\newrgbcolor{PST@COLOR43}{0.661 0.661 0.661}
\newrgbcolor{PST@COLOR44}{0.653 0.653 0.653}
\newrgbcolor{PST@COLOR45}{0.645 0.645 0.645}
\newrgbcolor{PST@COLOR46}{0.637 0.637 0.637}
\newrgbcolor{PST@COLOR47}{0.629 0.629 0.629}
\newrgbcolor{PST@COLOR48}{0.622 0.622 0.622}
\newrgbcolor{PST@COLOR49}{0.614 0.614 0.614}
\newrgbcolor{PST@COLOR50}{0.606 0.606 0.606}
\newrgbcolor{PST@COLOR51}{0.598 0.598 0.598}
\newrgbcolor{PST@COLOR52}{0.59 0.59 0.59}
\newrgbcolor{PST@COLOR53}{0.582 0.582 0.582}
\newrgbcolor{PST@COLOR54}{0.574 0.574 0.574}
\newrgbcolor{PST@COLOR55}{0.566 0.566 0.566}
\newrgbcolor{PST@COLOR56}{0.559 0.559 0.559}
\newrgbcolor{PST@COLOR57}{0.551 0.551 0.551}
\newrgbcolor{PST@COLOR58}{0.543 0.543 0.543}
\newrgbcolor{PST@COLOR59}{0.535 0.535 0.535}
\newrgbcolor{PST@COLOR60}{0.527 0.527 0.527}
\newrgbcolor{PST@COLOR61}{0.519 0.519 0.519}
\newrgbcolor{PST@COLOR62}{0.511 0.511 0.511}
\newrgbcolor{PST@COLOR63}{0.503 0.503 0.503}
\newrgbcolor{PST@COLOR64}{0.496 0.496 0.496}
\newrgbcolor{PST@COLOR65}{0.488 0.488 0.488}
\newrgbcolor{PST@COLOR66}{0.48 0.48 0.48}
\newrgbcolor{PST@COLOR67}{0.472 0.472 0.472}
\newrgbcolor{PST@COLOR68}{0.464 0.464 0.464}
\newrgbcolor{PST@COLOR69}{0.456 0.456 0.456}
\newrgbcolor{PST@COLOR70}{0.448 0.448 0.448}
\newrgbcolor{PST@COLOR71}{0.44 0.44 0.44}
\newrgbcolor{PST@COLOR72}{0.433 0.433 0.433}
\newrgbcolor{PST@COLOR73}{0.425 0.425 0.425}
\newrgbcolor{PST@COLOR74}{0.417 0.417 0.417}
\newrgbcolor{PST@COLOR75}{0.409 0.409 0.409}
\newrgbcolor{PST@COLOR76}{0.401 0.401 0.401}
\newrgbcolor{PST@COLOR77}{0.393 0.393 0.393}
\newrgbcolor{PST@COLOR78}{0.385 0.385 0.385}
\newrgbcolor{PST@COLOR79}{0.377 0.377 0.377}
\newrgbcolor{PST@COLOR80}{0.37 0.37 0.37}
\newrgbcolor{PST@COLOR81}{0.362 0.362 0.362}
\newrgbcolor{PST@COLOR82}{0.354 0.354 0.354}
\newrgbcolor{PST@COLOR83}{0.346 0.346 0.346}
\newrgbcolor{PST@COLOR84}{0.338 0.338 0.338}
\newrgbcolor{PST@COLOR85}{0.33 0.33 0.33}
\newrgbcolor{PST@COLOR86}{0.322 0.322 0.322}
\newrgbcolor{PST@COLOR87}{0.314 0.314 0.314}
\newrgbcolor{PST@COLOR88}{0.307 0.307 0.307}
\newrgbcolor{PST@COLOR89}{0.299 0.299 0.299}
\newrgbcolor{PST@COLOR90}{0.291 0.291 0.291}
\newrgbcolor{PST@COLOR91}{0.283 0.283 0.283}
\newrgbcolor{PST@COLOR92}{0.275 0.275 0.275}
\newrgbcolor{PST@COLOR93}{0.267 0.267 0.267}
\newrgbcolor{PST@COLOR94}{0.259 0.259 0.259}
\newrgbcolor{PST@COLOR95}{0.251 0.251 0.251}
\newrgbcolor{PST@COLOR96}{0.244 0.244 0.244}
\newrgbcolor{PST@COLOR97}{0.236 0.236 0.236}
\newrgbcolor{PST@COLOR98}{0.228 0.228 0.228}
\newrgbcolor{PST@COLOR99}{0.22 0.22 0.22}
\newrgbcolor{PST@COLOR100}{0.212 0.212 0.212}
\newrgbcolor{PST@COLOR101}{0.204 0.204 0.204}
\newrgbcolor{PST@COLOR102}{0.196 0.196 0.196}
\newrgbcolor{PST@COLOR103}{0.188 0.188 0.188}
\newrgbcolor{PST@COLOR104}{0.181 0.181 0.181}
\newrgbcolor{PST@COLOR105}{0.173 0.173 0.173}
\newrgbcolor{PST@COLOR106}{0.165 0.165 0.165}
\newrgbcolor{PST@COLOR107}{0.157 0.157 0.157}
\newrgbcolor{PST@COLOR108}{0.149 0.149 0.149}
\newrgbcolor{PST@COLOR109}{0.141 0.141 0.141}
\newrgbcolor{PST@COLOR110}{0.133 0.133 0.133}
\newrgbcolor{PST@COLOR111}{0.125 0.125 0.125}
\newrgbcolor{PST@COLOR112}{0.118 0.118 0.118}
\newrgbcolor{PST@COLOR113}{0.11 0.11 0.11}
\newrgbcolor{PST@COLOR114}{0.102 0.102 0.102}
\newrgbcolor{PST@COLOR115}{0.094 0.094 0.094}
\newrgbcolor{PST@COLOR116}{0.086 0.086 0.086}
\newrgbcolor{PST@COLOR117}{0.078 0.078 0.078}
\newrgbcolor{PST@COLOR118}{0.07 0.07 0.07}
\newrgbcolor{PST@COLOR119}{0.062 0.062 0.062}
\newrgbcolor{PST@COLOR120}{0.055 0.055 0.055}
\newrgbcolor{PST@COLOR121}{0.047 0.047 0.047}
\newrgbcolor{PST@COLOR122}{0.039 0.039 0.039}
\newrgbcolor{PST@COLOR123}{0.031 0.031 0.031}
\newrgbcolor{PST@COLOR124}{0.023 0.023 0.023}
\newrgbcolor{PST@COLOR125}{0.015 0.015 0.015}
\newrgbcolor{PST@COLOR126}{0.007 0.007 0.007}
\newrgbcolor{PST@COLOR127}{0 0 0}

\def\polypmIIId#1{\pspolygon[linestyle=none,fillstyle=solid,fillcolor=PST@COLOR#1]}

\polypmIIId{56} (0.0568,0.19)  (0.0,0.19)  (0.0,0.1078)(0.0568,0.1078)
\polypmIIId{92}  (0.0568,0.272) (0.0,0.272) (0.0,0.19)  (0.0568,0.19)
\polypmIIId{113}  (0.0568,0.3542)(0.0,0.3542)(0.0,0.272) (0.0568,0.272)

\polypmIIId{58} (0.1136,   0.19)  (0.0568,0.19)  (0.0568,0.1078)(0.1136,0.1078)
\polypmIIId{90}  (0.1136,   0.272) (0.0568,0.272) (0.0568,0.19)  (0.1136,0.19)
\polypmIIId{113}  (0.1136,   0.3542)(0.0568,0.3542)(0.0568,0.272) (0.1136,0.272)

\polypmIIId{57}(0.1704,0.19)  (0.1136,   0.19)  (0.1136,   0.1078)(0.1704,0.1078)
\polypmIIId{92} (0.1704,0.272) (0.1136,   0.272) (0.1136,   0.19)  (0.1704,0.19)
\polypmIIId{112}  (0.1704,0.3542)(0.1136,   0.3542)(0.1136,   0.272) (0.1704,0.272)

\polypmIIId{59}(0.2272,0.19)  (0.1704,0.19)  (0.1704,0.1078)(0.2272,0.1078)
\polypmIIId{96}  (0.2272,0.272) (0.1704,0.272) (0.1704,0.19)  (0.2272,0.19)
\polypmIIId{113}  (0.2272,0.3542)(0.1704,0.3542)(0.1704,0.272) (0.2272,0.272)

\rput(0.0284,0.07){3}
\rput(0.0852,0.07){4}
\rput(0.1420,0.07){5}
\rput(0.1988,0.07){6}
\rput(0.1136,0.0070){years}


\PST@Border(0.0,0.3542)
(0.0,0.1078)
(0.2272,0.1078)
(0.2272,0.3542)
(0.0,0.3542)

\catcode`@=12
\fi
\endpspicture}
    \subfloat[4 resources]{% GNUPLOT: LaTeX picture using PSTRICKS macros
% Define new PST objects, if not already defined
\ifx\PSTloaded\undefined
\def\PSTloaded{t}

\catcode`@=11

\newpsobject{PST@Border}{psline}{linewidth=.0015,linestyle=solid}

\catcode`@=12

\fi
\psset{unit=5.0in,xunit=5.0in,yunit=3.0in}
\pspicture(0.000000,0.000000)(0.3136,0.35)
\ifx\nofigs\undefined
\catcode`@=11

\newrgbcolor{PST@COLOR0}{1 1 1}
\newrgbcolor{PST@COLOR1}{0.992 0.992 0.992}
\newrgbcolor{PST@COLOR2}{0.984 0.984 0.984}
\newrgbcolor{PST@COLOR3}{0.976 0.976 0.976}
\newrgbcolor{PST@COLOR4}{0.968 0.968 0.968}
\newrgbcolor{PST@COLOR5}{0.96 0.96 0.96}
\newrgbcolor{PST@COLOR6}{0.952 0.952 0.952}
\newrgbcolor{PST@COLOR7}{0.944 0.944 0.944}
\newrgbcolor{PST@COLOR8}{0.937 0.937 0.937}
\newrgbcolor{PST@COLOR9}{0.929 0.929 0.929}
\newrgbcolor{PST@COLOR10}{0.921 0.921 0.921}
\newrgbcolor{PST@COLOR11}{0.913 0.913 0.913}
\newrgbcolor{PST@COLOR12}{0.905 0.905 0.905}
\newrgbcolor{PST@COLOR13}{0.897 0.897 0.897}
\newrgbcolor{PST@COLOR14}{0.889 0.889 0.889}
\newrgbcolor{PST@COLOR15}{0.881 0.881 0.881}
\newrgbcolor{PST@COLOR16}{0.874 0.874 0.874}
\newrgbcolor{PST@COLOR17}{0.866 0.866 0.866}
\newrgbcolor{PST@COLOR18}{0.858 0.858 0.858}
\newrgbcolor{PST@COLOR19}{0.85 0.85 0.85}
\newrgbcolor{PST@COLOR20}{0.842 0.842 0.842}
\newrgbcolor{PST@COLOR21}{0.834 0.834 0.834}
\newrgbcolor{PST@COLOR22}{0.826 0.826 0.826}
\newrgbcolor{PST@COLOR23}{0.818 0.818 0.818}
\newrgbcolor{PST@COLOR24}{0.811 0.811 0.811}
\newrgbcolor{PST@COLOR25}{0.803 0.803 0.803}
\newrgbcolor{PST@COLOR26}{0.795 0.795 0.795}
\newrgbcolor{PST@COLOR27}{0.787 0.787 0.787}
\newrgbcolor{PST@COLOR28}{0.779 0.779 0.779}
\newrgbcolor{PST@COLOR29}{0.771 0.771 0.771}
\newrgbcolor{PST@COLOR30}{0.763 0.763 0.763}
\newrgbcolor{PST@COLOR31}{0.755 0.755 0.755}
\newrgbcolor{PST@COLOR32}{0.748 0.748 0.748}
\newrgbcolor{PST@COLOR33}{0.74 0.74 0.74}
\newrgbcolor{PST@COLOR34}{0.732 0.732 0.732}
\newrgbcolor{PST@COLOR35}{0.724 0.724 0.724}
\newrgbcolor{PST@COLOR36}{0.716 0.716 0.716}
\newrgbcolor{PST@COLOR37}{0.708 0.708 0.708}
\newrgbcolor{PST@COLOR38}{0.7 0.7 0.7}
\newrgbcolor{PST@COLOR39}{0.692 0.692 0.692}
\newrgbcolor{PST@COLOR40}{0.685 0.685 0.685}
\newrgbcolor{PST@COLOR41}{0.677 0.677 0.677}
\newrgbcolor{PST@COLOR42}{0.669 0.669 0.669}
\newrgbcolor{PST@COLOR43}{0.661 0.661 0.661}
\newrgbcolor{PST@COLOR44}{0.653 0.653 0.653}
\newrgbcolor{PST@COLOR45}{0.645 0.645 0.645}
\newrgbcolor{PST@COLOR46}{0.637 0.637 0.637}
\newrgbcolor{PST@COLOR47}{0.629 0.629 0.629}
\newrgbcolor{PST@COLOR48}{0.622 0.622 0.622}
\newrgbcolor{PST@COLOR49}{0.614 0.614 0.614}
\newrgbcolor{PST@COLOR50}{0.606 0.606 0.606}
\newrgbcolor{PST@COLOR51}{0.598 0.598 0.598}
\newrgbcolor{PST@COLOR52}{0.59 0.59 0.59}
\newrgbcolor{PST@COLOR53}{0.582 0.582 0.582}
\newrgbcolor{PST@COLOR54}{0.574 0.574 0.574}
\newrgbcolor{PST@COLOR55}{0.566 0.566 0.566}
\newrgbcolor{PST@COLOR56}{0.559 0.559 0.559}
\newrgbcolor{PST@COLOR57}{0.551 0.551 0.551}
\newrgbcolor{PST@COLOR58}{0.543 0.543 0.543}
\newrgbcolor{PST@COLOR59}{0.535 0.535 0.535}
\newrgbcolor{PST@COLOR60}{0.527 0.527 0.527}
\newrgbcolor{PST@COLOR61}{0.519 0.519 0.519}
\newrgbcolor{PST@COLOR62}{0.511 0.511 0.511}
\newrgbcolor{PST@COLOR63}{0.503 0.503 0.503}
\newrgbcolor{PST@COLOR64}{0.496 0.496 0.496}
\newrgbcolor{PST@COLOR65}{0.488 0.488 0.488}
\newrgbcolor{PST@COLOR66}{0.48 0.48 0.48}
\newrgbcolor{PST@COLOR67}{0.472 0.472 0.472}
\newrgbcolor{PST@COLOR68}{0.464 0.464 0.464}
\newrgbcolor{PST@COLOR69}{0.456 0.456 0.456}
\newrgbcolor{PST@COLOR70}{0.448 0.448 0.448}
\newrgbcolor{PST@COLOR71}{0.44 0.44 0.44}
\newrgbcolor{PST@COLOR72}{0.433 0.433 0.433}
\newrgbcolor{PST@COLOR73}{0.425 0.425 0.425}
\newrgbcolor{PST@COLOR74}{0.417 0.417 0.417}
\newrgbcolor{PST@COLOR75}{0.409 0.409 0.409}
\newrgbcolor{PST@COLOR76}{0.401 0.401 0.401}
\newrgbcolor{PST@COLOR77}{0.393 0.393 0.393}
\newrgbcolor{PST@COLOR78}{0.385 0.385 0.385}
\newrgbcolor{PST@COLOR79}{0.377 0.377 0.377}
\newrgbcolor{PST@COLOR80}{0.37 0.37 0.37}
\newrgbcolor{PST@COLOR81}{0.362 0.362 0.362}
\newrgbcolor{PST@COLOR82}{0.354 0.354 0.354}
\newrgbcolor{PST@COLOR83}{0.346 0.346 0.346}
\newrgbcolor{PST@COLOR84}{0.338 0.338 0.338}
\newrgbcolor{PST@COLOR85}{0.33 0.33 0.33}
\newrgbcolor{PST@COLOR86}{0.322 0.322 0.322}
\newrgbcolor{PST@COLOR87}{0.314 0.314 0.314}
\newrgbcolor{PST@COLOR88}{0.307 0.307 0.307}
\newrgbcolor{PST@COLOR89}{0.299 0.299 0.299}
\newrgbcolor{PST@COLOR90}{0.291 0.291 0.291}
\newrgbcolor{PST@COLOR91}{0.283 0.283 0.283}
\newrgbcolor{PST@COLOR92}{0.275 0.275 0.275}
\newrgbcolor{PST@COLOR93}{0.267 0.267 0.267}
\newrgbcolor{PST@COLOR94}{0.259 0.259 0.259}
\newrgbcolor{PST@COLOR95}{0.251 0.251 0.251}
\newrgbcolor{PST@COLOR96}{0.244 0.244 0.244}
\newrgbcolor{PST@COLOR97}{0.236 0.236 0.236}
\newrgbcolor{PST@COLOR98}{0.228 0.228 0.228}
\newrgbcolor{PST@COLOR99}{0.22 0.22 0.22}
\newrgbcolor{PST@COLOR100}{0.212 0.212 0.212}
\newrgbcolor{PST@COLOR101}{0.204 0.204 0.204}
\newrgbcolor{PST@COLOR102}{0.196 0.196 0.196}
\newrgbcolor{PST@COLOR103}{0.188 0.188 0.188}
\newrgbcolor{PST@COLOR104}{0.181 0.181 0.181}
\newrgbcolor{PST@COLOR105}{0.173 0.173 0.173}
\newrgbcolor{PST@COLOR106}{0.165 0.165 0.165}
\newrgbcolor{PST@COLOR107}{0.157 0.157 0.157}
\newrgbcolor{PST@COLOR108}{0.149 0.149 0.149}
\newrgbcolor{PST@COLOR109}{0.141 0.141 0.141}
\newrgbcolor{PST@COLOR110}{0.133 0.133 0.133}
\newrgbcolor{PST@COLOR111}{0.125 0.125 0.125}
\newrgbcolor{PST@COLOR112}{0.118 0.118 0.118}
\newrgbcolor{PST@COLOR113}{0.11 0.11 0.11}
\newrgbcolor{PST@COLOR114}{0.102 0.102 0.102}
\newrgbcolor{PST@COLOR115}{0.094 0.094 0.094}
\newrgbcolor{PST@COLOR116}{0.086 0.086 0.086}
\newrgbcolor{PST@COLOR117}{0.078 0.078 0.078}
\newrgbcolor{PST@COLOR118}{0.07 0.07 0.07}
\newrgbcolor{PST@COLOR119}{0.062 0.062 0.062}
\newrgbcolor{PST@COLOR120}{0.055 0.055 0.055}
\newrgbcolor{PST@COLOR121}{0.047 0.047 0.047}
\newrgbcolor{PST@COLOR122}{0.039 0.039 0.039}
\newrgbcolor{PST@COLOR123}{0.031 0.031 0.031}
\newrgbcolor{PST@COLOR124}{0.023 0.023 0.023}
\newrgbcolor{PST@COLOR125}{0.015 0.015 0.015}
\newrgbcolor{PST@COLOR126}{0.007 0.007 0.007}
\newrgbcolor{PST@COLOR127}{0 0 0}

\def\polypmIIId#1{\pspolygon[linestyle=none,fillstyle=solid,fillcolor=PST@COLOR#1]}

\polypmIIId{13} (0.0568,0.19)  (0.0,0.19)  (0.0,0.1078)(0.0568,0.1078)
\polypmIIId{75}  (0.0568,0.272) (0.0,0.272) (0.0,0.19)  (0.0568,0.19)
\polypmIIId{98}  (0.0568,0.3542)(0.0,0.3542)(0.0,0.272) (0.0568,0.272)

\polypmIIId{13} (0.1136,   0.19)  (0.0568,0.19)  (0.0568,0.1078)(0.1136,0.1078)
\polypmIIId{72}  (0.1136,   0.272) (0.0568,0.272) (0.0568,0.19)  (0.1136,0.19)
\polypmIIId{100}  (0.1136,   0.3542)(0.0568,0.3542)(0.0568,0.272) (0.1136,0.272)

\polypmIIId{21}(0.1704,0.19)  (0.1136,   0.19)  (0.1136,   0.1078)(0.1704,0.1078)
\polypmIIId{72} (0.1704,0.272) (0.1136,   0.272) (0.1136,   0.19)  (0.1704,0.19)
\polypmIIId{99}  (0.1704,0.3542)(0.1136,   0.3542)(0.1136,   0.272) (0.1704,0.272)

\polypmIIId{19}(0.2272,0.19)  (0.1704,0.19)  (0.1704,0.1078)(0.2272,0.1078)
\polypmIIId{76}  (0.2272,0.272) (0.1704,0.272) (0.1704,0.19)  (0.2272,0.19)
\polypmIIId{101}  (0.2272,0.3542)(0.1704,0.3542)(0.1704,0.272) (0.2272,0.272)

\rput(0.0284,0.07){3}
\rput(0.0852,0.07){4}
\rput(0.1420,0.07){5}
\rput(0.1988,0.07){6}
\rput(0.1136,0.0070){years}

\PST@Border(0.0,0.3542)
(0.0,0.1078)
(0.2272,0.1078)
(0.2272,0.3542)
(0.0,0.3542)

\polypmIIId{0}(0.2329,0.1078)(0.2442,0.1078)(0.2442,0.1098)(0.2329,0.1098)
\polypmIIId{1}(0.2329,0.1097)(0.2442,0.1097)(0.2442,0.1117)(0.2329,0.1117)
\polypmIIId{2}(0.2329,0.1116)(0.2442,0.1116)(0.2442,0.1136)(0.2329,0.1136)
\polypmIIId{3}(0.2329,0.1135)(0.2442,0.1135)(0.2442,0.1156)(0.2329,0.1156)
\polypmIIId{4}(0.2329,0.1155)(0.2442,0.1155)(0.2442,0.1175)(0.2329,0.1175)
\polypmIIId{5}(0.2329,0.1174)(0.2442,0.1174)(0.2442,0.1194)(0.2329,0.1194)
\polypmIIId{6}(0.2329,0.1193)(0.2442,0.1193)(0.2442,0.1213)(0.2329,0.1213)
\polypmIIId{7}(0.2329,0.1212)(0.2442,0.1212)(0.2442,0.1233)(0.2329,0.1233)
\polypmIIId{8}(0.2329,0.1232)(0.2442,0.1232)(0.2442,0.1252)(0.2329,0.1252)
\polypmIIId{9}(0.2329,0.1251)(0.2442,0.1251)(0.2442,0.1271)(0.2329,0.1271)
\polypmIIId{10}(0.2329,0.127)(0.2442,0.127)(0.2442,0.129)(0.2329,0.129)
\polypmIIId{11}(0.2329,0.1289)(0.2442,0.1289)(0.2442,0.131)(0.2329,0.131)
\polypmIIId{12}(0.2329,0.1309)(0.2442,0.1309)(0.2442,0.1329)(0.2329,0.1329)
\polypmIIId{13}(0.2329,0.1328)(0.2442,0.1328)(0.2442,0.1348)(0.2329,0.1348)
\polypmIIId{14}(0.2329,0.1347)(0.2442,0.1347)(0.2442,0.1367)(0.2329,0.1367)
\polypmIIId{15}(0.2329,0.1366)(0.2442,0.1366)(0.2442,0.1387)(0.2329,0.1387)
\polypmIIId{16}(0.2329,0.1386)(0.2442,0.1386)(0.2442,0.1406)(0.2329,0.1406)
\polypmIIId{17}(0.2329,0.1405)(0.2442,0.1405)(0.2442,0.1425)(0.2329,0.1425)
\polypmIIId{18}(0.2329,0.1424)(0.2442,0.1424)(0.2442,0.1444)(0.2329,0.1444)
\polypmIIId{19}(0.2329,0.1443)(0.2442,0.1443)(0.2442,0.1464)(0.2329,0.1464)
\polypmIIId{20}(0.2329,0.1463)(0.2442,0.1463)(0.2442,0.1483)(0.2329,0.1483)
\polypmIIId{21}(0.2329,0.1482)(0.2442,0.1482)(0.2442,0.1502)(0.2329,0.1502)
\polypmIIId{22}(0.2329,0.1501)(0.2442,0.1501)(0.2442,0.1521)(0.2329,0.1521)
\polypmIIId{23}(0.2329,0.152)(0.2442,0.152)(0.2442,0.1541)(0.2329,0.1541)
\polypmIIId{24}(0.2329,0.154)(0.2442,0.154)(0.2442,0.156)(0.2329,0.156)
\polypmIIId{25}(0.2329,0.1559)(0.2442,0.1559)(0.2442,0.1579)(0.2329,0.1579)
\polypmIIId{26}(0.2329,0.1578)(0.2442,0.1578)(0.2442,0.1598)(0.2329,0.1598)
\polypmIIId{27}(0.2329,0.1597)(0.2442,0.1597)(0.2442,0.1618)(0.2329,0.1618)
\polypmIIId{28}(0.2329,0.1617)(0.2442,0.1617)(0.2442,0.1637)(0.2329,0.1637)
\polypmIIId{29}(0.2329,0.1636)(0.2442,0.1636)(0.2442,0.1656)(0.2329,0.1656)
\polypmIIId{30}(0.2329,0.1655)(0.2442,0.1655)(0.2442,0.1675)(0.2329,0.1675)
\polypmIIId{31}(0.2329,0.1674)(0.2442,0.1674)(0.2442,0.1695)(0.2329,0.1695)
\polypmIIId{32}(0.2329,0.1694)(0.2442,0.1694)(0.2442,0.1714)(0.2329,0.1714)
\polypmIIId{33}(0.2329,0.1713)(0.2442,0.1713)(0.2442,0.1733)(0.2329,0.1733)
\polypmIIId{34}(0.2329,0.1732)(0.2442,0.1732)(0.2442,0.1752)(0.2329,0.1752)
\polypmIIId{35}(0.2329,0.1751)(0.2442,0.1751)(0.2442,0.1772)(0.2329,0.1772)
\polypmIIId{36}(0.2329,0.1771)(0.2442,0.1771)(0.2442,0.1791)(0.2329,0.1791)
\polypmIIId{37}(0.2329,0.179)(0.2442,0.179)(0.2442,0.181)(0.2329,0.181)
\polypmIIId{38}(0.2329,0.1809)(0.2442,0.1809)(0.2442,0.1829)(0.2329,0.1829)
\polypmIIId{39}(0.2329,0.1828)(0.2442,0.1828)(0.2442,0.1849)(0.2329,0.1849)
\polypmIIId{40}(0.2329,0.1848)(0.2442,0.1848)(0.2442,0.1868)(0.2329,0.1868)
\polypmIIId{41}(0.2329,0.1867)(0.2442,0.1867)(0.2442,0.1887)(0.2329,0.1887)
\polypmIIId{42}(0.2329,0.1886)(0.2442,0.1886)(0.2442,0.1906)(0.2329,0.1906)
\polypmIIId{43}(0.2329,0.1905)(0.2442,0.1905)(0.2442,0.1926)(0.2329,0.1926)
\polypmIIId{44}(0.2329,0.1925)(0.2442,0.1925)(0.2442,0.1945)(0.2329,0.1945)
\polypmIIId{45}(0.2329,0.1944)(0.2442,0.1944)(0.2442,0.1964)(0.2329,0.1964)
\polypmIIId{46}(0.2329,0.1963)(0.2442,0.1963)(0.2442,0.1983)(0.2329,0.1983)
\polypmIIId{47}(0.2329,0.1982)(0.2442,0.1982)(0.2442,0.2003)(0.2329,0.2003)
\polypmIIId{48}(0.2329,0.2002)(0.2442,0.2002)(0.2442,0.2022)(0.2329,0.2022)
\polypmIIId{49}(0.2329,0.2021)(0.2442,0.2021)(0.2442,0.2041)(0.2329,0.2041)
\polypmIIId{50}(0.2329,0.204)(0.2442,0.204)(0.2442,0.206)(0.2329,0.206)
\polypmIIId{51}(0.2329,0.2059)(0.2442,0.2059)(0.2442,0.208)(0.2329,0.208)
\polypmIIId{52}(0.2329,0.2079)(0.2442,0.2079)(0.2442,0.2099)(0.2329,0.2099)
\polypmIIId{53}(0.2329,0.2098)(0.2442,0.2098)(0.2442,0.2118)(0.2329,0.2118)
\polypmIIId{54}(0.2329,0.2117)(0.2442,0.2117)(0.2442,0.2137)(0.2329,0.2137)
\polypmIIId{55}(0.2329,0.2136)(0.2442,0.2136)(0.2442,0.2157)(0.2329,0.2157)
\polypmIIId{56}(0.2329,0.2156)(0.2442,0.2156)(0.2442,0.2176)(0.2329,0.2176)
\polypmIIId{57}(0.2329,0.2175)(0.2442,0.2175)(0.2442,0.2195)(0.2329,0.2195)
\polypmIIId{58}(0.2329,0.2194)(0.2442,0.2194)(0.2442,0.2214)(0.2329,0.2214)
\polypmIIId{59}(0.2329,0.2213)(0.2442,0.2213)(0.2442,0.2234)(0.2329,0.2234)
\polypmIIId{60}(0.2329,0.2233)(0.2442,0.2233)(0.2442,0.2253)(0.2329,0.2253)
\polypmIIId{61}(0.2329,0.2252)(0.2442,0.2252)(0.2442,0.2272)(0.2329,0.2272)
\polypmIIId{62}(0.2329,0.2271)(0.2442,0.2271)(0.2442,0.2291)(0.2329,0.2291)
\polypmIIId{63}(0.2329,0.229)(0.2442,0.229)(0.2442,0.2311)(0.2329,0.2311)
\polypmIIId{64}(0.2329,0.231)(0.2442,0.231)(0.2442,0.233)(0.2329,0.233)
\polypmIIId{65}(0.2329,0.2329)(0.2442,0.2329)(0.2442,0.2349)(0.2329,0.2349)
\polypmIIId{66}(0.2329,0.2348)(0.2442,0.2348)(0.2442,0.2368)(0.2329,0.2368)
\polypmIIId{67}(0.2329,0.2367)(0.2442,0.2367)(0.2442,0.2388)(0.2329,0.2388)
\polypmIIId{68}(0.2329,0.2387)(0.2442,0.2387)(0.2442,0.2407)(0.2329,0.2407)
\polypmIIId{69}(0.2329,0.2406)(0.2442,0.2406)(0.2442,0.2426)(0.2329,0.2426)
\polypmIIId{70}(0.2329,0.2425)(0.2442,0.2425)(0.2442,0.2445)(0.2329,0.2445)
\polypmIIId{71}(0.2329,0.2444)(0.2442,0.2444)(0.2442,0.2465)(0.2329,0.2465)
\polypmIIId{72}(0.2329,0.2464)(0.2442,0.2464)(0.2442,0.2484)(0.2329,0.2484)
\polypmIIId{73}(0.2329,0.2483)(0.2442,0.2483)(0.2442,0.2503)(0.2329,0.2503)
\polypmIIId{74}(0.2329,0.2502)(0.2442,0.2502)(0.2442,0.2522)(0.2329,0.2522)
\polypmIIId{75}(0.2329,0.2521)(0.2442,0.2521)(0.2442,0.2542)(0.2329,0.2542)
\polypmIIId{76}(0.2329,0.2541)(0.2442,0.2541)(0.2442,0.2561)(0.2329,0.2561)
\polypmIIId{77}(0.2329,0.256)(0.2442,0.256)(0.2442,0.258)(0.2329,0.258)
\polypmIIId{78}(0.2329,0.2579)(0.2442,0.2579)(0.2442,0.2599)(0.2329,0.2599)
\polypmIIId{79}(0.2329,0.2598)(0.2442,0.2598)(0.2442,0.2619)(0.2329,0.2619)
\polypmIIId{80}(0.2329,0.2618)(0.2442,0.2618)(0.2442,0.2638)(0.2329,0.2638)
\polypmIIId{81}(0.2329,0.2637)(0.2442,0.2637)(0.2442,0.2657)(0.2329,0.2657)
\polypmIIId{82}(0.2329,0.2656)(0.2442,0.2656)(0.2442,0.2676)(0.2329,0.2676)
\polypmIIId{83}(0.2329,0.2675)(0.2442,0.2675)(0.2442,0.2696)(0.2329,0.2696)
\polypmIIId{84}(0.2329,0.2695)(0.2442,0.2695)(0.2442,0.2715)(0.2329,0.2715)
\polypmIIId{85}(0.2329,0.2714)(0.2442,0.2714)(0.2442,0.2734)(0.2329,0.2734)
\polypmIIId{86}(0.2329,0.2733)(0.2442,0.2733)(0.2442,0.2753)(0.2329,0.2753)
\polypmIIId{87}(0.2329,0.2752)(0.2442,0.2752)(0.2442,0.2773)(0.2329,0.2773)
\polypmIIId{88}(0.2329,0.2772)(0.2442,0.2772)(0.2442,0.2792)(0.2329,0.2792)
\polypmIIId{89}(0.2329,0.2791)(0.2442,0.2791)(0.2442,0.2811)(0.2329,0.2811)
\polypmIIId{90}(0.2329,0.281)(0.2442,0.281)(0.2442,0.283)(0.2329,0.283)
\polypmIIId{91}(0.2329,0.2829)(0.2442,0.2829)(0.2442,0.285)(0.2329,0.285)
\polypmIIId{92}(0.2329,0.2849)(0.2442,0.2849)(0.2442,0.2869)(0.2329,0.2869)
\polypmIIId{93}(0.2329,0.2868)(0.2442,0.2868)(0.2442,0.2888)(0.2329,0.2888)
\polypmIIId{94}(0.2329,0.2887)(0.2442,0.2887)(0.2442,0.2907)(0.2329,0.2907)
\polypmIIId{95}(0.2329,0.2906)(0.2442,0.2906)(0.2442,0.2927)(0.2329,0.2927)
\polypmIIId{96}(0.2329,0.2926)(0.2442,0.2926)(0.2442,0.2946)(0.2329,0.2946)
\polypmIIId{97}(0.2329,0.2945)(0.2442,0.2945)(0.2442,0.2965)(0.2329,0.2965)
\polypmIIId{98}(0.2329,0.2964)(0.2442,0.2964)(0.2442,0.2984)(0.2329,0.2984)
\polypmIIId{99}(0.2329,0.2983)(0.2442,0.2983)(0.2442,0.3004)(0.2329,0.3004)
\polypmIIId{100}(0.2329,0.3003)(0.2442,0.3003)(0.2442,0.3023)(0.2329,0.3023)
\polypmIIId{101}(0.2329,0.3022)(0.2442,0.3022)(0.2442,0.3042)(0.2329,0.3042)
\polypmIIId{102}(0.2329,0.3041)(0.2442,0.3041)(0.2442,0.3061)(0.2329,0.3061)
\polypmIIId{103}(0.2329,0.306)(0.2442,0.306)(0.2442,0.3081)(0.2329,0.3081)
\polypmIIId{104}(0.2329,0.308)(0.2442,0.308)(0.2442,0.31)(0.2329,0.31)
\polypmIIId{105}(0.2329,0.3099)(0.2442,0.3099)(0.2442,0.3119)(0.2329,0.3119)
\polypmIIId{106}(0.2329,0.3118)(0.2442,0.3118)(0.2442,0.3138)(0.2329,0.3138)
\polypmIIId{107}(0.2329,0.3137)(0.2442,0.3137)(0.2442,0.3158)(0.2329,0.3158)
\polypmIIId{108}(0.2329,0.3157)(0.2442,0.3157)(0.2442,0.3177)(0.2329,0.3177)
\polypmIIId{109}(0.2329,0.3176)(0.2442,0.3176)(0.2442,0.3196)(0.2329,0.3196)
\polypmIIId{110}(0.2329,0.3195)(0.2442,0.3195)(0.2442,0.3215)(0.2329,0.3215)
\polypmIIId{111}(0.2329,0.3214)(0.2442,0.3214)(0.2442,0.3235)(0.2329,0.3235)
\polypmIIId{112}(0.2329,0.3234)(0.2442,0.3234)(0.2442,0.3254)(0.2329,0.3254)
\polypmIIId{113}(0.2329,0.3253)(0.2442,0.3253)(0.2442,0.3273)(0.2329,0.3273)
\polypmIIId{114}(0.2329,0.3272)(0.2442,0.3272)(0.2442,0.3292)(0.2329,0.3292)
\polypmIIId{115}(0.2329,0.3291)(0.2442,0.3291)(0.2442,0.3312)(0.2329,0.3312)
\polypmIIId{116}(0.2329,0.3311)(0.2442,0.3311)(0.2442,0.3331)(0.2329,0.3331)
\polypmIIId{117}(0.2329,0.333)(0.2442,0.333)(0.2442,0.335)(0.2329,0.335)
\polypmIIId{118}(0.2329,0.3349)(0.2442,0.3349)(0.2442,0.3369)(0.2329,0.3369)
\polypmIIId{119}(0.2329,0.3368)(0.2442,0.3368)(0.2442,0.3389)(0.2329,0.3389)
\polypmIIId{120}(0.2329,0.3388)(0.2442,0.3388)(0.2442,0.3408)(0.2329,0.3408)
\polypmIIId{121}(0.2329,0.3407)(0.2442,0.3407)(0.2442,0.3427)(0.2329,0.3427)
\polypmIIId{122}(0.2329,0.3426)(0.2442,0.3426)(0.2442,0.3446)(0.2329,0.3446)
\polypmIIId{123}(0.2329,0.3445)(0.2442,0.3445)(0.2442,0.3466)(0.2329,0.3466)
\polypmIIId{124}(0.2329,0.3465)(0.2442,0.3465)(0.2442,0.3485)(0.2329,0.3485)
\polypmIIId{125}(0.2329,0.3484)(0.2442,0.3484)(0.2442,0.3504)(0.2329,0.3504)
\polypmIIId{126}(0.2329,0.3503)(0.2442,0.3503)(0.2442,0.3523)(0.2329,0.3523)
\polypmIIId{127}(0.2329,0.3522)(0.2442,0.3522)(0.2442,0.3542)(0.2329,0.3542)

\PST@Border(0.2329,0.1078)
(0.2442,0.1078)
(0.2442,0.3542)
(0.2329,0.3542)
(0.2329,0.1078)


\rput[l](0.2502,0.1301){0.997}
\rput[l](0.2502,0.2048){0.998}
\rput[l](0.2502,0.2795){0.999}
\rput[l](0.2502,0.3542){1}

\catcode`@=12
\fi
\endpspicture}
  }
  $\alpha = 0.1$
  %\label{fig:greedysolcomp01}
\end{figure}

\figspaces
\begin{figure}[H]
  \centering
  \resizebox{\columnwidth}{!}{%
    \subfloat[1 resource]{% GNUPLOT: LaTeX picture using PSTRICKS macros
% Define new PST objects, if not already defined
\ifx\PSTloaded\undefined
\def\PSTloaded{t}

\catcode`@=11

\newpsobject{PST@Border}{psline}{linewidth=.0015,linestyle=solid}

\catcode`@=12

\fi
\psset{unit=5.0in,xunit=5.0in,yunit=3.0in}
\pspicture(0.000000,0.000000)(0.31, 0.35)
\ifx\nofigs\undefined
\catcode`@=11

\newrgbcolor{PST@COLOR0}{1 1 1}
\newrgbcolor{PST@COLOR1}{0.992 0.992 0.992}
\newrgbcolor{PST@COLOR2}{0.984 0.984 0.984}
\newrgbcolor{PST@COLOR3}{0.976 0.976 0.976}
\newrgbcolor{PST@COLOR4}{0.968 0.968 0.968}
\newrgbcolor{PST@COLOR5}{0.96 0.96 0.96}
\newrgbcolor{PST@COLOR6}{0.952 0.952 0.952}
\newrgbcolor{PST@COLOR7}{0.944 0.944 0.944}
\newrgbcolor{PST@COLOR8}{0.937 0.937 0.937}
\newrgbcolor{PST@COLOR9}{0.929 0.929 0.929}
\newrgbcolor{PST@COLOR10}{0.921 0.921 0.921}
\newrgbcolor{PST@COLOR11}{0.913 0.913 0.913}
\newrgbcolor{PST@COLOR12}{0.905 0.905 0.905}
\newrgbcolor{PST@COLOR13}{0.897 0.897 0.897}
\newrgbcolor{PST@COLOR14}{0.889 0.889 0.889}
\newrgbcolor{PST@COLOR15}{0.881 0.881 0.881}
\newrgbcolor{PST@COLOR16}{0.874 0.874 0.874}
\newrgbcolor{PST@COLOR17}{0.866 0.866 0.866}
\newrgbcolor{PST@COLOR18}{0.858 0.858 0.858}
\newrgbcolor{PST@COLOR19}{0.85 0.85 0.85}
\newrgbcolor{PST@COLOR20}{0.842 0.842 0.842}
\newrgbcolor{PST@COLOR21}{0.834 0.834 0.834}
\newrgbcolor{PST@COLOR22}{0.826 0.826 0.826}
\newrgbcolor{PST@COLOR23}{0.818 0.818 0.818}
\newrgbcolor{PST@COLOR24}{0.811 0.811 0.811}
\newrgbcolor{PST@COLOR25}{0.803 0.803 0.803}
\newrgbcolor{PST@COLOR26}{0.795 0.795 0.795}
\newrgbcolor{PST@COLOR27}{0.787 0.787 0.787}
\newrgbcolor{PST@COLOR28}{0.779 0.779 0.779}
\newrgbcolor{PST@COLOR29}{0.771 0.771 0.771}
\newrgbcolor{PST@COLOR30}{0.763 0.763 0.763}
\newrgbcolor{PST@COLOR31}{0.755 0.755 0.755}
\newrgbcolor{PST@COLOR32}{0.748 0.748 0.748}
\newrgbcolor{PST@COLOR33}{0.74 0.74 0.74}
\newrgbcolor{PST@COLOR34}{0.732 0.732 0.732}
\newrgbcolor{PST@COLOR35}{0.724 0.724 0.724}
\newrgbcolor{PST@COLOR36}{0.716 0.716 0.716}
\newrgbcolor{PST@COLOR37}{0.708 0.708 0.708}
\newrgbcolor{PST@COLOR38}{0.7 0.7 0.7}
\newrgbcolor{PST@COLOR39}{0.692 0.692 0.692}
\newrgbcolor{PST@COLOR40}{0.685 0.685 0.685}
\newrgbcolor{PST@COLOR41}{0.677 0.677 0.677}
\newrgbcolor{PST@COLOR42}{0.669 0.669 0.669}
\newrgbcolor{PST@COLOR43}{0.661 0.661 0.661}
\newrgbcolor{PST@COLOR44}{0.653 0.653 0.653}
\newrgbcolor{PST@COLOR45}{0.645 0.645 0.645}
\newrgbcolor{PST@COLOR46}{0.637 0.637 0.637}
\newrgbcolor{PST@COLOR47}{0.629 0.629 0.629}
\newrgbcolor{PST@COLOR48}{0.622 0.622 0.622}
\newrgbcolor{PST@COLOR49}{0.614 0.614 0.614}
\newrgbcolor{PST@COLOR50}{0.606 0.606 0.606}
\newrgbcolor{PST@COLOR51}{0.598 0.598 0.598}
\newrgbcolor{PST@COLOR52}{0.59 0.59 0.59}
\newrgbcolor{PST@COLOR53}{0.582 0.582 0.582}
\newrgbcolor{PST@COLOR54}{0.574 0.574 0.574}
\newrgbcolor{PST@COLOR55}{0.566 0.566 0.566}
\newrgbcolor{PST@COLOR56}{0.559 0.559 0.559}
\newrgbcolor{PST@COLOR57}{0.551 0.551 0.551}
\newrgbcolor{PST@COLOR58}{0.543 0.543 0.543}
\newrgbcolor{PST@COLOR59}{0.535 0.535 0.535}
\newrgbcolor{PST@COLOR60}{0.527 0.527 0.527}
\newrgbcolor{PST@COLOR61}{0.519 0.519 0.519}
\newrgbcolor{PST@COLOR62}{0.511 0.511 0.511}
\newrgbcolor{PST@COLOR63}{0.503 0.503 0.503}
\newrgbcolor{PST@COLOR64}{0.496 0.496 0.496}
\newrgbcolor{PST@COLOR65}{0.488 0.488 0.488}
\newrgbcolor{PST@COLOR66}{0.48 0.48 0.48}
\newrgbcolor{PST@COLOR67}{0.472 0.472 0.472}
\newrgbcolor{PST@COLOR68}{0.464 0.464 0.464}
\newrgbcolor{PST@COLOR69}{0.456 0.456 0.456}
\newrgbcolor{PST@COLOR70}{0.448 0.448 0.448}
\newrgbcolor{PST@COLOR71}{0.44 0.44 0.44}
\newrgbcolor{PST@COLOR72}{0.433 0.433 0.433}
\newrgbcolor{PST@COLOR73}{0.425 0.425 0.425}
\newrgbcolor{PST@COLOR74}{0.417 0.417 0.417}
\newrgbcolor{PST@COLOR75}{0.409 0.409 0.409}
\newrgbcolor{PST@COLOR76}{0.401 0.401 0.401}
\newrgbcolor{PST@COLOR77}{0.393 0.393 0.393}
\newrgbcolor{PST@COLOR78}{0.385 0.385 0.385}
\newrgbcolor{PST@COLOR79}{0.377 0.377 0.377}
\newrgbcolor{PST@COLOR80}{0.37 0.37 0.37}
\newrgbcolor{PST@COLOR81}{0.362 0.362 0.362}
\newrgbcolor{PST@COLOR82}{0.354 0.354 0.354}
\newrgbcolor{PST@COLOR83}{0.346 0.346 0.346}
\newrgbcolor{PST@COLOR84}{0.338 0.338 0.338}
\newrgbcolor{PST@COLOR85}{0.33 0.33 0.33}
\newrgbcolor{PST@COLOR86}{0.322 0.322 0.322}
\newrgbcolor{PST@COLOR87}{0.314 0.314 0.314}
\newrgbcolor{PST@COLOR88}{0.307 0.307 0.307}
\newrgbcolor{PST@COLOR89}{0.299 0.299 0.299}
\newrgbcolor{PST@COLOR90}{0.291 0.291 0.291}
\newrgbcolor{PST@COLOR91}{0.283 0.283 0.283}
\newrgbcolor{PST@COLOR92}{0.275 0.275 0.275}
\newrgbcolor{PST@COLOR93}{0.267 0.267 0.267}
\newrgbcolor{PST@COLOR94}{0.259 0.259 0.259}
\newrgbcolor{PST@COLOR95}{0.251 0.251 0.251}
\newrgbcolor{PST@COLOR96}{0.244 0.244 0.244}
\newrgbcolor{PST@COLOR97}{0.236 0.236 0.236}
\newrgbcolor{PST@COLOR98}{0.228 0.228 0.228}
\newrgbcolor{PST@COLOR99}{0.22 0.22 0.22}
\newrgbcolor{PST@COLOR100}{0.212 0.212 0.212}
\newrgbcolor{PST@COLOR101}{0.204 0.204 0.204}
\newrgbcolor{PST@COLOR102}{0.196 0.196 0.196}
\newrgbcolor{PST@COLOR103}{0.188 0.188 0.188}
\newrgbcolor{PST@COLOR104}{0.181 0.181 0.181}
\newrgbcolor{PST@COLOR105}{0.173 0.173 0.173}
\newrgbcolor{PST@COLOR106}{0.165 0.165 0.165}
\newrgbcolor{PST@COLOR107}{0.157 0.157 0.157}
\newrgbcolor{PST@COLOR108}{0.149 0.149 0.149}
\newrgbcolor{PST@COLOR109}{0.141 0.141 0.141}
\newrgbcolor{PST@COLOR110}{0.133 0.133 0.133}
\newrgbcolor{PST@COLOR111}{0.125 0.125 0.125}
\newrgbcolor{PST@COLOR112}{0.118 0.118 0.118}
\newrgbcolor{PST@COLOR113}{0.11 0.11 0.11}
\newrgbcolor{PST@COLOR114}{0.102 0.102 0.102}
\newrgbcolor{PST@COLOR115}{0.094 0.094 0.094}
\newrgbcolor{PST@COLOR116}{0.086 0.086 0.086}
\newrgbcolor{PST@COLOR117}{0.078 0.078 0.078}
\newrgbcolor{PST@COLOR118}{0.07 0.07 0.07}
\newrgbcolor{PST@COLOR119}{0.062 0.062 0.062}
\newrgbcolor{PST@COLOR120}{0.055 0.055 0.055}
\newrgbcolor{PST@COLOR121}{0.047 0.047 0.047}
\newrgbcolor{PST@COLOR122}{0.039 0.039 0.039}
\newrgbcolor{PST@COLOR123}{0.031 0.031 0.031}
\newrgbcolor{PST@COLOR124}{0.023 0.023 0.023}
\newrgbcolor{PST@COLOR125}{0.015 0.015 0.015}
\newrgbcolor{PST@COLOR126}{0.007 0.007 0.007}
\newrgbcolor{PST@COLOR127}{0 0 0}


\def\polypmIIId#1{\pspolygon[linestyle=none,fillstyle=solid,fillcolor=PST@COLOR#1]}

\polypmIIId{116}(0.1432,0.19)(0.0864,0.19)(0.0864,0.1078)(0.1432,0.1078)
\polypmIIId{122}(0.1432,0.272)(0.0864,0.272)(0.0864,0.19)(0.1432,0.19)
\polypmIIId{125}(0.1432,0.3542)(0.0864,0.3542)(0.0864,0.272)(0.1432,0.272)

\polypmIIId{118}(0.2,0.19)(0.1432,0.19)(0.1432,0.1078)(0.2,0.1078)
\polypmIIId{122}(0.2,0.272)(0.1432,0.272)(0.1432,0.19)(0.2,0.19)
\polypmIIId{125}(0.2,0.3542)(0.1432,0.3542)(0.1432,0.272)(0.2,0.272)

\polypmIIId{119}(0.2568,0.19)(0.2,0.19)(0.2,0.1078)(0.2568,0.1078)
\polypmIIId{124}(0.2568,0.272)(0.2,0.272)(0.2,0.19)(0.2568,0.19)
\polypmIIId{125}(0.2568,0.3542)(0.2,0.3542)(0.2,0.272)(0.2568,0.272)

\polypmIIId{120}(0.3136,0.19)(0.2568,0.19)(0.2568,0.1078)(0.3136,0.1078)
\polypmIIId{125}(0.3136,0.272)(0.2568,0.272)(0.2568,0.19)(0.3136,0.19)
\polypmIIId{125}(0.3136,0.3542)(0.2568,0.3542)(0.2568,0.272)(0.3136,0.272)

\rput(0.1148,0.07){3}
\rput(0.1716,0.07){4}
\rput(0.2284,0.07){5}
\rput(0.2852,0.07){6}
\rput(0.2000,0.0070){years}

\rput[r](0.0806,0.1489){25}
\rput[r](0.0806,0.2310){50}
\rput[r](0.0806,0.3131){100}
\rput{L}(0.0096,0.2310){actions}

\PST@Border(0.0864,0.3542)
(0.0864,0.1078)
(0.3136,0.1078)
(0.3136,0.3542)
(0.0864,0.3542)

\catcode`@=12
\fi
\endpspicture} 
    \subfloat[2 resources]{% GNUPLOT: LaTeX picture using PSTRICKS macros
% Define new PST objects, if not already defined
\ifx\PSTloaded\undefined
\def\PSTloaded{t}

\catcode`@=11

\newpsobject{PST@Border}{psline}{linewidth=.0015,linestyle=solid}

\catcode`@=12

\fi
\psset{unit=5.0in,xunit=5.0in,yunit=3.0in}
\pspicture(0.000000,0.000000)(0.225000,0.35)
\ifx\nofigs\undefined
\catcode`@=11

\newrgbcolor{PST@COLOR0}{1 1 1}
\newrgbcolor{PST@COLOR1}{0.992 0.992 0.992}
\newrgbcolor{PST@COLOR2}{0.984 0.984 0.984}
\newrgbcolor{PST@COLOR3}{0.976 0.976 0.976}
\newrgbcolor{PST@COLOR4}{0.968 0.968 0.968}
\newrgbcolor{PST@COLOR5}{0.96 0.96 0.96}
\newrgbcolor{PST@COLOR6}{0.952 0.952 0.952}
\newrgbcolor{PST@COLOR7}{0.944 0.944 0.944}
\newrgbcolor{PST@COLOR8}{0.937 0.937 0.937}
\newrgbcolor{PST@COLOR9}{0.929 0.929 0.929}
\newrgbcolor{PST@COLOR10}{0.921 0.921 0.921}
\newrgbcolor{PST@COLOR11}{0.913 0.913 0.913}
\newrgbcolor{PST@COLOR12}{0.905 0.905 0.905}
\newrgbcolor{PST@COLOR13}{0.897 0.897 0.897}
\newrgbcolor{PST@COLOR14}{0.889 0.889 0.889}
\newrgbcolor{PST@COLOR15}{0.881 0.881 0.881}
\newrgbcolor{PST@COLOR16}{0.874 0.874 0.874}
\newrgbcolor{PST@COLOR17}{0.866 0.866 0.866}
\newrgbcolor{PST@COLOR18}{0.858 0.858 0.858}
\newrgbcolor{PST@COLOR19}{0.85 0.85 0.85}
\newrgbcolor{PST@COLOR20}{0.842 0.842 0.842}
\newrgbcolor{PST@COLOR21}{0.834 0.834 0.834}
\newrgbcolor{PST@COLOR22}{0.826 0.826 0.826}
\newrgbcolor{PST@COLOR23}{0.818 0.818 0.818}
\newrgbcolor{PST@COLOR24}{0.811 0.811 0.811}
\newrgbcolor{PST@COLOR25}{0.803 0.803 0.803}
\newrgbcolor{PST@COLOR26}{0.795 0.795 0.795}
\newrgbcolor{PST@COLOR27}{0.787 0.787 0.787}
\newrgbcolor{PST@COLOR28}{0.779 0.779 0.779}
\newrgbcolor{PST@COLOR29}{0.771 0.771 0.771}
\newrgbcolor{PST@COLOR30}{0.763 0.763 0.763}
\newrgbcolor{PST@COLOR31}{0.755 0.755 0.755}
\newrgbcolor{PST@COLOR32}{0.748 0.748 0.748}
\newrgbcolor{PST@COLOR33}{0.74 0.74 0.74}
\newrgbcolor{PST@COLOR34}{0.732 0.732 0.732}
\newrgbcolor{PST@COLOR35}{0.724 0.724 0.724}
\newrgbcolor{PST@COLOR36}{0.716 0.716 0.716}
\newrgbcolor{PST@COLOR37}{0.708 0.708 0.708}
\newrgbcolor{PST@COLOR38}{0.7 0.7 0.7}
\newrgbcolor{PST@COLOR39}{0.692 0.692 0.692}
\newrgbcolor{PST@COLOR40}{0.685 0.685 0.685}
\newrgbcolor{PST@COLOR41}{0.677 0.677 0.677}
\newrgbcolor{PST@COLOR42}{0.669 0.669 0.669}
\newrgbcolor{PST@COLOR43}{0.661 0.661 0.661}
\newrgbcolor{PST@COLOR44}{0.653 0.653 0.653}
\newrgbcolor{PST@COLOR45}{0.645 0.645 0.645}
\newrgbcolor{PST@COLOR46}{0.637 0.637 0.637}
\newrgbcolor{PST@COLOR47}{0.629 0.629 0.629}
\newrgbcolor{PST@COLOR48}{0.622 0.622 0.622}
\newrgbcolor{PST@COLOR49}{0.614 0.614 0.614}
\newrgbcolor{PST@COLOR50}{0.606 0.606 0.606}
\newrgbcolor{PST@COLOR51}{0.598 0.598 0.598}
\newrgbcolor{PST@COLOR52}{0.59 0.59 0.59}
\newrgbcolor{PST@COLOR53}{0.582 0.582 0.582}
\newrgbcolor{PST@COLOR54}{0.574 0.574 0.574}
\newrgbcolor{PST@COLOR55}{0.566 0.566 0.566}
\newrgbcolor{PST@COLOR56}{0.559 0.559 0.559}
\newrgbcolor{PST@COLOR57}{0.551 0.551 0.551}
\newrgbcolor{PST@COLOR58}{0.543 0.543 0.543}
\newrgbcolor{PST@COLOR59}{0.535 0.535 0.535}
\newrgbcolor{PST@COLOR60}{0.527 0.527 0.527}
\newrgbcolor{PST@COLOR61}{0.519 0.519 0.519}
\newrgbcolor{PST@COLOR62}{0.511 0.511 0.511}
\newrgbcolor{PST@COLOR63}{0.503 0.503 0.503}
\newrgbcolor{PST@COLOR64}{0.496 0.496 0.496}
\newrgbcolor{PST@COLOR65}{0.488 0.488 0.488}
\newrgbcolor{PST@COLOR66}{0.48 0.48 0.48}
\newrgbcolor{PST@COLOR67}{0.472 0.472 0.472}
\newrgbcolor{PST@COLOR68}{0.464 0.464 0.464}
\newrgbcolor{PST@COLOR69}{0.456 0.456 0.456}
\newrgbcolor{PST@COLOR70}{0.448 0.448 0.448}
\newrgbcolor{PST@COLOR71}{0.44 0.44 0.44}
\newrgbcolor{PST@COLOR72}{0.433 0.433 0.433}
\newrgbcolor{PST@COLOR73}{0.425 0.425 0.425}
\newrgbcolor{PST@COLOR74}{0.417 0.417 0.417}
\newrgbcolor{PST@COLOR75}{0.409 0.409 0.409}
\newrgbcolor{PST@COLOR76}{0.401 0.401 0.401}
\newrgbcolor{PST@COLOR77}{0.393 0.393 0.393}
\newrgbcolor{PST@COLOR78}{0.385 0.385 0.385}
\newrgbcolor{PST@COLOR79}{0.377 0.377 0.377}
\newrgbcolor{PST@COLOR80}{0.37 0.37 0.37}
\newrgbcolor{PST@COLOR81}{0.362 0.362 0.362}
\newrgbcolor{PST@COLOR82}{0.354 0.354 0.354}
\newrgbcolor{PST@COLOR83}{0.346 0.346 0.346}
\newrgbcolor{PST@COLOR84}{0.338 0.338 0.338}
\newrgbcolor{PST@COLOR85}{0.33 0.33 0.33}
\newrgbcolor{PST@COLOR86}{0.322 0.322 0.322}
\newrgbcolor{PST@COLOR87}{0.314 0.314 0.314}
\newrgbcolor{PST@COLOR88}{0.307 0.307 0.307}
\newrgbcolor{PST@COLOR89}{0.299 0.299 0.299}
\newrgbcolor{PST@COLOR90}{0.291 0.291 0.291}
\newrgbcolor{PST@COLOR91}{0.283 0.283 0.283}
\newrgbcolor{PST@COLOR92}{0.275 0.275 0.275}
\newrgbcolor{PST@COLOR93}{0.267 0.267 0.267}
\newrgbcolor{PST@COLOR94}{0.259 0.259 0.259}
\newrgbcolor{PST@COLOR95}{0.251 0.251 0.251}
\newrgbcolor{PST@COLOR96}{0.244 0.244 0.244}
\newrgbcolor{PST@COLOR97}{0.236 0.236 0.236}
\newrgbcolor{PST@COLOR98}{0.228 0.228 0.228}
\newrgbcolor{PST@COLOR99}{0.22 0.22 0.22}
\newrgbcolor{PST@COLOR100}{0.212 0.212 0.212}
\newrgbcolor{PST@COLOR101}{0.204 0.204 0.204}
\newrgbcolor{PST@COLOR102}{0.196 0.196 0.196}
\newrgbcolor{PST@COLOR103}{0.188 0.188 0.188}
\newrgbcolor{PST@COLOR104}{0.181 0.181 0.181}
\newrgbcolor{PST@COLOR105}{0.173 0.173 0.173}
\newrgbcolor{PST@COLOR106}{0.165 0.165 0.165}
\newrgbcolor{PST@COLOR107}{0.157 0.157 0.157}
\newrgbcolor{PST@COLOR108}{0.149 0.149 0.149}
\newrgbcolor{PST@COLOR109}{0.141 0.141 0.141}
\newrgbcolor{PST@COLOR110}{0.133 0.133 0.133}
\newrgbcolor{PST@COLOR111}{0.125 0.125 0.125}
\newrgbcolor{PST@COLOR112}{0.118 0.118 0.118}
\newrgbcolor{PST@COLOR113}{0.11 0.11 0.11}
\newrgbcolor{PST@COLOR114}{0.102 0.102 0.102}
\newrgbcolor{PST@COLOR115}{0.094 0.094 0.094}
\newrgbcolor{PST@COLOR116}{0.086 0.086 0.086}
\newrgbcolor{PST@COLOR117}{0.078 0.078 0.078}
\newrgbcolor{PST@COLOR118}{0.07 0.07 0.07}
\newrgbcolor{PST@COLOR119}{0.062 0.062 0.062}
\newrgbcolor{PST@COLOR120}{0.055 0.055 0.055}
\newrgbcolor{PST@COLOR121}{0.047 0.047 0.047}
\newrgbcolor{PST@COLOR122}{0.039 0.039 0.039}
\newrgbcolor{PST@COLOR123}{0.031 0.031 0.031}
\newrgbcolor{PST@COLOR124}{0.023 0.023 0.023}
\newrgbcolor{PST@COLOR125}{0.015 0.015 0.015}
\newrgbcolor{PST@COLOR126}{0.007 0.007 0.007}
\newrgbcolor{PST@COLOR127}{0 0 0}

\def\polypmIIId#1{\pspolygon[linestyle=none,fillstyle=solid,fillcolor=PST@COLOR#1]}

\polypmIIId{112} (0.0568,0.19)  (0.0,0.19)  (0.0,0.1078)(0.0568,0.1078)
\polypmIIId{121}  (0.0568,0.272) (0.0,0.272) (0.0,0.19)  (0.0568,0.19)
\polypmIIId{123}  (0.0568,0.3542)(0.0,0.3542)(0.0,0.272) (0.0568,0.272)

\polypmIIId{109} (0.1136,   0.19)  (0.0568,0.19)  (0.0568,0.1078)(0.1136,0.1078)
\polypmIIId{120}  (0.1136,   0.272) (0.0568,0.272) (0.0568,0.19)  (0.1136,0.19)
\polypmIIId{123}  (0.1136,   0.3542)(0.0568,0.3542)(0.0568,0.272) (0.1136,0.272)

\polypmIIId{111}(0.1704,0.19)  (0.1136,   0.19)  (0.1136,   0.1078)(0.1704,0.1078)
\polypmIIId{120} (0.1704,0.272) (0.1136,   0.272) (0.1136,   0.19)  (0.1704,0.19)
\polypmIIId{123}  (0.1704,0.3542)(0.1136,   0.3542)(0.1136,   0.272) (0.1704,0.272)

\polypmIIId{110}(0.2272,0.19)  (0.1704,0.19)  (0.1704,0.1078)(0.2272,0.1078)
\polypmIIId{120}  (0.2272,0.272) (0.1704,0.272) (0.1704,0.19)  (0.2272,0.19)
\polypmIIId{123}  (0.2272,0.3542)(0.1704,0.3542)(0.1704,0.272) (0.2272,0.272)

\rput(0.0284,0.07){3}
\rput(0.0852,0.07){4}
\rput(0.1420,0.07){5}
\rput(0.1988,0.07){6}
\rput(0.1136,0.0070){years}


\PST@Border(0.0,0.3542)
(0.0,0.1078)
(0.2272,0.1078)
(0.2272,0.3542)
(0.0,0.3542)

\catcode`@=12
\fi
\endpspicture}
    \subfloat[4 resources]{% GNUPLOT: LaTeX picture using PSTRICKS macros
% Define new PST objects, if not already defined
\ifx\PSTloaded\undefined
\def\PSTloaded{t}

\catcode`@=11

\newpsobject{PST@Border}{psline}{linewidth=.0015,linestyle=solid}

\catcode`@=12

\fi
\psset{unit=5.0in,xunit=5.0in,yunit=3.0in}
\pspicture(0.000000,0.000000)(0.3136,0.35)
\ifx\nofigs\undefined
\catcode`@=11

\newrgbcolor{PST@COLOR0}{1 1 1}
\newrgbcolor{PST@COLOR1}{0.992 0.992 0.992}
\newrgbcolor{PST@COLOR2}{0.984 0.984 0.984}
\newrgbcolor{PST@COLOR3}{0.976 0.976 0.976}
\newrgbcolor{PST@COLOR4}{0.968 0.968 0.968}
\newrgbcolor{PST@COLOR5}{0.96 0.96 0.96}
\newrgbcolor{PST@COLOR6}{0.952 0.952 0.952}
\newrgbcolor{PST@COLOR7}{0.944 0.944 0.944}
\newrgbcolor{PST@COLOR8}{0.937 0.937 0.937}
\newrgbcolor{PST@COLOR9}{0.929 0.929 0.929}
\newrgbcolor{PST@COLOR10}{0.921 0.921 0.921}
\newrgbcolor{PST@COLOR11}{0.913 0.913 0.913}
\newrgbcolor{PST@COLOR12}{0.905 0.905 0.905}
\newrgbcolor{PST@COLOR13}{0.897 0.897 0.897}
\newrgbcolor{PST@COLOR14}{0.889 0.889 0.889}
\newrgbcolor{PST@COLOR15}{0.881 0.881 0.881}
\newrgbcolor{PST@COLOR16}{0.874 0.874 0.874}
\newrgbcolor{PST@COLOR17}{0.866 0.866 0.866}
\newrgbcolor{PST@COLOR18}{0.858 0.858 0.858}
\newrgbcolor{PST@COLOR19}{0.85 0.85 0.85}
\newrgbcolor{PST@COLOR20}{0.842 0.842 0.842}
\newrgbcolor{PST@COLOR21}{0.834 0.834 0.834}
\newrgbcolor{PST@COLOR22}{0.826 0.826 0.826}
\newrgbcolor{PST@COLOR23}{0.818 0.818 0.818}
\newrgbcolor{PST@COLOR24}{0.811 0.811 0.811}
\newrgbcolor{PST@COLOR25}{0.803 0.803 0.803}
\newrgbcolor{PST@COLOR26}{0.795 0.795 0.795}
\newrgbcolor{PST@COLOR27}{0.787 0.787 0.787}
\newrgbcolor{PST@COLOR28}{0.779 0.779 0.779}
\newrgbcolor{PST@COLOR29}{0.771 0.771 0.771}
\newrgbcolor{PST@COLOR30}{0.763 0.763 0.763}
\newrgbcolor{PST@COLOR31}{0.755 0.755 0.755}
\newrgbcolor{PST@COLOR32}{0.748 0.748 0.748}
\newrgbcolor{PST@COLOR33}{0.74 0.74 0.74}
\newrgbcolor{PST@COLOR34}{0.732 0.732 0.732}
\newrgbcolor{PST@COLOR35}{0.724 0.724 0.724}
\newrgbcolor{PST@COLOR36}{0.716 0.716 0.716}
\newrgbcolor{PST@COLOR37}{0.708 0.708 0.708}
\newrgbcolor{PST@COLOR38}{0.7 0.7 0.7}
\newrgbcolor{PST@COLOR39}{0.692 0.692 0.692}
\newrgbcolor{PST@COLOR40}{0.685 0.685 0.685}
\newrgbcolor{PST@COLOR41}{0.677 0.677 0.677}
\newrgbcolor{PST@COLOR42}{0.669 0.669 0.669}
\newrgbcolor{PST@COLOR43}{0.661 0.661 0.661}
\newrgbcolor{PST@COLOR44}{0.653 0.653 0.653}
\newrgbcolor{PST@COLOR45}{0.645 0.645 0.645}
\newrgbcolor{PST@COLOR46}{0.637 0.637 0.637}
\newrgbcolor{PST@COLOR47}{0.629 0.629 0.629}
\newrgbcolor{PST@COLOR48}{0.622 0.622 0.622}
\newrgbcolor{PST@COLOR49}{0.614 0.614 0.614}
\newrgbcolor{PST@COLOR50}{0.606 0.606 0.606}
\newrgbcolor{PST@COLOR51}{0.598 0.598 0.598}
\newrgbcolor{PST@COLOR52}{0.59 0.59 0.59}
\newrgbcolor{PST@COLOR53}{0.582 0.582 0.582}
\newrgbcolor{PST@COLOR54}{0.574 0.574 0.574}
\newrgbcolor{PST@COLOR55}{0.566 0.566 0.566}
\newrgbcolor{PST@COLOR56}{0.559 0.559 0.559}
\newrgbcolor{PST@COLOR57}{0.551 0.551 0.551}
\newrgbcolor{PST@COLOR58}{0.543 0.543 0.543}
\newrgbcolor{PST@COLOR59}{0.535 0.535 0.535}
\newrgbcolor{PST@COLOR60}{0.527 0.527 0.527}
\newrgbcolor{PST@COLOR61}{0.519 0.519 0.519}
\newrgbcolor{PST@COLOR62}{0.511 0.511 0.511}
\newrgbcolor{PST@COLOR63}{0.503 0.503 0.503}
\newrgbcolor{PST@COLOR64}{0.496 0.496 0.496}
\newrgbcolor{PST@COLOR65}{0.488 0.488 0.488}
\newrgbcolor{PST@COLOR66}{0.48 0.48 0.48}
\newrgbcolor{PST@COLOR67}{0.472 0.472 0.472}
\newrgbcolor{PST@COLOR68}{0.464 0.464 0.464}
\newrgbcolor{PST@COLOR69}{0.456 0.456 0.456}
\newrgbcolor{PST@COLOR70}{0.448 0.448 0.448}
\newrgbcolor{PST@COLOR71}{0.44 0.44 0.44}
\newrgbcolor{PST@COLOR72}{0.433 0.433 0.433}
\newrgbcolor{PST@COLOR73}{0.425 0.425 0.425}
\newrgbcolor{PST@COLOR74}{0.417 0.417 0.417}
\newrgbcolor{PST@COLOR75}{0.409 0.409 0.409}
\newrgbcolor{PST@COLOR76}{0.401 0.401 0.401}
\newrgbcolor{PST@COLOR77}{0.393 0.393 0.393}
\newrgbcolor{PST@COLOR78}{0.385 0.385 0.385}
\newrgbcolor{PST@COLOR79}{0.377 0.377 0.377}
\newrgbcolor{PST@COLOR80}{0.37 0.37 0.37}
\newrgbcolor{PST@COLOR81}{0.362 0.362 0.362}
\newrgbcolor{PST@COLOR82}{0.354 0.354 0.354}
\newrgbcolor{PST@COLOR83}{0.346 0.346 0.346}
\newrgbcolor{PST@COLOR84}{0.338 0.338 0.338}
\newrgbcolor{PST@COLOR85}{0.33 0.33 0.33}
\newrgbcolor{PST@COLOR86}{0.322 0.322 0.322}
\newrgbcolor{PST@COLOR87}{0.314 0.314 0.314}
\newrgbcolor{PST@COLOR88}{0.307 0.307 0.307}
\newrgbcolor{PST@COLOR89}{0.299 0.299 0.299}
\newrgbcolor{PST@COLOR90}{0.291 0.291 0.291}
\newrgbcolor{PST@COLOR91}{0.283 0.283 0.283}
\newrgbcolor{PST@COLOR92}{0.275 0.275 0.275}
\newrgbcolor{PST@COLOR93}{0.267 0.267 0.267}
\newrgbcolor{PST@COLOR94}{0.259 0.259 0.259}
\newrgbcolor{PST@COLOR95}{0.251 0.251 0.251}
\newrgbcolor{PST@COLOR96}{0.244 0.244 0.244}
\newrgbcolor{PST@COLOR97}{0.236 0.236 0.236}
\newrgbcolor{PST@COLOR98}{0.228 0.228 0.228}
\newrgbcolor{PST@COLOR99}{0.22 0.22 0.22}
\newrgbcolor{PST@COLOR100}{0.212 0.212 0.212}
\newrgbcolor{PST@COLOR101}{0.204 0.204 0.204}
\newrgbcolor{PST@COLOR102}{0.196 0.196 0.196}
\newrgbcolor{PST@COLOR103}{0.188 0.188 0.188}
\newrgbcolor{PST@COLOR104}{0.181 0.181 0.181}
\newrgbcolor{PST@COLOR105}{0.173 0.173 0.173}
\newrgbcolor{PST@COLOR106}{0.165 0.165 0.165}
\newrgbcolor{PST@COLOR107}{0.157 0.157 0.157}
\newrgbcolor{PST@COLOR108}{0.149 0.149 0.149}
\newrgbcolor{PST@COLOR109}{0.141 0.141 0.141}
\newrgbcolor{PST@COLOR110}{0.133 0.133 0.133}
\newrgbcolor{PST@COLOR111}{0.125 0.125 0.125}
\newrgbcolor{PST@COLOR112}{0.118 0.118 0.118}
\newrgbcolor{PST@COLOR113}{0.11 0.11 0.11}
\newrgbcolor{PST@COLOR114}{0.102 0.102 0.102}
\newrgbcolor{PST@COLOR115}{0.094 0.094 0.094}
\newrgbcolor{PST@COLOR116}{0.086 0.086 0.086}
\newrgbcolor{PST@COLOR117}{0.078 0.078 0.078}
\newrgbcolor{PST@COLOR118}{0.07 0.07 0.07}
\newrgbcolor{PST@COLOR119}{0.062 0.062 0.062}
\newrgbcolor{PST@COLOR120}{0.055 0.055 0.055}
\newrgbcolor{PST@COLOR121}{0.047 0.047 0.047}
\newrgbcolor{PST@COLOR122}{0.039 0.039 0.039}
\newrgbcolor{PST@COLOR123}{0.031 0.031 0.031}
\newrgbcolor{PST@COLOR124}{0.023 0.023 0.023}
\newrgbcolor{PST@COLOR125}{0.015 0.015 0.015}
\newrgbcolor{PST@COLOR126}{0.007 0.007 0.007}
\newrgbcolor{PST@COLOR127}{0 0 0}

\def\polypmIIId#1{\pspolygon[linestyle=none,fillstyle=solid,fillcolor=PST@COLOR#1]}

\polypmIIId{99} (0.0568,0.19)  (0.0,0.19)  (0.0,0.1078)(0.0568,0.1078)
\polypmIIId{111}  (0.0568,0.272) (0.0,0.272) (0.0,0.19)  (0.0568,0.19)
\polypmIIId{121}  (0.0568,0.3542)(0.0,0.3542)(0.0,0.272) (0.0568,0.272)

\polypmIIId{96} (0.1136,   0.19)  (0.0568,0.19)  (0.0568,0.1078)(0.1136,0.1078)
\polypmIIId{115}  (0.1136,   0.272) (0.0568,0.272) (0.0568,0.19)  (0.1136,0.19)
\polypmIIId{120}  (0.1136,   0.3542)(0.0568,0.3542)(0.0568,0.272) (0.1136,0.272)

\polypmIIId{99}(0.1704,0.19)  (0.1136,   0.19)  (0.1136,   0.1078)(0.1704,0.1078)
\polypmIIId{113} (0.1704,0.272) (0.1136,   0.272) (0.1136,   0.19)  (0.1704,0.19)
\polypmIIId{121}  (0.1704,0.3542)(0.1136,   0.3542)(0.1136,   0.272) (0.1704,0.272)

\polypmIIId{100}(0.2272,0.19)  (0.1704,0.19)  (0.1704,0.1078)(0.2272,0.1078)
\polypmIIId{113}  (0.2272,0.272) (0.1704,0.272) (0.1704,0.19)  (0.2272,0.19)
\polypmIIId{121}  (0.2272,0.3542)(0.1704,0.3542)(0.1704,0.272) (0.2272,0.272)

\rput(0.0284,0.07){3}
\rput(0.0852,0.07){4}
\rput(0.1420,0.07){5}
\rput(0.1988,0.07){6}
\rput(0.1136,0.0070){years}

\PST@Border(0.0,0.3542)
(0.0,0.1078)
(0.2272,0.1078)
(0.2272,0.3542)
(0.0,0.3542)

\polypmIIId{0}(0.2329,0.1078)(0.2442,0.1078)(0.2442,0.1098)(0.2329,0.1098)
\polypmIIId{1}(0.2329,0.1097)(0.2442,0.1097)(0.2442,0.1117)(0.2329,0.1117)
\polypmIIId{2}(0.2329,0.1116)(0.2442,0.1116)(0.2442,0.1136)(0.2329,0.1136)
\polypmIIId{3}(0.2329,0.1135)(0.2442,0.1135)(0.2442,0.1156)(0.2329,0.1156)
\polypmIIId{4}(0.2329,0.1155)(0.2442,0.1155)(0.2442,0.1175)(0.2329,0.1175)
\polypmIIId{5}(0.2329,0.1174)(0.2442,0.1174)(0.2442,0.1194)(0.2329,0.1194)
\polypmIIId{6}(0.2329,0.1193)(0.2442,0.1193)(0.2442,0.1213)(0.2329,0.1213)
\polypmIIId{7}(0.2329,0.1212)(0.2442,0.1212)(0.2442,0.1233)(0.2329,0.1233)
\polypmIIId{8}(0.2329,0.1232)(0.2442,0.1232)(0.2442,0.1252)(0.2329,0.1252)
\polypmIIId{9}(0.2329,0.1251)(0.2442,0.1251)(0.2442,0.1271)(0.2329,0.1271)
\polypmIIId{10}(0.2329,0.127)(0.2442,0.127)(0.2442,0.129)(0.2329,0.129)
\polypmIIId{11}(0.2329,0.1289)(0.2442,0.1289)(0.2442,0.131)(0.2329,0.131)
\polypmIIId{12}(0.2329,0.1309)(0.2442,0.1309)(0.2442,0.1329)(0.2329,0.1329)
\polypmIIId{13}(0.2329,0.1328)(0.2442,0.1328)(0.2442,0.1348)(0.2329,0.1348)
\polypmIIId{14}(0.2329,0.1347)(0.2442,0.1347)(0.2442,0.1367)(0.2329,0.1367)
\polypmIIId{15}(0.2329,0.1366)(0.2442,0.1366)(0.2442,0.1387)(0.2329,0.1387)
\polypmIIId{16}(0.2329,0.1386)(0.2442,0.1386)(0.2442,0.1406)(0.2329,0.1406)
\polypmIIId{17}(0.2329,0.1405)(0.2442,0.1405)(0.2442,0.1425)(0.2329,0.1425)
\polypmIIId{18}(0.2329,0.1424)(0.2442,0.1424)(0.2442,0.1444)(0.2329,0.1444)
\polypmIIId{19}(0.2329,0.1443)(0.2442,0.1443)(0.2442,0.1464)(0.2329,0.1464)
\polypmIIId{20}(0.2329,0.1463)(0.2442,0.1463)(0.2442,0.1483)(0.2329,0.1483)
\polypmIIId{21}(0.2329,0.1482)(0.2442,0.1482)(0.2442,0.1502)(0.2329,0.1502)
\polypmIIId{22}(0.2329,0.1501)(0.2442,0.1501)(0.2442,0.1521)(0.2329,0.1521)
\polypmIIId{23}(0.2329,0.152)(0.2442,0.152)(0.2442,0.1541)(0.2329,0.1541)
\polypmIIId{24}(0.2329,0.154)(0.2442,0.154)(0.2442,0.156)(0.2329,0.156)
\polypmIIId{25}(0.2329,0.1559)(0.2442,0.1559)(0.2442,0.1579)(0.2329,0.1579)
\polypmIIId{26}(0.2329,0.1578)(0.2442,0.1578)(0.2442,0.1598)(0.2329,0.1598)
\polypmIIId{27}(0.2329,0.1597)(0.2442,0.1597)(0.2442,0.1618)(0.2329,0.1618)
\polypmIIId{28}(0.2329,0.1617)(0.2442,0.1617)(0.2442,0.1637)(0.2329,0.1637)
\polypmIIId{29}(0.2329,0.1636)(0.2442,0.1636)(0.2442,0.1656)(0.2329,0.1656)
\polypmIIId{30}(0.2329,0.1655)(0.2442,0.1655)(0.2442,0.1675)(0.2329,0.1675)
\polypmIIId{31}(0.2329,0.1674)(0.2442,0.1674)(0.2442,0.1695)(0.2329,0.1695)
\polypmIIId{32}(0.2329,0.1694)(0.2442,0.1694)(0.2442,0.1714)(0.2329,0.1714)
\polypmIIId{33}(0.2329,0.1713)(0.2442,0.1713)(0.2442,0.1733)(0.2329,0.1733)
\polypmIIId{34}(0.2329,0.1732)(0.2442,0.1732)(0.2442,0.1752)(0.2329,0.1752)
\polypmIIId{35}(0.2329,0.1751)(0.2442,0.1751)(0.2442,0.1772)(0.2329,0.1772)
\polypmIIId{36}(0.2329,0.1771)(0.2442,0.1771)(0.2442,0.1791)(0.2329,0.1791)
\polypmIIId{37}(0.2329,0.179)(0.2442,0.179)(0.2442,0.181)(0.2329,0.181)
\polypmIIId{38}(0.2329,0.1809)(0.2442,0.1809)(0.2442,0.1829)(0.2329,0.1829)
\polypmIIId{39}(0.2329,0.1828)(0.2442,0.1828)(0.2442,0.1849)(0.2329,0.1849)
\polypmIIId{40}(0.2329,0.1848)(0.2442,0.1848)(0.2442,0.1868)(0.2329,0.1868)
\polypmIIId{41}(0.2329,0.1867)(0.2442,0.1867)(0.2442,0.1887)(0.2329,0.1887)
\polypmIIId{42}(0.2329,0.1886)(0.2442,0.1886)(0.2442,0.1906)(0.2329,0.1906)
\polypmIIId{43}(0.2329,0.1905)(0.2442,0.1905)(0.2442,0.1926)(0.2329,0.1926)
\polypmIIId{44}(0.2329,0.1925)(0.2442,0.1925)(0.2442,0.1945)(0.2329,0.1945)
\polypmIIId{45}(0.2329,0.1944)(0.2442,0.1944)(0.2442,0.1964)(0.2329,0.1964)
\polypmIIId{46}(0.2329,0.1963)(0.2442,0.1963)(0.2442,0.1983)(0.2329,0.1983)
\polypmIIId{47}(0.2329,0.1982)(0.2442,0.1982)(0.2442,0.2003)(0.2329,0.2003)
\polypmIIId{48}(0.2329,0.2002)(0.2442,0.2002)(0.2442,0.2022)(0.2329,0.2022)
\polypmIIId{49}(0.2329,0.2021)(0.2442,0.2021)(0.2442,0.2041)(0.2329,0.2041)
\polypmIIId{50}(0.2329,0.204)(0.2442,0.204)(0.2442,0.206)(0.2329,0.206)
\polypmIIId{51}(0.2329,0.2059)(0.2442,0.2059)(0.2442,0.208)(0.2329,0.208)
\polypmIIId{52}(0.2329,0.2079)(0.2442,0.2079)(0.2442,0.2099)(0.2329,0.2099)
\polypmIIId{53}(0.2329,0.2098)(0.2442,0.2098)(0.2442,0.2118)(0.2329,0.2118)
\polypmIIId{54}(0.2329,0.2117)(0.2442,0.2117)(0.2442,0.2137)(0.2329,0.2137)
\polypmIIId{55}(0.2329,0.2136)(0.2442,0.2136)(0.2442,0.2157)(0.2329,0.2157)
\polypmIIId{56}(0.2329,0.2156)(0.2442,0.2156)(0.2442,0.2176)(0.2329,0.2176)
\polypmIIId{57}(0.2329,0.2175)(0.2442,0.2175)(0.2442,0.2195)(0.2329,0.2195)
\polypmIIId{58}(0.2329,0.2194)(0.2442,0.2194)(0.2442,0.2214)(0.2329,0.2214)
\polypmIIId{59}(0.2329,0.2213)(0.2442,0.2213)(0.2442,0.2234)(0.2329,0.2234)
\polypmIIId{60}(0.2329,0.2233)(0.2442,0.2233)(0.2442,0.2253)(0.2329,0.2253)
\polypmIIId{61}(0.2329,0.2252)(0.2442,0.2252)(0.2442,0.2272)(0.2329,0.2272)
\polypmIIId{62}(0.2329,0.2271)(0.2442,0.2271)(0.2442,0.2291)(0.2329,0.2291)
\polypmIIId{63}(0.2329,0.229)(0.2442,0.229)(0.2442,0.2311)(0.2329,0.2311)
\polypmIIId{64}(0.2329,0.231)(0.2442,0.231)(0.2442,0.233)(0.2329,0.233)
\polypmIIId{65}(0.2329,0.2329)(0.2442,0.2329)(0.2442,0.2349)(0.2329,0.2349)
\polypmIIId{66}(0.2329,0.2348)(0.2442,0.2348)(0.2442,0.2368)(0.2329,0.2368)
\polypmIIId{67}(0.2329,0.2367)(0.2442,0.2367)(0.2442,0.2388)(0.2329,0.2388)
\polypmIIId{68}(0.2329,0.2387)(0.2442,0.2387)(0.2442,0.2407)(0.2329,0.2407)
\polypmIIId{69}(0.2329,0.2406)(0.2442,0.2406)(0.2442,0.2426)(0.2329,0.2426)
\polypmIIId{70}(0.2329,0.2425)(0.2442,0.2425)(0.2442,0.2445)(0.2329,0.2445)
\polypmIIId{71}(0.2329,0.2444)(0.2442,0.2444)(0.2442,0.2465)(0.2329,0.2465)
\polypmIIId{72}(0.2329,0.2464)(0.2442,0.2464)(0.2442,0.2484)(0.2329,0.2484)
\polypmIIId{73}(0.2329,0.2483)(0.2442,0.2483)(0.2442,0.2503)(0.2329,0.2503)
\polypmIIId{74}(0.2329,0.2502)(0.2442,0.2502)(0.2442,0.2522)(0.2329,0.2522)
\polypmIIId{75}(0.2329,0.2521)(0.2442,0.2521)(0.2442,0.2542)(0.2329,0.2542)
\polypmIIId{76}(0.2329,0.2541)(0.2442,0.2541)(0.2442,0.2561)(0.2329,0.2561)
\polypmIIId{77}(0.2329,0.256)(0.2442,0.256)(0.2442,0.258)(0.2329,0.258)
\polypmIIId{78}(0.2329,0.2579)(0.2442,0.2579)(0.2442,0.2599)(0.2329,0.2599)
\polypmIIId{79}(0.2329,0.2598)(0.2442,0.2598)(0.2442,0.2619)(0.2329,0.2619)
\polypmIIId{80}(0.2329,0.2618)(0.2442,0.2618)(0.2442,0.2638)(0.2329,0.2638)
\polypmIIId{81}(0.2329,0.2637)(0.2442,0.2637)(0.2442,0.2657)(0.2329,0.2657)
\polypmIIId{82}(0.2329,0.2656)(0.2442,0.2656)(0.2442,0.2676)(0.2329,0.2676)
\polypmIIId{83}(0.2329,0.2675)(0.2442,0.2675)(0.2442,0.2696)(0.2329,0.2696)
\polypmIIId{84}(0.2329,0.2695)(0.2442,0.2695)(0.2442,0.2715)(0.2329,0.2715)
\polypmIIId{85}(0.2329,0.2714)(0.2442,0.2714)(0.2442,0.2734)(0.2329,0.2734)
\polypmIIId{86}(0.2329,0.2733)(0.2442,0.2733)(0.2442,0.2753)(0.2329,0.2753)
\polypmIIId{87}(0.2329,0.2752)(0.2442,0.2752)(0.2442,0.2773)(0.2329,0.2773)
\polypmIIId{88}(0.2329,0.2772)(0.2442,0.2772)(0.2442,0.2792)(0.2329,0.2792)
\polypmIIId{89}(0.2329,0.2791)(0.2442,0.2791)(0.2442,0.2811)(0.2329,0.2811)
\polypmIIId{90}(0.2329,0.281)(0.2442,0.281)(0.2442,0.283)(0.2329,0.283)
\polypmIIId{91}(0.2329,0.2829)(0.2442,0.2829)(0.2442,0.285)(0.2329,0.285)
\polypmIIId{92}(0.2329,0.2849)(0.2442,0.2849)(0.2442,0.2869)(0.2329,0.2869)
\polypmIIId{93}(0.2329,0.2868)(0.2442,0.2868)(0.2442,0.2888)(0.2329,0.2888)
\polypmIIId{94}(0.2329,0.2887)(0.2442,0.2887)(0.2442,0.2907)(0.2329,0.2907)
\polypmIIId{95}(0.2329,0.2906)(0.2442,0.2906)(0.2442,0.2927)(0.2329,0.2927)
\polypmIIId{96}(0.2329,0.2926)(0.2442,0.2926)(0.2442,0.2946)(0.2329,0.2946)
\polypmIIId{97}(0.2329,0.2945)(0.2442,0.2945)(0.2442,0.2965)(0.2329,0.2965)
\polypmIIId{98}(0.2329,0.2964)(0.2442,0.2964)(0.2442,0.2984)(0.2329,0.2984)
\polypmIIId{99}(0.2329,0.2983)(0.2442,0.2983)(0.2442,0.3004)(0.2329,0.3004)
\polypmIIId{100}(0.2329,0.3003)(0.2442,0.3003)(0.2442,0.3023)(0.2329,0.3023)
\polypmIIId{101}(0.2329,0.3022)(0.2442,0.3022)(0.2442,0.3042)(0.2329,0.3042)
\polypmIIId{102}(0.2329,0.3041)(0.2442,0.3041)(0.2442,0.3061)(0.2329,0.3061)
\polypmIIId{103}(0.2329,0.306)(0.2442,0.306)(0.2442,0.3081)(0.2329,0.3081)
\polypmIIId{104}(0.2329,0.308)(0.2442,0.308)(0.2442,0.31)(0.2329,0.31)
\polypmIIId{105}(0.2329,0.3099)(0.2442,0.3099)(0.2442,0.3119)(0.2329,0.3119)
\polypmIIId{106}(0.2329,0.3118)(0.2442,0.3118)(0.2442,0.3138)(0.2329,0.3138)
\polypmIIId{107}(0.2329,0.3137)(0.2442,0.3137)(0.2442,0.3158)(0.2329,0.3158)
\polypmIIId{108}(0.2329,0.3157)(0.2442,0.3157)(0.2442,0.3177)(0.2329,0.3177)
\polypmIIId{109}(0.2329,0.3176)(0.2442,0.3176)(0.2442,0.3196)(0.2329,0.3196)
\polypmIIId{110}(0.2329,0.3195)(0.2442,0.3195)(0.2442,0.3215)(0.2329,0.3215)
\polypmIIId{111}(0.2329,0.3214)(0.2442,0.3214)(0.2442,0.3235)(0.2329,0.3235)
\polypmIIId{112}(0.2329,0.3234)(0.2442,0.3234)(0.2442,0.3254)(0.2329,0.3254)
\polypmIIId{113}(0.2329,0.3253)(0.2442,0.3253)(0.2442,0.3273)(0.2329,0.3273)
\polypmIIId{114}(0.2329,0.3272)(0.2442,0.3272)(0.2442,0.3292)(0.2329,0.3292)
\polypmIIId{115}(0.2329,0.3291)(0.2442,0.3291)(0.2442,0.3312)(0.2329,0.3312)
\polypmIIId{116}(0.2329,0.3311)(0.2442,0.3311)(0.2442,0.3331)(0.2329,0.3331)
\polypmIIId{117}(0.2329,0.333)(0.2442,0.333)(0.2442,0.335)(0.2329,0.335)
\polypmIIId{118}(0.2329,0.3349)(0.2442,0.3349)(0.2442,0.3369)(0.2329,0.3369)
\polypmIIId{119}(0.2329,0.3368)(0.2442,0.3368)(0.2442,0.3389)(0.2329,0.3389)
\polypmIIId{120}(0.2329,0.3388)(0.2442,0.3388)(0.2442,0.3408)(0.2329,0.3408)
\polypmIIId{121}(0.2329,0.3407)(0.2442,0.3407)(0.2442,0.3427)(0.2329,0.3427)
\polypmIIId{122}(0.2329,0.3426)(0.2442,0.3426)(0.2442,0.3446)(0.2329,0.3446)
\polypmIIId{123}(0.2329,0.3445)(0.2442,0.3445)(0.2442,0.3466)(0.2329,0.3466)
\polypmIIId{124}(0.2329,0.3465)(0.2442,0.3465)(0.2442,0.3485)(0.2329,0.3485)
\polypmIIId{125}(0.2329,0.3484)(0.2442,0.3484)(0.2442,0.3504)(0.2329,0.3504)
\polypmIIId{126}(0.2329,0.3503)(0.2442,0.3503)(0.2442,0.3523)(0.2329,0.3523)
\polypmIIId{127}(0.2329,0.3522)(0.2442,0.3522)(0.2442,0.3542)(0.2329,0.3542)

\PST@Border(0.2329,0.1078)
(0.2442,0.1078)
(0.2442,0.3542)
(0.2329,0.3542)
(0.2329,0.1078)


\rput[l](0.2502,0.1301){0.997}
\rput[l](0.2502,0.2048){0.998}
\rput[l](0.2502,0.2795){0.999}
\rput[l](0.2502,0.3542){1}

\catcode`@=12
\fi
\endpspicture}
  }
  $\alpha = 1.0$
  \caption{GALP solution quality on uncorrelated instances.}
  \label{fig:greedysolcomp10}
\end{figure}

Observing the figure a comparison between the two heuristics can also be made.
By comparing figure~\ref{fig:tabusolcomp10} to 
figure~\ref{fig:greedysolcomp10}, the first group 
of figure show a darker tone than the second. It indicates that on average the TSLP was able to find better
solutions than the GALP on the tested instances. All the experiments results are available under request to the authors.

Even considering that the heuristics starting point was already a good solution, since it was on average
99.652\% of the best known solution, the heuristics were able to improve the solutions. On average, 
the solutions found by the GALP were 99.912\% of the best known solutions, and the TSLP obtained 
even better results, with it's solutions being 99.967\% of the best known ones, on average.

%Regarding the execution time,
The TSLP had a maximum execution time of 2 minutes, with an average of 30 seconds.
GALP had a negligible running time, below 1 second for all instances.


%This works presents a modeling of a relevant problem to Electricity Distribution Companies in developing countries.
%The modeling yields a challenging optimization problem. We propose a exact technique to solve easier instances
%of the problem and two heuristics to tackle the more difficult instances.

%We conclude that our incarnation  of the classical Knapsack is sensible to the correlation between weight and cost, as
%predicted by the literature on the problem.

%We have tested two heuristic algorithm, namely Tabu Searh using the Linear Problem solution as an initial search point
%(TSLP) and Gradient Ascent using the Linear Problem solution as an initial search point (GALP), both achieving good solutions and
%execution times. In particular, the TSLP algorithm was statically better than the GALP algorithm
%in our test instances, with bigger running times.

%Future work includes the investigation an algorithm for the complete version of the problem and
%a deeper study of what makes some instances take more time than others despite 
%having the same amount of years, actions and $\alpha$ and being randomly generated
%in a similar way.

%\vspace{1cm}


\section*{Acknowledgement}
The authors wish to thank EDP Energias do Brasil S.A. for cooperation in this research.

% Can use something like this to put references on a page
% by themselves when using endfloat and the captionsoff option.
\ifCLASSOPTIONcaptionsoff
  \newpage
\fi

%\section*{References}

\bibliographystyle{ieeetr}
\bibliography{mybibfile}

\flushend

\end{document}


