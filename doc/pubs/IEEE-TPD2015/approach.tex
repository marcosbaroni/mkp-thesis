\section{Computational Approach}
\label{sec:approach}

To solve the POMMBKPDSA we used three approaches:

\begin{itemize}
  \item An exact approach, using integer programming techniques;
  \item a heuristic based on a greedy strategy, using the truncated solution of the linear relaxation of the problem, called GALP;
  \item a Tabu Search based heuristic, also using the truncated solution of the linear relaxation of the problem, called TSLP.
\end{itemize}

The first approach used, the exact approach, was to solve the formulation of the problem described in 
equation~\ref{eq:pmmlpo++} using the \textit{branch-and-cut} method~\cite{padberg1991branch} implemented on the
CPLEX solver.
%CPLEX solver~\cite{Cplex}.

Before explaining the other two approaches, it is important to explain the reason why both heuristics used 
the truncated solution of the LP-relaxed version of the problem as starting points. During our tests, we tried 
to use other alternatives as starting points for the heuristics, like empty solutions or randomly generated 
solutions. Still those alternatives never yielded better results than starting the heuristics from the truncated
solution of the LP-relaxed version of the problem. In fact, we found out that this solution was on the worst case
97.7\% of the solution of the LP-relaxed version of the problem, which is an upper bound, and at 99.65\% on the average case.
Being such a good starting point, and so easily obtainable by modern solvers, it seems a good choice to start the 
heuristic from this solution.

The second applied technique, the GALP, is a local search heuristic using a greedy strategy. Generally, greedy strategies have the disadvantage 
of not being able to escape any local optimum they reach. On the other hand, they usually converge faster than other methods, 
making them suitable alternatives when fast viable solutions are desired. Also, they are usually easy to understand and implement.

Fig~\ref{alg:galp} describes the greedy algorithm developed for the POMMBKPDSA. The heuristic receives as initial solution 
the truncated solution of the LP-relaxed version of the problem. It then tries to improve it by iteratively adding items to the 
knapsacks until no more additions are possible. It is divided in three phases: On the first phase, the algorithm generates a list
of all possible allocations which may be made on a solution. Those alocations are all the possible combinations of item vs. knapsack, 
and represent the addition of a certain item to a knapsack. That list is then sorted by decreasing order of each alocation's efficiency,
i.e. how much each alocation contributes to the increase of the objective function. In the last phase, that sorted list of alocations
is used to iterativelly improve the initial solution.

\begin{figure}

\begin{algorithmic}[1]
\REQUIRE Initial Solution: $S_{initial}$
\ENSURE Best Solution: $S_{best}$
\STATE $S_{best} \gets S_{initial}$ 
\STATE $AlocationList \gets GenerateAlocations()$
  \label{alg:galp:gera}
\STATE $AlocationList \gets DecreasingOrder(AlocationList)$
  \label{alg:galp:ordena}
\STATE $Improved \gets \TRUE$ 
\WHILE {$Improved$}
  \label{alg:galp:inicioloop}
  \STATE $Improved \gets \FALSE$
  \FOR{$Alocation \in AlocationList \ \&\& \ not\  Improved$}
    \label{alg:galp:inicioloopinterno}
    \IF{$isViable(S_{best}, Alocation)$}
      \STATE $S_{best} \gets S_{best} + Alocation$
	    \label{alg:galp:melhoria}
      \STATE $Improved \gets \TRUE$ 
    \ENDIF
  \ENDFOR
  \label{alg:galp:fimloopinterno}
\ENDWHILE
\label{alg:galp:fimloop}
\RETURN $S_{best}$
\end{algorithmic}

\caption{Greedy Algorithm}
\label{alg:galp}
\end{figure}

At line~\ref{alg:galp:gera}, GenerateAlocations() function generates a list with all the possible alocations of items in knapsacks. 
For example, for an instance of the problem with two knapsacks and three items, the list generated would be \{(1,1), (1,2), (2,1), (2,2), (3,1), (3,2)\},
where each pair from the list represents adding one copy of an item to a knapsack. On the sequence, on line~\ref{alg:galp:ordena} that list is 
ordered by descending order of efficiency, i.e. how much each item contributes to the increase on the objective function's value. The initial solution is 
then iteratively improved on the loop between lines~\ref{alg:galp:inicioloop} 
and~\ref{alg:galp:fimloop}. At each step, the alocation list is searched for the alocation with the highest efficiency which leads to a viable solution.
Once that alocation is found, the currently best solution is updated with the alocation (line~\ref{alg:galp:melhoria}) and the loop continues, until no
alocation can lead to a viable solution. At that point, the algorithm stops and returns the best solution found.

The last applied heuristic, the TSLP, is an implementation of the metaheuristic Tabu Search~\cite{glover1989tabup1}. This metaheuristic has been heavilly used on the
last decades to tackle Combinatorial Optimization problems, has already been applied on the dKP in~\cite{danmeyer1993} and for some time the Tabu Search presented at~\cite{vasquez2001} 
and improved at~\cite{vasquez2005} was the method which obtained the best results for the dKP intances commonly used in the litarature.

Like the greedy algorithm previously presented, the Tabu Search is also essentially a local search method, so it should also present a tendency to 
become stuck at local optima. To work around this problem, the Tabu Search allows that during the search, when no solution better than the current one is found,
a worse solution may be chosen with the spectation that this choice will lead to better solutions on the long run. Also, in this method, the characteristics
of all the choices made during the search may be taken in consideration when making the next choice. Because of those characteristics, the Tabu Search is usually
able to escape local optima and find better solutions than other simple local searches.

Before explaining the implemented heuristic, it's useful to explain some terms which will be used ahead. The first of them are the 
\textit{moves}. Moves are the operations applied on a solution which takes them to another solution. The set of solutions which can be reached
from solution $x$ by applying one of the possible movements is called x's \textit{Neighborhood}. One characteristic component of the Tabu Search is the
\textit{Tabu List}. This list usually contains moves or characteristics that lead to unwanted solutions, and is used during the search to forbid certain moves
as a way to escape local optima and hopefully guide the search towards better solutions.

Fig.~\ref{alg:tabu} describes the implemented heuristic. It receives the 
initial solution and tries to improve that solution by iteractively searching its neighborhood.

\begin{figure}
\begin{algorithmic}[1]
\REQUIRE Initial Solution: $S_{initial}$
\ENSURE Best Solution: $S_{best}$
\STATE $S_{current} \gets S_{initial}$
\STATE $S_{best} \gets S_{initial}$ 
\STATE $ListaTabu \gets \emptyset$
\WHILE{$\neg stoppingCondition$}
  \label{alg:tabu:inicioloop}
  \STATE $moves \gets Neighborhood(S_{current})$
    \label{alg:tabu:vizinhanca}
  \STATE $move \gets BestNonTabuMove(moves)$
    \label{alg:tabu:melhormovimento}
  \STATE $S_{current} \gets S_{current} + move$
    \label{alg:tabu:movimento}
  \STATE $TabuList \gets TabuList + move$
    \label{alg:tabu:tabu}
  \IF{$S_{current} \geq S_{best}$}
    \STATE $S_{best} \gets S_{current}$
      \label{alg:tabu:melhor}
     \STATE $ResetTabuList()$
  \ENDIF
\ENDWHILE
  \label{alg:tabu:fimloop}
\RETURN $S_{best}$
\end{algorithmic}
\caption{Tabu Search}
\label{alg:tabu}
\end{figure}

The most relevant part of the heuristic happens on the loop between lines~\ref{alg:tabu:inicioloop}
and~\ref{alg:tabu:fimloop}. Firstly, all the moves leading to neighbours of the current solutions are obtained at 
line~\ref{alg:tabu:vizinhanca}. At line~\ref{alg:tabu:melhormovimento} the function \textit{BestMoveNotTabu()} searches
the movement list for the move which will lead to the greatest value for the objective function and is not forbidden
by the Tabu List. The current solution is then updated with the chosen move (line~\ref{alg:tabu:movimento}), 
the Tabu List is also updated with the chosen move (line ~\ref{alg:tabu:tabu}) 
and finally if the current solution is better than the best solution found so far, the best solution is updated 
with the current solution (line~\ref{alg:tabu:melhor}) and the tabu list is reset.

An important aspect to be considered when implementing a Tabu Search is how to implement and manage the Tabu List (TL).
In this work, a dynamic TL management method was implemented based on the Reverse Elimination Method (REM)~\cite{glover1990tabup2}. This method leads to
an exact tabu status, meaning that any move present in the TL leads to a solution that was already visited by the search. The method works
by adding the move made at each iteraction to the Tabu List. When a new move is added to the Tabu List, the list is traced in reverse order
to build an Active Tabu List (ATL), which contains every move that will lead to a solution already visited during the search. Now, in order to check if a move
is forbidden, all we have to do is check if it is contained in the ATL. When a new global optimum is found during the search, the TL is cleared.
For more details on the Reverse Elimination Method the reader is directed to~\cite{glover1990tabup2}.






















