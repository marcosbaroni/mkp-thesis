\begin{abstract}
Electricity losses during distribution due to non-technical sources are among the major causes of 
profit loss for electricity distribution companies (EDCOs). In Brasil, part of that profit loss can be
recovered through an increase in the energy bill. However, the maximum value of that increase is 
limited by the regulatory agency in the form of non-technical loss reduction goals. The optimization 
problem addressed in this work treats the loss reduction problem from the EDCOs' point of view. 
In order to achieve the reduction goals established by the regulatory agency, EDCOs
have several loss reduction actions which are allocated in multiyear loss reduction plans. 
These plans try to achieve the goals without exceeding some predefined budgets,
always aiming to obtain the highest possible profit for the EDCO. This work approaches
the problem of the plan's definition as a generalization of the Knapsack Problem.
A formal model is defined as an integer programming problem, and it's hardness is analysed
through computational experiments using a generic solver applied to a variety of
artificial instances. Two heuristics are then proposed, the first
based in a greedy approach and the second based on the Tabu Search metaheuristic, and applied
to the problem. Finally, the three approaches are compared considering the quality of the
solutions found.
\end{abstract}

\begin{keyword}
OR in energy, Combinatorial optimization, Metaheuristics
\end{keyword}
