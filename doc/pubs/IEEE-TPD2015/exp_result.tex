\section{Experimental Results}
\label{sec:exp_results}

In order to evaluate the hardness of the instances created with the generator and the effectiveness of the proposed
solution approaches, computational tests were conducted using both heuristics and the generic solver CPLEX~\cite{Cplex}.

Firstly, the created instances were solved with the CPLEX solver version 12.5.0, in order to verify their hardness. 
The solver was used in its default configuration. Then, the GALP and TSLP algorithms were executed on the same instances.
While GALP doesn't use any adjustable parameter, the TSLP has the stopping condition, which was 10000 iterations in our
tests. All the tests were executed on Intel Core i5-3570 @ 3.40 GHz machines with 8GB RAM memory and executing Ubuntu 13.04.
Both heuristics were implemented in Java and run on the 1.7 JVM.

To execute the tests a suite of artificial benchmark instances was created using the generator presented at the last session.
The instances were created for all combinations of 3 to 6 years, 25, 50 and 100 actions, 1, 2 and 4 resources and
correlation levels ($\alpha$) of 0.0, 0.1 and 1.0. Those numbers of years, actions and resources were chosen to simulate 
the dimensions of the instances expected to be found in practice. The $\alpha$ values were chosen to verify if this 
generalization of the KP is also sensitive to the correlation level on the items profits and costs, as predicted
by the KP literature~\cite{pisinger2005}, and represent respectivelly strongly, weakly and uncorrelated instances. Therefore, 100 instances 
for each possible parameter combination were created, totaling 10800 instances.

Next session shows a hardness analysis using the results of the CPLEX tests. Then, the results of the tests with
the two implemented heuristics are discussed.

\subsection{Problem Hardness}

The first battery of tests involved solving the generated instances with the CPLEX solver, to check if they were
indeed hard to solve. During preliminary tests, while the CPEX solver was able to fastly solve some instances, it took 
a lot of time to solve some others. So to be able to continue the experiments, an alternative strategy was used: instances
that took more than 20 minutes to be solved were considered prohibitive for the continuity of the experiment, so when an 
instance reached this threshold, it was interrupted and the current solution saved.

Figures~\ref{fig:result00},~\ref{fig:result01} and~\ref{fig:result10} show the results of the first tests, 
considering the interruption rate. Each of the three figures represents a level of correlation, while each heatmap 
shows the instances with certain number of resources. The colors on the heatmaps represent the ratio of interrupted 
instances during the CPLEX execution, varying from the combinations of actions/years where no instance was interrupted 
(paler squares) to the combinations where all instances were interrupted (darker squares).

\begin{figure}[H]
  \centering
    \subfloat[1 resource]{% GNUPLOT: LaTeX picture using PSTRICKS macros
% Define new PST objects, if not already defined
\ifx\PSTloaded\undefined
\def\PSTloaded{t}

\catcode`@=11

\newpsobject{PST@Border}{psline}{linewidth=.0015,linestyle=solid}

\catcode`@=12

\fi
\psset{unit=5.0in,xunit=5.0in,yunit=3.0in}
\pspicture(0.000000,0.000000)(0.31, 0.35)
\ifx\nofigs\undefined
\catcode`@=11

\newrgbcolor{PST@COLOR0}{1 1 1}
\newrgbcolor{PST@COLOR1}{0.992 0.992 0.992}
\newrgbcolor{PST@COLOR3}{0.976 0.976 0.976}
\newrgbcolor{PST@COLOR6}{0.952 0.952 0.952}


\def\polypmIIId#1{\pspolygon[linestyle=none,fillstyle=solid,fillcolor=PST@COLOR#1]}

\polypmIIId{1}(0.1432,0.19)(0.0864,0.19)(0.0864,0.1078)(0.1432,0.1078)
\polypmIIId{0}(0.1432,0.272)(0.0864,0.272)(0.0864,0.19)(0.1432,0.19)
\polypmIIId{0}(0.1432,0.3542)(0.0864,0.3542)(0.0864,0.272)(0.1432,0.272)

\polypmIIId{1}(0.2,0.19)(0.1432,0.19)(0.1432,0.1078)(0.2,0.1078)
\polypmIIId{0}(0.2,0.272)(0.1432,0.272)(0.1432,0.19)(0.2,0.19)
\polypmIIId{0}(0.2,0.3542)(0.1432,0.3542)(0.1432,0.272)(0.2,0.272)

\polypmIIId{3}(0.2568,0.19)(0.2,0.19)(0.2,0.1078)(0.2568,0.1078)
\polypmIIId{0}(0.2568,0.272)(0.2,0.272)(0.2,0.19)(0.2568,0.19)
\polypmIIId{0}(0.2568,0.3542)(0.2,0.3542)(0.2,0.272)(0.2568,0.272)

\polypmIIId{6}(0.3136,0.19)(0.2568,0.19)(0.2568,0.1078)(0.3136,0.1078)
\polypmIIId{0}(0.3136,0.272)(0.2568,0.272)(0.2568,0.19)(0.3136,0.19)
\polypmIIId{0}(0.3136,0.3542)(0.2568,0.3542)(0.2568,0.272)(0.3136,0.272)

\rput(0.1148,0.07){3}
\rput(0.1716,0.07){4}
\rput(0.2284,0.07){5}
\rput(0.2852,0.07){6}
\rput(0.2000,0.0070){years}

\rput[r](0.0806,0.1489){25}
\rput[r](0.0806,0.2310){50}
\rput[r](0.0806,0.3131){100}
\rput{L}(0.0096,0.2310){actions}

\PST@Border(0.0864,0.3542)
(0.0864,0.1078)
(0.3136,0.1078)
(0.3136,0.3542)
(0.0864,0.3542)

\catcode`@=12
\fi
\endpspicture
}
    \subfloat[2 resources]{% GNUPLOT: LaTeX picture using PSTRICKS macros
% Define new PST objects, if not already defined
\ifx\PSTloaded\undefined
\def\PSTloaded{t}

\catcode`@=11

\newpsobject{PST@Border}{psline}{linewidth=.0015,linestyle=solid}

\catcode`@=12

\fi
\psset{unit=5.0in,xunit=5.0in,yunit=3.0in}
\pspicture(0.000000,0.000000)(0.225000,0.35)
\ifx\nofigs\undefined
\catcode`@=11

\newrgbcolor{PST@COLOR0}{1 1 1}
\newrgbcolor{PST@COLOR2}{0.984 0.984 0.984}
\newrgbcolor{PST@COLOR8}{0.937 0.937 0.937}
\newrgbcolor{PST@COLOR10}{0.921 0.921 0.921}
\newrgbcolor{PST@COLOR55}{0.566 0.566 0.566}
\newrgbcolor{PST@COLOR96}{0.244 0.244 0.244}
\newrgbcolor{PST@COLOR117}{0.078 0.078 0.078}

\def\polypmIIId#1{\pspolygon[linestyle=none,fillstyle=solid,fillcolor=PST@COLOR#1]}

\polypmIIId{55} (0.0568,0.19)  (0.0,0.19)  (0.0,0.1078)(0.0568,0.1078)
\polypmIIId{0}  (0.0568,0.272) (0.0,0.272) (0.0,0.19)  (0.0568,0.19)
\polypmIIId{0}  (0.0568,0.3542)(0.0,0.3542)(0.0,0.272) (0.0568,0.272)

\polypmIIId{96} (0.1136,   0.19)  (0.0568,0.19)  (0.0568,0.1078)(0.1136,0.1078)
\polypmIIId{2}  (0.1136,   0.272) (0.0568,0.272) (0.0568,0.19)  (0.1136,0.19)
\polypmIIId{0}  (0.1136,   0.3542)(0.0568,0.3542)(0.0568,0.272) (0.1136,0.272)

\polypmIIId{117}(0.1704,0.19)  (0.1136,   0.19)  (0.1136,   0.1078)(0.1704,0.1078)
\polypmIIId{10} (0.1704,0.272) (0.1136,   0.272) (0.1136,   0.19)  (0.1704,0.19)
\polypmIIId{0}  (0.1704,0.3542)(0.1136,   0.3542)(0.1136,   0.272) (0.1704,0.272)

\polypmIIId{117}(0.2272,0.19)  (0.1704,0.19)  (0.1704,0.1078)(0.2272,0.1078)
\polypmIIId{8}  (0.2272,0.272) (0.1704,0.272) (0.1704,0.19)  (0.2272,0.19)
\polypmIIId{0}  (0.2272,0.3542)(0.1704,0.3542)(0.1704,0.272) (0.2272,0.272)

\rput(0.0284,0.07){3}
\rput(0.0852,0.07){4}
\rput(0.1420,0.07){5}
\rput(0.1988,0.07){6}
\rput(0.1136,0.0070){years}


\PST@Border(0.0,0.3542)
(0.0,0.1078)
(0.2272,0.1078)
(0.2272,0.3542)
(0.0,0.3542)

\catcode`@=12
\fi
\endpspicture
}
    \subfloat[4 resources]{% GNUPLOT: LaTeX picture using PSTRICKS macros
% Define new PST objects, if not already defined
\ifx\PSTloaded\undefined
\def\PSTloaded{t}

\catcode`@=11

\newpsobject{PST@Border}{psline}{linewidth=.0015,linestyle=solid}

\catcode`@=12

\fi
\psset{unit=5.0in,xunit=5.0in,yunit=3.0in}
\pspicture(0.000000,0.000000)(0.3136,0.35)
\ifx\nofigs\undefined
\catcode`@=11

\newrgbcolor{PST@COLOR0}{1 1 1}
\newrgbcolor{PST@COLOR1}{0.992 0.992 0.992}
\newrgbcolor{PST@COLOR2}{0.984 0.984 0.984}
\newrgbcolor{PST@COLOR3}{0.976 0.976 0.976}
\newrgbcolor{PST@COLOR4}{0.968 0.968 0.968}
\newrgbcolor{PST@COLOR5}{0.96 0.96 0.96}
\newrgbcolor{PST@COLOR6}{0.952 0.952 0.952}
\newrgbcolor{PST@COLOR7}{0.944 0.944 0.944}
\newrgbcolor{PST@COLOR8}{0.937 0.937 0.937}
\newrgbcolor{PST@COLOR9}{0.929 0.929 0.929}
\newrgbcolor{PST@COLOR10}{0.921 0.921 0.921}
\newrgbcolor{PST@COLOR11}{0.913 0.913 0.913}
\newrgbcolor{PST@COLOR12}{0.905 0.905 0.905}
\newrgbcolor{PST@COLOR13}{0.897 0.897 0.897}
\newrgbcolor{PST@COLOR14}{0.889 0.889 0.889}
\newrgbcolor{PST@COLOR15}{0.881 0.881 0.881}
\newrgbcolor{PST@COLOR16}{0.874 0.874 0.874}
\newrgbcolor{PST@COLOR17}{0.866 0.866 0.866}
\newrgbcolor{PST@COLOR18}{0.858 0.858 0.858}
\newrgbcolor{PST@COLOR19}{0.85 0.85 0.85}
\newrgbcolor{PST@COLOR20}{0.842 0.842 0.842}
\newrgbcolor{PST@COLOR21}{0.834 0.834 0.834}
\newrgbcolor{PST@COLOR22}{0.826 0.826 0.826}
\newrgbcolor{PST@COLOR23}{0.818 0.818 0.818}
\newrgbcolor{PST@COLOR24}{0.811 0.811 0.811}
\newrgbcolor{PST@COLOR25}{0.803 0.803 0.803}
\newrgbcolor{PST@COLOR26}{0.795 0.795 0.795}
\newrgbcolor{PST@COLOR27}{0.787 0.787 0.787}
\newrgbcolor{PST@COLOR28}{0.779 0.779 0.779}
\newrgbcolor{PST@COLOR29}{0.771 0.771 0.771}
\newrgbcolor{PST@COLOR30}{0.763 0.763 0.763}
\newrgbcolor{PST@COLOR31}{0.755 0.755 0.755}
\newrgbcolor{PST@COLOR32}{0.748 0.748 0.748}
\newrgbcolor{PST@COLOR33}{0.74 0.74 0.74}
\newrgbcolor{PST@COLOR34}{0.732 0.732 0.732}
\newrgbcolor{PST@COLOR35}{0.724 0.724 0.724}
\newrgbcolor{PST@COLOR36}{0.716 0.716 0.716}
\newrgbcolor{PST@COLOR37}{0.708 0.708 0.708}
\newrgbcolor{PST@COLOR38}{0.7 0.7 0.7}
\newrgbcolor{PST@COLOR39}{0.692 0.692 0.692}
\newrgbcolor{PST@COLOR40}{0.685 0.685 0.685}
\newrgbcolor{PST@COLOR41}{0.677 0.677 0.677}
\newrgbcolor{PST@COLOR42}{0.669 0.669 0.669}
\newrgbcolor{PST@COLOR43}{0.661 0.661 0.661}
\newrgbcolor{PST@COLOR44}{0.653 0.653 0.653}
\newrgbcolor{PST@COLOR45}{0.645 0.645 0.645}
\newrgbcolor{PST@COLOR46}{0.637 0.637 0.637}
\newrgbcolor{PST@COLOR47}{0.629 0.629 0.629}
\newrgbcolor{PST@COLOR48}{0.622 0.622 0.622}
\newrgbcolor{PST@COLOR49}{0.614 0.614 0.614}
\newrgbcolor{PST@COLOR50}{0.606 0.606 0.606}
\newrgbcolor{PST@COLOR51}{0.598 0.598 0.598}
\newrgbcolor{PST@COLOR52}{0.59 0.59 0.59}
\newrgbcolor{PST@COLOR53}{0.582 0.582 0.582}
\newrgbcolor{PST@COLOR54}{0.574 0.574 0.574}
\newrgbcolor{PST@COLOR55}{0.566 0.566 0.566}
\newrgbcolor{PST@COLOR56}{0.559 0.559 0.559}
\newrgbcolor{PST@COLOR57}{0.551 0.551 0.551}
\newrgbcolor{PST@COLOR58}{0.543 0.543 0.543}
\newrgbcolor{PST@COLOR59}{0.535 0.535 0.535}
\newrgbcolor{PST@COLOR60}{0.527 0.527 0.527}
\newrgbcolor{PST@COLOR61}{0.519 0.519 0.519}
\newrgbcolor{PST@COLOR62}{0.511 0.511 0.511}
\newrgbcolor{PST@COLOR63}{0.503 0.503 0.503}
\newrgbcolor{PST@COLOR64}{0.496 0.496 0.496}
\newrgbcolor{PST@COLOR65}{0.488 0.488 0.488}
\newrgbcolor{PST@COLOR66}{0.48 0.48 0.48}
\newrgbcolor{PST@COLOR67}{0.472 0.472 0.472}
\newrgbcolor{PST@COLOR68}{0.464 0.464 0.464}
\newrgbcolor{PST@COLOR69}{0.456 0.456 0.456}
\newrgbcolor{PST@COLOR70}{0.448 0.448 0.448}
\newrgbcolor{PST@COLOR71}{0.44 0.44 0.44}
\newrgbcolor{PST@COLOR72}{0.433 0.433 0.433}
\newrgbcolor{PST@COLOR73}{0.425 0.425 0.425}
\newrgbcolor{PST@COLOR74}{0.417 0.417 0.417}
\newrgbcolor{PST@COLOR75}{0.409 0.409 0.409}
\newrgbcolor{PST@COLOR76}{0.401 0.401 0.401}
\newrgbcolor{PST@COLOR77}{0.393 0.393 0.393}
\newrgbcolor{PST@COLOR78}{0.385 0.385 0.385}
\newrgbcolor{PST@COLOR79}{0.377 0.377 0.377}
\newrgbcolor{PST@COLOR80}{0.37 0.37 0.37}
\newrgbcolor{PST@COLOR81}{0.362 0.362 0.362}
\newrgbcolor{PST@COLOR82}{0.354 0.354 0.354}
\newrgbcolor{PST@COLOR83}{0.346 0.346 0.346}
\newrgbcolor{PST@COLOR84}{0.338 0.338 0.338}
\newrgbcolor{PST@COLOR85}{0.33 0.33 0.33}
\newrgbcolor{PST@COLOR86}{0.322 0.322 0.322}
\newrgbcolor{PST@COLOR87}{0.314 0.314 0.314}
\newrgbcolor{PST@COLOR88}{0.307 0.307 0.307}
\newrgbcolor{PST@COLOR89}{0.299 0.299 0.299}
\newrgbcolor{PST@COLOR90}{0.291 0.291 0.291}
\newrgbcolor{PST@COLOR91}{0.283 0.283 0.283}
\newrgbcolor{PST@COLOR92}{0.275 0.275 0.275}
\newrgbcolor{PST@COLOR93}{0.267 0.267 0.267}
\newrgbcolor{PST@COLOR94}{0.259 0.259 0.259}
\newrgbcolor{PST@COLOR95}{0.251 0.251 0.251}
\newrgbcolor{PST@COLOR96}{0.244 0.244 0.244}
\newrgbcolor{PST@COLOR97}{0.236 0.236 0.236}
\newrgbcolor{PST@COLOR98}{0.228 0.228 0.228}
\newrgbcolor{PST@COLOR99}{0.22 0.22 0.22}
\newrgbcolor{PST@COLOR100}{0.212 0.212 0.212}
\newrgbcolor{PST@COLOR101}{0.204 0.204 0.204}
\newrgbcolor{PST@COLOR102}{0.196 0.196 0.196}
\newrgbcolor{PST@COLOR103}{0.188 0.188 0.188}
\newrgbcolor{PST@COLOR104}{0.181 0.181 0.181}
\newrgbcolor{PST@COLOR105}{0.173 0.173 0.173}
\newrgbcolor{PST@COLOR106}{0.165 0.165 0.165}
\newrgbcolor{PST@COLOR107}{0.157 0.157 0.157}
\newrgbcolor{PST@COLOR108}{0.149 0.149 0.149}
\newrgbcolor{PST@COLOR109}{0.141 0.141 0.141}
\newrgbcolor{PST@COLOR110}{0.133 0.133 0.133}
\newrgbcolor{PST@COLOR111}{0.125 0.125 0.125}
\newrgbcolor{PST@COLOR112}{0.118 0.118 0.118}
\newrgbcolor{PST@COLOR113}{0.11 0.11 0.11}
\newrgbcolor{PST@COLOR114}{0.102 0.102 0.102}
\newrgbcolor{PST@COLOR115}{0.094 0.094 0.094}
\newrgbcolor{PST@COLOR116}{0.086 0.086 0.086}
\newrgbcolor{PST@COLOR117}{0.078 0.078 0.078}
\newrgbcolor{PST@COLOR118}{0.07 0.07 0.07}
\newrgbcolor{PST@COLOR119}{0.062 0.062 0.062}
\newrgbcolor{PST@COLOR120}{0.055 0.055 0.055}
\newrgbcolor{PST@COLOR121}{0.047 0.047 0.047}
\newrgbcolor{PST@COLOR122}{0.039 0.039 0.039}
\newrgbcolor{PST@COLOR123}{0.031 0.031 0.031}
\newrgbcolor{PST@COLOR124}{0.023 0.023 0.023}
\newrgbcolor{PST@COLOR125}{0.015 0.015 0.015}
\newrgbcolor{PST@COLOR126}{0.007 0.007 0.007}
\newrgbcolor{PST@COLOR127}{0 0 0}

\def\polypmIIId#1{\pspolygon[linestyle=none,fillstyle=solid,fillcolor=PST@COLOR#1]}

\polypmIIId{127} (0.0568,0.19)  (0.0,0.19)  (0.0,0.1078)(0.0568,0.1078)
\polypmIIId{124}  (0.0568,0.272) (0.0,0.272) (0.0,0.19)  (0.0568,0.19)
\polypmIIId{0}  (0.0568,0.3542)(0.0,0.3542)(0.0,0.272) (0.0568,0.272)

\polypmIIId{127} (0.1136,   0.19)  (0.0568,0.19)  (0.0568,0.1078)(0.1136,0.1078)
\polypmIIId{127}  (0.1136,   0.272) (0.0568,0.272) (0.0568,0.19)  (0.1136,0.19)
\polypmIIId{3}  (0.1136,   0.3542)(0.0568,0.3542)(0.0568,0.272) (0.1136,0.272)

\polypmIIId{127}(0.1704,0.19)  (0.1136,   0.19)  (0.1136,   0.1078)(0.1704,0.1078)
\polypmIIId{127} (0.1704,0.272) (0.1136,   0.272) (0.1136,   0.19)  (0.1704,0.19)
\polypmIIId{10}  (0.1704,0.3542)(0.1136,   0.3542)(0.1136,   0.272) (0.1704,0.272)

\polypmIIId{127}(0.2272,0.19)  (0.1704,0.19)  (0.1704,0.1078)(0.2272,0.1078)
\polypmIIId{127}  (0.2272,0.272) (0.1704,0.272) (0.1704,0.19)  (0.2272,0.19)
\polypmIIId{15}  (0.2272,0.3542)(0.1704,0.3542)(0.1704,0.272) (0.2272,0.272)

\rput(0.0284,0.07){3}
\rput(0.0852,0.07){4}
\rput(0.1420,0.07){5}
\rput(0.1988,0.07){6}
\rput(0.1136,0.0070){years}

\PST@Border(0.0,0.3542)
(0.0,0.1078)
(0.2272,0.1078)
(0.2272,0.3542)
(0.0,0.3542)

\polypmIIId{0}(0.2329,0.1078)(0.2442,0.1078)(0.2442,0.1098)(0.2329,0.1098)
\polypmIIId{1}(0.2329,0.1097)(0.2442,0.1097)(0.2442,0.1117)(0.2329,0.1117)
\polypmIIId{2}(0.2329,0.1116)(0.2442,0.1116)(0.2442,0.1136)(0.2329,0.1136)
\polypmIIId{3}(0.2329,0.1135)(0.2442,0.1135)(0.2442,0.1156)(0.2329,0.1156)
\polypmIIId{4}(0.2329,0.1155)(0.2442,0.1155)(0.2442,0.1175)(0.2329,0.1175)
\polypmIIId{5}(0.2329,0.1174)(0.2442,0.1174)(0.2442,0.1194)(0.2329,0.1194)
\polypmIIId{6}(0.2329,0.1193)(0.2442,0.1193)(0.2442,0.1213)(0.2329,0.1213)
\polypmIIId{7}(0.2329,0.1212)(0.2442,0.1212)(0.2442,0.1233)(0.2329,0.1233)
\polypmIIId{8}(0.2329,0.1232)(0.2442,0.1232)(0.2442,0.1252)(0.2329,0.1252)
\polypmIIId{9}(0.2329,0.1251)(0.2442,0.1251)(0.2442,0.1271)(0.2329,0.1271)
\polypmIIId{10}(0.2329,0.127)(0.2442,0.127)(0.2442,0.129)(0.2329,0.129)
\polypmIIId{11}(0.2329,0.1289)(0.2442,0.1289)(0.2442,0.131)(0.2329,0.131)
\polypmIIId{12}(0.2329,0.1309)(0.2442,0.1309)(0.2442,0.1329)(0.2329,0.1329)
\polypmIIId{13}(0.2329,0.1328)(0.2442,0.1328)(0.2442,0.1348)(0.2329,0.1348)
\polypmIIId{14}(0.2329,0.1347)(0.2442,0.1347)(0.2442,0.1367)(0.2329,0.1367)
\polypmIIId{15}(0.2329,0.1366)(0.2442,0.1366)(0.2442,0.1387)(0.2329,0.1387)
\polypmIIId{16}(0.2329,0.1386)(0.2442,0.1386)(0.2442,0.1406)(0.2329,0.1406)
\polypmIIId{17}(0.2329,0.1405)(0.2442,0.1405)(0.2442,0.1425)(0.2329,0.1425)
\polypmIIId{18}(0.2329,0.1424)(0.2442,0.1424)(0.2442,0.1444)(0.2329,0.1444)
\polypmIIId{19}(0.2329,0.1443)(0.2442,0.1443)(0.2442,0.1464)(0.2329,0.1464)
\polypmIIId{20}(0.2329,0.1463)(0.2442,0.1463)(0.2442,0.1483)(0.2329,0.1483)
\polypmIIId{21}(0.2329,0.1482)(0.2442,0.1482)(0.2442,0.1502)(0.2329,0.1502)
\polypmIIId{22}(0.2329,0.1501)(0.2442,0.1501)(0.2442,0.1521)(0.2329,0.1521)
\polypmIIId{23}(0.2329,0.152)(0.2442,0.152)(0.2442,0.1541)(0.2329,0.1541)
\polypmIIId{24}(0.2329,0.154)(0.2442,0.154)(0.2442,0.156)(0.2329,0.156)
\polypmIIId{25}(0.2329,0.1559)(0.2442,0.1559)(0.2442,0.1579)(0.2329,0.1579)
\polypmIIId{26}(0.2329,0.1578)(0.2442,0.1578)(0.2442,0.1598)(0.2329,0.1598)
\polypmIIId{27}(0.2329,0.1597)(0.2442,0.1597)(0.2442,0.1618)(0.2329,0.1618)
\polypmIIId{28}(0.2329,0.1617)(0.2442,0.1617)(0.2442,0.1637)(0.2329,0.1637)
\polypmIIId{29}(0.2329,0.1636)(0.2442,0.1636)(0.2442,0.1656)(0.2329,0.1656)
\polypmIIId{30}(0.2329,0.1655)(0.2442,0.1655)(0.2442,0.1675)(0.2329,0.1675)
\polypmIIId{31}(0.2329,0.1674)(0.2442,0.1674)(0.2442,0.1695)(0.2329,0.1695)
\polypmIIId{32}(0.2329,0.1694)(0.2442,0.1694)(0.2442,0.1714)(0.2329,0.1714)
\polypmIIId{33}(0.2329,0.1713)(0.2442,0.1713)(0.2442,0.1733)(0.2329,0.1733)
\polypmIIId{34}(0.2329,0.1732)(0.2442,0.1732)(0.2442,0.1752)(0.2329,0.1752)
\polypmIIId{35}(0.2329,0.1751)(0.2442,0.1751)(0.2442,0.1772)(0.2329,0.1772)
\polypmIIId{36}(0.2329,0.1771)(0.2442,0.1771)(0.2442,0.1791)(0.2329,0.1791)
\polypmIIId{37}(0.2329,0.179)(0.2442,0.179)(0.2442,0.181)(0.2329,0.181)
\polypmIIId{38}(0.2329,0.1809)(0.2442,0.1809)(0.2442,0.1829)(0.2329,0.1829)
\polypmIIId{39}(0.2329,0.1828)(0.2442,0.1828)(0.2442,0.1849)(0.2329,0.1849)
\polypmIIId{40}(0.2329,0.1848)(0.2442,0.1848)(0.2442,0.1868)(0.2329,0.1868)
\polypmIIId{41}(0.2329,0.1867)(0.2442,0.1867)(0.2442,0.1887)(0.2329,0.1887)
\polypmIIId{42}(0.2329,0.1886)(0.2442,0.1886)(0.2442,0.1906)(0.2329,0.1906)
\polypmIIId{43}(0.2329,0.1905)(0.2442,0.1905)(0.2442,0.1926)(0.2329,0.1926)
\polypmIIId{44}(0.2329,0.1925)(0.2442,0.1925)(0.2442,0.1945)(0.2329,0.1945)
\polypmIIId{45}(0.2329,0.1944)(0.2442,0.1944)(0.2442,0.1964)(0.2329,0.1964)
\polypmIIId{46}(0.2329,0.1963)(0.2442,0.1963)(0.2442,0.1983)(0.2329,0.1983)
\polypmIIId{47}(0.2329,0.1982)(0.2442,0.1982)(0.2442,0.2003)(0.2329,0.2003)
\polypmIIId{48}(0.2329,0.2002)(0.2442,0.2002)(0.2442,0.2022)(0.2329,0.2022)
\polypmIIId{49}(0.2329,0.2021)(0.2442,0.2021)(0.2442,0.2041)(0.2329,0.2041)
\polypmIIId{50}(0.2329,0.204)(0.2442,0.204)(0.2442,0.206)(0.2329,0.206)
\polypmIIId{51}(0.2329,0.2059)(0.2442,0.2059)(0.2442,0.208)(0.2329,0.208)
\polypmIIId{52}(0.2329,0.2079)(0.2442,0.2079)(0.2442,0.2099)(0.2329,0.2099)
\polypmIIId{53}(0.2329,0.2098)(0.2442,0.2098)(0.2442,0.2118)(0.2329,0.2118)
\polypmIIId{54}(0.2329,0.2117)(0.2442,0.2117)(0.2442,0.2137)(0.2329,0.2137)
\polypmIIId{55}(0.2329,0.2136)(0.2442,0.2136)(0.2442,0.2157)(0.2329,0.2157)
\polypmIIId{56}(0.2329,0.2156)(0.2442,0.2156)(0.2442,0.2176)(0.2329,0.2176)
\polypmIIId{57}(0.2329,0.2175)(0.2442,0.2175)(0.2442,0.2195)(0.2329,0.2195)
\polypmIIId{58}(0.2329,0.2194)(0.2442,0.2194)(0.2442,0.2214)(0.2329,0.2214)
\polypmIIId{59}(0.2329,0.2213)(0.2442,0.2213)(0.2442,0.2234)(0.2329,0.2234)
\polypmIIId{60}(0.2329,0.2233)(0.2442,0.2233)(0.2442,0.2253)(0.2329,0.2253)
\polypmIIId{61}(0.2329,0.2252)(0.2442,0.2252)(0.2442,0.2272)(0.2329,0.2272)
\polypmIIId{62}(0.2329,0.2271)(0.2442,0.2271)(0.2442,0.2291)(0.2329,0.2291)
\polypmIIId{63}(0.2329,0.229)(0.2442,0.229)(0.2442,0.2311)(0.2329,0.2311)
\polypmIIId{64}(0.2329,0.231)(0.2442,0.231)(0.2442,0.233)(0.2329,0.233)
\polypmIIId{65}(0.2329,0.2329)(0.2442,0.2329)(0.2442,0.2349)(0.2329,0.2349)
\polypmIIId{66}(0.2329,0.2348)(0.2442,0.2348)(0.2442,0.2368)(0.2329,0.2368)
\polypmIIId{67}(0.2329,0.2367)(0.2442,0.2367)(0.2442,0.2388)(0.2329,0.2388)
\polypmIIId{68}(0.2329,0.2387)(0.2442,0.2387)(0.2442,0.2407)(0.2329,0.2407)
\polypmIIId{69}(0.2329,0.2406)(0.2442,0.2406)(0.2442,0.2426)(0.2329,0.2426)
\polypmIIId{70}(0.2329,0.2425)(0.2442,0.2425)(0.2442,0.2445)(0.2329,0.2445)
\polypmIIId{71}(0.2329,0.2444)(0.2442,0.2444)(0.2442,0.2465)(0.2329,0.2465)
\polypmIIId{72}(0.2329,0.2464)(0.2442,0.2464)(0.2442,0.2484)(0.2329,0.2484)
\polypmIIId{73}(0.2329,0.2483)(0.2442,0.2483)(0.2442,0.2503)(0.2329,0.2503)
\polypmIIId{74}(0.2329,0.2502)(0.2442,0.2502)(0.2442,0.2522)(0.2329,0.2522)
\polypmIIId{75}(0.2329,0.2521)(0.2442,0.2521)(0.2442,0.2542)(0.2329,0.2542)
\polypmIIId{76}(0.2329,0.2541)(0.2442,0.2541)(0.2442,0.2561)(0.2329,0.2561)
\polypmIIId{77}(0.2329,0.256)(0.2442,0.256)(0.2442,0.258)(0.2329,0.258)
\polypmIIId{78}(0.2329,0.2579)(0.2442,0.2579)(0.2442,0.2599)(0.2329,0.2599)
\polypmIIId{79}(0.2329,0.2598)(0.2442,0.2598)(0.2442,0.2619)(0.2329,0.2619)
\polypmIIId{80}(0.2329,0.2618)(0.2442,0.2618)(0.2442,0.2638)(0.2329,0.2638)
\polypmIIId{81}(0.2329,0.2637)(0.2442,0.2637)(0.2442,0.2657)(0.2329,0.2657)
\polypmIIId{82}(0.2329,0.2656)(0.2442,0.2656)(0.2442,0.2676)(0.2329,0.2676)
\polypmIIId{83}(0.2329,0.2675)(0.2442,0.2675)(0.2442,0.2696)(0.2329,0.2696)
\polypmIIId{84}(0.2329,0.2695)(0.2442,0.2695)(0.2442,0.2715)(0.2329,0.2715)
\polypmIIId{85}(0.2329,0.2714)(0.2442,0.2714)(0.2442,0.2734)(0.2329,0.2734)
\polypmIIId{86}(0.2329,0.2733)(0.2442,0.2733)(0.2442,0.2753)(0.2329,0.2753)
\polypmIIId{87}(0.2329,0.2752)(0.2442,0.2752)(0.2442,0.2773)(0.2329,0.2773)
\polypmIIId{88}(0.2329,0.2772)(0.2442,0.2772)(0.2442,0.2792)(0.2329,0.2792)
\polypmIIId{89}(0.2329,0.2791)(0.2442,0.2791)(0.2442,0.2811)(0.2329,0.2811)
\polypmIIId{90}(0.2329,0.281)(0.2442,0.281)(0.2442,0.283)(0.2329,0.283)
\polypmIIId{91}(0.2329,0.2829)(0.2442,0.2829)(0.2442,0.285)(0.2329,0.285)
\polypmIIId{92}(0.2329,0.2849)(0.2442,0.2849)(0.2442,0.2869)(0.2329,0.2869)
\polypmIIId{93}(0.2329,0.2868)(0.2442,0.2868)(0.2442,0.2888)(0.2329,0.2888)
\polypmIIId{94}(0.2329,0.2887)(0.2442,0.2887)(0.2442,0.2907)(0.2329,0.2907)
\polypmIIId{95}(0.2329,0.2906)(0.2442,0.2906)(0.2442,0.2927)(0.2329,0.2927)
\polypmIIId{96}(0.2329,0.2926)(0.2442,0.2926)(0.2442,0.2946)(0.2329,0.2946)
\polypmIIId{97}(0.2329,0.2945)(0.2442,0.2945)(0.2442,0.2965)(0.2329,0.2965)
\polypmIIId{98}(0.2329,0.2964)(0.2442,0.2964)(0.2442,0.2984)(0.2329,0.2984)
\polypmIIId{99}(0.2329,0.2983)(0.2442,0.2983)(0.2442,0.3004)(0.2329,0.3004)
\polypmIIId{100}(0.2329,0.3003)(0.2442,0.3003)(0.2442,0.3023)(0.2329,0.3023)
\polypmIIId{101}(0.2329,0.3022)(0.2442,0.3022)(0.2442,0.3042)(0.2329,0.3042)
\polypmIIId{102}(0.2329,0.3041)(0.2442,0.3041)(0.2442,0.3061)(0.2329,0.3061)
\polypmIIId{103}(0.2329,0.306)(0.2442,0.306)(0.2442,0.3081)(0.2329,0.3081)
\polypmIIId{104}(0.2329,0.308)(0.2442,0.308)(0.2442,0.31)(0.2329,0.31)
\polypmIIId{105}(0.2329,0.3099)(0.2442,0.3099)(0.2442,0.3119)(0.2329,0.3119)
\polypmIIId{106}(0.2329,0.3118)(0.2442,0.3118)(0.2442,0.3138)(0.2329,0.3138)
\polypmIIId{107}(0.2329,0.3137)(0.2442,0.3137)(0.2442,0.3158)(0.2329,0.3158)
\polypmIIId{108}(0.2329,0.3157)(0.2442,0.3157)(0.2442,0.3177)(0.2329,0.3177)
\polypmIIId{109}(0.2329,0.3176)(0.2442,0.3176)(0.2442,0.3196)(0.2329,0.3196)
\polypmIIId{110}(0.2329,0.3195)(0.2442,0.3195)(0.2442,0.3215)(0.2329,0.3215)
\polypmIIId{111}(0.2329,0.3214)(0.2442,0.3214)(0.2442,0.3235)(0.2329,0.3235)
\polypmIIId{112}(0.2329,0.3234)(0.2442,0.3234)(0.2442,0.3254)(0.2329,0.3254)
\polypmIIId{113}(0.2329,0.3253)(0.2442,0.3253)(0.2442,0.3273)(0.2329,0.3273)
\polypmIIId{114}(0.2329,0.3272)(0.2442,0.3272)(0.2442,0.3292)(0.2329,0.3292)
\polypmIIId{115}(0.2329,0.3291)(0.2442,0.3291)(0.2442,0.3312)(0.2329,0.3312)
\polypmIIId{116}(0.2329,0.3311)(0.2442,0.3311)(0.2442,0.3331)(0.2329,0.3331)
\polypmIIId{117}(0.2329,0.333)(0.2442,0.333)(0.2442,0.335)(0.2329,0.335)
\polypmIIId{118}(0.2329,0.3349)(0.2442,0.3349)(0.2442,0.3369)(0.2329,0.3369)
\polypmIIId{119}(0.2329,0.3368)(0.2442,0.3368)(0.2442,0.3389)(0.2329,0.3389)
\polypmIIId{120}(0.2329,0.3388)(0.2442,0.3388)(0.2442,0.3408)(0.2329,0.3408)
\polypmIIId{121}(0.2329,0.3407)(0.2442,0.3407)(0.2442,0.3427)(0.2329,0.3427)
\polypmIIId{122}(0.2329,0.3426)(0.2442,0.3426)(0.2442,0.3446)(0.2329,0.3446)
\polypmIIId{123}(0.2329,0.3445)(0.2442,0.3445)(0.2442,0.3466)(0.2329,0.3466)
\polypmIIId{124}(0.2329,0.3465)(0.2442,0.3465)(0.2442,0.3485)(0.2329,0.3485)
\polypmIIId{125}(0.2329,0.3484)(0.2442,0.3484)(0.2442,0.3504)(0.2329,0.3504)
\polypmIIId{126}(0.2329,0.3503)(0.2442,0.3503)(0.2442,0.3523)(0.2329,0.3523)
\polypmIIId{127}(0.2329,0.3522)(0.2442,0.3522)(0.2442,0.3542)(0.2329,0.3542)

\PST@Border(0.2329,0.1078)
(0.2442,0.1078)
(0.2442,0.3542)
(0.2329,0.3542)
(0.2329,0.1078)

\rput[l](0.2502,0.1078){0}
\rput[l](0.2502,0.1570){0.2}
\rput[l](0.2502,0.2063){0.4}
\rput[l](0.2502,0.2556){0.6}
\rput[l](0.2502,0.3049){0.8}
\rput[l](0.2502,0.3542){1}

\catcode`@=12
\fi
\endpspicture
}
  \caption{Ratio of strongly correlated instances interrupted ($\alpha = 0.0$).}
  \label{fig:result00}
\end{figure}

\begin{figure}[H]
  \centering
    \subfloat[1 resource]{% GNUPLOT: LaTeX picture using PSTRICKS macros
% Define new PST objects, if not already defined
\ifx\PSTloaded\undefined
\def\PSTloaded{t}

\catcode`@=11

\newpsobject{PST@Border}{psline}{linewidth=.0015,linestyle=solid}

\catcode`@=12

\fi
\psset{unit=5.0in,xunit=5.0in,yunit=3.0in}
\pspicture(0.000000,0.000000)(0.31, 0.35)
\ifx\nofigs\undefined
\catcode`@=11

\newrgbcolor{PST@COLOR0}{1 1 1}
\newrgbcolor{PST@COLOR1}{0.992 0.992 0.992}
\newrgbcolor{PST@COLOR3}{0.976 0.976 0.976}
\newrgbcolor{PST@COLOR6}{0.952 0.952 0.952}


\def\polypmIIId#1{\pspolygon[linestyle=none,fillstyle=solid,fillcolor=PST@COLOR#1]}

\polypmIIId{0}(0.1432,0.19)(0.0864,0.19)(0.0864,0.1078)(0.1432,0.1078)
\polypmIIId{0}(0.1432,0.272)(0.0864,0.272)(0.0864,0.19)(0.1432,0.19)
\polypmIIId{0}(0.1432,0.3542)(0.0864,0.3542)(0.0864,0.272)(0.1432,0.272)

\polypmIIId{0}(0.2,0.19)(0.1432,0.19)(0.1432,0.1078)(0.2,0.1078)
\polypmIIId{0}(0.2,0.272)(0.1432,0.272)(0.1432,0.19)(0.2,0.19)
\polypmIIId{0}(0.2,0.3542)(0.1432,0.3542)(0.1432,0.272)(0.2,0.272)

\polypmIIId{1}(0.2568,0.19)(0.2,0.19)(0.2,0.1078)(0.2568,0.1078)
\polypmIIId{0}(0.2568,0.272)(0.2,0.272)(0.2,0.19)(0.2568,0.19)
\polypmIIId{0}(0.2568,0.3542)(0.2,0.3542)(0.2,0.272)(0.2568,0.272)

\polypmIIId{0}(0.3136,0.19)(0.2568,0.19)(0.2568,0.1078)(0.3136,0.1078)
\polypmIIId{0}(0.3136,0.272)(0.2568,0.272)(0.2568,0.19)(0.3136,0.19)
\polypmIIId{0}(0.3136,0.3542)(0.2568,0.3542)(0.2568,0.272)(0.3136,0.272)

\rput(0.1148,0.07){3}
\rput(0.1716,0.07){4}
\rput(0.2284,0.07){5}
\rput(0.2852,0.07){6}
\rput(0.2000,0.0070){years}

\rput[r](0.0806,0.1489){25}
\rput[r](0.0806,0.2310){50}
\rput[r](0.0806,0.3131){100}
\rput{L}(0.0096,0.2310){actions}

\PST@Border(0.0864,0.3542)
(0.0864,0.1078)
(0.3136,0.1078)
(0.3136,0.3542)
(0.0864,0.3542)

\catcode`@=12
\fi
\endpspicture}
    \subfloat[2 resources]{% GNUPLOT: LaTeX picture using PSTRICKS macros
% Define new PST objects, if not already defined
\ifx\PSTloaded\undefined
\def\PSTloaded{t}

\catcode`@=11

\newpsobject{PST@Border}{psline}{linewidth=.0015,linestyle=solid}

\catcode`@=12

\fi
\psset{unit=5.0in,xunit=5.0in,yunit=3.0in}
\pspicture(0.000000,0.000000)(0.225000,0.35)
\ifx\nofigs\undefined
\catcode`@=11

\newrgbcolor{PST@COLOR0}{1 1 1}
\newrgbcolor{PST@COLOR2}{0.984 0.984 0.984}
\newrgbcolor{PST@COLOR8}{0.937 0.937 0.937}
\newrgbcolor{PST@COLOR10}{0.921 0.921 0.921}
\newrgbcolor{PST@COLOR55}{0.566 0.566 0.566}
\newrgbcolor{PST@COLOR96}{0.244 0.244 0.244}
\newrgbcolor{PST@COLOR117}{0.078 0.078 0.078}

\def\polypmIIId#1{\pspolygon[linestyle=none,fillstyle=solid,fillcolor=PST@COLOR#1]}

\polypmIIId{32} (0.0568,0.19)  (0.0,0.19)  (0.0,0.1078)(0.0568,0.1078)
\polypmIIId{6}  (0.0568,0.272) (0.0,0.272) (0.0,0.19)  (0.0568,0.19)
\polypmIIId{0}  (0.0568,0.3942)(0.0,0.3942)(0.0,0.272) (0.0568,0.272)

\polypmIIId{83} (0.1136,   0.19)  (0.0568,0.19)  (0.0568,0.1078)(0.1136,0.1078)
\polypmIIId{25}  (0.1136,   0.272) (0.0568,0.272) (0.0568,0.19)  (0.1136,0.19)
\polypmIIId{0}  (0.1136,   0.3942)(0.0568,0.3942)(0.0568,0.272) (0.1136,0.272)

\polypmIIId{108}(0.1704,0.19)  (0.1136,   0.19)  (0.1136,   0.1078)(0.1704,0.1078)
\polypmIIId{46} (0.1704,0.272) (0.1136,   0.272) (0.1136,   0.19)  (0.1704,0.19)
\polypmIIId{0}  (0.1704,0.3942)(0.1136,   0.3942)(0.1136,   0.272) (0.1704,0.272)

\polypmIIId{124}(0.2272,0.19)  (0.1704,0.19)  (0.1704,0.1078)(0.2272,0.1078)
\polypmIIId{65}  (0.2272,0.272) (0.1704,0.272) (0.1704,0.19)  (0.2272,0.19)
\polypmIIId{0}  (0.2272,0.3942)(0.1704,0.3942)(0.1704,0.272) (0.2272,0.272)


\rput(0.0284,0.07){3}
\rput(0.0852,0.07){4}
\rput(0.1420,0.07){5}
\rput(0.1988,0.07){6}
\rput(0.1136,0.0070){years}


\PST@Border(0.0,0.3542)
(0.0,0.1078)
(0.2272,0.1078)
(0.2272,0.3542)
(0.0,0.3542)

\catcode`@=12
\fi
\endpspicture}
    \subfloat[4 resources]{% GNUPLOT: LaTeX picture using PSTRICKS macros
% Define new PST objects, if not already defined
\ifx\PSTloaded\undefined
\def\PSTloaded{t}

\catcode`@=11

\newpsobject{PST@Border}{psline}{linewidth=.0015,linestyle=solid}

\catcode`@=12

\fi
\psset{unit=5.0in,xunit=5.0in,yunit=3.0in}
\pspicture(0.000000,0.000000)(0.3136,0.35)
\ifx\nofigs\undefined
\catcode`@=11

\newrgbcolor{PST@COLOR0}{1 1 1}
\newrgbcolor{PST@COLOR1}{0.992 0.992 0.992}
\newrgbcolor{PST@COLOR2}{0.984 0.984 0.984}
\newrgbcolor{PST@COLOR3}{0.976 0.976 0.976}
\newrgbcolor{PST@COLOR4}{0.968 0.968 0.968}
\newrgbcolor{PST@COLOR5}{0.96 0.96 0.96}
\newrgbcolor{PST@COLOR6}{0.952 0.952 0.952}
\newrgbcolor{PST@COLOR7}{0.944 0.944 0.944}
\newrgbcolor{PST@COLOR8}{0.937 0.937 0.937}
\newrgbcolor{PST@COLOR9}{0.929 0.929 0.929}
\newrgbcolor{PST@COLOR10}{0.921 0.921 0.921}
\newrgbcolor{PST@COLOR11}{0.913 0.913 0.913}
\newrgbcolor{PST@COLOR12}{0.905 0.905 0.905}
\newrgbcolor{PST@COLOR13}{0.897 0.897 0.897}
\newrgbcolor{PST@COLOR14}{0.889 0.889 0.889}
\newrgbcolor{PST@COLOR15}{0.881 0.881 0.881}
\newrgbcolor{PST@COLOR16}{0.874 0.874 0.874}
\newrgbcolor{PST@COLOR17}{0.866 0.866 0.866}
\newrgbcolor{PST@COLOR18}{0.858 0.858 0.858}
\newrgbcolor{PST@COLOR19}{0.85 0.85 0.85}
\newrgbcolor{PST@COLOR20}{0.842 0.842 0.842}
\newrgbcolor{PST@COLOR21}{0.834 0.834 0.834}
\newrgbcolor{PST@COLOR22}{0.826 0.826 0.826}
\newrgbcolor{PST@COLOR23}{0.818 0.818 0.818}
\newrgbcolor{PST@COLOR24}{0.811 0.811 0.811}
\newrgbcolor{PST@COLOR25}{0.803 0.803 0.803}
\newrgbcolor{PST@COLOR26}{0.795 0.795 0.795}
\newrgbcolor{PST@COLOR27}{0.787 0.787 0.787}
\newrgbcolor{PST@COLOR28}{0.779 0.779 0.779}
\newrgbcolor{PST@COLOR29}{0.771 0.771 0.771}
\newrgbcolor{PST@COLOR30}{0.763 0.763 0.763}
\newrgbcolor{PST@COLOR31}{0.755 0.755 0.755}
\newrgbcolor{PST@COLOR32}{0.748 0.748 0.748}
\newrgbcolor{PST@COLOR33}{0.74 0.74 0.74}
\newrgbcolor{PST@COLOR34}{0.732 0.732 0.732}
\newrgbcolor{PST@COLOR35}{0.724 0.724 0.724}
\newrgbcolor{PST@COLOR36}{0.716 0.716 0.716}
\newrgbcolor{PST@COLOR37}{0.708 0.708 0.708}
\newrgbcolor{PST@COLOR38}{0.7 0.7 0.7}
\newrgbcolor{PST@COLOR39}{0.692 0.692 0.692}
\newrgbcolor{PST@COLOR40}{0.685 0.685 0.685}
\newrgbcolor{PST@COLOR41}{0.677 0.677 0.677}
\newrgbcolor{PST@COLOR42}{0.669 0.669 0.669}
\newrgbcolor{PST@COLOR43}{0.661 0.661 0.661}
\newrgbcolor{PST@COLOR44}{0.653 0.653 0.653}
\newrgbcolor{PST@COLOR45}{0.645 0.645 0.645}
\newrgbcolor{PST@COLOR46}{0.637 0.637 0.637}
\newrgbcolor{PST@COLOR47}{0.629 0.629 0.629}
\newrgbcolor{PST@COLOR48}{0.622 0.622 0.622}
\newrgbcolor{PST@COLOR49}{0.614 0.614 0.614}
\newrgbcolor{PST@COLOR50}{0.606 0.606 0.606}
\newrgbcolor{PST@COLOR51}{0.598 0.598 0.598}
\newrgbcolor{PST@COLOR52}{0.59 0.59 0.59}
\newrgbcolor{PST@COLOR53}{0.582 0.582 0.582}
\newrgbcolor{PST@COLOR54}{0.574 0.574 0.574}
\newrgbcolor{PST@COLOR55}{0.566 0.566 0.566}
\newrgbcolor{PST@COLOR56}{0.559 0.559 0.559}
\newrgbcolor{PST@COLOR57}{0.551 0.551 0.551}
\newrgbcolor{PST@COLOR58}{0.543 0.543 0.543}
\newrgbcolor{PST@COLOR59}{0.535 0.535 0.535}
\newrgbcolor{PST@COLOR60}{0.527 0.527 0.527}
\newrgbcolor{PST@COLOR61}{0.519 0.519 0.519}
\newrgbcolor{PST@COLOR62}{0.511 0.511 0.511}
\newrgbcolor{PST@COLOR63}{0.503 0.503 0.503}
\newrgbcolor{PST@COLOR64}{0.496 0.496 0.496}
\newrgbcolor{PST@COLOR65}{0.488 0.488 0.488}
\newrgbcolor{PST@COLOR66}{0.48 0.48 0.48}
\newrgbcolor{PST@COLOR67}{0.472 0.472 0.472}
\newrgbcolor{PST@COLOR68}{0.464 0.464 0.464}
\newrgbcolor{PST@COLOR69}{0.456 0.456 0.456}
\newrgbcolor{PST@COLOR70}{0.448 0.448 0.448}
\newrgbcolor{PST@COLOR71}{0.44 0.44 0.44}
\newrgbcolor{PST@COLOR72}{0.433 0.433 0.433}
\newrgbcolor{PST@COLOR73}{0.425 0.425 0.425}
\newrgbcolor{PST@COLOR74}{0.417 0.417 0.417}
\newrgbcolor{PST@COLOR75}{0.409 0.409 0.409}
\newrgbcolor{PST@COLOR76}{0.401 0.401 0.401}
\newrgbcolor{PST@COLOR77}{0.393 0.393 0.393}
\newrgbcolor{PST@COLOR78}{0.385 0.385 0.385}
\newrgbcolor{PST@COLOR79}{0.377 0.377 0.377}
\newrgbcolor{PST@COLOR80}{0.37 0.37 0.37}
\newrgbcolor{PST@COLOR81}{0.362 0.362 0.362}
\newrgbcolor{PST@COLOR82}{0.354 0.354 0.354}
\newrgbcolor{PST@COLOR83}{0.346 0.346 0.346}
\newrgbcolor{PST@COLOR84}{0.338 0.338 0.338}
\newrgbcolor{PST@COLOR85}{0.33 0.33 0.33}
\newrgbcolor{PST@COLOR86}{0.322 0.322 0.322}
\newrgbcolor{PST@COLOR87}{0.314 0.314 0.314}
\newrgbcolor{PST@COLOR88}{0.307 0.307 0.307}
\newrgbcolor{PST@COLOR89}{0.299 0.299 0.299}
\newrgbcolor{PST@COLOR90}{0.291 0.291 0.291}
\newrgbcolor{PST@COLOR91}{0.283 0.283 0.283}
\newrgbcolor{PST@COLOR92}{0.275 0.275 0.275}
\newrgbcolor{PST@COLOR93}{0.267 0.267 0.267}
\newrgbcolor{PST@COLOR94}{0.259 0.259 0.259}
\newrgbcolor{PST@COLOR95}{0.251 0.251 0.251}
\newrgbcolor{PST@COLOR96}{0.244 0.244 0.244}
\newrgbcolor{PST@COLOR97}{0.236 0.236 0.236}
\newrgbcolor{PST@COLOR98}{0.228 0.228 0.228}
\newrgbcolor{PST@COLOR99}{0.22 0.22 0.22}
\newrgbcolor{PST@COLOR100}{0.212 0.212 0.212}
\newrgbcolor{PST@COLOR101}{0.204 0.204 0.204}
\newrgbcolor{PST@COLOR102}{0.196 0.196 0.196}
\newrgbcolor{PST@COLOR103}{0.188 0.188 0.188}
\newrgbcolor{PST@COLOR104}{0.181 0.181 0.181}
\newrgbcolor{PST@COLOR105}{0.173 0.173 0.173}
\newrgbcolor{PST@COLOR106}{0.165 0.165 0.165}
\newrgbcolor{PST@COLOR107}{0.157 0.157 0.157}
\newrgbcolor{PST@COLOR108}{0.149 0.149 0.149}
\newrgbcolor{PST@COLOR109}{0.141 0.141 0.141}
\newrgbcolor{PST@COLOR110}{0.133 0.133 0.133}
\newrgbcolor{PST@COLOR111}{0.125 0.125 0.125}
\newrgbcolor{PST@COLOR112}{0.118 0.118 0.118}
\newrgbcolor{PST@COLOR113}{0.11 0.11 0.11}
\newrgbcolor{PST@COLOR114}{0.102 0.102 0.102}
\newrgbcolor{PST@COLOR115}{0.094 0.094 0.094}
\newrgbcolor{PST@COLOR116}{0.086 0.086 0.086}
\newrgbcolor{PST@COLOR117}{0.078 0.078 0.078}
\newrgbcolor{PST@COLOR118}{0.07 0.07 0.07}
\newrgbcolor{PST@COLOR119}{0.062 0.062 0.062}
\newrgbcolor{PST@COLOR120}{0.055 0.055 0.055}
\newrgbcolor{PST@COLOR121}{0.047 0.047 0.047}
\newrgbcolor{PST@COLOR122}{0.039 0.039 0.039}
\newrgbcolor{PST@COLOR123}{0.031 0.031 0.031}
\newrgbcolor{PST@COLOR124}{0.023 0.023 0.023}
\newrgbcolor{PST@COLOR125}{0.015 0.015 0.015}
\newrgbcolor{PST@COLOR126}{0.007 0.007 0.007}
\newrgbcolor{PST@COLOR127}{0 0 0}

\def\polypmIIId#1{\pspolygon[linestyle=none,fillstyle=solid,fillcolor=PST@COLOR#1]}

\polypmIIId{125} (0.0568,0.19)  (0.0,0.19)  (0.0,0.1078)(0.0568,0.1078)
\polypmIIId{127}  (0.0568,0.272) (0.0,0.272) (0.0,0.19)  (0.0568,0.19)
\polypmIIId{74}  (0.0568,0.3542)(0.0,0.3542)(0.0,0.272) (0.0568,0.272)

\polypmIIId{127} (0.1136,   0.19)  (0.0568,0.19)  (0.0568,0.1078)(0.1136,0.1078)
\polypmIIId{127}  (0.1136,   0.272) (0.0568,0.272) (0.0568,0.19)  (0.1136,0.19)
\polypmIIId{74}  (0.1136,   0.3542)(0.0568,0.3542)(0.0568,0.272) (0.1136,0.272)

\polypmIIId{127}(0.1704,0.19)  (0.1136,   0.19)  (0.1136,   0.1078)(0.1704,0.1078)
\polypmIIId{127} (0.1704,0.272) (0.1136,   0.272) (0.1136,   0.19)  (0.1704,0.19)
\polypmIIId{125}  (0.1704,0.3542)(0.1136,   0.3542)(0.1136,   0.272) (0.1704,0.272)

\polypmIIId{127}(0.2272,0.19)  (0.1704,0.19)  (0.1704,0.1078)(0.2272,0.1078)
\polypmIIId{127}  (0.2272,0.272) (0.1704,0.272) (0.1704,0.19)  (0.2272,0.19)
\polypmIIId{127}  (0.2272,0.3542)(0.1704,0.3542)(0.1704,0.272) (0.2272,0.272)

\rput(0.0284,0.07){3}
\rput(0.0852,0.07){4}
\rput(0.1420,0.07){5}
\rput(0.1988,0.07){6}
\rput(0.1136,0.0070){years}

\PST@Border(0.0,0.3542)
(0.0,0.1078)
(0.2272,0.1078)
(0.2272,0.3542)
(0.0,0.3542)

\polypmIIId{0}(0.2329,0.1078)(0.2442,0.1078)(0.2442,0.1098)(0.2329,0.1098)
\polypmIIId{1}(0.2329,0.1097)(0.2442,0.1097)(0.2442,0.1117)(0.2329,0.1117)
\polypmIIId{2}(0.2329,0.1116)(0.2442,0.1116)(0.2442,0.1136)(0.2329,0.1136)
\polypmIIId{3}(0.2329,0.1135)(0.2442,0.1135)(0.2442,0.1156)(0.2329,0.1156)
\polypmIIId{4}(0.2329,0.1155)(0.2442,0.1155)(0.2442,0.1175)(0.2329,0.1175)
\polypmIIId{5}(0.2329,0.1174)(0.2442,0.1174)(0.2442,0.1194)(0.2329,0.1194)
\polypmIIId{6}(0.2329,0.1193)(0.2442,0.1193)(0.2442,0.1213)(0.2329,0.1213)
\polypmIIId{7}(0.2329,0.1212)(0.2442,0.1212)(0.2442,0.1233)(0.2329,0.1233)
\polypmIIId{8}(0.2329,0.1232)(0.2442,0.1232)(0.2442,0.1252)(0.2329,0.1252)
\polypmIIId{9}(0.2329,0.1251)(0.2442,0.1251)(0.2442,0.1271)(0.2329,0.1271)
\polypmIIId{10}(0.2329,0.127)(0.2442,0.127)(0.2442,0.129)(0.2329,0.129)
\polypmIIId{11}(0.2329,0.1289)(0.2442,0.1289)(0.2442,0.131)(0.2329,0.131)
\polypmIIId{12}(0.2329,0.1309)(0.2442,0.1309)(0.2442,0.1329)(0.2329,0.1329)
\polypmIIId{13}(0.2329,0.1328)(0.2442,0.1328)(0.2442,0.1348)(0.2329,0.1348)
\polypmIIId{14}(0.2329,0.1347)(0.2442,0.1347)(0.2442,0.1367)(0.2329,0.1367)
\polypmIIId{15}(0.2329,0.1366)(0.2442,0.1366)(0.2442,0.1387)(0.2329,0.1387)
\polypmIIId{16}(0.2329,0.1386)(0.2442,0.1386)(0.2442,0.1406)(0.2329,0.1406)
\polypmIIId{17}(0.2329,0.1405)(0.2442,0.1405)(0.2442,0.1425)(0.2329,0.1425)
\polypmIIId{18}(0.2329,0.1424)(0.2442,0.1424)(0.2442,0.1444)(0.2329,0.1444)
\polypmIIId{19}(0.2329,0.1443)(0.2442,0.1443)(0.2442,0.1464)(0.2329,0.1464)
\polypmIIId{20}(0.2329,0.1463)(0.2442,0.1463)(0.2442,0.1483)(0.2329,0.1483)
\polypmIIId{21}(0.2329,0.1482)(0.2442,0.1482)(0.2442,0.1502)(0.2329,0.1502)
\polypmIIId{22}(0.2329,0.1501)(0.2442,0.1501)(0.2442,0.1521)(0.2329,0.1521)
\polypmIIId{23}(0.2329,0.152)(0.2442,0.152)(0.2442,0.1541)(0.2329,0.1541)
\polypmIIId{24}(0.2329,0.154)(0.2442,0.154)(0.2442,0.156)(0.2329,0.156)
\polypmIIId{25}(0.2329,0.1559)(0.2442,0.1559)(0.2442,0.1579)(0.2329,0.1579)
\polypmIIId{26}(0.2329,0.1578)(0.2442,0.1578)(0.2442,0.1598)(0.2329,0.1598)
\polypmIIId{27}(0.2329,0.1597)(0.2442,0.1597)(0.2442,0.1618)(0.2329,0.1618)
\polypmIIId{28}(0.2329,0.1617)(0.2442,0.1617)(0.2442,0.1637)(0.2329,0.1637)
\polypmIIId{29}(0.2329,0.1636)(0.2442,0.1636)(0.2442,0.1656)(0.2329,0.1656)
\polypmIIId{30}(0.2329,0.1655)(0.2442,0.1655)(0.2442,0.1675)(0.2329,0.1675)
\polypmIIId{31}(0.2329,0.1674)(0.2442,0.1674)(0.2442,0.1695)(0.2329,0.1695)
\polypmIIId{32}(0.2329,0.1694)(0.2442,0.1694)(0.2442,0.1714)(0.2329,0.1714)
\polypmIIId{33}(0.2329,0.1713)(0.2442,0.1713)(0.2442,0.1733)(0.2329,0.1733)
\polypmIIId{34}(0.2329,0.1732)(0.2442,0.1732)(0.2442,0.1752)(0.2329,0.1752)
\polypmIIId{35}(0.2329,0.1751)(0.2442,0.1751)(0.2442,0.1772)(0.2329,0.1772)
\polypmIIId{36}(0.2329,0.1771)(0.2442,0.1771)(0.2442,0.1791)(0.2329,0.1791)
\polypmIIId{37}(0.2329,0.179)(0.2442,0.179)(0.2442,0.181)(0.2329,0.181)
\polypmIIId{38}(0.2329,0.1809)(0.2442,0.1809)(0.2442,0.1829)(0.2329,0.1829)
\polypmIIId{39}(0.2329,0.1828)(0.2442,0.1828)(0.2442,0.1849)(0.2329,0.1849)
\polypmIIId{40}(0.2329,0.1848)(0.2442,0.1848)(0.2442,0.1868)(0.2329,0.1868)
\polypmIIId{41}(0.2329,0.1867)(0.2442,0.1867)(0.2442,0.1887)(0.2329,0.1887)
\polypmIIId{42}(0.2329,0.1886)(0.2442,0.1886)(0.2442,0.1906)(0.2329,0.1906)
\polypmIIId{43}(0.2329,0.1905)(0.2442,0.1905)(0.2442,0.1926)(0.2329,0.1926)
\polypmIIId{44}(0.2329,0.1925)(0.2442,0.1925)(0.2442,0.1945)(0.2329,0.1945)
\polypmIIId{45}(0.2329,0.1944)(0.2442,0.1944)(0.2442,0.1964)(0.2329,0.1964)
\polypmIIId{46}(0.2329,0.1963)(0.2442,0.1963)(0.2442,0.1983)(0.2329,0.1983)
\polypmIIId{47}(0.2329,0.1982)(0.2442,0.1982)(0.2442,0.2003)(0.2329,0.2003)
\polypmIIId{48}(0.2329,0.2002)(0.2442,0.2002)(0.2442,0.2022)(0.2329,0.2022)
\polypmIIId{49}(0.2329,0.2021)(0.2442,0.2021)(0.2442,0.2041)(0.2329,0.2041)
\polypmIIId{50}(0.2329,0.204)(0.2442,0.204)(0.2442,0.206)(0.2329,0.206)
\polypmIIId{51}(0.2329,0.2059)(0.2442,0.2059)(0.2442,0.208)(0.2329,0.208)
\polypmIIId{52}(0.2329,0.2079)(0.2442,0.2079)(0.2442,0.2099)(0.2329,0.2099)
\polypmIIId{53}(0.2329,0.2098)(0.2442,0.2098)(0.2442,0.2118)(0.2329,0.2118)
\polypmIIId{54}(0.2329,0.2117)(0.2442,0.2117)(0.2442,0.2137)(0.2329,0.2137)
\polypmIIId{55}(0.2329,0.2136)(0.2442,0.2136)(0.2442,0.2157)(0.2329,0.2157)
\polypmIIId{56}(0.2329,0.2156)(0.2442,0.2156)(0.2442,0.2176)(0.2329,0.2176)
\polypmIIId{57}(0.2329,0.2175)(0.2442,0.2175)(0.2442,0.2195)(0.2329,0.2195)
\polypmIIId{58}(0.2329,0.2194)(0.2442,0.2194)(0.2442,0.2214)(0.2329,0.2214)
\polypmIIId{59}(0.2329,0.2213)(0.2442,0.2213)(0.2442,0.2234)(0.2329,0.2234)
\polypmIIId{60}(0.2329,0.2233)(0.2442,0.2233)(0.2442,0.2253)(0.2329,0.2253)
\polypmIIId{61}(0.2329,0.2252)(0.2442,0.2252)(0.2442,0.2272)(0.2329,0.2272)
\polypmIIId{62}(0.2329,0.2271)(0.2442,0.2271)(0.2442,0.2291)(0.2329,0.2291)
\polypmIIId{63}(0.2329,0.229)(0.2442,0.229)(0.2442,0.2311)(0.2329,0.2311)
\polypmIIId{64}(0.2329,0.231)(0.2442,0.231)(0.2442,0.233)(0.2329,0.233)
\polypmIIId{65}(0.2329,0.2329)(0.2442,0.2329)(0.2442,0.2349)(0.2329,0.2349)
\polypmIIId{66}(0.2329,0.2348)(0.2442,0.2348)(0.2442,0.2368)(0.2329,0.2368)
\polypmIIId{67}(0.2329,0.2367)(0.2442,0.2367)(0.2442,0.2388)(0.2329,0.2388)
\polypmIIId{68}(0.2329,0.2387)(0.2442,0.2387)(0.2442,0.2407)(0.2329,0.2407)
\polypmIIId{69}(0.2329,0.2406)(0.2442,0.2406)(0.2442,0.2426)(0.2329,0.2426)
\polypmIIId{70}(0.2329,0.2425)(0.2442,0.2425)(0.2442,0.2445)(0.2329,0.2445)
\polypmIIId{71}(0.2329,0.2444)(0.2442,0.2444)(0.2442,0.2465)(0.2329,0.2465)
\polypmIIId{72}(0.2329,0.2464)(0.2442,0.2464)(0.2442,0.2484)(0.2329,0.2484)
\polypmIIId{73}(0.2329,0.2483)(0.2442,0.2483)(0.2442,0.2503)(0.2329,0.2503)
\polypmIIId{74}(0.2329,0.2502)(0.2442,0.2502)(0.2442,0.2522)(0.2329,0.2522)
\polypmIIId{75}(0.2329,0.2521)(0.2442,0.2521)(0.2442,0.2542)(0.2329,0.2542)
\polypmIIId{76}(0.2329,0.2541)(0.2442,0.2541)(0.2442,0.2561)(0.2329,0.2561)
\polypmIIId{77}(0.2329,0.256)(0.2442,0.256)(0.2442,0.258)(0.2329,0.258)
\polypmIIId{78}(0.2329,0.2579)(0.2442,0.2579)(0.2442,0.2599)(0.2329,0.2599)
\polypmIIId{79}(0.2329,0.2598)(0.2442,0.2598)(0.2442,0.2619)(0.2329,0.2619)
\polypmIIId{80}(0.2329,0.2618)(0.2442,0.2618)(0.2442,0.2638)(0.2329,0.2638)
\polypmIIId{81}(0.2329,0.2637)(0.2442,0.2637)(0.2442,0.2657)(0.2329,0.2657)
\polypmIIId{82}(0.2329,0.2656)(0.2442,0.2656)(0.2442,0.2676)(0.2329,0.2676)
\polypmIIId{83}(0.2329,0.2675)(0.2442,0.2675)(0.2442,0.2696)(0.2329,0.2696)
\polypmIIId{84}(0.2329,0.2695)(0.2442,0.2695)(0.2442,0.2715)(0.2329,0.2715)
\polypmIIId{85}(0.2329,0.2714)(0.2442,0.2714)(0.2442,0.2734)(0.2329,0.2734)
\polypmIIId{86}(0.2329,0.2733)(0.2442,0.2733)(0.2442,0.2753)(0.2329,0.2753)
\polypmIIId{87}(0.2329,0.2752)(0.2442,0.2752)(0.2442,0.2773)(0.2329,0.2773)
\polypmIIId{88}(0.2329,0.2772)(0.2442,0.2772)(0.2442,0.2792)(0.2329,0.2792)
\polypmIIId{89}(0.2329,0.2791)(0.2442,0.2791)(0.2442,0.2811)(0.2329,0.2811)
\polypmIIId{90}(0.2329,0.281)(0.2442,0.281)(0.2442,0.283)(0.2329,0.283)
\polypmIIId{91}(0.2329,0.2829)(0.2442,0.2829)(0.2442,0.285)(0.2329,0.285)
\polypmIIId{92}(0.2329,0.2849)(0.2442,0.2849)(0.2442,0.2869)(0.2329,0.2869)
\polypmIIId{93}(0.2329,0.2868)(0.2442,0.2868)(0.2442,0.2888)(0.2329,0.2888)
\polypmIIId{94}(0.2329,0.2887)(0.2442,0.2887)(0.2442,0.2907)(0.2329,0.2907)
\polypmIIId{95}(0.2329,0.2906)(0.2442,0.2906)(0.2442,0.2927)(0.2329,0.2927)
\polypmIIId{96}(0.2329,0.2926)(0.2442,0.2926)(0.2442,0.2946)(0.2329,0.2946)
\polypmIIId{97}(0.2329,0.2945)(0.2442,0.2945)(0.2442,0.2965)(0.2329,0.2965)
\polypmIIId{98}(0.2329,0.2964)(0.2442,0.2964)(0.2442,0.2984)(0.2329,0.2984)
\polypmIIId{99}(0.2329,0.2983)(0.2442,0.2983)(0.2442,0.3004)(0.2329,0.3004)
\polypmIIId{100}(0.2329,0.3003)(0.2442,0.3003)(0.2442,0.3023)(0.2329,0.3023)
\polypmIIId{101}(0.2329,0.3022)(0.2442,0.3022)(0.2442,0.3042)(0.2329,0.3042)
\polypmIIId{102}(0.2329,0.3041)(0.2442,0.3041)(0.2442,0.3061)(0.2329,0.3061)
\polypmIIId{103}(0.2329,0.306)(0.2442,0.306)(0.2442,0.3081)(0.2329,0.3081)
\polypmIIId{104}(0.2329,0.308)(0.2442,0.308)(0.2442,0.31)(0.2329,0.31)
\polypmIIId{105}(0.2329,0.3099)(0.2442,0.3099)(0.2442,0.3119)(0.2329,0.3119)
\polypmIIId{106}(0.2329,0.3118)(0.2442,0.3118)(0.2442,0.3138)(0.2329,0.3138)
\polypmIIId{107}(0.2329,0.3137)(0.2442,0.3137)(0.2442,0.3158)(0.2329,0.3158)
\polypmIIId{108}(0.2329,0.3157)(0.2442,0.3157)(0.2442,0.3177)(0.2329,0.3177)
\polypmIIId{109}(0.2329,0.3176)(0.2442,0.3176)(0.2442,0.3196)(0.2329,0.3196)
\polypmIIId{110}(0.2329,0.3195)(0.2442,0.3195)(0.2442,0.3215)(0.2329,0.3215)
\polypmIIId{111}(0.2329,0.3214)(0.2442,0.3214)(0.2442,0.3235)(0.2329,0.3235)
\polypmIIId{112}(0.2329,0.3234)(0.2442,0.3234)(0.2442,0.3254)(0.2329,0.3254)
\polypmIIId{113}(0.2329,0.3253)(0.2442,0.3253)(0.2442,0.3273)(0.2329,0.3273)
\polypmIIId{114}(0.2329,0.3272)(0.2442,0.3272)(0.2442,0.3292)(0.2329,0.3292)
\polypmIIId{115}(0.2329,0.3291)(0.2442,0.3291)(0.2442,0.3312)(0.2329,0.3312)
\polypmIIId{116}(0.2329,0.3311)(0.2442,0.3311)(0.2442,0.3331)(0.2329,0.3331)
\polypmIIId{117}(0.2329,0.333)(0.2442,0.333)(0.2442,0.335)(0.2329,0.335)
\polypmIIId{118}(0.2329,0.3349)(0.2442,0.3349)(0.2442,0.3369)(0.2329,0.3369)
\polypmIIId{119}(0.2329,0.3368)(0.2442,0.3368)(0.2442,0.3389)(0.2329,0.3389)
\polypmIIId{120}(0.2329,0.3388)(0.2442,0.3388)(0.2442,0.3408)(0.2329,0.3408)
\polypmIIId{121}(0.2329,0.3407)(0.2442,0.3407)(0.2442,0.3427)(0.2329,0.3427)
\polypmIIId{122}(0.2329,0.3426)(0.2442,0.3426)(0.2442,0.3446)(0.2329,0.3446)
\polypmIIId{123}(0.2329,0.3445)(0.2442,0.3445)(0.2442,0.3466)(0.2329,0.3466)
\polypmIIId{124}(0.2329,0.3465)(0.2442,0.3465)(0.2442,0.3485)(0.2329,0.3485)
\polypmIIId{125}(0.2329,0.3484)(0.2442,0.3484)(0.2442,0.3504)(0.2329,0.3504)
\polypmIIId{126}(0.2329,0.3503)(0.2442,0.3503)(0.2442,0.3523)(0.2329,0.3523)
\polypmIIId{127}(0.2329,0.3522)(0.2442,0.3522)(0.2442,0.3542)(0.2329,0.3542)

\PST@Border(0.2329,0.1078)
(0.2442,0.1078)
(0.2442,0.3542)
(0.2329,0.3542)
(0.2329,0.1078)

\rput[l](0.2502,0.1078){0}
\rput[l](0.2502,0.1570){0.2}
\rput[l](0.2502,0.2063){0.4}
\rput[l](0.2502,0.2556){0.6}
\rput[l](0.2502,0.3049){0.8}
\rput[l](0.2502,0.3542){1}

\catcode`@=12
\fi
\endpspicture
}
  \caption{Ratio of weakly correlated instances interrupted ($\alpha = 0.1$).}
  \label{fig:result01}
\end{figure}

\begin{figure}[H]
  \centering
    \subfloat[1 resource]{% GNUPLOT: LaTeX picture using PSTRICKS macros
% Define new PST objects, if not already defined
\ifx\PSTloaded\undefined
\def\PSTloaded{t}

\catcode`@=11

\newpsobject{PST@Border}{psline}{linewidth=.0015,linestyle=solid}

\catcode`@=12

\fi
\psset{unit=5.0in,xunit=5.0in,yunit=3.0in}
\pspicture(0.000000,0.000000)(0.31, 0.35)
\ifx\nofigs\undefined
\catcode`@=11

\newrgbcolor{PST@COLOR0}{1 1 1}
\newrgbcolor{PST@COLOR1}{0.992 0.992 0.992}
\newrgbcolor{PST@COLOR3}{0.976 0.976 0.976}
\newrgbcolor{PST@COLOR6}{0.952 0.952 0.952}


\def\polypmIIId#1{\pspolygon[linestyle=none,fillstyle=solid,fillcolor=PST@COLOR#1]}

\polypmIIId{0}(0.1432,0.19)(0.0864,0.19)(0.0864,0.1078)(0.1432,0.1078)
\polypmIIId{0}(0.1432,0.272)(0.0864,0.272)(0.0864,0.19)(0.1432,0.19)
\polypmIIId{0}(0.1432,0.3542)(0.0864,0.3542)(0.0864,0.272)(0.1432,0.272)

\polypmIIId{0}(0.2,0.19)(0.1432,0.19)(0.1432,0.1078)(0.2,0.1078)
\polypmIIId{0}(0.2,0.272)(0.1432,0.272)(0.1432,0.19)(0.2,0.19)
\polypmIIId{0}(0.2,0.3542)(0.1432,0.3542)(0.1432,0.272)(0.2,0.272)

\polypmIIId{0}(0.2568,0.19)(0.2,0.19)(0.2,0.1078)(0.2568,0.1078)
\polypmIIId{0}(0.2568,0.272)(0.2,0.272)(0.2,0.19)(0.2568,0.19)
\polypmIIId{0}(0.2568,0.3542)(0.2,0.3542)(0.2,0.272)(0.2568,0.272)

\polypmIIId{0}(0.3136,0.19)(0.2568,0.19)(0.2568,0.1078)(0.3136,0.1078)
\polypmIIId{0}(0.3136,0.272)(0.2568,0.272)(0.2568,0.19)(0.3136,0.19)
\polypmIIId{0}(0.3136,0.3542)(0.2568,0.3542)(0.2568,0.272)(0.3136,0.272)

\rput(0.1148,0.07){3}
\rput(0.1716,0.07){4}
\rput(0.2284,0.07){5}
\rput(0.2852,0.07){6}
\rput(0.2000,0.0070){years}

\rput[r](0.0806,0.1489){25}
\rput[r](0.0806,0.2310){50}
\rput[r](0.0806,0.3131){100}
\rput{L}(0.0096,0.2310){actions}

\PST@Border(0.0864,0.3542)
(0.0864,0.1078)
(0.3136,0.1078)
(0.3136,0.3542)
(0.0864,0.3542)

\catcode`@=12
\fi
\endpspicture}
    \subfloat[2 resources]{% GNUPLOT: LaTeX picture using PSTRICKS macros
% Define new PST objects, if not already defined
\ifx\PSTloaded\undefined
\def\PSTloaded{t}

\catcode`@=11

\newpsobject{PST@Border}{psline}{linewidth=.0015,linestyle=solid}

\catcode`@=12

\fi
\psset{unit=5.0in,xunit=5.0in,yunit=3.0in}
\pspicture(0.000000,0.000000)(0.225000,0.35)
\ifx\nofigs\undefined
\catcode`@=11

\newrgbcolor{PST@COLOR0}{1 1 1}
\newrgbcolor{PST@COLOR2}{0.984 0.984 0.984}
\newrgbcolor{PST@COLOR8}{0.937 0.937 0.937}
\newrgbcolor{PST@COLOR10}{0.921 0.921 0.921}
\newrgbcolor{PST@COLOR55}{0.566 0.566 0.566}
\newrgbcolor{PST@COLOR96}{0.244 0.244 0.244}
\newrgbcolor{PST@COLOR117}{0.078 0.078 0.078}

\def\polypmIIId#1{\pspolygon[linestyle=none,fillstyle=solid,fillcolor=PST@COLOR#1]}

\polypmIIId{0} (0.0568,0.19)  (0.0,0.19)  (0.0,0.1078)(0.0568,0.1078)
\polypmIIId{0}  (0.0568,0.272) (0.0,0.272) (0.0,0.19)  (0.0568,0.19)
\polypmIIId{0}  (0.0568,0.3942)(0.0,0.3942)(0.0,0.272) (0.0568,0.272)

\polypmIIId{0} (0.1136,   0.19)  (0.0568,0.19)  (0.0568,0.1078)(0.1136,0.1078)
\polypmIIId{0}  (0.1136,   0.272) (0.0568,0.272) (0.0568,0.19)  (0.1136,0.19)
\polypmIIId{0}  (0.1136,   0.3942)(0.0568,0.3942)(0.0568,0.272) (0.1136,0.272)

\polypmIIId{0}(0.1704,0.19)  (0.1136,   0.19)  (0.1136,   0.1078)(0.1704,0.1078)
\polypmIIId{0} (0.1704,0.272) (0.1136,   0.272) (0.1136,   0.19)  (0.1704,0.19)
\polypmIIId{0}  (0.1704,0.3942)(0.1136,   0.3942)(0.1136,   0.272) (0.1704,0.272)

\polypmIIId{1}(0.2272,0.19)  (0.1704,0.19)  (0.1704,0.1078)(0.2272,0.1078)
\polypmIIId{0}  (0.2272,0.272) (0.1704,0.272) (0.1704,0.19)  (0.2272,0.19)
\polypmIIId{0}  (0.2272,0.3942)(0.1704,0.3942)(0.1704,0.272) (0.2272,0.272)

\rput(0.0284,0.07){3}
\rput(0.0852,0.07){4}
\rput(0.1420,0.07){5}
\rput(0.1988,0.07){6}
\rput(0.1136,0.0070){years}


\PST@Border(0.0,0.3542)
(0.0,0.1078)
(0.2272,0.1078)
(0.2272,0.3542)
(0.0,0.3542)

\catcode`@=12
\fi
\endpspicture}
    \subfloat[4 resources]{% GNUPLOT: LaTeX picture using PSTRICKS macros
% Define new PST objects, if not already defined
\ifx\PSTloaded\undefined
\def\PSTloaded{t}

\catcode`@=11

\newpsobject{PST@Border}{psline}{linewidth=.0015,linestyle=solid}

\catcode`@=12

\fi
\psset{unit=5.0in,xunit=5.0in,yunit=3.0in}
\pspicture(0.000000,0.000000)(0.3136,0.35)
\ifx\nofigs\undefined
\catcode`@=11

\newrgbcolor{PST@COLOR0}{1 1 1}
\newrgbcolor{PST@COLOR1}{0.992 0.992 0.992}
\newrgbcolor{PST@COLOR2}{0.984 0.984 0.984}
\newrgbcolor{PST@COLOR3}{0.976 0.976 0.976}
\newrgbcolor{PST@COLOR4}{0.968 0.968 0.968}
\newrgbcolor{PST@COLOR5}{0.96 0.96 0.96}
\newrgbcolor{PST@COLOR6}{0.952 0.952 0.952}
\newrgbcolor{PST@COLOR7}{0.944 0.944 0.944}
\newrgbcolor{PST@COLOR8}{0.937 0.937 0.937}
\newrgbcolor{PST@COLOR9}{0.929 0.929 0.929}
\newrgbcolor{PST@COLOR10}{0.921 0.921 0.921}
\newrgbcolor{PST@COLOR11}{0.913 0.913 0.913}
\newrgbcolor{PST@COLOR12}{0.905 0.905 0.905}
\newrgbcolor{PST@COLOR13}{0.897 0.897 0.897}
\newrgbcolor{PST@COLOR14}{0.889 0.889 0.889}
\newrgbcolor{PST@COLOR15}{0.881 0.881 0.881}
\newrgbcolor{PST@COLOR16}{0.874 0.874 0.874}
\newrgbcolor{PST@COLOR17}{0.866 0.866 0.866}
\newrgbcolor{PST@COLOR18}{0.858 0.858 0.858}
\newrgbcolor{PST@COLOR19}{0.85 0.85 0.85}
\newrgbcolor{PST@COLOR20}{0.842 0.842 0.842}
\newrgbcolor{PST@COLOR21}{0.834 0.834 0.834}
\newrgbcolor{PST@COLOR22}{0.826 0.826 0.826}
\newrgbcolor{PST@COLOR23}{0.818 0.818 0.818}
\newrgbcolor{PST@COLOR24}{0.811 0.811 0.811}
\newrgbcolor{PST@COLOR25}{0.803 0.803 0.803}
\newrgbcolor{PST@COLOR26}{0.795 0.795 0.795}
\newrgbcolor{PST@COLOR27}{0.787 0.787 0.787}
\newrgbcolor{PST@COLOR28}{0.779 0.779 0.779}
\newrgbcolor{PST@COLOR29}{0.771 0.771 0.771}
\newrgbcolor{PST@COLOR30}{0.763 0.763 0.763}
\newrgbcolor{PST@COLOR31}{0.755 0.755 0.755}
\newrgbcolor{PST@COLOR32}{0.748 0.748 0.748}
\newrgbcolor{PST@COLOR33}{0.74 0.74 0.74}
\newrgbcolor{PST@COLOR34}{0.732 0.732 0.732}
\newrgbcolor{PST@COLOR35}{0.724 0.724 0.724}
\newrgbcolor{PST@COLOR36}{0.716 0.716 0.716}
\newrgbcolor{PST@COLOR37}{0.708 0.708 0.708}
\newrgbcolor{PST@COLOR38}{0.7 0.7 0.7}
\newrgbcolor{PST@COLOR39}{0.692 0.692 0.692}
\newrgbcolor{PST@COLOR40}{0.685 0.685 0.685}
\newrgbcolor{PST@COLOR41}{0.677 0.677 0.677}
\newrgbcolor{PST@COLOR42}{0.669 0.669 0.669}
\newrgbcolor{PST@COLOR43}{0.661 0.661 0.661}
\newrgbcolor{PST@COLOR44}{0.653 0.653 0.653}
\newrgbcolor{PST@COLOR45}{0.645 0.645 0.645}
\newrgbcolor{PST@COLOR46}{0.637 0.637 0.637}
\newrgbcolor{PST@COLOR47}{0.629 0.629 0.629}
\newrgbcolor{PST@COLOR48}{0.622 0.622 0.622}
\newrgbcolor{PST@COLOR49}{0.614 0.614 0.614}
\newrgbcolor{PST@COLOR50}{0.606 0.606 0.606}
\newrgbcolor{PST@COLOR51}{0.598 0.598 0.598}
\newrgbcolor{PST@COLOR52}{0.59 0.59 0.59}
\newrgbcolor{PST@COLOR53}{0.582 0.582 0.582}
\newrgbcolor{PST@COLOR54}{0.574 0.574 0.574}
\newrgbcolor{PST@COLOR55}{0.566 0.566 0.566}
\newrgbcolor{PST@COLOR56}{0.559 0.559 0.559}
\newrgbcolor{PST@COLOR57}{0.551 0.551 0.551}
\newrgbcolor{PST@COLOR58}{0.543 0.543 0.543}
\newrgbcolor{PST@COLOR59}{0.535 0.535 0.535}
\newrgbcolor{PST@COLOR60}{0.527 0.527 0.527}
\newrgbcolor{PST@COLOR61}{0.519 0.519 0.519}
\newrgbcolor{PST@COLOR62}{0.511 0.511 0.511}
\newrgbcolor{PST@COLOR63}{0.503 0.503 0.503}
\newrgbcolor{PST@COLOR64}{0.496 0.496 0.496}
\newrgbcolor{PST@COLOR65}{0.488 0.488 0.488}
\newrgbcolor{PST@COLOR66}{0.48 0.48 0.48}
\newrgbcolor{PST@COLOR67}{0.472 0.472 0.472}
\newrgbcolor{PST@COLOR68}{0.464 0.464 0.464}
\newrgbcolor{PST@COLOR69}{0.456 0.456 0.456}
\newrgbcolor{PST@COLOR70}{0.448 0.448 0.448}
\newrgbcolor{PST@COLOR71}{0.44 0.44 0.44}
\newrgbcolor{PST@COLOR72}{0.433 0.433 0.433}
\newrgbcolor{PST@COLOR73}{0.425 0.425 0.425}
\newrgbcolor{PST@COLOR74}{0.417 0.417 0.417}
\newrgbcolor{PST@COLOR75}{0.409 0.409 0.409}
\newrgbcolor{PST@COLOR76}{0.401 0.401 0.401}
\newrgbcolor{PST@COLOR77}{0.393 0.393 0.393}
\newrgbcolor{PST@COLOR78}{0.385 0.385 0.385}
\newrgbcolor{PST@COLOR79}{0.377 0.377 0.377}
\newrgbcolor{PST@COLOR80}{0.37 0.37 0.37}
\newrgbcolor{PST@COLOR81}{0.362 0.362 0.362}
\newrgbcolor{PST@COLOR82}{0.354 0.354 0.354}
\newrgbcolor{PST@COLOR83}{0.346 0.346 0.346}
\newrgbcolor{PST@COLOR84}{0.338 0.338 0.338}
\newrgbcolor{PST@COLOR85}{0.33 0.33 0.33}
\newrgbcolor{PST@COLOR86}{0.322 0.322 0.322}
\newrgbcolor{PST@COLOR87}{0.314 0.314 0.314}
\newrgbcolor{PST@COLOR88}{0.307 0.307 0.307}
\newrgbcolor{PST@COLOR89}{0.299 0.299 0.299}
\newrgbcolor{PST@COLOR90}{0.291 0.291 0.291}
\newrgbcolor{PST@COLOR91}{0.283 0.283 0.283}
\newrgbcolor{PST@COLOR92}{0.275 0.275 0.275}
\newrgbcolor{PST@COLOR93}{0.267 0.267 0.267}
\newrgbcolor{PST@COLOR94}{0.259 0.259 0.259}
\newrgbcolor{PST@COLOR95}{0.251 0.251 0.251}
\newrgbcolor{PST@COLOR96}{0.244 0.244 0.244}
\newrgbcolor{PST@COLOR97}{0.236 0.236 0.236}
\newrgbcolor{PST@COLOR98}{0.228 0.228 0.228}
\newrgbcolor{PST@COLOR99}{0.22 0.22 0.22}
\newrgbcolor{PST@COLOR100}{0.212 0.212 0.212}
\newrgbcolor{PST@COLOR101}{0.204 0.204 0.204}
\newrgbcolor{PST@COLOR102}{0.196 0.196 0.196}
\newrgbcolor{PST@COLOR103}{0.188 0.188 0.188}
\newrgbcolor{PST@COLOR104}{0.181 0.181 0.181}
\newrgbcolor{PST@COLOR105}{0.173 0.173 0.173}
\newrgbcolor{PST@COLOR106}{0.165 0.165 0.165}
\newrgbcolor{PST@COLOR107}{0.157 0.157 0.157}
\newrgbcolor{PST@COLOR108}{0.149 0.149 0.149}
\newrgbcolor{PST@COLOR109}{0.141 0.141 0.141}
\newrgbcolor{PST@COLOR110}{0.133 0.133 0.133}
\newrgbcolor{PST@COLOR111}{0.125 0.125 0.125}
\newrgbcolor{PST@COLOR112}{0.118 0.118 0.118}
\newrgbcolor{PST@COLOR113}{0.11 0.11 0.11}
\newrgbcolor{PST@COLOR114}{0.102 0.102 0.102}
\newrgbcolor{PST@COLOR115}{0.094 0.094 0.094}
\newrgbcolor{PST@COLOR116}{0.086 0.086 0.086}
\newrgbcolor{PST@COLOR117}{0.078 0.078 0.078}
\newrgbcolor{PST@COLOR118}{0.07 0.07 0.07}
\newrgbcolor{PST@COLOR119}{0.062 0.062 0.062}
\newrgbcolor{PST@COLOR120}{0.055 0.055 0.055}
\newrgbcolor{PST@COLOR121}{0.047 0.047 0.047}
\newrgbcolor{PST@COLOR122}{0.039 0.039 0.039}
\newrgbcolor{PST@COLOR123}{0.031 0.031 0.031}
\newrgbcolor{PST@COLOR124}{0.023 0.023 0.023}
\newrgbcolor{PST@COLOR125}{0.015 0.015 0.015}
\newrgbcolor{PST@COLOR126}{0.007 0.007 0.007}
\newrgbcolor{PST@COLOR127}{0 0 0}

\def\polypmIIId#1{\pspolygon[linestyle=none,fillstyle=solid,fillcolor=PST@COLOR#1]}

\polypmIIId{0} (0.0568,0.19)  (0.0,0.19)  (0.0,0.1078)(0.0568,0.1078)
\polypmIIId{0}  (0.0568,0.272) (0.0,0.272) (0.0,0.19)  (0.0568,0.19)
\polypmIIId{0}  (0.0568,0.3542)(0.0,0.3542)(0.0,0.272) (0.0568,0.272)

\polypmIIId{0} (0.1136,   0.19)  (0.0568,0.19)  (0.0568,0.1078)(0.1136,0.1078)
\polypmIIId{0}  (0.1136,   0.272) (0.0568,0.272) (0.0568,0.19)  (0.1136,0.19)
\polypmIIId{0}  (0.1136,   0.3542)(0.0568,0.3542)(0.0568,0.272) (0.1136,0.272)

\polypmIIId{3}(0.1704,0.19)  (0.1136,   0.19)  (0.1136,   0.1078)(0.1704,0.1078)
\polypmIIId{2} (0.1704,0.272) (0.1136,   0.272) (0.1136,   0.19)  (0.1704,0.19)
\polypmIIId{1}  (0.1704,0.3542)(0.1136,   0.3542)(0.1136,   0.272) (0.1704,0.272)

\polypmIIId{7}(0.2272,0.19)  (0.1704,0.19)  (0.1704,0.1078)(0.2272,0.1078)
\polypmIIId{7}  (0.2272,0.272) (0.1704,0.272) (0.1704,0.19)  (0.2272,0.19)
\polypmIIId{0}  (0.2272,0.3542)(0.1704,0.3542)(0.1704,0.272) (0.2272,0.272)

\rput(0.0284,0.07){3}
\rput(0.0852,0.07){4}
\rput(0.1420,0.07){5}
\rput(0.1988,0.07){6}
\rput(0.1136,0.0070){years}

\PST@Border(0.0,0.3542)
(0.0,0.1078)
(0.2272,0.1078)
(0.2272,0.3542)
(0.0,0.3542)

\polypmIIId{0}(0.2329,0.1078)(0.2442,0.1078)(0.2442,0.1098)(0.2329,0.1098)
\polypmIIId{1}(0.2329,0.1097)(0.2442,0.1097)(0.2442,0.1117)(0.2329,0.1117)
\polypmIIId{2}(0.2329,0.1116)(0.2442,0.1116)(0.2442,0.1136)(0.2329,0.1136)
\polypmIIId{3}(0.2329,0.1135)(0.2442,0.1135)(0.2442,0.1156)(0.2329,0.1156)
\polypmIIId{4}(0.2329,0.1155)(0.2442,0.1155)(0.2442,0.1175)(0.2329,0.1175)
\polypmIIId{5}(0.2329,0.1174)(0.2442,0.1174)(0.2442,0.1194)(0.2329,0.1194)
\polypmIIId{6}(0.2329,0.1193)(0.2442,0.1193)(0.2442,0.1213)(0.2329,0.1213)
\polypmIIId{7}(0.2329,0.1212)(0.2442,0.1212)(0.2442,0.1233)(0.2329,0.1233)
\polypmIIId{8}(0.2329,0.1232)(0.2442,0.1232)(0.2442,0.1252)(0.2329,0.1252)
\polypmIIId{9}(0.2329,0.1251)(0.2442,0.1251)(0.2442,0.1271)(0.2329,0.1271)
\polypmIIId{10}(0.2329,0.127)(0.2442,0.127)(0.2442,0.129)(0.2329,0.129)
\polypmIIId{11}(0.2329,0.1289)(0.2442,0.1289)(0.2442,0.131)(0.2329,0.131)
\polypmIIId{12}(0.2329,0.1309)(0.2442,0.1309)(0.2442,0.1329)(0.2329,0.1329)
\polypmIIId{13}(0.2329,0.1328)(0.2442,0.1328)(0.2442,0.1348)(0.2329,0.1348)
\polypmIIId{14}(0.2329,0.1347)(0.2442,0.1347)(0.2442,0.1367)(0.2329,0.1367)
\polypmIIId{15}(0.2329,0.1366)(0.2442,0.1366)(0.2442,0.1387)(0.2329,0.1387)
\polypmIIId{16}(0.2329,0.1386)(0.2442,0.1386)(0.2442,0.1406)(0.2329,0.1406)
\polypmIIId{17}(0.2329,0.1405)(0.2442,0.1405)(0.2442,0.1425)(0.2329,0.1425)
\polypmIIId{18}(0.2329,0.1424)(0.2442,0.1424)(0.2442,0.1444)(0.2329,0.1444)
\polypmIIId{19}(0.2329,0.1443)(0.2442,0.1443)(0.2442,0.1464)(0.2329,0.1464)
\polypmIIId{20}(0.2329,0.1463)(0.2442,0.1463)(0.2442,0.1483)(0.2329,0.1483)
\polypmIIId{21}(0.2329,0.1482)(0.2442,0.1482)(0.2442,0.1502)(0.2329,0.1502)
\polypmIIId{22}(0.2329,0.1501)(0.2442,0.1501)(0.2442,0.1521)(0.2329,0.1521)
\polypmIIId{23}(0.2329,0.152)(0.2442,0.152)(0.2442,0.1541)(0.2329,0.1541)
\polypmIIId{24}(0.2329,0.154)(0.2442,0.154)(0.2442,0.156)(0.2329,0.156)
\polypmIIId{25}(0.2329,0.1559)(0.2442,0.1559)(0.2442,0.1579)(0.2329,0.1579)
\polypmIIId{26}(0.2329,0.1578)(0.2442,0.1578)(0.2442,0.1598)(0.2329,0.1598)
\polypmIIId{27}(0.2329,0.1597)(0.2442,0.1597)(0.2442,0.1618)(0.2329,0.1618)
\polypmIIId{28}(0.2329,0.1617)(0.2442,0.1617)(0.2442,0.1637)(0.2329,0.1637)
\polypmIIId{29}(0.2329,0.1636)(0.2442,0.1636)(0.2442,0.1656)(0.2329,0.1656)
\polypmIIId{30}(0.2329,0.1655)(0.2442,0.1655)(0.2442,0.1675)(0.2329,0.1675)
\polypmIIId{31}(0.2329,0.1674)(0.2442,0.1674)(0.2442,0.1695)(0.2329,0.1695)
\polypmIIId{32}(0.2329,0.1694)(0.2442,0.1694)(0.2442,0.1714)(0.2329,0.1714)
\polypmIIId{33}(0.2329,0.1713)(0.2442,0.1713)(0.2442,0.1733)(0.2329,0.1733)
\polypmIIId{34}(0.2329,0.1732)(0.2442,0.1732)(0.2442,0.1752)(0.2329,0.1752)
\polypmIIId{35}(0.2329,0.1751)(0.2442,0.1751)(0.2442,0.1772)(0.2329,0.1772)
\polypmIIId{36}(0.2329,0.1771)(0.2442,0.1771)(0.2442,0.1791)(0.2329,0.1791)
\polypmIIId{37}(0.2329,0.179)(0.2442,0.179)(0.2442,0.181)(0.2329,0.181)
\polypmIIId{38}(0.2329,0.1809)(0.2442,0.1809)(0.2442,0.1829)(0.2329,0.1829)
\polypmIIId{39}(0.2329,0.1828)(0.2442,0.1828)(0.2442,0.1849)(0.2329,0.1849)
\polypmIIId{40}(0.2329,0.1848)(0.2442,0.1848)(0.2442,0.1868)(0.2329,0.1868)
\polypmIIId{41}(0.2329,0.1867)(0.2442,0.1867)(0.2442,0.1887)(0.2329,0.1887)
\polypmIIId{42}(0.2329,0.1886)(0.2442,0.1886)(0.2442,0.1906)(0.2329,0.1906)
\polypmIIId{43}(0.2329,0.1905)(0.2442,0.1905)(0.2442,0.1926)(0.2329,0.1926)
\polypmIIId{44}(0.2329,0.1925)(0.2442,0.1925)(0.2442,0.1945)(0.2329,0.1945)
\polypmIIId{45}(0.2329,0.1944)(0.2442,0.1944)(0.2442,0.1964)(0.2329,0.1964)
\polypmIIId{46}(0.2329,0.1963)(0.2442,0.1963)(0.2442,0.1983)(0.2329,0.1983)
\polypmIIId{47}(0.2329,0.1982)(0.2442,0.1982)(0.2442,0.2003)(0.2329,0.2003)
\polypmIIId{48}(0.2329,0.2002)(0.2442,0.2002)(0.2442,0.2022)(0.2329,0.2022)
\polypmIIId{49}(0.2329,0.2021)(0.2442,0.2021)(0.2442,0.2041)(0.2329,0.2041)
\polypmIIId{50}(0.2329,0.204)(0.2442,0.204)(0.2442,0.206)(0.2329,0.206)
\polypmIIId{51}(0.2329,0.2059)(0.2442,0.2059)(0.2442,0.208)(0.2329,0.208)
\polypmIIId{52}(0.2329,0.2079)(0.2442,0.2079)(0.2442,0.2099)(0.2329,0.2099)
\polypmIIId{53}(0.2329,0.2098)(0.2442,0.2098)(0.2442,0.2118)(0.2329,0.2118)
\polypmIIId{54}(0.2329,0.2117)(0.2442,0.2117)(0.2442,0.2137)(0.2329,0.2137)
\polypmIIId{55}(0.2329,0.2136)(0.2442,0.2136)(0.2442,0.2157)(0.2329,0.2157)
\polypmIIId{56}(0.2329,0.2156)(0.2442,0.2156)(0.2442,0.2176)(0.2329,0.2176)
\polypmIIId{57}(0.2329,0.2175)(0.2442,0.2175)(0.2442,0.2195)(0.2329,0.2195)
\polypmIIId{58}(0.2329,0.2194)(0.2442,0.2194)(0.2442,0.2214)(0.2329,0.2214)
\polypmIIId{59}(0.2329,0.2213)(0.2442,0.2213)(0.2442,0.2234)(0.2329,0.2234)
\polypmIIId{60}(0.2329,0.2233)(0.2442,0.2233)(0.2442,0.2253)(0.2329,0.2253)
\polypmIIId{61}(0.2329,0.2252)(0.2442,0.2252)(0.2442,0.2272)(0.2329,0.2272)
\polypmIIId{62}(0.2329,0.2271)(0.2442,0.2271)(0.2442,0.2291)(0.2329,0.2291)
\polypmIIId{63}(0.2329,0.229)(0.2442,0.229)(0.2442,0.2311)(0.2329,0.2311)
\polypmIIId{64}(0.2329,0.231)(0.2442,0.231)(0.2442,0.233)(0.2329,0.233)
\polypmIIId{65}(0.2329,0.2329)(0.2442,0.2329)(0.2442,0.2349)(0.2329,0.2349)
\polypmIIId{66}(0.2329,0.2348)(0.2442,0.2348)(0.2442,0.2368)(0.2329,0.2368)
\polypmIIId{67}(0.2329,0.2367)(0.2442,0.2367)(0.2442,0.2388)(0.2329,0.2388)
\polypmIIId{68}(0.2329,0.2387)(0.2442,0.2387)(0.2442,0.2407)(0.2329,0.2407)
\polypmIIId{69}(0.2329,0.2406)(0.2442,0.2406)(0.2442,0.2426)(0.2329,0.2426)
\polypmIIId{70}(0.2329,0.2425)(0.2442,0.2425)(0.2442,0.2445)(0.2329,0.2445)
\polypmIIId{71}(0.2329,0.2444)(0.2442,0.2444)(0.2442,0.2465)(0.2329,0.2465)
\polypmIIId{72}(0.2329,0.2464)(0.2442,0.2464)(0.2442,0.2484)(0.2329,0.2484)
\polypmIIId{73}(0.2329,0.2483)(0.2442,0.2483)(0.2442,0.2503)(0.2329,0.2503)
\polypmIIId{74}(0.2329,0.2502)(0.2442,0.2502)(0.2442,0.2522)(0.2329,0.2522)
\polypmIIId{75}(0.2329,0.2521)(0.2442,0.2521)(0.2442,0.2542)(0.2329,0.2542)
\polypmIIId{76}(0.2329,0.2541)(0.2442,0.2541)(0.2442,0.2561)(0.2329,0.2561)
\polypmIIId{77}(0.2329,0.256)(0.2442,0.256)(0.2442,0.258)(0.2329,0.258)
\polypmIIId{78}(0.2329,0.2579)(0.2442,0.2579)(0.2442,0.2599)(0.2329,0.2599)
\polypmIIId{79}(0.2329,0.2598)(0.2442,0.2598)(0.2442,0.2619)(0.2329,0.2619)
\polypmIIId{80}(0.2329,0.2618)(0.2442,0.2618)(0.2442,0.2638)(0.2329,0.2638)
\polypmIIId{81}(0.2329,0.2637)(0.2442,0.2637)(0.2442,0.2657)(0.2329,0.2657)
\polypmIIId{82}(0.2329,0.2656)(0.2442,0.2656)(0.2442,0.2676)(0.2329,0.2676)
\polypmIIId{83}(0.2329,0.2675)(0.2442,0.2675)(0.2442,0.2696)(0.2329,0.2696)
\polypmIIId{84}(0.2329,0.2695)(0.2442,0.2695)(0.2442,0.2715)(0.2329,0.2715)
\polypmIIId{85}(0.2329,0.2714)(0.2442,0.2714)(0.2442,0.2734)(0.2329,0.2734)
\polypmIIId{86}(0.2329,0.2733)(0.2442,0.2733)(0.2442,0.2753)(0.2329,0.2753)
\polypmIIId{87}(0.2329,0.2752)(0.2442,0.2752)(0.2442,0.2773)(0.2329,0.2773)
\polypmIIId{88}(0.2329,0.2772)(0.2442,0.2772)(0.2442,0.2792)(0.2329,0.2792)
\polypmIIId{89}(0.2329,0.2791)(0.2442,0.2791)(0.2442,0.2811)(0.2329,0.2811)
\polypmIIId{90}(0.2329,0.281)(0.2442,0.281)(0.2442,0.283)(0.2329,0.283)
\polypmIIId{91}(0.2329,0.2829)(0.2442,0.2829)(0.2442,0.285)(0.2329,0.285)
\polypmIIId{92}(0.2329,0.2849)(0.2442,0.2849)(0.2442,0.2869)(0.2329,0.2869)
\polypmIIId{93}(0.2329,0.2868)(0.2442,0.2868)(0.2442,0.2888)(0.2329,0.2888)
\polypmIIId{94}(0.2329,0.2887)(0.2442,0.2887)(0.2442,0.2907)(0.2329,0.2907)
\polypmIIId{95}(0.2329,0.2906)(0.2442,0.2906)(0.2442,0.2927)(0.2329,0.2927)
\polypmIIId{96}(0.2329,0.2926)(0.2442,0.2926)(0.2442,0.2946)(0.2329,0.2946)
\polypmIIId{97}(0.2329,0.2945)(0.2442,0.2945)(0.2442,0.2965)(0.2329,0.2965)
\polypmIIId{98}(0.2329,0.2964)(0.2442,0.2964)(0.2442,0.2984)(0.2329,0.2984)
\polypmIIId{99}(0.2329,0.2983)(0.2442,0.2983)(0.2442,0.3004)(0.2329,0.3004)
\polypmIIId{100}(0.2329,0.3003)(0.2442,0.3003)(0.2442,0.3023)(0.2329,0.3023)
\polypmIIId{101}(0.2329,0.3022)(0.2442,0.3022)(0.2442,0.3042)(0.2329,0.3042)
\polypmIIId{102}(0.2329,0.3041)(0.2442,0.3041)(0.2442,0.3061)(0.2329,0.3061)
\polypmIIId{103}(0.2329,0.306)(0.2442,0.306)(0.2442,0.3081)(0.2329,0.3081)
\polypmIIId{104}(0.2329,0.308)(0.2442,0.308)(0.2442,0.31)(0.2329,0.31)
\polypmIIId{105}(0.2329,0.3099)(0.2442,0.3099)(0.2442,0.3119)(0.2329,0.3119)
\polypmIIId{106}(0.2329,0.3118)(0.2442,0.3118)(0.2442,0.3138)(0.2329,0.3138)
\polypmIIId{107}(0.2329,0.3137)(0.2442,0.3137)(0.2442,0.3158)(0.2329,0.3158)
\polypmIIId{108}(0.2329,0.3157)(0.2442,0.3157)(0.2442,0.3177)(0.2329,0.3177)
\polypmIIId{109}(0.2329,0.3176)(0.2442,0.3176)(0.2442,0.3196)(0.2329,0.3196)
\polypmIIId{110}(0.2329,0.3195)(0.2442,0.3195)(0.2442,0.3215)(0.2329,0.3215)
\polypmIIId{111}(0.2329,0.3214)(0.2442,0.3214)(0.2442,0.3235)(0.2329,0.3235)
\polypmIIId{112}(0.2329,0.3234)(0.2442,0.3234)(0.2442,0.3254)(0.2329,0.3254)
\polypmIIId{113}(0.2329,0.3253)(0.2442,0.3253)(0.2442,0.3273)(0.2329,0.3273)
\polypmIIId{114}(0.2329,0.3272)(0.2442,0.3272)(0.2442,0.3292)(0.2329,0.3292)
\polypmIIId{115}(0.2329,0.3291)(0.2442,0.3291)(0.2442,0.3312)(0.2329,0.3312)
\polypmIIId{116}(0.2329,0.3311)(0.2442,0.3311)(0.2442,0.3331)(0.2329,0.3331)
\polypmIIId{117}(0.2329,0.333)(0.2442,0.333)(0.2442,0.335)(0.2329,0.335)
\polypmIIId{118}(0.2329,0.3349)(0.2442,0.3349)(0.2442,0.3369)(0.2329,0.3369)
\polypmIIId{119}(0.2329,0.3368)(0.2442,0.3368)(0.2442,0.3389)(0.2329,0.3389)
\polypmIIId{120}(0.2329,0.3388)(0.2442,0.3388)(0.2442,0.3408)(0.2329,0.3408)
\polypmIIId{121}(0.2329,0.3407)(0.2442,0.3407)(0.2442,0.3427)(0.2329,0.3427)
\polypmIIId{122}(0.2329,0.3426)(0.2442,0.3426)(0.2442,0.3446)(0.2329,0.3446)
\polypmIIId{123}(0.2329,0.3445)(0.2442,0.3445)(0.2442,0.3466)(0.2329,0.3466)
\polypmIIId{124}(0.2329,0.3465)(0.2442,0.3465)(0.2442,0.3485)(0.2329,0.3485)
\polypmIIId{125}(0.2329,0.3484)(0.2442,0.3484)(0.2442,0.3504)(0.2329,0.3504)
\polypmIIId{126}(0.2329,0.3503)(0.2442,0.3503)(0.2442,0.3523)(0.2329,0.3523)
\polypmIIId{127}(0.2329,0.3522)(0.2442,0.3522)(0.2442,0.3542)(0.2329,0.3542)

\PST@Border(0.2329,0.1078)
(0.2442,0.1078)
(0.2442,0.3542)
(0.2329,0.3542)
(0.2329,0.1078)

\rput[l](0.2502,0.1078){0}
\rput[l](0.2502,0.1570){0.2}
\rput[l](0.2502,0.2063){0.4}
\rput[l](0.2502,0.2556){0.6}
\rput[l](0.2502,0.3049){0.8}
\rput[l](0.2502,0.3542){1}

\catcode`@=12
\fi
\endpspicture}
  \caption{Ratio of uncorrelated instances interrupted ($\alpha = 1.0$).}
  \label{fig:result10}
\end{figure}

Comparing figures~\ref{fig:result00} and~\ref{fig:result01} to figure~\ref{fig:result10} one can see the first two figures are darker 
than the last, meaning that the correlation level indeed affects the instances hardness. As expected, uncorrelated instances
are easier for the CPLEX to solve, with just a few instances with bigger dimensions being interrupted. However, comparing only 
figure~\ref{fig:result00} to figure~\ref{fig:result01}, it appears that weakly correlated instances are harder to solve than
strongly correlated instances, as opposed to the expected.

Concerning quantity of years and resources of the instances, the obtained results confirm what was expected, since 
figures~\ref{fig:result00} to~\ref{fig:result10}
show that the bigger the instances are on this two parameters, the harder it gets for the CPLEX solver to solve them to optimality. 
Still, comparing the three figures observing the hardness in relation to the variation on the number of actions, it seems that instances with fewer actions are 
harder to solve to optimallity.

In relation to the time taken to solve the instances, the CPLEX took on average 380 seconds to solve the instances. The average gap was 
0.01\%, meaning that the CPLEX found solutions that were, in average, at least 99.99\% of the optimal ones.

\subsection{Solutions Quality}

To analyse the quality of the solutions obtained by the heuristics, the same instances previously solved with CPLEX were now solved with
the GALP and TSLP, and the solutions found were compared to the ones obtained by CPLEX before reaching the time limit. 
Figures~\ref{fig:tabusolcomp00} to \ref{fig:tabusolcomp10} show the results for the TSLP. The heatmaps follow the same configuration 
of the CPLEX tests, but now showing the average quality of the solutions. The color scale represents
the average ratio between the solutions found by the TSLP and the best known solution, in other words, darker tones indicate solutions
with better quality were found by the TSLP. The same representations are used on figures~\ref{fig:greedysolcomp00} to \ref{fig:greedysolcomp10} 
to present the results obtained with GALP.

\begin{figure}[H]
  \centering
    \subfloat[1 resource]{% GNUPLOT: LaTeX picture using PSTRICKS macros
% Define new PST objects, if not already defined
\ifx\PSTloaded\undefined
\def\PSTloaded{t}

\catcode`@=11

\newpsobject{PST@Border}{psline}{linewidth=.0015,linestyle=solid}

\catcode`@=12

\fi
\psset{unit=5.0in,xunit=5.0in,yunit=3.0in}
\pspicture(0.000000,0.000000)(0.31, 0.35)
\ifx\nofigs\undefined
\catcode`@=11

\newrgbcolor{PST@COLOR0}{1 1 1}
\newrgbcolor{PST@COLOR1}{0.992 0.992 0.992}
\newrgbcolor{PST@COLOR2}{0.984 0.984 0.984}
\newrgbcolor{PST@COLOR3}{0.976 0.976 0.976}
\newrgbcolor{PST@COLOR4}{0.968 0.968 0.968}
\newrgbcolor{PST@COLOR5}{0.96 0.96 0.96}
\newrgbcolor{PST@COLOR6}{0.952 0.952 0.952}
\newrgbcolor{PST@COLOR7}{0.944 0.944 0.944}
\newrgbcolor{PST@COLOR8}{0.937 0.937 0.937}
\newrgbcolor{PST@COLOR9}{0.929 0.929 0.929}
\newrgbcolor{PST@COLOR10}{0.921 0.921 0.921}
\newrgbcolor{PST@COLOR11}{0.913 0.913 0.913}
\newrgbcolor{PST@COLOR12}{0.905 0.905 0.905}
\newrgbcolor{PST@COLOR13}{0.897 0.897 0.897}
\newrgbcolor{PST@COLOR14}{0.889 0.889 0.889}
\newrgbcolor{PST@COLOR15}{0.881 0.881 0.881}
\newrgbcolor{PST@COLOR16}{0.874 0.874 0.874}
\newrgbcolor{PST@COLOR17}{0.866 0.866 0.866}
\newrgbcolor{PST@COLOR18}{0.858 0.858 0.858}
\newrgbcolor{PST@COLOR19}{0.85 0.85 0.85}
\newrgbcolor{PST@COLOR20}{0.842 0.842 0.842}
\newrgbcolor{PST@COLOR21}{0.834 0.834 0.834}
\newrgbcolor{PST@COLOR22}{0.826 0.826 0.826}
\newrgbcolor{PST@COLOR23}{0.818 0.818 0.818}
\newrgbcolor{PST@COLOR24}{0.811 0.811 0.811}
\newrgbcolor{PST@COLOR25}{0.803 0.803 0.803}
\newrgbcolor{PST@COLOR26}{0.795 0.795 0.795}
\newrgbcolor{PST@COLOR27}{0.787 0.787 0.787}
\newrgbcolor{PST@COLOR28}{0.779 0.779 0.779}
\newrgbcolor{PST@COLOR29}{0.771 0.771 0.771}
\newrgbcolor{PST@COLOR30}{0.763 0.763 0.763}
\newrgbcolor{PST@COLOR31}{0.755 0.755 0.755}
\newrgbcolor{PST@COLOR32}{0.748 0.748 0.748}
\newrgbcolor{PST@COLOR33}{0.74 0.74 0.74}
\newrgbcolor{PST@COLOR34}{0.732 0.732 0.732}
\newrgbcolor{PST@COLOR35}{0.724 0.724 0.724}
\newrgbcolor{PST@COLOR36}{0.716 0.716 0.716}
\newrgbcolor{PST@COLOR37}{0.708 0.708 0.708}
\newrgbcolor{PST@COLOR38}{0.7 0.7 0.7}
\newrgbcolor{PST@COLOR39}{0.692 0.692 0.692}
\newrgbcolor{PST@COLOR40}{0.685 0.685 0.685}
\newrgbcolor{PST@COLOR41}{0.677 0.677 0.677}
\newrgbcolor{PST@COLOR42}{0.669 0.669 0.669}
\newrgbcolor{PST@COLOR43}{0.661 0.661 0.661}
\newrgbcolor{PST@COLOR44}{0.653 0.653 0.653}
\newrgbcolor{PST@COLOR45}{0.645 0.645 0.645}
\newrgbcolor{PST@COLOR46}{0.637 0.637 0.637}
\newrgbcolor{PST@COLOR47}{0.629 0.629 0.629}
\newrgbcolor{PST@COLOR48}{0.622 0.622 0.622}
\newrgbcolor{PST@COLOR49}{0.614 0.614 0.614}
\newrgbcolor{PST@COLOR50}{0.606 0.606 0.606}
\newrgbcolor{PST@COLOR51}{0.598 0.598 0.598}
\newrgbcolor{PST@COLOR52}{0.59 0.59 0.59}
\newrgbcolor{PST@COLOR53}{0.582 0.582 0.582}
\newrgbcolor{PST@COLOR54}{0.574 0.574 0.574}
\newrgbcolor{PST@COLOR55}{0.566 0.566 0.566}
\newrgbcolor{PST@COLOR56}{0.559 0.559 0.559}
\newrgbcolor{PST@COLOR57}{0.551 0.551 0.551}
\newrgbcolor{PST@COLOR58}{0.543 0.543 0.543}
\newrgbcolor{PST@COLOR59}{0.535 0.535 0.535}
\newrgbcolor{PST@COLOR60}{0.527 0.527 0.527}
\newrgbcolor{PST@COLOR61}{0.519 0.519 0.519}
\newrgbcolor{PST@COLOR62}{0.511 0.511 0.511}
\newrgbcolor{PST@COLOR63}{0.503 0.503 0.503}
\newrgbcolor{PST@COLOR64}{0.496 0.496 0.496}
\newrgbcolor{PST@COLOR65}{0.488 0.488 0.488}
\newrgbcolor{PST@COLOR66}{0.48 0.48 0.48}
\newrgbcolor{PST@COLOR67}{0.472 0.472 0.472}
\newrgbcolor{PST@COLOR68}{0.464 0.464 0.464}
\newrgbcolor{PST@COLOR69}{0.456 0.456 0.456}
\newrgbcolor{PST@COLOR70}{0.448 0.448 0.448}
\newrgbcolor{PST@COLOR71}{0.44 0.44 0.44}
\newrgbcolor{PST@COLOR72}{0.433 0.433 0.433}
\newrgbcolor{PST@COLOR73}{0.425 0.425 0.425}
\newrgbcolor{PST@COLOR74}{0.417 0.417 0.417}
\newrgbcolor{PST@COLOR75}{0.409 0.409 0.409}
\newrgbcolor{PST@COLOR76}{0.401 0.401 0.401}
\newrgbcolor{PST@COLOR77}{0.393 0.393 0.393}
\newrgbcolor{PST@COLOR78}{0.385 0.385 0.385}
\newrgbcolor{PST@COLOR79}{0.377 0.377 0.377}
\newrgbcolor{PST@COLOR80}{0.37 0.37 0.37}
\newrgbcolor{PST@COLOR81}{0.362 0.362 0.362}
\newrgbcolor{PST@COLOR82}{0.354 0.354 0.354}
\newrgbcolor{PST@COLOR83}{0.346 0.346 0.346}
\newrgbcolor{PST@COLOR84}{0.338 0.338 0.338}
\newrgbcolor{PST@COLOR85}{0.33 0.33 0.33}
\newrgbcolor{PST@COLOR86}{0.322 0.322 0.322}
\newrgbcolor{PST@COLOR87}{0.314 0.314 0.314}
\newrgbcolor{PST@COLOR88}{0.307 0.307 0.307}
\newrgbcolor{PST@COLOR89}{0.299 0.299 0.299}
\newrgbcolor{PST@COLOR90}{0.291 0.291 0.291}
\newrgbcolor{PST@COLOR91}{0.283 0.283 0.283}
\newrgbcolor{PST@COLOR92}{0.275 0.275 0.275}
\newrgbcolor{PST@COLOR93}{0.267 0.267 0.267}
\newrgbcolor{PST@COLOR94}{0.259 0.259 0.259}
\newrgbcolor{PST@COLOR95}{0.251 0.251 0.251}
\newrgbcolor{PST@COLOR96}{0.244 0.244 0.244}
\newrgbcolor{PST@COLOR97}{0.236 0.236 0.236}
\newrgbcolor{PST@COLOR98}{0.228 0.228 0.228}
\newrgbcolor{PST@COLOR99}{0.22 0.22 0.22}
\newrgbcolor{PST@COLOR100}{0.212 0.212 0.212}
\newrgbcolor{PST@COLOR101}{0.204 0.204 0.204}
\newrgbcolor{PST@COLOR102}{0.196 0.196 0.196}
\newrgbcolor{PST@COLOR103}{0.188 0.188 0.188}
\newrgbcolor{PST@COLOR104}{0.181 0.181 0.181}
\newrgbcolor{PST@COLOR105}{0.173 0.173 0.173}
\newrgbcolor{PST@COLOR106}{0.165 0.165 0.165}
\newrgbcolor{PST@COLOR107}{0.157 0.157 0.157}
\newrgbcolor{PST@COLOR108}{0.149 0.149 0.149}
\newrgbcolor{PST@COLOR109}{0.141 0.141 0.141}
\newrgbcolor{PST@COLOR110}{0.133 0.133 0.133}
\newrgbcolor{PST@COLOR111}{0.125 0.125 0.125}
\newrgbcolor{PST@COLOR112}{0.118 0.118 0.118}
\newrgbcolor{PST@COLOR113}{0.11 0.11 0.11}
\newrgbcolor{PST@COLOR114}{0.102 0.102 0.102}
\newrgbcolor{PST@COLOR115}{0.094 0.094 0.094}
\newrgbcolor{PST@COLOR116}{0.086 0.086 0.086}
\newrgbcolor{PST@COLOR117}{0.078 0.078 0.078}
\newrgbcolor{PST@COLOR118}{0.07 0.07 0.07}
\newrgbcolor{PST@COLOR119}{0.062 0.062 0.062}
\newrgbcolor{PST@COLOR120}{0.055 0.055 0.055}
\newrgbcolor{PST@COLOR121}{0.047 0.047 0.047}
\newrgbcolor{PST@COLOR122}{0.039 0.039 0.039}
\newrgbcolor{PST@COLOR123}{0.031 0.031 0.031}
\newrgbcolor{PST@COLOR124}{0.023 0.023 0.023}
\newrgbcolor{PST@COLOR125}{0.015 0.015 0.015}
\newrgbcolor{PST@COLOR126}{0.007 0.007 0.007}
\newrgbcolor{PST@COLOR127}{0 0 0}


\def\polypmIIId#1{\pspolygon[linestyle=none,fillstyle=solid,fillcolor=PST@COLOR#1]}

\polypmIIId{116}(0.1432,0.19)(0.0864,0.19)(0.0864,0.1078)(0.1432,0.1078)
\polypmIIId{121}(0.1432,0.272)(0.0864,0.272)(0.0864,0.19)(0.1432,0.19)
\polypmIIId{125}(0.1432,0.3542)(0.0864,0.3542)(0.0864,0.272)(0.1432,0.272)

\polypmIIId{116}(0.2,0.19)(0.1432,0.19)(0.1432,0.1078)(0.2,0.1078)
\polypmIIId{121}(0.2,0.272)(0.1432,0.272)(0.1432,0.19)(0.2,0.19)
\polypmIIId{125}(0.2,0.3542)(0.1432,0.3542)(0.1432,0.272)(0.2,0.272)

\polypmIIId{117}(0.2568,0.19)(0.2,0.19)(0.2,0.1078)(0.2568,0.1078)
\polypmIIId{122}(0.2568,0.272)(0.2,0.272)(0.2,0.19)(0.2568,0.19)
\polypmIIId{125}(0.2568,0.3542)(0.2,0.3542)(0.2,0.272)(0.2568,0.272)

\polypmIIId{116}(0.3136,0.19)(0.2568,0.19)(0.2568,0.1078)(0.3136,0.1078)
\polypmIIId{122}(0.3136,0.272)(0.2568,0.272)(0.2568,0.19)(0.3136,0.19)
\polypmIIId{126}(0.3136,0.3542)(0.2568,0.3542)(0.2568,0.272)(0.3136,0.272)

\rput(0.1148,0.07){3}
\rput(0.1716,0.07){4}
\rput(0.2284,0.07){5}
\rput(0.2852,0.07){6}
\rput(0.2000,0.0070){years}

\rput[r](0.0806,0.1489){25}
\rput[r](0.0806,0.2310){50}
\rput[r](0.0806,0.3131){100}
\rput{L}(0.0096,0.2310){actions}

\PST@Border(0.0864,0.3542)
(0.0864,0.1078)
(0.3136,0.1078)
(0.3136,0.3542)
(0.0864,0.3542)

\catcode`@=12
\fi
\endpspicture}
    \subfloat[2 resources]{% GNUPLOT: LaTeX picture using PSTRICKS macros
% Define new PST objects, if not already defined
\ifx\PSTloaded\undefined
\def\PSTloaded{t}

\catcode`@=11

\newpsobject{PST@Border}{psline}{linewidth=.0015,linestyle=solid}

\catcode`@=12

\fi
\psset{unit=5.0in,xunit=5.0in,yunit=3.0in}
\pspicture(0.000000,0.000000)(0.225000,0.35)
\ifx\nofigs\undefined
\catcode`@=11

\newrgbcolor{PST@COLOR0}{1 1 1}
\newrgbcolor{PST@COLOR1}{0.992 0.992 0.992}
\newrgbcolor{PST@COLOR2}{0.984 0.984 0.984}
\newrgbcolor{PST@COLOR3}{0.976 0.976 0.976}
\newrgbcolor{PST@COLOR4}{0.968 0.968 0.968}
\newrgbcolor{PST@COLOR5}{0.96 0.96 0.96}
\newrgbcolor{PST@COLOR6}{0.952 0.952 0.952}
\newrgbcolor{PST@COLOR7}{0.944 0.944 0.944}
\newrgbcolor{PST@COLOR8}{0.937 0.937 0.937}
\newrgbcolor{PST@COLOR9}{0.929 0.929 0.929}
\newrgbcolor{PST@COLOR10}{0.921 0.921 0.921}
\newrgbcolor{PST@COLOR11}{0.913 0.913 0.913}
\newrgbcolor{PST@COLOR12}{0.905 0.905 0.905}
\newrgbcolor{PST@COLOR13}{0.897 0.897 0.897}
\newrgbcolor{PST@COLOR14}{0.889 0.889 0.889}
\newrgbcolor{PST@COLOR15}{0.881 0.881 0.881}
\newrgbcolor{PST@COLOR16}{0.874 0.874 0.874}
\newrgbcolor{PST@COLOR17}{0.866 0.866 0.866}
\newrgbcolor{PST@COLOR18}{0.858 0.858 0.858}
\newrgbcolor{PST@COLOR19}{0.85 0.85 0.85}
\newrgbcolor{PST@COLOR20}{0.842 0.842 0.842}
\newrgbcolor{PST@COLOR21}{0.834 0.834 0.834}
\newrgbcolor{PST@COLOR22}{0.826 0.826 0.826}
\newrgbcolor{PST@COLOR23}{0.818 0.818 0.818}
\newrgbcolor{PST@COLOR24}{0.811 0.811 0.811}
\newrgbcolor{PST@COLOR25}{0.803 0.803 0.803}
\newrgbcolor{PST@COLOR26}{0.795 0.795 0.795}
\newrgbcolor{PST@COLOR27}{0.787 0.787 0.787}
\newrgbcolor{PST@COLOR28}{0.779 0.779 0.779}
\newrgbcolor{PST@COLOR29}{0.771 0.771 0.771}
\newrgbcolor{PST@COLOR30}{0.763 0.763 0.763}
\newrgbcolor{PST@COLOR31}{0.755 0.755 0.755}
\newrgbcolor{PST@COLOR32}{0.748 0.748 0.748}
\newrgbcolor{PST@COLOR33}{0.74 0.74 0.74}
\newrgbcolor{PST@COLOR34}{0.732 0.732 0.732}
\newrgbcolor{PST@COLOR35}{0.724 0.724 0.724}
\newrgbcolor{PST@COLOR36}{0.716 0.716 0.716}
\newrgbcolor{PST@COLOR37}{0.708 0.708 0.708}
\newrgbcolor{PST@COLOR38}{0.7 0.7 0.7}
\newrgbcolor{PST@COLOR39}{0.692 0.692 0.692}
\newrgbcolor{PST@COLOR40}{0.685 0.685 0.685}
\newrgbcolor{PST@COLOR41}{0.677 0.677 0.677}
\newrgbcolor{PST@COLOR42}{0.669 0.669 0.669}
\newrgbcolor{PST@COLOR43}{0.661 0.661 0.661}
\newrgbcolor{PST@COLOR44}{0.653 0.653 0.653}
\newrgbcolor{PST@COLOR45}{0.645 0.645 0.645}
\newrgbcolor{PST@COLOR46}{0.637 0.637 0.637}
\newrgbcolor{PST@COLOR47}{0.629 0.629 0.629}
\newrgbcolor{PST@COLOR48}{0.622 0.622 0.622}
\newrgbcolor{PST@COLOR49}{0.614 0.614 0.614}
\newrgbcolor{PST@COLOR50}{0.606 0.606 0.606}
\newrgbcolor{PST@COLOR51}{0.598 0.598 0.598}
\newrgbcolor{PST@COLOR52}{0.59 0.59 0.59}
\newrgbcolor{PST@COLOR53}{0.582 0.582 0.582}
\newrgbcolor{PST@COLOR54}{0.574 0.574 0.574}
\newrgbcolor{PST@COLOR55}{0.566 0.566 0.566}
\newrgbcolor{PST@COLOR56}{0.559 0.559 0.559}
\newrgbcolor{PST@COLOR57}{0.551 0.551 0.551}
\newrgbcolor{PST@COLOR58}{0.543 0.543 0.543}
\newrgbcolor{PST@COLOR59}{0.535 0.535 0.535}
\newrgbcolor{PST@COLOR60}{0.527 0.527 0.527}
\newrgbcolor{PST@COLOR61}{0.519 0.519 0.519}
\newrgbcolor{PST@COLOR62}{0.511 0.511 0.511}
\newrgbcolor{PST@COLOR63}{0.503 0.503 0.503}
\newrgbcolor{PST@COLOR64}{0.496 0.496 0.496}
\newrgbcolor{PST@COLOR65}{0.488 0.488 0.488}
\newrgbcolor{PST@COLOR66}{0.48 0.48 0.48}
\newrgbcolor{PST@COLOR67}{0.472 0.472 0.472}
\newrgbcolor{PST@COLOR68}{0.464 0.464 0.464}
\newrgbcolor{PST@COLOR69}{0.456 0.456 0.456}
\newrgbcolor{PST@COLOR70}{0.448 0.448 0.448}
\newrgbcolor{PST@COLOR71}{0.44 0.44 0.44}
\newrgbcolor{PST@COLOR72}{0.433 0.433 0.433}
\newrgbcolor{PST@COLOR73}{0.425 0.425 0.425}
\newrgbcolor{PST@COLOR74}{0.417 0.417 0.417}
\newrgbcolor{PST@COLOR75}{0.409 0.409 0.409}
\newrgbcolor{PST@COLOR76}{0.401 0.401 0.401}
\newrgbcolor{PST@COLOR77}{0.393 0.393 0.393}
\newrgbcolor{PST@COLOR78}{0.385 0.385 0.385}
\newrgbcolor{PST@COLOR79}{0.377 0.377 0.377}
\newrgbcolor{PST@COLOR80}{0.37 0.37 0.37}
\newrgbcolor{PST@COLOR81}{0.362 0.362 0.362}
\newrgbcolor{PST@COLOR82}{0.354 0.354 0.354}
\newrgbcolor{PST@COLOR83}{0.346 0.346 0.346}
\newrgbcolor{PST@COLOR84}{0.338 0.338 0.338}
\newrgbcolor{PST@COLOR85}{0.33 0.33 0.33}
\newrgbcolor{PST@COLOR86}{0.322 0.322 0.322}
\newrgbcolor{PST@COLOR87}{0.314 0.314 0.314}
\newrgbcolor{PST@COLOR88}{0.307 0.307 0.307}
\newrgbcolor{PST@COLOR89}{0.299 0.299 0.299}
\newrgbcolor{PST@COLOR90}{0.291 0.291 0.291}
\newrgbcolor{PST@COLOR91}{0.283 0.283 0.283}
\newrgbcolor{PST@COLOR92}{0.275 0.275 0.275}
\newrgbcolor{PST@COLOR93}{0.267 0.267 0.267}
\newrgbcolor{PST@COLOR94}{0.259 0.259 0.259}
\newrgbcolor{PST@COLOR95}{0.251 0.251 0.251}
\newrgbcolor{PST@COLOR96}{0.244 0.244 0.244}
\newrgbcolor{PST@COLOR97}{0.236 0.236 0.236}
\newrgbcolor{PST@COLOR98}{0.228 0.228 0.228}
\newrgbcolor{PST@COLOR99}{0.22 0.22 0.22}
\newrgbcolor{PST@COLOR100}{0.212 0.212 0.212}
\newrgbcolor{PST@COLOR101}{0.204 0.204 0.204}
\newrgbcolor{PST@COLOR102}{0.196 0.196 0.196}
\newrgbcolor{PST@COLOR103}{0.188 0.188 0.188}
\newrgbcolor{PST@COLOR104}{0.181 0.181 0.181}
\newrgbcolor{PST@COLOR105}{0.173 0.173 0.173}
\newrgbcolor{PST@COLOR106}{0.165 0.165 0.165}
\newrgbcolor{PST@COLOR107}{0.157 0.157 0.157}
\newrgbcolor{PST@COLOR108}{0.149 0.149 0.149}
\newrgbcolor{PST@COLOR109}{0.141 0.141 0.141}
\newrgbcolor{PST@COLOR110}{0.133 0.133 0.133}
\newrgbcolor{PST@COLOR111}{0.125 0.125 0.125}
\newrgbcolor{PST@COLOR112}{0.118 0.118 0.118}
\newrgbcolor{PST@COLOR113}{0.11 0.11 0.11}
\newrgbcolor{PST@COLOR114}{0.102 0.102 0.102}
\newrgbcolor{PST@COLOR115}{0.094 0.094 0.094}
\newrgbcolor{PST@COLOR116}{0.086 0.086 0.086}
\newrgbcolor{PST@COLOR117}{0.078 0.078 0.078}
\newrgbcolor{PST@COLOR118}{0.07 0.07 0.07}
\newrgbcolor{PST@COLOR119}{0.062 0.062 0.062}
\newrgbcolor{PST@COLOR120}{0.055 0.055 0.055}
\newrgbcolor{PST@COLOR121}{0.047 0.047 0.047}
\newrgbcolor{PST@COLOR122}{0.039 0.039 0.039}
\newrgbcolor{PST@COLOR123}{0.031 0.031 0.031}
\newrgbcolor{PST@COLOR124}{0.023 0.023 0.023}
\newrgbcolor{PST@COLOR125}{0.015 0.015 0.015}
\newrgbcolor{PST@COLOR126}{0.007 0.007 0.007}
\newrgbcolor{PST@COLOR127}{0 0 0}

\def\polypmIIId#1{\pspolygon[linestyle=none,fillstyle=solid,fillcolor=PST@COLOR#1]}

\polypmIIId{111} (0.0568,0.19)  (0.0,0.19)  (0.0,0.1078)(0.0568,0.1078)
\polypmIIId{119}  (0.0568,0.272) (0.0,0.272) (0.0,0.19)  (0.0568,0.19)
\polypmIIId{124}  (0.0568,0.3542)(0.0,0.3542)(0.0,0.272) (0.0568,0.272)

\polypmIIId{111} (0.1136,   0.19)  (0.0568,0.19)  (0.0568,0.1078)(0.1136,0.1078)
\polypmIIId{120}  (0.1136,   0.272) (0.0568,0.272) (0.0568,0.19)  (0.1136,0.19)
\polypmIIId{124}  (0.1136,   0.3542)(0.0568,0.3542)(0.0568,0.272) (0.1136,0.272)

\polypmIIId{112}(0.1704,0.19)  (0.1136,   0.19)  (0.1136,   0.1078)(0.1704,0.1078)
\polypmIIId{120} (0.1704,0.272) (0.1136,   0.272) (0.1136,   0.19)  (0.1704,0.19)
\polypmIIId{125}  (0.1704,0.3542)(0.1136,   0.3542)(0.1136,   0.272) (0.1704,0.272)

\polypmIIId{112}(0.2272,0.19)  (0.1704,0.19)  (0.1704,0.1078)(0.2272,0.1078)
\polypmIIId{120}  (0.2272,0.272) (0.1704,0.272) (0.1704,0.19)  (0.2272,0.19)
\polypmIIId{125}  (0.2272,0.3542)(0.1704,0.3542)(0.1704,0.272) (0.2272,0.272)

\rput(0.0284,0.07){3}
\rput(0.0852,0.07){4}
\rput(0.1420,0.07){5}
\rput(0.1988,0.07){6}
\rput(0.1136,0.0070){years}


\PST@Border(0.0,0.3542)
(0.0,0.1078)
(0.2272,0.1078)
(0.2272,0.3542)
(0.0,0.3542)

\catcode`@=12
\fi
\endpspicture}
    \subfloat[4 resources]{% GNUPLOT: LaTeX picture using PSTRICKS macros
% Define new PST objects, if not already defined
\ifx\PSTloaded\undefined
\def\PSTloaded{t}

\catcode`@=11

\newpsobject{PST@Border}{psline}{linewidth=.0015,linestyle=solid}

\catcode`@=12

\fi
\psset{unit=5.0in,xunit=5.0in,yunit=3.0in}
\pspicture(0.000000,0.000000)(0.3136,0.35)
\ifx\nofigs\undefined
\catcode`@=11

\newrgbcolor{PST@COLOR0}{1 1 1}
\newrgbcolor{PST@COLOR1}{0.992 0.992 0.992}
\newrgbcolor{PST@COLOR2}{0.984 0.984 0.984}
\newrgbcolor{PST@COLOR3}{0.976 0.976 0.976}
\newrgbcolor{PST@COLOR4}{0.968 0.968 0.968}
\newrgbcolor{PST@COLOR5}{0.96 0.96 0.96}
\newrgbcolor{PST@COLOR6}{0.952 0.952 0.952}
\newrgbcolor{PST@COLOR7}{0.944 0.944 0.944}
\newrgbcolor{PST@COLOR8}{0.937 0.937 0.937}
\newrgbcolor{PST@COLOR9}{0.929 0.929 0.929}
\newrgbcolor{PST@COLOR10}{0.921 0.921 0.921}
\newrgbcolor{PST@COLOR11}{0.913 0.913 0.913}
\newrgbcolor{PST@COLOR12}{0.905 0.905 0.905}
\newrgbcolor{PST@COLOR13}{0.897 0.897 0.897}
\newrgbcolor{PST@COLOR14}{0.889 0.889 0.889}
\newrgbcolor{PST@COLOR15}{0.881 0.881 0.881}
\newrgbcolor{PST@COLOR16}{0.874 0.874 0.874}
\newrgbcolor{PST@COLOR17}{0.866 0.866 0.866}
\newrgbcolor{PST@COLOR18}{0.858 0.858 0.858}
\newrgbcolor{PST@COLOR19}{0.85 0.85 0.85}
\newrgbcolor{PST@COLOR20}{0.842 0.842 0.842}
\newrgbcolor{PST@COLOR21}{0.834 0.834 0.834}
\newrgbcolor{PST@COLOR22}{0.826 0.826 0.826}
\newrgbcolor{PST@COLOR23}{0.818 0.818 0.818}
\newrgbcolor{PST@COLOR24}{0.811 0.811 0.811}
\newrgbcolor{PST@COLOR25}{0.803 0.803 0.803}
\newrgbcolor{PST@COLOR26}{0.795 0.795 0.795}
\newrgbcolor{PST@COLOR27}{0.787 0.787 0.787}
\newrgbcolor{PST@COLOR28}{0.779 0.779 0.779}
\newrgbcolor{PST@COLOR29}{0.771 0.771 0.771}
\newrgbcolor{PST@COLOR30}{0.763 0.763 0.763}
\newrgbcolor{PST@COLOR31}{0.755 0.755 0.755}
\newrgbcolor{PST@COLOR32}{0.748 0.748 0.748}
\newrgbcolor{PST@COLOR33}{0.74 0.74 0.74}
\newrgbcolor{PST@COLOR34}{0.732 0.732 0.732}
\newrgbcolor{PST@COLOR35}{0.724 0.724 0.724}
\newrgbcolor{PST@COLOR36}{0.716 0.716 0.716}
\newrgbcolor{PST@COLOR37}{0.708 0.708 0.708}
\newrgbcolor{PST@COLOR38}{0.7 0.7 0.7}
\newrgbcolor{PST@COLOR39}{0.692 0.692 0.692}
\newrgbcolor{PST@COLOR40}{0.685 0.685 0.685}
\newrgbcolor{PST@COLOR41}{0.677 0.677 0.677}
\newrgbcolor{PST@COLOR42}{0.669 0.669 0.669}
\newrgbcolor{PST@COLOR43}{0.661 0.661 0.661}
\newrgbcolor{PST@COLOR44}{0.653 0.653 0.653}
\newrgbcolor{PST@COLOR45}{0.645 0.645 0.645}
\newrgbcolor{PST@COLOR46}{0.637 0.637 0.637}
\newrgbcolor{PST@COLOR47}{0.629 0.629 0.629}
\newrgbcolor{PST@COLOR48}{0.622 0.622 0.622}
\newrgbcolor{PST@COLOR49}{0.614 0.614 0.614}
\newrgbcolor{PST@COLOR50}{0.606 0.606 0.606}
\newrgbcolor{PST@COLOR51}{0.598 0.598 0.598}
\newrgbcolor{PST@COLOR52}{0.59 0.59 0.59}
\newrgbcolor{PST@COLOR53}{0.582 0.582 0.582}
\newrgbcolor{PST@COLOR54}{0.574 0.574 0.574}
\newrgbcolor{PST@COLOR55}{0.566 0.566 0.566}
\newrgbcolor{PST@COLOR56}{0.559 0.559 0.559}
\newrgbcolor{PST@COLOR57}{0.551 0.551 0.551}
\newrgbcolor{PST@COLOR58}{0.543 0.543 0.543}
\newrgbcolor{PST@COLOR59}{0.535 0.535 0.535}
\newrgbcolor{PST@COLOR60}{0.527 0.527 0.527}
\newrgbcolor{PST@COLOR61}{0.519 0.519 0.519}
\newrgbcolor{PST@COLOR62}{0.511 0.511 0.511}
\newrgbcolor{PST@COLOR63}{0.503 0.503 0.503}
\newrgbcolor{PST@COLOR64}{0.496 0.496 0.496}
\newrgbcolor{PST@COLOR65}{0.488 0.488 0.488}
\newrgbcolor{PST@COLOR66}{0.48 0.48 0.48}
\newrgbcolor{PST@COLOR67}{0.472 0.472 0.472}
\newrgbcolor{PST@COLOR68}{0.464 0.464 0.464}
\newrgbcolor{PST@COLOR69}{0.456 0.456 0.456}
\newrgbcolor{PST@COLOR70}{0.448 0.448 0.448}
\newrgbcolor{PST@COLOR71}{0.44 0.44 0.44}
\newrgbcolor{PST@COLOR72}{0.433 0.433 0.433}
\newrgbcolor{PST@COLOR73}{0.425 0.425 0.425}
\newrgbcolor{PST@COLOR74}{0.417 0.417 0.417}
\newrgbcolor{PST@COLOR75}{0.409 0.409 0.409}
\newrgbcolor{PST@COLOR76}{0.401 0.401 0.401}
\newrgbcolor{PST@COLOR77}{0.393 0.393 0.393}
\newrgbcolor{PST@COLOR78}{0.385 0.385 0.385}
\newrgbcolor{PST@COLOR79}{0.377 0.377 0.377}
\newrgbcolor{PST@COLOR80}{0.37 0.37 0.37}
\newrgbcolor{PST@COLOR81}{0.362 0.362 0.362}
\newrgbcolor{PST@COLOR82}{0.354 0.354 0.354}
\newrgbcolor{PST@COLOR83}{0.346 0.346 0.346}
\newrgbcolor{PST@COLOR84}{0.338 0.338 0.338}
\newrgbcolor{PST@COLOR85}{0.33 0.33 0.33}
\newrgbcolor{PST@COLOR86}{0.322 0.322 0.322}
\newrgbcolor{PST@COLOR87}{0.314 0.314 0.314}
\newrgbcolor{PST@COLOR88}{0.307 0.307 0.307}
\newrgbcolor{PST@COLOR89}{0.299 0.299 0.299}
\newrgbcolor{PST@COLOR90}{0.291 0.291 0.291}
\newrgbcolor{PST@COLOR91}{0.283 0.283 0.283}
\newrgbcolor{PST@COLOR92}{0.275 0.275 0.275}
\newrgbcolor{PST@COLOR93}{0.267 0.267 0.267}
\newrgbcolor{PST@COLOR94}{0.259 0.259 0.259}
\newrgbcolor{PST@COLOR95}{0.251 0.251 0.251}
\newrgbcolor{PST@COLOR96}{0.244 0.244 0.244}
\newrgbcolor{PST@COLOR97}{0.236 0.236 0.236}
\newrgbcolor{PST@COLOR98}{0.228 0.228 0.228}
\newrgbcolor{PST@COLOR99}{0.22 0.22 0.22}
\newrgbcolor{PST@COLOR100}{0.212 0.212 0.212}
\newrgbcolor{PST@COLOR101}{0.204 0.204 0.204}
\newrgbcolor{PST@COLOR102}{0.196 0.196 0.196}
\newrgbcolor{PST@COLOR103}{0.188 0.188 0.188}
\newrgbcolor{PST@COLOR104}{0.181 0.181 0.181}
\newrgbcolor{PST@COLOR105}{0.173 0.173 0.173}
\newrgbcolor{PST@COLOR106}{0.165 0.165 0.165}
\newrgbcolor{PST@COLOR107}{0.157 0.157 0.157}
\newrgbcolor{PST@COLOR108}{0.149 0.149 0.149}
\newrgbcolor{PST@COLOR109}{0.141 0.141 0.141}
\newrgbcolor{PST@COLOR110}{0.133 0.133 0.133}
\newrgbcolor{PST@COLOR111}{0.125 0.125 0.125}
\newrgbcolor{PST@COLOR112}{0.118 0.118 0.118}
\newrgbcolor{PST@COLOR113}{0.11 0.11 0.11}
\newrgbcolor{PST@COLOR114}{0.102 0.102 0.102}
\newrgbcolor{PST@COLOR115}{0.094 0.094 0.094}
\newrgbcolor{PST@COLOR116}{0.086 0.086 0.086}
\newrgbcolor{PST@COLOR117}{0.078 0.078 0.078}
\newrgbcolor{PST@COLOR118}{0.07 0.07 0.07}
\newrgbcolor{PST@COLOR119}{0.062 0.062 0.062}
\newrgbcolor{PST@COLOR120}{0.055 0.055 0.055}
\newrgbcolor{PST@COLOR121}{0.047 0.047 0.047}
\newrgbcolor{PST@COLOR122}{0.039 0.039 0.039}
\newrgbcolor{PST@COLOR123}{0.031 0.031 0.031}
\newrgbcolor{PST@COLOR124}{0.023 0.023 0.023}
\newrgbcolor{PST@COLOR125}{0.015 0.015 0.015}
\newrgbcolor{PST@COLOR126}{0.007 0.007 0.007}
\newrgbcolor{PST@COLOR127}{0 0 0}

\def\polypmIIId#1{\pspolygon[linestyle=none,fillstyle=solid,fillcolor=PST@COLOR#1]}

\polypmIIId{101} (0.0568,0.19)  (0.0,0.19)  (0.0,0.1078)(0.0568,0.1078)
\polypmIIId{113}  (0.0568,0.272) (0.0,0.272) (0.0,0.19)  (0.0568,0.19)
\polypmIIId{121}  (0.0568,0.3542)(0.0,0.3542)(0.0,0.272) (0.0568,0.272)

\polypmIIId{101} (0.1136,   0.19)  (0.0568,0.19)  (0.0568,0.1078)(0.1136,0.1078)
\polypmIIId{113}  (0.1136,   0.272) (0.0568,0.272) (0.0568,0.19)  (0.1136,0.19)
\polypmIIId{121}  (0.1136,   0.3542)(0.0568,0.3542)(0.0568,0.272) (0.1136,0.272)

\polypmIIId{102}(0.1704,0.19)  (0.1136,   0.19)  (0.1136,   0.1078)(0.1704,0.1078)
\polypmIIId{114} (0.1704,0.272) (0.1136,   0.272) (0.1136,   0.19)  (0.1704,0.19)
\polypmIIId{121}  (0.1704,0.3542)(0.1136,   0.3542)(0.1136,   0.272) (0.1704,0.272)

\polypmIIId{102}(0.2272,0.19)  (0.1704,0.19)  (0.1704,0.1078)(0.2272,0.1078)
\polypmIIId{114}  (0.2272,0.272) (0.1704,0.272) (0.1704,0.19)  (0.2272,0.19)
\polypmIIId{122}  (0.2272,0.3542)(0.1704,0.3542)(0.1704,0.272) (0.2272,0.272)

\rput(0.0284,0.07){3}
\rput(0.0852,0.07){4}
\rput(0.1420,0.07){5}
\rput(0.1988,0.07){6}
\rput(0.1136,0.0070){years}

\PST@Border(0.0,0.3542)
(0.0,0.1078)
(0.2272,0.1078)
(0.2272,0.3542)
(0.0,0.3542)

\polypmIIId{0}(0.2329,0.1078)(0.2442,0.1078)(0.2442,0.1098)(0.2329,0.1098)
\polypmIIId{1}(0.2329,0.1097)(0.2442,0.1097)(0.2442,0.1117)(0.2329,0.1117)
\polypmIIId{2}(0.2329,0.1116)(0.2442,0.1116)(0.2442,0.1136)(0.2329,0.1136)
\polypmIIId{3}(0.2329,0.1135)(0.2442,0.1135)(0.2442,0.1156)(0.2329,0.1156)
\polypmIIId{4}(0.2329,0.1155)(0.2442,0.1155)(0.2442,0.1175)(0.2329,0.1175)
\polypmIIId{5}(0.2329,0.1174)(0.2442,0.1174)(0.2442,0.1194)(0.2329,0.1194)
\polypmIIId{6}(0.2329,0.1193)(0.2442,0.1193)(0.2442,0.1213)(0.2329,0.1213)
\polypmIIId{7}(0.2329,0.1212)(0.2442,0.1212)(0.2442,0.1233)(0.2329,0.1233)
\polypmIIId{8}(0.2329,0.1232)(0.2442,0.1232)(0.2442,0.1252)(0.2329,0.1252)
\polypmIIId{9}(0.2329,0.1251)(0.2442,0.1251)(0.2442,0.1271)(0.2329,0.1271)
\polypmIIId{10}(0.2329,0.127)(0.2442,0.127)(0.2442,0.129)(0.2329,0.129)
\polypmIIId{11}(0.2329,0.1289)(0.2442,0.1289)(0.2442,0.131)(0.2329,0.131)
\polypmIIId{12}(0.2329,0.1309)(0.2442,0.1309)(0.2442,0.1329)(0.2329,0.1329)
\polypmIIId{13}(0.2329,0.1328)(0.2442,0.1328)(0.2442,0.1348)(0.2329,0.1348)
\polypmIIId{14}(0.2329,0.1347)(0.2442,0.1347)(0.2442,0.1367)(0.2329,0.1367)
\polypmIIId{15}(0.2329,0.1366)(0.2442,0.1366)(0.2442,0.1387)(0.2329,0.1387)
\polypmIIId{16}(0.2329,0.1386)(0.2442,0.1386)(0.2442,0.1406)(0.2329,0.1406)
\polypmIIId{17}(0.2329,0.1405)(0.2442,0.1405)(0.2442,0.1425)(0.2329,0.1425)
\polypmIIId{18}(0.2329,0.1424)(0.2442,0.1424)(0.2442,0.1444)(0.2329,0.1444)
\polypmIIId{19}(0.2329,0.1443)(0.2442,0.1443)(0.2442,0.1464)(0.2329,0.1464)
\polypmIIId{20}(0.2329,0.1463)(0.2442,0.1463)(0.2442,0.1483)(0.2329,0.1483)
\polypmIIId{21}(0.2329,0.1482)(0.2442,0.1482)(0.2442,0.1502)(0.2329,0.1502)
\polypmIIId{22}(0.2329,0.1501)(0.2442,0.1501)(0.2442,0.1521)(0.2329,0.1521)
\polypmIIId{23}(0.2329,0.152)(0.2442,0.152)(0.2442,0.1541)(0.2329,0.1541)
\polypmIIId{24}(0.2329,0.154)(0.2442,0.154)(0.2442,0.156)(0.2329,0.156)
\polypmIIId{25}(0.2329,0.1559)(0.2442,0.1559)(0.2442,0.1579)(0.2329,0.1579)
\polypmIIId{26}(0.2329,0.1578)(0.2442,0.1578)(0.2442,0.1598)(0.2329,0.1598)
\polypmIIId{27}(0.2329,0.1597)(0.2442,0.1597)(0.2442,0.1618)(0.2329,0.1618)
\polypmIIId{28}(0.2329,0.1617)(0.2442,0.1617)(0.2442,0.1637)(0.2329,0.1637)
\polypmIIId{29}(0.2329,0.1636)(0.2442,0.1636)(0.2442,0.1656)(0.2329,0.1656)
\polypmIIId{30}(0.2329,0.1655)(0.2442,0.1655)(0.2442,0.1675)(0.2329,0.1675)
\polypmIIId{31}(0.2329,0.1674)(0.2442,0.1674)(0.2442,0.1695)(0.2329,0.1695)
\polypmIIId{32}(0.2329,0.1694)(0.2442,0.1694)(0.2442,0.1714)(0.2329,0.1714)
\polypmIIId{33}(0.2329,0.1713)(0.2442,0.1713)(0.2442,0.1733)(0.2329,0.1733)
\polypmIIId{34}(0.2329,0.1732)(0.2442,0.1732)(0.2442,0.1752)(0.2329,0.1752)
\polypmIIId{35}(0.2329,0.1751)(0.2442,0.1751)(0.2442,0.1772)(0.2329,0.1772)
\polypmIIId{36}(0.2329,0.1771)(0.2442,0.1771)(0.2442,0.1791)(0.2329,0.1791)
\polypmIIId{37}(0.2329,0.179)(0.2442,0.179)(0.2442,0.181)(0.2329,0.181)
\polypmIIId{38}(0.2329,0.1809)(0.2442,0.1809)(0.2442,0.1829)(0.2329,0.1829)
\polypmIIId{39}(0.2329,0.1828)(0.2442,0.1828)(0.2442,0.1849)(0.2329,0.1849)
\polypmIIId{40}(0.2329,0.1848)(0.2442,0.1848)(0.2442,0.1868)(0.2329,0.1868)
\polypmIIId{41}(0.2329,0.1867)(0.2442,0.1867)(0.2442,0.1887)(0.2329,0.1887)
\polypmIIId{42}(0.2329,0.1886)(0.2442,0.1886)(0.2442,0.1906)(0.2329,0.1906)
\polypmIIId{43}(0.2329,0.1905)(0.2442,0.1905)(0.2442,0.1926)(0.2329,0.1926)
\polypmIIId{44}(0.2329,0.1925)(0.2442,0.1925)(0.2442,0.1945)(0.2329,0.1945)
\polypmIIId{45}(0.2329,0.1944)(0.2442,0.1944)(0.2442,0.1964)(0.2329,0.1964)
\polypmIIId{46}(0.2329,0.1963)(0.2442,0.1963)(0.2442,0.1983)(0.2329,0.1983)
\polypmIIId{47}(0.2329,0.1982)(0.2442,0.1982)(0.2442,0.2003)(0.2329,0.2003)
\polypmIIId{48}(0.2329,0.2002)(0.2442,0.2002)(0.2442,0.2022)(0.2329,0.2022)
\polypmIIId{49}(0.2329,0.2021)(0.2442,0.2021)(0.2442,0.2041)(0.2329,0.2041)
\polypmIIId{50}(0.2329,0.204)(0.2442,0.204)(0.2442,0.206)(0.2329,0.206)
\polypmIIId{51}(0.2329,0.2059)(0.2442,0.2059)(0.2442,0.208)(0.2329,0.208)
\polypmIIId{52}(0.2329,0.2079)(0.2442,0.2079)(0.2442,0.2099)(0.2329,0.2099)
\polypmIIId{53}(0.2329,0.2098)(0.2442,0.2098)(0.2442,0.2118)(0.2329,0.2118)
\polypmIIId{54}(0.2329,0.2117)(0.2442,0.2117)(0.2442,0.2137)(0.2329,0.2137)
\polypmIIId{55}(0.2329,0.2136)(0.2442,0.2136)(0.2442,0.2157)(0.2329,0.2157)
\polypmIIId{56}(0.2329,0.2156)(0.2442,0.2156)(0.2442,0.2176)(0.2329,0.2176)
\polypmIIId{57}(0.2329,0.2175)(0.2442,0.2175)(0.2442,0.2195)(0.2329,0.2195)
\polypmIIId{58}(0.2329,0.2194)(0.2442,0.2194)(0.2442,0.2214)(0.2329,0.2214)
\polypmIIId{59}(0.2329,0.2213)(0.2442,0.2213)(0.2442,0.2234)(0.2329,0.2234)
\polypmIIId{60}(0.2329,0.2233)(0.2442,0.2233)(0.2442,0.2253)(0.2329,0.2253)
\polypmIIId{61}(0.2329,0.2252)(0.2442,0.2252)(0.2442,0.2272)(0.2329,0.2272)
\polypmIIId{62}(0.2329,0.2271)(0.2442,0.2271)(0.2442,0.2291)(0.2329,0.2291)
\polypmIIId{63}(0.2329,0.229)(0.2442,0.229)(0.2442,0.2311)(0.2329,0.2311)
\polypmIIId{64}(0.2329,0.231)(0.2442,0.231)(0.2442,0.233)(0.2329,0.233)
\polypmIIId{65}(0.2329,0.2329)(0.2442,0.2329)(0.2442,0.2349)(0.2329,0.2349)
\polypmIIId{66}(0.2329,0.2348)(0.2442,0.2348)(0.2442,0.2368)(0.2329,0.2368)
\polypmIIId{67}(0.2329,0.2367)(0.2442,0.2367)(0.2442,0.2388)(0.2329,0.2388)
\polypmIIId{68}(0.2329,0.2387)(0.2442,0.2387)(0.2442,0.2407)(0.2329,0.2407)
\polypmIIId{69}(0.2329,0.2406)(0.2442,0.2406)(0.2442,0.2426)(0.2329,0.2426)
\polypmIIId{70}(0.2329,0.2425)(0.2442,0.2425)(0.2442,0.2445)(0.2329,0.2445)
\polypmIIId{71}(0.2329,0.2444)(0.2442,0.2444)(0.2442,0.2465)(0.2329,0.2465)
\polypmIIId{72}(0.2329,0.2464)(0.2442,0.2464)(0.2442,0.2484)(0.2329,0.2484)
\polypmIIId{73}(0.2329,0.2483)(0.2442,0.2483)(0.2442,0.2503)(0.2329,0.2503)
\polypmIIId{74}(0.2329,0.2502)(0.2442,0.2502)(0.2442,0.2522)(0.2329,0.2522)
\polypmIIId{75}(0.2329,0.2521)(0.2442,0.2521)(0.2442,0.2542)(0.2329,0.2542)
\polypmIIId{76}(0.2329,0.2541)(0.2442,0.2541)(0.2442,0.2561)(0.2329,0.2561)
\polypmIIId{77}(0.2329,0.256)(0.2442,0.256)(0.2442,0.258)(0.2329,0.258)
\polypmIIId{78}(0.2329,0.2579)(0.2442,0.2579)(0.2442,0.2599)(0.2329,0.2599)
\polypmIIId{79}(0.2329,0.2598)(0.2442,0.2598)(0.2442,0.2619)(0.2329,0.2619)
\polypmIIId{80}(0.2329,0.2618)(0.2442,0.2618)(0.2442,0.2638)(0.2329,0.2638)
\polypmIIId{81}(0.2329,0.2637)(0.2442,0.2637)(0.2442,0.2657)(0.2329,0.2657)
\polypmIIId{82}(0.2329,0.2656)(0.2442,0.2656)(0.2442,0.2676)(0.2329,0.2676)
\polypmIIId{83}(0.2329,0.2675)(0.2442,0.2675)(0.2442,0.2696)(0.2329,0.2696)
\polypmIIId{84}(0.2329,0.2695)(0.2442,0.2695)(0.2442,0.2715)(0.2329,0.2715)
\polypmIIId{85}(0.2329,0.2714)(0.2442,0.2714)(0.2442,0.2734)(0.2329,0.2734)
\polypmIIId{86}(0.2329,0.2733)(0.2442,0.2733)(0.2442,0.2753)(0.2329,0.2753)
\polypmIIId{87}(0.2329,0.2752)(0.2442,0.2752)(0.2442,0.2773)(0.2329,0.2773)
\polypmIIId{88}(0.2329,0.2772)(0.2442,0.2772)(0.2442,0.2792)(0.2329,0.2792)
\polypmIIId{89}(0.2329,0.2791)(0.2442,0.2791)(0.2442,0.2811)(0.2329,0.2811)
\polypmIIId{90}(0.2329,0.281)(0.2442,0.281)(0.2442,0.283)(0.2329,0.283)
\polypmIIId{91}(0.2329,0.2829)(0.2442,0.2829)(0.2442,0.285)(0.2329,0.285)
\polypmIIId{92}(0.2329,0.2849)(0.2442,0.2849)(0.2442,0.2869)(0.2329,0.2869)
\polypmIIId{93}(0.2329,0.2868)(0.2442,0.2868)(0.2442,0.2888)(0.2329,0.2888)
\polypmIIId{94}(0.2329,0.2887)(0.2442,0.2887)(0.2442,0.2907)(0.2329,0.2907)
\polypmIIId{95}(0.2329,0.2906)(0.2442,0.2906)(0.2442,0.2927)(0.2329,0.2927)
\polypmIIId{96}(0.2329,0.2926)(0.2442,0.2926)(0.2442,0.2946)(0.2329,0.2946)
\polypmIIId{97}(0.2329,0.2945)(0.2442,0.2945)(0.2442,0.2965)(0.2329,0.2965)
\polypmIIId{98}(0.2329,0.2964)(0.2442,0.2964)(0.2442,0.2984)(0.2329,0.2984)
\polypmIIId{99}(0.2329,0.2983)(0.2442,0.2983)(0.2442,0.3004)(0.2329,0.3004)
\polypmIIId{100}(0.2329,0.3003)(0.2442,0.3003)(0.2442,0.3023)(0.2329,0.3023)
\polypmIIId{101}(0.2329,0.3022)(0.2442,0.3022)(0.2442,0.3042)(0.2329,0.3042)
\polypmIIId{102}(0.2329,0.3041)(0.2442,0.3041)(0.2442,0.3061)(0.2329,0.3061)
\polypmIIId{103}(0.2329,0.306)(0.2442,0.306)(0.2442,0.3081)(0.2329,0.3081)
\polypmIIId{104}(0.2329,0.308)(0.2442,0.308)(0.2442,0.31)(0.2329,0.31)
\polypmIIId{105}(0.2329,0.3099)(0.2442,0.3099)(0.2442,0.3119)(0.2329,0.3119)
\polypmIIId{106}(0.2329,0.3118)(0.2442,0.3118)(0.2442,0.3138)(0.2329,0.3138)
\polypmIIId{107}(0.2329,0.3137)(0.2442,0.3137)(0.2442,0.3158)(0.2329,0.3158)
\polypmIIId{108}(0.2329,0.3157)(0.2442,0.3157)(0.2442,0.3177)(0.2329,0.3177)
\polypmIIId{109}(0.2329,0.3176)(0.2442,0.3176)(0.2442,0.3196)(0.2329,0.3196)
\polypmIIId{110}(0.2329,0.3195)(0.2442,0.3195)(0.2442,0.3215)(0.2329,0.3215)
\polypmIIId{111}(0.2329,0.3214)(0.2442,0.3214)(0.2442,0.3235)(0.2329,0.3235)
\polypmIIId{112}(0.2329,0.3234)(0.2442,0.3234)(0.2442,0.3254)(0.2329,0.3254)
\polypmIIId{113}(0.2329,0.3253)(0.2442,0.3253)(0.2442,0.3273)(0.2329,0.3273)
\polypmIIId{114}(0.2329,0.3272)(0.2442,0.3272)(0.2442,0.3292)(0.2329,0.3292)
\polypmIIId{115}(0.2329,0.3291)(0.2442,0.3291)(0.2442,0.3312)(0.2329,0.3312)
\polypmIIId{116}(0.2329,0.3311)(0.2442,0.3311)(0.2442,0.3331)(0.2329,0.3331)
\polypmIIId{117}(0.2329,0.333)(0.2442,0.333)(0.2442,0.335)(0.2329,0.335)
\polypmIIId{118}(0.2329,0.3349)(0.2442,0.3349)(0.2442,0.3369)(0.2329,0.3369)
\polypmIIId{119}(0.2329,0.3368)(0.2442,0.3368)(0.2442,0.3389)(0.2329,0.3389)
\polypmIIId{120}(0.2329,0.3388)(0.2442,0.3388)(0.2442,0.3408)(0.2329,0.3408)
\polypmIIId{121}(0.2329,0.3407)(0.2442,0.3407)(0.2442,0.3427)(0.2329,0.3427)
\polypmIIId{122}(0.2329,0.3426)(0.2442,0.3426)(0.2442,0.3446)(0.2329,0.3446)
\polypmIIId{123}(0.2329,0.3445)(0.2442,0.3445)(0.2442,0.3466)(0.2329,0.3466)
\polypmIIId{124}(0.2329,0.3465)(0.2442,0.3465)(0.2442,0.3485)(0.2329,0.3485)
\polypmIIId{125}(0.2329,0.3484)(0.2442,0.3484)(0.2442,0.3504)(0.2329,0.3504)
\polypmIIId{126}(0.2329,0.3503)(0.2442,0.3503)(0.2442,0.3523)(0.2329,0.3523)
\polypmIIId{127}(0.2329,0.3522)(0.2442,0.3522)(0.2442,0.3542)(0.2329,0.3542)

\PST@Border(0.2329,0.1078)
(0.2442,0.1078)
(0.2442,0.3542)
(0.2329,0.3542)
(0.2329,0.1078)


\rput[l](0.2502,0.1301){0.997}
\rput[l](0.2502,0.2048){0.998}
\rput[l](0.2502,0.2795){0.999}
\rput[l](0.2502,0.3542){1}

\catcode`@=12
\fi
\endpspicture}
  \caption{TSLP solution quality on strongly correlated instances ($\alpha = 0.0$).}
  \label{fig:tabusolcomp00}
\end{figure}

\begin{figure}[H]
  \centering
    \subfloat[1 resource]{% GNUPLOT: LaTeX picture using PSTRICKS macros
% Define new PST objects, if not already defined
\ifx\PSTloaded\undefined
\def\PSTloaded{t}

\catcode`@=11

\newpsobject{PST@Border}{psline}{linewidth=.0015,linestyle=solid}

\catcode`@=12

\fi
\psset{unit=5.0in,xunit=5.0in,yunit=3.0in}
\pspicture(0.000000,0.000000)(0.31, 0.35)
\ifx\nofigs\undefined
\catcode`@=11

\newrgbcolor{PST@COLOR0}{1 1 1}
\newrgbcolor{PST@COLOR1}{0.992 0.992 0.992}
\newrgbcolor{PST@COLOR2}{0.984 0.984 0.984}
\newrgbcolor{PST@COLOR3}{0.976 0.976 0.976}
\newrgbcolor{PST@COLOR4}{0.968 0.968 0.968}
\newrgbcolor{PST@COLOR5}{0.96 0.96 0.96}
\newrgbcolor{PST@COLOR6}{0.952 0.952 0.952}
\newrgbcolor{PST@COLOR7}{0.944 0.944 0.944}
\newrgbcolor{PST@COLOR8}{0.937 0.937 0.937}
\newrgbcolor{PST@COLOR9}{0.929 0.929 0.929}
\newrgbcolor{PST@COLOR10}{0.921 0.921 0.921}
\newrgbcolor{PST@COLOR11}{0.913 0.913 0.913}
\newrgbcolor{PST@COLOR12}{0.905 0.905 0.905}
\newrgbcolor{PST@COLOR13}{0.897 0.897 0.897}
\newrgbcolor{PST@COLOR14}{0.889 0.889 0.889}
\newrgbcolor{PST@COLOR15}{0.881 0.881 0.881}
\newrgbcolor{PST@COLOR16}{0.874 0.874 0.874}
\newrgbcolor{PST@COLOR17}{0.866 0.866 0.866}
\newrgbcolor{PST@COLOR18}{0.858 0.858 0.858}
\newrgbcolor{PST@COLOR19}{0.85 0.85 0.85}
\newrgbcolor{PST@COLOR20}{0.842 0.842 0.842}
\newrgbcolor{PST@COLOR21}{0.834 0.834 0.834}
\newrgbcolor{PST@COLOR22}{0.826 0.826 0.826}
\newrgbcolor{PST@COLOR23}{0.818 0.818 0.818}
\newrgbcolor{PST@COLOR24}{0.811 0.811 0.811}
\newrgbcolor{PST@COLOR25}{0.803 0.803 0.803}
\newrgbcolor{PST@COLOR26}{0.795 0.795 0.795}
\newrgbcolor{PST@COLOR27}{0.787 0.787 0.787}
\newrgbcolor{PST@COLOR28}{0.779 0.779 0.779}
\newrgbcolor{PST@COLOR29}{0.771 0.771 0.771}
\newrgbcolor{PST@COLOR30}{0.763 0.763 0.763}
\newrgbcolor{PST@COLOR31}{0.755 0.755 0.755}
\newrgbcolor{PST@COLOR32}{0.748 0.748 0.748}
\newrgbcolor{PST@COLOR33}{0.74 0.74 0.74}
\newrgbcolor{PST@COLOR34}{0.732 0.732 0.732}
\newrgbcolor{PST@COLOR35}{0.724 0.724 0.724}
\newrgbcolor{PST@COLOR36}{0.716 0.716 0.716}
\newrgbcolor{PST@COLOR37}{0.708 0.708 0.708}
\newrgbcolor{PST@COLOR38}{0.7 0.7 0.7}
\newrgbcolor{PST@COLOR39}{0.692 0.692 0.692}
\newrgbcolor{PST@COLOR40}{0.685 0.685 0.685}
\newrgbcolor{PST@COLOR41}{0.677 0.677 0.677}
\newrgbcolor{PST@COLOR42}{0.669 0.669 0.669}
\newrgbcolor{PST@COLOR43}{0.661 0.661 0.661}
\newrgbcolor{PST@COLOR44}{0.653 0.653 0.653}
\newrgbcolor{PST@COLOR45}{0.645 0.645 0.645}
\newrgbcolor{PST@COLOR46}{0.637 0.637 0.637}
\newrgbcolor{PST@COLOR47}{0.629 0.629 0.629}
\newrgbcolor{PST@COLOR48}{0.622 0.622 0.622}
\newrgbcolor{PST@COLOR49}{0.614 0.614 0.614}
\newrgbcolor{PST@COLOR50}{0.606 0.606 0.606}
\newrgbcolor{PST@COLOR51}{0.598 0.598 0.598}
\newrgbcolor{PST@COLOR52}{0.59 0.59 0.59}
\newrgbcolor{PST@COLOR53}{0.582 0.582 0.582}
\newrgbcolor{PST@COLOR54}{0.574 0.574 0.574}
\newrgbcolor{PST@COLOR55}{0.566 0.566 0.566}
\newrgbcolor{PST@COLOR56}{0.559 0.559 0.559}
\newrgbcolor{PST@COLOR57}{0.551 0.551 0.551}
\newrgbcolor{PST@COLOR58}{0.543 0.543 0.543}
\newrgbcolor{PST@COLOR59}{0.535 0.535 0.535}
\newrgbcolor{PST@COLOR60}{0.527 0.527 0.527}
\newrgbcolor{PST@COLOR61}{0.519 0.519 0.519}
\newrgbcolor{PST@COLOR62}{0.511 0.511 0.511}
\newrgbcolor{PST@COLOR63}{0.503 0.503 0.503}
\newrgbcolor{PST@COLOR64}{0.496 0.496 0.496}
\newrgbcolor{PST@COLOR65}{0.488 0.488 0.488}
\newrgbcolor{PST@COLOR66}{0.48 0.48 0.48}
\newrgbcolor{PST@COLOR67}{0.472 0.472 0.472}
\newrgbcolor{PST@COLOR68}{0.464 0.464 0.464}
\newrgbcolor{PST@COLOR69}{0.456 0.456 0.456}
\newrgbcolor{PST@COLOR70}{0.448 0.448 0.448}
\newrgbcolor{PST@COLOR71}{0.44 0.44 0.44}
\newrgbcolor{PST@COLOR72}{0.433 0.433 0.433}
\newrgbcolor{PST@COLOR73}{0.425 0.425 0.425}
\newrgbcolor{PST@COLOR74}{0.417 0.417 0.417}
\newrgbcolor{PST@COLOR75}{0.409 0.409 0.409}
\newrgbcolor{PST@COLOR76}{0.401 0.401 0.401}
\newrgbcolor{PST@COLOR77}{0.393 0.393 0.393}
\newrgbcolor{PST@COLOR78}{0.385 0.385 0.385}
\newrgbcolor{PST@COLOR79}{0.377 0.377 0.377}
\newrgbcolor{PST@COLOR80}{0.37 0.37 0.37}
\newrgbcolor{PST@COLOR81}{0.362 0.362 0.362}
\newrgbcolor{PST@COLOR82}{0.354 0.354 0.354}
\newrgbcolor{PST@COLOR83}{0.346 0.346 0.346}
\newrgbcolor{PST@COLOR84}{0.338 0.338 0.338}
\newrgbcolor{PST@COLOR85}{0.33 0.33 0.33}
\newrgbcolor{PST@COLOR86}{0.322 0.322 0.322}
\newrgbcolor{PST@COLOR87}{0.314 0.314 0.314}
\newrgbcolor{PST@COLOR88}{0.307 0.307 0.307}
\newrgbcolor{PST@COLOR89}{0.299 0.299 0.299}
\newrgbcolor{PST@COLOR90}{0.291 0.291 0.291}
\newrgbcolor{PST@COLOR91}{0.283 0.283 0.283}
\newrgbcolor{PST@COLOR92}{0.275 0.275 0.275}
\newrgbcolor{PST@COLOR93}{0.267 0.267 0.267}
\newrgbcolor{PST@COLOR94}{0.259 0.259 0.259}
\newrgbcolor{PST@COLOR95}{0.251 0.251 0.251}
\newrgbcolor{PST@COLOR96}{0.244 0.244 0.244}
\newrgbcolor{PST@COLOR97}{0.236 0.236 0.236}
\newrgbcolor{PST@COLOR98}{0.228 0.228 0.228}
\newrgbcolor{PST@COLOR99}{0.22 0.22 0.22}
\newrgbcolor{PST@COLOR100}{0.212 0.212 0.212}
\newrgbcolor{PST@COLOR101}{0.204 0.204 0.204}
\newrgbcolor{PST@COLOR102}{0.196 0.196 0.196}
\newrgbcolor{PST@COLOR103}{0.188 0.188 0.188}
\newrgbcolor{PST@COLOR104}{0.181 0.181 0.181}
\newrgbcolor{PST@COLOR105}{0.173 0.173 0.173}
\newrgbcolor{PST@COLOR106}{0.165 0.165 0.165}
\newrgbcolor{PST@COLOR107}{0.157 0.157 0.157}
\newrgbcolor{PST@COLOR108}{0.149 0.149 0.149}
\newrgbcolor{PST@COLOR109}{0.141 0.141 0.141}
\newrgbcolor{PST@COLOR110}{0.133 0.133 0.133}
\newrgbcolor{PST@COLOR111}{0.125 0.125 0.125}
\newrgbcolor{PST@COLOR112}{0.118 0.118 0.118}
\newrgbcolor{PST@COLOR113}{0.11 0.11 0.11}
\newrgbcolor{PST@COLOR114}{0.102 0.102 0.102}
\newrgbcolor{PST@COLOR115}{0.094 0.094 0.094}
\newrgbcolor{PST@COLOR116}{0.086 0.086 0.086}
\newrgbcolor{PST@COLOR117}{0.078 0.078 0.078}
\newrgbcolor{PST@COLOR118}{0.07 0.07 0.07}
\newrgbcolor{PST@COLOR119}{0.062 0.062 0.062}
\newrgbcolor{PST@COLOR120}{0.055 0.055 0.055}
\newrgbcolor{PST@COLOR121}{0.047 0.047 0.047}
\newrgbcolor{PST@COLOR122}{0.039 0.039 0.039}
\newrgbcolor{PST@COLOR123}{0.031 0.031 0.031}
\newrgbcolor{PST@COLOR124}{0.023 0.023 0.023}
\newrgbcolor{PST@COLOR125}{0.015 0.015 0.015}
\newrgbcolor{PST@COLOR126}{0.007 0.007 0.007}
\newrgbcolor{PST@COLOR127}{0 0 0}


\def\polypmIIId#1{\pspolygon[linestyle=none,fillstyle=solid,fillcolor=PST@COLOR#1]}

\polypmIIId{110}(0.1432,0.19)(0.0864,0.19)(0.0864,0.1078)(0.1432,0.1078)
\polypmIIId{117}(0.1432,0.272)(0.0864,0.272)(0.0864,0.19)(0.1432,0.19)
\polypmIIId{122}(0.1432,0.3542)(0.0864,0.3542)(0.0864,0.272)(0.1432,0.272)

\polypmIIId{107}(0.2,0.19)(0.1432,0.19)(0.1432,0.1078)(0.2,0.1078)
\polypmIIId{117}(0.2,0.272)(0.1432,0.272)(0.1432,0.19)(0.2,0.19)
\polypmIIId{122}(0.2,0.3542)(0.1432,0.3542)(0.1432,0.272)(0.2,0.272)

\polypmIIId{108}(0.2568,0.19)(0.2,0.19)(0.2,0.1078)(0.2568,0.1078)
\polypmIIId{116}(0.2568,0.272)(0.2,0.272)(0.2,0.19)(0.2568,0.19)
\polypmIIId{123}(0.2568,0.3542)(0.2,0.3542)(0.2,0.272)(0.2568,0.272)

\polypmIIId{107}(0.3136,0.19)(0.2568,0.19)(0.2568,0.1078)(0.3136,0.1078)
\polypmIIId{117}(0.3136,0.272)(0.2568,0.272)(0.2568,0.19)(0.3136,0.19)
\polypmIIId{123}(0.3136,0.3542)(0.2568,0.3542)(0.2568,0.272)(0.3136,0.272)

\rput(0.1148,0.07){3}
\rput(0.1716,0.07){4}
\rput(0.2284,0.07){5}
\rput(0.2852,0.07){6}
\rput(0.2000,0.0070){years}

\rput[r](0.0806,0.1489){25}
\rput[r](0.0806,0.2310){50}
\rput[r](0.0806,0.3131){100}
\rput{L}(0.0096,0.2310){actions}

\PST@Border(0.0864,0.3542)
(0.0864,0.1078)
(0.3136,0.1078)
(0.3136,0.3542)
(0.0864,0.3542)

\catcode`@=12
\fi
\endpspicture}
    \subfloat[2 resources]{% GNUPLOT: LaTeX picture using PSTRICKS macros
% Define new PST objects, if not already defined
\ifx\PSTloaded\undefined
\def\PSTloaded{t}

\catcode`@=11

\newpsobject{PST@Border}{psline}{linewidth=.0015,linestyle=solid}

\catcode`@=12

\fi
\psset{unit=5.0in,xunit=5.0in,yunit=3.0in}
\pspicture(0.000000,0.000000)(0.225000,0.35)
\ifx\nofigs\undefined
\catcode`@=11

\newrgbcolor{PST@COLOR0}{1 1 1}
\newrgbcolor{PST@COLOR1}{0.992 0.992 0.992}
\newrgbcolor{PST@COLOR2}{0.984 0.984 0.984}
\newrgbcolor{PST@COLOR3}{0.976 0.976 0.976}
\newrgbcolor{PST@COLOR4}{0.968 0.968 0.968}
\newrgbcolor{PST@COLOR5}{0.96 0.96 0.96}
\newrgbcolor{PST@COLOR6}{0.952 0.952 0.952}
\newrgbcolor{PST@COLOR7}{0.944 0.944 0.944}
\newrgbcolor{PST@COLOR8}{0.937 0.937 0.937}
\newrgbcolor{PST@COLOR9}{0.929 0.929 0.929}
\newrgbcolor{PST@COLOR10}{0.921 0.921 0.921}
\newrgbcolor{PST@COLOR11}{0.913 0.913 0.913}
\newrgbcolor{PST@COLOR12}{0.905 0.905 0.905}
\newrgbcolor{PST@COLOR13}{0.897 0.897 0.897}
\newrgbcolor{PST@COLOR14}{0.889 0.889 0.889}
\newrgbcolor{PST@COLOR15}{0.881 0.881 0.881}
\newrgbcolor{PST@COLOR16}{0.874 0.874 0.874}
\newrgbcolor{PST@COLOR17}{0.866 0.866 0.866}
\newrgbcolor{PST@COLOR18}{0.858 0.858 0.858}
\newrgbcolor{PST@COLOR19}{0.85 0.85 0.85}
\newrgbcolor{PST@COLOR20}{0.842 0.842 0.842}
\newrgbcolor{PST@COLOR21}{0.834 0.834 0.834}
\newrgbcolor{PST@COLOR22}{0.826 0.826 0.826}
\newrgbcolor{PST@COLOR23}{0.818 0.818 0.818}
\newrgbcolor{PST@COLOR24}{0.811 0.811 0.811}
\newrgbcolor{PST@COLOR25}{0.803 0.803 0.803}
\newrgbcolor{PST@COLOR26}{0.795 0.795 0.795}
\newrgbcolor{PST@COLOR27}{0.787 0.787 0.787}
\newrgbcolor{PST@COLOR28}{0.779 0.779 0.779}
\newrgbcolor{PST@COLOR29}{0.771 0.771 0.771}
\newrgbcolor{PST@COLOR30}{0.763 0.763 0.763}
\newrgbcolor{PST@COLOR31}{0.755 0.755 0.755}
\newrgbcolor{PST@COLOR32}{0.748 0.748 0.748}
\newrgbcolor{PST@COLOR33}{0.74 0.74 0.74}
\newrgbcolor{PST@COLOR34}{0.732 0.732 0.732}
\newrgbcolor{PST@COLOR35}{0.724 0.724 0.724}
\newrgbcolor{PST@COLOR36}{0.716 0.716 0.716}
\newrgbcolor{PST@COLOR37}{0.708 0.708 0.708}
\newrgbcolor{PST@COLOR38}{0.7 0.7 0.7}
\newrgbcolor{PST@COLOR39}{0.692 0.692 0.692}
\newrgbcolor{PST@COLOR40}{0.685 0.685 0.685}
\newrgbcolor{PST@COLOR41}{0.677 0.677 0.677}
\newrgbcolor{PST@COLOR42}{0.669 0.669 0.669}
\newrgbcolor{PST@COLOR43}{0.661 0.661 0.661}
\newrgbcolor{PST@COLOR44}{0.653 0.653 0.653}
\newrgbcolor{PST@COLOR45}{0.645 0.645 0.645}
\newrgbcolor{PST@COLOR46}{0.637 0.637 0.637}
\newrgbcolor{PST@COLOR47}{0.629 0.629 0.629}
\newrgbcolor{PST@COLOR48}{0.622 0.622 0.622}
\newrgbcolor{PST@COLOR49}{0.614 0.614 0.614}
\newrgbcolor{PST@COLOR50}{0.606 0.606 0.606}
\newrgbcolor{PST@COLOR51}{0.598 0.598 0.598}
\newrgbcolor{PST@COLOR52}{0.59 0.59 0.59}
\newrgbcolor{PST@COLOR53}{0.582 0.582 0.582}
\newrgbcolor{PST@COLOR54}{0.574 0.574 0.574}
\newrgbcolor{PST@COLOR55}{0.566 0.566 0.566}
\newrgbcolor{PST@COLOR56}{0.559 0.559 0.559}
\newrgbcolor{PST@COLOR57}{0.551 0.551 0.551}
\newrgbcolor{PST@COLOR58}{0.543 0.543 0.543}
\newrgbcolor{PST@COLOR59}{0.535 0.535 0.535}
\newrgbcolor{PST@COLOR60}{0.527 0.527 0.527}
\newrgbcolor{PST@COLOR61}{0.519 0.519 0.519}
\newrgbcolor{PST@COLOR62}{0.511 0.511 0.511}
\newrgbcolor{PST@COLOR63}{0.503 0.503 0.503}
\newrgbcolor{PST@COLOR64}{0.496 0.496 0.496}
\newrgbcolor{PST@COLOR65}{0.488 0.488 0.488}
\newrgbcolor{PST@COLOR66}{0.48 0.48 0.48}
\newrgbcolor{PST@COLOR67}{0.472 0.472 0.472}
\newrgbcolor{PST@COLOR68}{0.464 0.464 0.464}
\newrgbcolor{PST@COLOR69}{0.456 0.456 0.456}
\newrgbcolor{PST@COLOR70}{0.448 0.448 0.448}
\newrgbcolor{PST@COLOR71}{0.44 0.44 0.44}
\newrgbcolor{PST@COLOR72}{0.433 0.433 0.433}
\newrgbcolor{PST@COLOR73}{0.425 0.425 0.425}
\newrgbcolor{PST@COLOR74}{0.417 0.417 0.417}
\newrgbcolor{PST@COLOR75}{0.409 0.409 0.409}
\newrgbcolor{PST@COLOR76}{0.401 0.401 0.401}
\newrgbcolor{PST@COLOR77}{0.393 0.393 0.393}
\newrgbcolor{PST@COLOR78}{0.385 0.385 0.385}
\newrgbcolor{PST@COLOR79}{0.377 0.377 0.377}
\newrgbcolor{PST@COLOR80}{0.37 0.37 0.37}
\newrgbcolor{PST@COLOR81}{0.362 0.362 0.362}
\newrgbcolor{PST@COLOR82}{0.354 0.354 0.354}
\newrgbcolor{PST@COLOR83}{0.346 0.346 0.346}
\newrgbcolor{PST@COLOR84}{0.338 0.338 0.338}
\newrgbcolor{PST@COLOR85}{0.33 0.33 0.33}
\newrgbcolor{PST@COLOR86}{0.322 0.322 0.322}
\newrgbcolor{PST@COLOR87}{0.314 0.314 0.314}
\newrgbcolor{PST@COLOR88}{0.307 0.307 0.307}
\newrgbcolor{PST@COLOR89}{0.299 0.299 0.299}
\newrgbcolor{PST@COLOR90}{0.291 0.291 0.291}
\newrgbcolor{PST@COLOR91}{0.283 0.283 0.283}
\newrgbcolor{PST@COLOR92}{0.275 0.275 0.275}
\newrgbcolor{PST@COLOR93}{0.267 0.267 0.267}
\newrgbcolor{PST@COLOR94}{0.259 0.259 0.259}
\newrgbcolor{PST@COLOR95}{0.251 0.251 0.251}
\newrgbcolor{PST@COLOR96}{0.244 0.244 0.244}
\newrgbcolor{PST@COLOR97}{0.236 0.236 0.236}
\newrgbcolor{PST@COLOR98}{0.228 0.228 0.228}
\newrgbcolor{PST@COLOR99}{0.22 0.22 0.22}
\newrgbcolor{PST@COLOR100}{0.212 0.212 0.212}
\newrgbcolor{PST@COLOR101}{0.204 0.204 0.204}
\newrgbcolor{PST@COLOR102}{0.196 0.196 0.196}
\newrgbcolor{PST@COLOR103}{0.188 0.188 0.188}
\newrgbcolor{PST@COLOR104}{0.181 0.181 0.181}
\newrgbcolor{PST@COLOR105}{0.173 0.173 0.173}
\newrgbcolor{PST@COLOR106}{0.165 0.165 0.165}
\newrgbcolor{PST@COLOR107}{0.157 0.157 0.157}
\newrgbcolor{PST@COLOR108}{0.149 0.149 0.149}
\newrgbcolor{PST@COLOR109}{0.141 0.141 0.141}
\newrgbcolor{PST@COLOR110}{0.133 0.133 0.133}
\newrgbcolor{PST@COLOR111}{0.125 0.125 0.125}
\newrgbcolor{PST@COLOR112}{0.118 0.118 0.118}
\newrgbcolor{PST@COLOR113}{0.11 0.11 0.11}
\newrgbcolor{PST@COLOR114}{0.102 0.102 0.102}
\newrgbcolor{PST@COLOR115}{0.094 0.094 0.094}
\newrgbcolor{PST@COLOR116}{0.086 0.086 0.086}
\newrgbcolor{PST@COLOR117}{0.078 0.078 0.078}
\newrgbcolor{PST@COLOR118}{0.07 0.07 0.07}
\newrgbcolor{PST@COLOR119}{0.062 0.062 0.062}
\newrgbcolor{PST@COLOR120}{0.055 0.055 0.055}
\newrgbcolor{PST@COLOR121}{0.047 0.047 0.047}
\newrgbcolor{PST@COLOR122}{0.039 0.039 0.039}
\newrgbcolor{PST@COLOR123}{0.031 0.031 0.031}
\newrgbcolor{PST@COLOR124}{0.023 0.023 0.023}
\newrgbcolor{PST@COLOR125}{0.015 0.015 0.015}
\newrgbcolor{PST@COLOR126}{0.007 0.007 0.007}
\newrgbcolor{PST@COLOR127}{0 0 0}

\def\polypmIIId#1{\pspolygon[linestyle=none,fillstyle=solid,fillcolor=PST@COLOR#1]}

\polypmIIId{94} (0.0568,0.19)  (0.0,0.19)  (0.0,0.1078)(0.0568,0.1078)
\polypmIIId{106}  (0.0568,0.272) (0.0,0.272) (0.0,0.19)  (0.0568,0.19)
\polypmIIId{117}  (0.0568,0.3542)(0.0,0.3542)(0.0,0.272) (0.0568,0.272)

\polypmIIId{93} (0.1136,   0.19)  (0.0568,0.19)  (0.0568,0.1078)(0.1136,0.1078)
\polypmIIId{106}  (0.1136,   0.272) (0.0568,0.272) (0.0568,0.19)  (0.1136,0.19)
\polypmIIId{118}  (0.1136,   0.3542)(0.0568,0.3542)(0.0568,0.272) (0.1136,0.272)

\polypmIIId{93}(0.1704,0.19)  (0.1136,   0.19)  (0.1136,   0.1078)(0.1704,0.1078)
\polypmIIId{108} (0.1704,0.272) (0.1136,   0.272) (0.1136,   0.19)  (0.1704,0.19)
\polypmIIId{118}  (0.1704,0.3542)(0.1136,   0.3542)(0.1136,   0.272) (0.1704,0.272)

\polypmIIId{95}(0.2272,0.19)  (0.1704,0.19)  (0.1704,0.1078)(0.2272,0.1078)
\polypmIIId{109}  (0.2272,0.272) (0.1704,0.272) (0.1704,0.19)  (0.2272,0.19)
\polypmIIId{119}  (0.2272,0.3542)(0.1704,0.3542)(0.1704,0.272) (0.2272,0.272)

\rput(0.0284,0.07){3}
\rput(0.0852,0.07){4}
\rput(0.1420,0.07){5}
\rput(0.1988,0.07){6}
\rput(0.1136,0.0070){years}


\PST@Border(0.0,0.3542)
(0.0,0.1078)
(0.2272,0.1078)
(0.2272,0.3542)
(0.0,0.3542)

\catcode`@=12
\fi
\endpspicture}
    \subfloat[4 resources]{% GNUPLOT: LaTeX picture using PSTRICKS macros
% Define new PST objects, if not already defined
\ifx\PSTloaded\undefined
\def\PSTloaded{t}

\catcode`@=11

\newpsobject{PST@Border}{psline}{linewidth=.0015,linestyle=solid}

\catcode`@=12

\fi
\psset{unit=5.0in,xunit=5.0in,yunit=3.0in}
\pspicture(0.000000,0.000000)(0.3136,0.35)
\ifx\nofigs\undefined
\catcode`@=11

\newrgbcolor{PST@COLOR0}{1 1 1}
\newrgbcolor{PST@COLOR1}{0.992 0.992 0.992}
\newrgbcolor{PST@COLOR2}{0.984 0.984 0.984}
\newrgbcolor{PST@COLOR3}{0.976 0.976 0.976}
\newrgbcolor{PST@COLOR4}{0.968 0.968 0.968}
\newrgbcolor{PST@COLOR5}{0.96 0.96 0.96}
\newrgbcolor{PST@COLOR6}{0.952 0.952 0.952}
\newrgbcolor{PST@COLOR7}{0.944 0.944 0.944}
\newrgbcolor{PST@COLOR8}{0.937 0.937 0.937}
\newrgbcolor{PST@COLOR9}{0.929 0.929 0.929}
\newrgbcolor{PST@COLOR10}{0.921 0.921 0.921}
\newrgbcolor{PST@COLOR11}{0.913 0.913 0.913}
\newrgbcolor{PST@COLOR12}{0.905 0.905 0.905}
\newrgbcolor{PST@COLOR13}{0.897 0.897 0.897}
\newrgbcolor{PST@COLOR14}{0.889 0.889 0.889}
\newrgbcolor{PST@COLOR15}{0.881 0.881 0.881}
\newrgbcolor{PST@COLOR16}{0.874 0.874 0.874}
\newrgbcolor{PST@COLOR17}{0.866 0.866 0.866}
\newrgbcolor{PST@COLOR18}{0.858 0.858 0.858}
\newrgbcolor{PST@COLOR19}{0.85 0.85 0.85}
\newrgbcolor{PST@COLOR20}{0.842 0.842 0.842}
\newrgbcolor{PST@COLOR21}{0.834 0.834 0.834}
\newrgbcolor{PST@COLOR22}{0.826 0.826 0.826}
\newrgbcolor{PST@COLOR23}{0.818 0.818 0.818}
\newrgbcolor{PST@COLOR24}{0.811 0.811 0.811}
\newrgbcolor{PST@COLOR25}{0.803 0.803 0.803}
\newrgbcolor{PST@COLOR26}{0.795 0.795 0.795}
\newrgbcolor{PST@COLOR27}{0.787 0.787 0.787}
\newrgbcolor{PST@COLOR28}{0.779 0.779 0.779}
\newrgbcolor{PST@COLOR29}{0.771 0.771 0.771}
\newrgbcolor{PST@COLOR30}{0.763 0.763 0.763}
\newrgbcolor{PST@COLOR31}{0.755 0.755 0.755}
\newrgbcolor{PST@COLOR32}{0.748 0.748 0.748}
\newrgbcolor{PST@COLOR33}{0.74 0.74 0.74}
\newrgbcolor{PST@COLOR34}{0.732 0.732 0.732}
\newrgbcolor{PST@COLOR35}{0.724 0.724 0.724}
\newrgbcolor{PST@COLOR36}{0.716 0.716 0.716}
\newrgbcolor{PST@COLOR37}{0.708 0.708 0.708}
\newrgbcolor{PST@COLOR38}{0.7 0.7 0.7}
\newrgbcolor{PST@COLOR39}{0.692 0.692 0.692}
\newrgbcolor{PST@COLOR40}{0.685 0.685 0.685}
\newrgbcolor{PST@COLOR41}{0.677 0.677 0.677}
\newrgbcolor{PST@COLOR42}{0.669 0.669 0.669}
\newrgbcolor{PST@COLOR43}{0.661 0.661 0.661}
\newrgbcolor{PST@COLOR44}{0.653 0.653 0.653}
\newrgbcolor{PST@COLOR45}{0.645 0.645 0.645}
\newrgbcolor{PST@COLOR46}{0.637 0.637 0.637}
\newrgbcolor{PST@COLOR47}{0.629 0.629 0.629}
\newrgbcolor{PST@COLOR48}{0.622 0.622 0.622}
\newrgbcolor{PST@COLOR49}{0.614 0.614 0.614}
\newrgbcolor{PST@COLOR50}{0.606 0.606 0.606}
\newrgbcolor{PST@COLOR51}{0.598 0.598 0.598}
\newrgbcolor{PST@COLOR52}{0.59 0.59 0.59}
\newrgbcolor{PST@COLOR53}{0.582 0.582 0.582}
\newrgbcolor{PST@COLOR54}{0.574 0.574 0.574}
\newrgbcolor{PST@COLOR55}{0.566 0.566 0.566}
\newrgbcolor{PST@COLOR56}{0.559 0.559 0.559}
\newrgbcolor{PST@COLOR57}{0.551 0.551 0.551}
\newrgbcolor{PST@COLOR58}{0.543 0.543 0.543}
\newrgbcolor{PST@COLOR59}{0.535 0.535 0.535}
\newrgbcolor{PST@COLOR60}{0.527 0.527 0.527}
\newrgbcolor{PST@COLOR61}{0.519 0.519 0.519}
\newrgbcolor{PST@COLOR62}{0.511 0.511 0.511}
\newrgbcolor{PST@COLOR63}{0.503 0.503 0.503}
\newrgbcolor{PST@COLOR64}{0.496 0.496 0.496}
\newrgbcolor{PST@COLOR65}{0.488 0.488 0.488}
\newrgbcolor{PST@COLOR66}{0.48 0.48 0.48}
\newrgbcolor{PST@COLOR67}{0.472 0.472 0.472}
\newrgbcolor{PST@COLOR68}{0.464 0.464 0.464}
\newrgbcolor{PST@COLOR69}{0.456 0.456 0.456}
\newrgbcolor{PST@COLOR70}{0.448 0.448 0.448}
\newrgbcolor{PST@COLOR71}{0.44 0.44 0.44}
\newrgbcolor{PST@COLOR72}{0.433 0.433 0.433}
\newrgbcolor{PST@COLOR73}{0.425 0.425 0.425}
\newrgbcolor{PST@COLOR74}{0.417 0.417 0.417}
\newrgbcolor{PST@COLOR75}{0.409 0.409 0.409}
\newrgbcolor{PST@COLOR76}{0.401 0.401 0.401}
\newrgbcolor{PST@COLOR77}{0.393 0.393 0.393}
\newrgbcolor{PST@COLOR78}{0.385 0.385 0.385}
\newrgbcolor{PST@COLOR79}{0.377 0.377 0.377}
\newrgbcolor{PST@COLOR80}{0.37 0.37 0.37}
\newrgbcolor{PST@COLOR81}{0.362 0.362 0.362}
\newrgbcolor{PST@COLOR82}{0.354 0.354 0.354}
\newrgbcolor{PST@COLOR83}{0.346 0.346 0.346}
\newrgbcolor{PST@COLOR84}{0.338 0.338 0.338}
\newrgbcolor{PST@COLOR85}{0.33 0.33 0.33}
\newrgbcolor{PST@COLOR86}{0.322 0.322 0.322}
\newrgbcolor{PST@COLOR87}{0.314 0.314 0.314}
\newrgbcolor{PST@COLOR88}{0.307 0.307 0.307}
\newrgbcolor{PST@COLOR89}{0.299 0.299 0.299}
\newrgbcolor{PST@COLOR90}{0.291 0.291 0.291}
\newrgbcolor{PST@COLOR91}{0.283 0.283 0.283}
\newrgbcolor{PST@COLOR92}{0.275 0.275 0.275}
\newrgbcolor{PST@COLOR93}{0.267 0.267 0.267}
\newrgbcolor{PST@COLOR94}{0.259 0.259 0.259}
\newrgbcolor{PST@COLOR95}{0.251 0.251 0.251}
\newrgbcolor{PST@COLOR96}{0.244 0.244 0.244}
\newrgbcolor{PST@COLOR97}{0.236 0.236 0.236}
\newrgbcolor{PST@COLOR98}{0.228 0.228 0.228}
\newrgbcolor{PST@COLOR99}{0.22 0.22 0.22}
\newrgbcolor{PST@COLOR100}{0.212 0.212 0.212}
\newrgbcolor{PST@COLOR101}{0.204 0.204 0.204}
\newrgbcolor{PST@COLOR102}{0.196 0.196 0.196}
\newrgbcolor{PST@COLOR103}{0.188 0.188 0.188}
\newrgbcolor{PST@COLOR104}{0.181 0.181 0.181}
\newrgbcolor{PST@COLOR105}{0.173 0.173 0.173}
\newrgbcolor{PST@COLOR106}{0.165 0.165 0.165}
\newrgbcolor{PST@COLOR107}{0.157 0.157 0.157}
\newrgbcolor{PST@COLOR108}{0.149 0.149 0.149}
\newrgbcolor{PST@COLOR109}{0.141 0.141 0.141}
\newrgbcolor{PST@COLOR110}{0.133 0.133 0.133}
\newrgbcolor{PST@COLOR111}{0.125 0.125 0.125}
\newrgbcolor{PST@COLOR112}{0.118 0.118 0.118}
\newrgbcolor{PST@COLOR113}{0.11 0.11 0.11}
\newrgbcolor{PST@COLOR114}{0.102 0.102 0.102}
\newrgbcolor{PST@COLOR115}{0.094 0.094 0.094}
\newrgbcolor{PST@COLOR116}{0.086 0.086 0.086}
\newrgbcolor{PST@COLOR117}{0.078 0.078 0.078}
\newrgbcolor{PST@COLOR118}{0.07 0.07 0.07}
\newrgbcolor{PST@COLOR119}{0.062 0.062 0.062}
\newrgbcolor{PST@COLOR120}{0.055 0.055 0.055}
\newrgbcolor{PST@COLOR121}{0.047 0.047 0.047}
\newrgbcolor{PST@COLOR122}{0.039 0.039 0.039}
\newrgbcolor{PST@COLOR123}{0.031 0.031 0.031}
\newrgbcolor{PST@COLOR124}{0.023 0.023 0.023}
\newrgbcolor{PST@COLOR125}{0.015 0.015 0.015}
\newrgbcolor{PST@COLOR126}{0.007 0.007 0.007}
\newrgbcolor{PST@COLOR127}{0 0 0}

\def\polypmIIId#1{\pspolygon[linestyle=none,fillstyle=solid,fillcolor=PST@COLOR#1]}

\polypmIIId{74} (0.0568,0.19)  (0.0,0.19)  (0.0,0.1078)(0.0568,0.1078)
\polypmIIId{93}  (0.0568,0.272) (0.0,0.272) (0.0,0.19)  (0.0568,0.19)
\polypmIIId{106}  (0.0568,0.3542)(0.0,0.3542)(0.0,0.272) (0.0568,0.272)

\polypmIIId{75} (0.1136,   0.19)  (0.0568,0.19)  (0.0568,0.1078)(0.1136,0.1078)
\polypmIIId{92}  (0.1136,   0.272) (0.0568,0.272) (0.0568,0.19)  (0.1136,0.19)
\polypmIIId{108}  (0.1136,   0.3542)(0.0568,0.3542)(0.0568,0.272) (0.1136,0.272)

\polypmIIId{71}(0.1704,0.19)  (0.1136,   0.19)  (0.1136,   0.1078)(0.1704,0.1078)
\polypmIIId{95} (0.1704,0.272) (0.1136,   0.272) (0.1136,   0.19)  (0.1704,0.19)
\polypmIIId{108}  (0.1704,0.3542)(0.1136,   0.3542)(0.1136,   0.272) (0.1704,0.272)

\polypmIIId{71}(0.2272,0.19)  (0.1704,0.19)  (0.1704,0.1078)(0.2272,0.1078)
\polypmIIId{97}  (0.2272,0.272) (0.1704,0.272) (0.1704,0.19)  (0.2272,0.19)
\polypmIIId{109}  (0.2272,0.3542)(0.1704,0.3542)(0.1704,0.272) (0.2272,0.272)

\rput(0.0284,0.07){3}
\rput(0.0852,0.07){4}
\rput(0.1420,0.07){5}
\rput(0.1988,0.07){6}
\rput(0.1136,0.0070){years}

\PST@Border(0.0,0.3542)
(0.0,0.1078)
(0.2272,0.1078)
(0.2272,0.3542)
(0.0,0.3542)

\polypmIIId{0}(0.2329,0.1078)(0.2442,0.1078)(0.2442,0.1098)(0.2329,0.1098)
\polypmIIId{1}(0.2329,0.1097)(0.2442,0.1097)(0.2442,0.1117)(0.2329,0.1117)
\polypmIIId{2}(0.2329,0.1116)(0.2442,0.1116)(0.2442,0.1136)(0.2329,0.1136)
\polypmIIId{3}(0.2329,0.1135)(0.2442,0.1135)(0.2442,0.1156)(0.2329,0.1156)
\polypmIIId{4}(0.2329,0.1155)(0.2442,0.1155)(0.2442,0.1175)(0.2329,0.1175)
\polypmIIId{5}(0.2329,0.1174)(0.2442,0.1174)(0.2442,0.1194)(0.2329,0.1194)
\polypmIIId{6}(0.2329,0.1193)(0.2442,0.1193)(0.2442,0.1213)(0.2329,0.1213)
\polypmIIId{7}(0.2329,0.1212)(0.2442,0.1212)(0.2442,0.1233)(0.2329,0.1233)
\polypmIIId{8}(0.2329,0.1232)(0.2442,0.1232)(0.2442,0.1252)(0.2329,0.1252)
\polypmIIId{9}(0.2329,0.1251)(0.2442,0.1251)(0.2442,0.1271)(0.2329,0.1271)
\polypmIIId{10}(0.2329,0.127)(0.2442,0.127)(0.2442,0.129)(0.2329,0.129)
\polypmIIId{11}(0.2329,0.1289)(0.2442,0.1289)(0.2442,0.131)(0.2329,0.131)
\polypmIIId{12}(0.2329,0.1309)(0.2442,0.1309)(0.2442,0.1329)(0.2329,0.1329)
\polypmIIId{13}(0.2329,0.1328)(0.2442,0.1328)(0.2442,0.1348)(0.2329,0.1348)
\polypmIIId{14}(0.2329,0.1347)(0.2442,0.1347)(0.2442,0.1367)(0.2329,0.1367)
\polypmIIId{15}(0.2329,0.1366)(0.2442,0.1366)(0.2442,0.1387)(0.2329,0.1387)
\polypmIIId{16}(0.2329,0.1386)(0.2442,0.1386)(0.2442,0.1406)(0.2329,0.1406)
\polypmIIId{17}(0.2329,0.1405)(0.2442,0.1405)(0.2442,0.1425)(0.2329,0.1425)
\polypmIIId{18}(0.2329,0.1424)(0.2442,0.1424)(0.2442,0.1444)(0.2329,0.1444)
\polypmIIId{19}(0.2329,0.1443)(0.2442,0.1443)(0.2442,0.1464)(0.2329,0.1464)
\polypmIIId{20}(0.2329,0.1463)(0.2442,0.1463)(0.2442,0.1483)(0.2329,0.1483)
\polypmIIId{21}(0.2329,0.1482)(0.2442,0.1482)(0.2442,0.1502)(0.2329,0.1502)
\polypmIIId{22}(0.2329,0.1501)(0.2442,0.1501)(0.2442,0.1521)(0.2329,0.1521)
\polypmIIId{23}(0.2329,0.152)(0.2442,0.152)(0.2442,0.1541)(0.2329,0.1541)
\polypmIIId{24}(0.2329,0.154)(0.2442,0.154)(0.2442,0.156)(0.2329,0.156)
\polypmIIId{25}(0.2329,0.1559)(0.2442,0.1559)(0.2442,0.1579)(0.2329,0.1579)
\polypmIIId{26}(0.2329,0.1578)(0.2442,0.1578)(0.2442,0.1598)(0.2329,0.1598)
\polypmIIId{27}(0.2329,0.1597)(0.2442,0.1597)(0.2442,0.1618)(0.2329,0.1618)
\polypmIIId{28}(0.2329,0.1617)(0.2442,0.1617)(0.2442,0.1637)(0.2329,0.1637)
\polypmIIId{29}(0.2329,0.1636)(0.2442,0.1636)(0.2442,0.1656)(0.2329,0.1656)
\polypmIIId{30}(0.2329,0.1655)(0.2442,0.1655)(0.2442,0.1675)(0.2329,0.1675)
\polypmIIId{31}(0.2329,0.1674)(0.2442,0.1674)(0.2442,0.1695)(0.2329,0.1695)
\polypmIIId{32}(0.2329,0.1694)(0.2442,0.1694)(0.2442,0.1714)(0.2329,0.1714)
\polypmIIId{33}(0.2329,0.1713)(0.2442,0.1713)(0.2442,0.1733)(0.2329,0.1733)
\polypmIIId{34}(0.2329,0.1732)(0.2442,0.1732)(0.2442,0.1752)(0.2329,0.1752)
\polypmIIId{35}(0.2329,0.1751)(0.2442,0.1751)(0.2442,0.1772)(0.2329,0.1772)
\polypmIIId{36}(0.2329,0.1771)(0.2442,0.1771)(0.2442,0.1791)(0.2329,0.1791)
\polypmIIId{37}(0.2329,0.179)(0.2442,0.179)(0.2442,0.181)(0.2329,0.181)
\polypmIIId{38}(0.2329,0.1809)(0.2442,0.1809)(0.2442,0.1829)(0.2329,0.1829)
\polypmIIId{39}(0.2329,0.1828)(0.2442,0.1828)(0.2442,0.1849)(0.2329,0.1849)
\polypmIIId{40}(0.2329,0.1848)(0.2442,0.1848)(0.2442,0.1868)(0.2329,0.1868)
\polypmIIId{41}(0.2329,0.1867)(0.2442,0.1867)(0.2442,0.1887)(0.2329,0.1887)
\polypmIIId{42}(0.2329,0.1886)(0.2442,0.1886)(0.2442,0.1906)(0.2329,0.1906)
\polypmIIId{43}(0.2329,0.1905)(0.2442,0.1905)(0.2442,0.1926)(0.2329,0.1926)
\polypmIIId{44}(0.2329,0.1925)(0.2442,0.1925)(0.2442,0.1945)(0.2329,0.1945)
\polypmIIId{45}(0.2329,0.1944)(0.2442,0.1944)(0.2442,0.1964)(0.2329,0.1964)
\polypmIIId{46}(0.2329,0.1963)(0.2442,0.1963)(0.2442,0.1983)(0.2329,0.1983)
\polypmIIId{47}(0.2329,0.1982)(0.2442,0.1982)(0.2442,0.2003)(0.2329,0.2003)
\polypmIIId{48}(0.2329,0.2002)(0.2442,0.2002)(0.2442,0.2022)(0.2329,0.2022)
\polypmIIId{49}(0.2329,0.2021)(0.2442,0.2021)(0.2442,0.2041)(0.2329,0.2041)
\polypmIIId{50}(0.2329,0.204)(0.2442,0.204)(0.2442,0.206)(0.2329,0.206)
\polypmIIId{51}(0.2329,0.2059)(0.2442,0.2059)(0.2442,0.208)(0.2329,0.208)
\polypmIIId{52}(0.2329,0.2079)(0.2442,0.2079)(0.2442,0.2099)(0.2329,0.2099)
\polypmIIId{53}(0.2329,0.2098)(0.2442,0.2098)(0.2442,0.2118)(0.2329,0.2118)
\polypmIIId{54}(0.2329,0.2117)(0.2442,0.2117)(0.2442,0.2137)(0.2329,0.2137)
\polypmIIId{55}(0.2329,0.2136)(0.2442,0.2136)(0.2442,0.2157)(0.2329,0.2157)
\polypmIIId{56}(0.2329,0.2156)(0.2442,0.2156)(0.2442,0.2176)(0.2329,0.2176)
\polypmIIId{57}(0.2329,0.2175)(0.2442,0.2175)(0.2442,0.2195)(0.2329,0.2195)
\polypmIIId{58}(0.2329,0.2194)(0.2442,0.2194)(0.2442,0.2214)(0.2329,0.2214)
\polypmIIId{59}(0.2329,0.2213)(0.2442,0.2213)(0.2442,0.2234)(0.2329,0.2234)
\polypmIIId{60}(0.2329,0.2233)(0.2442,0.2233)(0.2442,0.2253)(0.2329,0.2253)
\polypmIIId{61}(0.2329,0.2252)(0.2442,0.2252)(0.2442,0.2272)(0.2329,0.2272)
\polypmIIId{62}(0.2329,0.2271)(0.2442,0.2271)(0.2442,0.2291)(0.2329,0.2291)
\polypmIIId{63}(0.2329,0.229)(0.2442,0.229)(0.2442,0.2311)(0.2329,0.2311)
\polypmIIId{64}(0.2329,0.231)(0.2442,0.231)(0.2442,0.233)(0.2329,0.233)
\polypmIIId{65}(0.2329,0.2329)(0.2442,0.2329)(0.2442,0.2349)(0.2329,0.2349)
\polypmIIId{66}(0.2329,0.2348)(0.2442,0.2348)(0.2442,0.2368)(0.2329,0.2368)
\polypmIIId{67}(0.2329,0.2367)(0.2442,0.2367)(0.2442,0.2388)(0.2329,0.2388)
\polypmIIId{68}(0.2329,0.2387)(0.2442,0.2387)(0.2442,0.2407)(0.2329,0.2407)
\polypmIIId{69}(0.2329,0.2406)(0.2442,0.2406)(0.2442,0.2426)(0.2329,0.2426)
\polypmIIId{70}(0.2329,0.2425)(0.2442,0.2425)(0.2442,0.2445)(0.2329,0.2445)
\polypmIIId{71}(0.2329,0.2444)(0.2442,0.2444)(0.2442,0.2465)(0.2329,0.2465)
\polypmIIId{72}(0.2329,0.2464)(0.2442,0.2464)(0.2442,0.2484)(0.2329,0.2484)
\polypmIIId{73}(0.2329,0.2483)(0.2442,0.2483)(0.2442,0.2503)(0.2329,0.2503)
\polypmIIId{74}(0.2329,0.2502)(0.2442,0.2502)(0.2442,0.2522)(0.2329,0.2522)
\polypmIIId{75}(0.2329,0.2521)(0.2442,0.2521)(0.2442,0.2542)(0.2329,0.2542)
\polypmIIId{76}(0.2329,0.2541)(0.2442,0.2541)(0.2442,0.2561)(0.2329,0.2561)
\polypmIIId{77}(0.2329,0.256)(0.2442,0.256)(0.2442,0.258)(0.2329,0.258)
\polypmIIId{78}(0.2329,0.2579)(0.2442,0.2579)(0.2442,0.2599)(0.2329,0.2599)
\polypmIIId{79}(0.2329,0.2598)(0.2442,0.2598)(0.2442,0.2619)(0.2329,0.2619)
\polypmIIId{80}(0.2329,0.2618)(0.2442,0.2618)(0.2442,0.2638)(0.2329,0.2638)
\polypmIIId{81}(0.2329,0.2637)(0.2442,0.2637)(0.2442,0.2657)(0.2329,0.2657)
\polypmIIId{82}(0.2329,0.2656)(0.2442,0.2656)(0.2442,0.2676)(0.2329,0.2676)
\polypmIIId{83}(0.2329,0.2675)(0.2442,0.2675)(0.2442,0.2696)(0.2329,0.2696)
\polypmIIId{84}(0.2329,0.2695)(0.2442,0.2695)(0.2442,0.2715)(0.2329,0.2715)
\polypmIIId{85}(0.2329,0.2714)(0.2442,0.2714)(0.2442,0.2734)(0.2329,0.2734)
\polypmIIId{86}(0.2329,0.2733)(0.2442,0.2733)(0.2442,0.2753)(0.2329,0.2753)
\polypmIIId{87}(0.2329,0.2752)(0.2442,0.2752)(0.2442,0.2773)(0.2329,0.2773)
\polypmIIId{88}(0.2329,0.2772)(0.2442,0.2772)(0.2442,0.2792)(0.2329,0.2792)
\polypmIIId{89}(0.2329,0.2791)(0.2442,0.2791)(0.2442,0.2811)(0.2329,0.2811)
\polypmIIId{90}(0.2329,0.281)(0.2442,0.281)(0.2442,0.283)(0.2329,0.283)
\polypmIIId{91}(0.2329,0.2829)(0.2442,0.2829)(0.2442,0.285)(0.2329,0.285)
\polypmIIId{92}(0.2329,0.2849)(0.2442,0.2849)(0.2442,0.2869)(0.2329,0.2869)
\polypmIIId{93}(0.2329,0.2868)(0.2442,0.2868)(0.2442,0.2888)(0.2329,0.2888)
\polypmIIId{94}(0.2329,0.2887)(0.2442,0.2887)(0.2442,0.2907)(0.2329,0.2907)
\polypmIIId{95}(0.2329,0.2906)(0.2442,0.2906)(0.2442,0.2927)(0.2329,0.2927)
\polypmIIId{96}(0.2329,0.2926)(0.2442,0.2926)(0.2442,0.2946)(0.2329,0.2946)
\polypmIIId{97}(0.2329,0.2945)(0.2442,0.2945)(0.2442,0.2965)(0.2329,0.2965)
\polypmIIId{98}(0.2329,0.2964)(0.2442,0.2964)(0.2442,0.2984)(0.2329,0.2984)
\polypmIIId{99}(0.2329,0.2983)(0.2442,0.2983)(0.2442,0.3004)(0.2329,0.3004)
\polypmIIId{100}(0.2329,0.3003)(0.2442,0.3003)(0.2442,0.3023)(0.2329,0.3023)
\polypmIIId{101}(0.2329,0.3022)(0.2442,0.3022)(0.2442,0.3042)(0.2329,0.3042)
\polypmIIId{102}(0.2329,0.3041)(0.2442,0.3041)(0.2442,0.3061)(0.2329,0.3061)
\polypmIIId{103}(0.2329,0.306)(0.2442,0.306)(0.2442,0.3081)(0.2329,0.3081)
\polypmIIId{104}(0.2329,0.308)(0.2442,0.308)(0.2442,0.31)(0.2329,0.31)
\polypmIIId{105}(0.2329,0.3099)(0.2442,0.3099)(0.2442,0.3119)(0.2329,0.3119)
\polypmIIId{106}(0.2329,0.3118)(0.2442,0.3118)(0.2442,0.3138)(0.2329,0.3138)
\polypmIIId{107}(0.2329,0.3137)(0.2442,0.3137)(0.2442,0.3158)(0.2329,0.3158)
\polypmIIId{108}(0.2329,0.3157)(0.2442,0.3157)(0.2442,0.3177)(0.2329,0.3177)
\polypmIIId{109}(0.2329,0.3176)(0.2442,0.3176)(0.2442,0.3196)(0.2329,0.3196)
\polypmIIId{110}(0.2329,0.3195)(0.2442,0.3195)(0.2442,0.3215)(0.2329,0.3215)
\polypmIIId{111}(0.2329,0.3214)(0.2442,0.3214)(0.2442,0.3235)(0.2329,0.3235)
\polypmIIId{112}(0.2329,0.3234)(0.2442,0.3234)(0.2442,0.3254)(0.2329,0.3254)
\polypmIIId{113}(0.2329,0.3253)(0.2442,0.3253)(0.2442,0.3273)(0.2329,0.3273)
\polypmIIId{114}(0.2329,0.3272)(0.2442,0.3272)(0.2442,0.3292)(0.2329,0.3292)
\polypmIIId{115}(0.2329,0.3291)(0.2442,0.3291)(0.2442,0.3312)(0.2329,0.3312)
\polypmIIId{116}(0.2329,0.3311)(0.2442,0.3311)(0.2442,0.3331)(0.2329,0.3331)
\polypmIIId{117}(0.2329,0.333)(0.2442,0.333)(0.2442,0.335)(0.2329,0.335)
\polypmIIId{118}(0.2329,0.3349)(0.2442,0.3349)(0.2442,0.3369)(0.2329,0.3369)
\polypmIIId{119}(0.2329,0.3368)(0.2442,0.3368)(0.2442,0.3389)(0.2329,0.3389)
\polypmIIId{120}(0.2329,0.3388)(0.2442,0.3388)(0.2442,0.3408)(0.2329,0.3408)
\polypmIIId{121}(0.2329,0.3407)(0.2442,0.3407)(0.2442,0.3427)(0.2329,0.3427)
\polypmIIId{122}(0.2329,0.3426)(0.2442,0.3426)(0.2442,0.3446)(0.2329,0.3446)
\polypmIIId{123}(0.2329,0.3445)(0.2442,0.3445)(0.2442,0.3466)(0.2329,0.3466)
\polypmIIId{124}(0.2329,0.3465)(0.2442,0.3465)(0.2442,0.3485)(0.2329,0.3485)
\polypmIIId{125}(0.2329,0.3484)(0.2442,0.3484)(0.2442,0.3504)(0.2329,0.3504)
\polypmIIId{126}(0.2329,0.3503)(0.2442,0.3503)(0.2442,0.3523)(0.2329,0.3523)
\polypmIIId{127}(0.2329,0.3522)(0.2442,0.3522)(0.2442,0.3542)(0.2329,0.3542)

\PST@Border(0.2329,0.1078)
(0.2442,0.1078)
(0.2442,0.3542)
(0.2329,0.3542)
(0.2329,0.1078)


\rput[l](0.2502,0.1301){0.997}
\rput[l](0.2502,0.2048){0.998}
\rput[l](0.2502,0.2795){0.999}
\rput[l](0.2502,0.3542){1}

\catcode`@=12
\fi
\endpspicture}
  \caption{TSLP solution quality on weakly correlated instances ($\alpha = 0.1$).}
  \label{fig:tabusolcomp01}
\end{figure}

\begin{figure}[H]
  \centering
    \subfloat[1 resource]{% GNUPLOT: LaTeX picture using PSTRICKS macros
% Define new PST objects, if not already defined
\ifx\PSTloaded\undefined
\def\PSTloaded{t}

\catcode`@=11

\newpsobject{PST@Border}{psline}{linewidth=.0015,linestyle=solid}

\catcode`@=12

\fi
\psset{unit=5.0in,xunit=5.0in,yunit=3.0in}
\pspicture(0.000000,0.000000)(0.31, 0.35)
\ifx\nofigs\undefined
\catcode`@=11

\newrgbcolor{PST@COLOR0}{1 1 1}
\newrgbcolor{PST@COLOR1}{0.992 0.992 0.992}
\newrgbcolor{PST@COLOR2}{0.984 0.984 0.984}
\newrgbcolor{PST@COLOR3}{0.976 0.976 0.976}
\newrgbcolor{PST@COLOR4}{0.968 0.968 0.968}
\newrgbcolor{PST@COLOR5}{0.96 0.96 0.96}
\newrgbcolor{PST@COLOR6}{0.952 0.952 0.952}
\newrgbcolor{PST@COLOR7}{0.944 0.944 0.944}
\newrgbcolor{PST@COLOR8}{0.937 0.937 0.937}
\newrgbcolor{PST@COLOR9}{0.929 0.929 0.929}
\newrgbcolor{PST@COLOR10}{0.921 0.921 0.921}
\newrgbcolor{PST@COLOR11}{0.913 0.913 0.913}
\newrgbcolor{PST@COLOR12}{0.905 0.905 0.905}
\newrgbcolor{PST@COLOR13}{0.897 0.897 0.897}
\newrgbcolor{PST@COLOR14}{0.889 0.889 0.889}
\newrgbcolor{PST@COLOR15}{0.881 0.881 0.881}
\newrgbcolor{PST@COLOR16}{0.874 0.874 0.874}
\newrgbcolor{PST@COLOR17}{0.866 0.866 0.866}
\newrgbcolor{PST@COLOR18}{0.858 0.858 0.858}
\newrgbcolor{PST@COLOR19}{0.85 0.85 0.85}
\newrgbcolor{PST@COLOR20}{0.842 0.842 0.842}
\newrgbcolor{PST@COLOR21}{0.834 0.834 0.834}
\newrgbcolor{PST@COLOR22}{0.826 0.826 0.826}
\newrgbcolor{PST@COLOR23}{0.818 0.818 0.818}
\newrgbcolor{PST@COLOR24}{0.811 0.811 0.811}
\newrgbcolor{PST@COLOR25}{0.803 0.803 0.803}
\newrgbcolor{PST@COLOR26}{0.795 0.795 0.795}
\newrgbcolor{PST@COLOR27}{0.787 0.787 0.787}
\newrgbcolor{PST@COLOR28}{0.779 0.779 0.779}
\newrgbcolor{PST@COLOR29}{0.771 0.771 0.771}
\newrgbcolor{PST@COLOR30}{0.763 0.763 0.763}
\newrgbcolor{PST@COLOR31}{0.755 0.755 0.755}
\newrgbcolor{PST@COLOR32}{0.748 0.748 0.748}
\newrgbcolor{PST@COLOR33}{0.74 0.74 0.74}
\newrgbcolor{PST@COLOR34}{0.732 0.732 0.732}
\newrgbcolor{PST@COLOR35}{0.724 0.724 0.724}
\newrgbcolor{PST@COLOR36}{0.716 0.716 0.716}
\newrgbcolor{PST@COLOR37}{0.708 0.708 0.708}
\newrgbcolor{PST@COLOR38}{0.7 0.7 0.7}
\newrgbcolor{PST@COLOR39}{0.692 0.692 0.692}
\newrgbcolor{PST@COLOR40}{0.685 0.685 0.685}
\newrgbcolor{PST@COLOR41}{0.677 0.677 0.677}
\newrgbcolor{PST@COLOR42}{0.669 0.669 0.669}
\newrgbcolor{PST@COLOR43}{0.661 0.661 0.661}
\newrgbcolor{PST@COLOR44}{0.653 0.653 0.653}
\newrgbcolor{PST@COLOR45}{0.645 0.645 0.645}
\newrgbcolor{PST@COLOR46}{0.637 0.637 0.637}
\newrgbcolor{PST@COLOR47}{0.629 0.629 0.629}
\newrgbcolor{PST@COLOR48}{0.622 0.622 0.622}
\newrgbcolor{PST@COLOR49}{0.614 0.614 0.614}
\newrgbcolor{PST@COLOR50}{0.606 0.606 0.606}
\newrgbcolor{PST@COLOR51}{0.598 0.598 0.598}
\newrgbcolor{PST@COLOR52}{0.59 0.59 0.59}
\newrgbcolor{PST@COLOR53}{0.582 0.582 0.582}
\newrgbcolor{PST@COLOR54}{0.574 0.574 0.574}
\newrgbcolor{PST@COLOR55}{0.566 0.566 0.566}
\newrgbcolor{PST@COLOR56}{0.559 0.559 0.559}
\newrgbcolor{PST@COLOR57}{0.551 0.551 0.551}
\newrgbcolor{PST@COLOR58}{0.543 0.543 0.543}
\newrgbcolor{PST@COLOR59}{0.535 0.535 0.535}
\newrgbcolor{PST@COLOR60}{0.527 0.527 0.527}
\newrgbcolor{PST@COLOR61}{0.519 0.519 0.519}
\newrgbcolor{PST@COLOR62}{0.511 0.511 0.511}
\newrgbcolor{PST@COLOR63}{0.503 0.503 0.503}
\newrgbcolor{PST@COLOR64}{0.496 0.496 0.496}
\newrgbcolor{PST@COLOR65}{0.488 0.488 0.488}
\newrgbcolor{PST@COLOR66}{0.48 0.48 0.48}
\newrgbcolor{PST@COLOR67}{0.472 0.472 0.472}
\newrgbcolor{PST@COLOR68}{0.464 0.464 0.464}
\newrgbcolor{PST@COLOR69}{0.456 0.456 0.456}
\newrgbcolor{PST@COLOR70}{0.448 0.448 0.448}
\newrgbcolor{PST@COLOR71}{0.44 0.44 0.44}
\newrgbcolor{PST@COLOR72}{0.433 0.433 0.433}
\newrgbcolor{PST@COLOR73}{0.425 0.425 0.425}
\newrgbcolor{PST@COLOR74}{0.417 0.417 0.417}
\newrgbcolor{PST@COLOR75}{0.409 0.409 0.409}
\newrgbcolor{PST@COLOR76}{0.401 0.401 0.401}
\newrgbcolor{PST@COLOR77}{0.393 0.393 0.393}
\newrgbcolor{PST@COLOR78}{0.385 0.385 0.385}
\newrgbcolor{PST@COLOR79}{0.377 0.377 0.377}
\newrgbcolor{PST@COLOR80}{0.37 0.37 0.37}
\newrgbcolor{PST@COLOR81}{0.362 0.362 0.362}
\newrgbcolor{PST@COLOR82}{0.354 0.354 0.354}
\newrgbcolor{PST@COLOR83}{0.346 0.346 0.346}
\newrgbcolor{PST@COLOR84}{0.338 0.338 0.338}
\newrgbcolor{PST@COLOR85}{0.33 0.33 0.33}
\newrgbcolor{PST@COLOR86}{0.322 0.322 0.322}
\newrgbcolor{PST@COLOR87}{0.314 0.314 0.314}
\newrgbcolor{PST@COLOR88}{0.307 0.307 0.307}
\newrgbcolor{PST@COLOR89}{0.299 0.299 0.299}
\newrgbcolor{PST@COLOR90}{0.291 0.291 0.291}
\newrgbcolor{PST@COLOR91}{0.283 0.283 0.283}
\newrgbcolor{PST@COLOR92}{0.275 0.275 0.275}
\newrgbcolor{PST@COLOR93}{0.267 0.267 0.267}
\newrgbcolor{PST@COLOR94}{0.259 0.259 0.259}
\newrgbcolor{PST@COLOR95}{0.251 0.251 0.251}
\newrgbcolor{PST@COLOR96}{0.244 0.244 0.244}
\newrgbcolor{PST@COLOR97}{0.236 0.236 0.236}
\newrgbcolor{PST@COLOR98}{0.228 0.228 0.228}
\newrgbcolor{PST@COLOR99}{0.22 0.22 0.22}
\newrgbcolor{PST@COLOR100}{0.212 0.212 0.212}
\newrgbcolor{PST@COLOR101}{0.204 0.204 0.204}
\newrgbcolor{PST@COLOR102}{0.196 0.196 0.196}
\newrgbcolor{PST@COLOR103}{0.188 0.188 0.188}
\newrgbcolor{PST@COLOR104}{0.181 0.181 0.181}
\newrgbcolor{PST@COLOR105}{0.173 0.173 0.173}
\newrgbcolor{PST@COLOR106}{0.165 0.165 0.165}
\newrgbcolor{PST@COLOR107}{0.157 0.157 0.157}
\newrgbcolor{PST@COLOR108}{0.149 0.149 0.149}
\newrgbcolor{PST@COLOR109}{0.141 0.141 0.141}
\newrgbcolor{PST@COLOR110}{0.133 0.133 0.133}
\newrgbcolor{PST@COLOR111}{0.125 0.125 0.125}
\newrgbcolor{PST@COLOR112}{0.118 0.118 0.118}
\newrgbcolor{PST@COLOR113}{0.11 0.11 0.11}
\newrgbcolor{PST@COLOR114}{0.102 0.102 0.102}
\newrgbcolor{PST@COLOR115}{0.094 0.094 0.094}
\newrgbcolor{PST@COLOR116}{0.086 0.086 0.086}
\newrgbcolor{PST@COLOR117}{0.078 0.078 0.078}
\newrgbcolor{PST@COLOR118}{0.07 0.07 0.07}
\newrgbcolor{PST@COLOR119}{0.062 0.062 0.062}
\newrgbcolor{PST@COLOR120}{0.055 0.055 0.055}
\newrgbcolor{PST@COLOR121}{0.047 0.047 0.047}
\newrgbcolor{PST@COLOR122}{0.039 0.039 0.039}
\newrgbcolor{PST@COLOR123}{0.031 0.031 0.031}
\newrgbcolor{PST@COLOR124}{0.023 0.023 0.023}
\newrgbcolor{PST@COLOR125}{0.015 0.015 0.015}
\newrgbcolor{PST@COLOR126}{0.007 0.007 0.007}
\newrgbcolor{PST@COLOR127}{0 0 0}


\def\polypmIIId#1{\pspolygon[linestyle=none,fillstyle=solid,fillcolor=PST@COLOR#1]}

\polypmIIId{124}(0.1432,0.19)(0.0864,0.19)(0.0864,0.1078)(0.1432,0.1078)
\polypmIIId{126}(0.1432,0.272)(0.0864,0.272)(0.0864,0.19)(0.1432,0.19)
\polypmIIId{127}(0.1432,0.3542)(0.0864,0.3542)(0.0864,0.272)(0.1432,0.272)

\polypmIIId{125}(0.2,0.19)(0.1432,0.19)(0.1432,0.1078)(0.2,0.1078)
\polypmIIId{126}(0.2,0.272)(0.1432,0.272)(0.1432,0.19)(0.2,0.19)
\polypmIIId{127}(0.2,0.3542)(0.1432,0.3542)(0.1432,0.272)(0.2,0.272)

\polypmIIId{125}(0.2568,0.19)(0.2,0.19)(0.2,0.1078)(0.2568,0.1078)
\polypmIIId{126}(0.2568,0.272)(0.2,0.272)(0.2,0.19)(0.2568,0.19)
\polypmIIId{127}(0.2568,0.3542)(0.2,0.3542)(0.2,0.272)(0.2568,0.272)

\polypmIIId{124}(0.3136,0.19)(0.2568,0.19)(0.2568,0.1078)(0.3136,0.1078)
\polypmIIId{126}(0.3136,0.272)(0.2568,0.272)(0.2568,0.19)(0.3136,0.19)
\polypmIIId{127}(0.3136,0.3542)(0.2568,0.3542)(0.2568,0.272)(0.3136,0.272)

\rput(0.1148,0.07){3}
\rput(0.1716,0.07){4}
\rput(0.2284,0.07){5}
\rput(0.2852,0.07){6}
\rput(0.2000,0.0070){years}

\rput[r](0.0806,0.1489){25}
\rput[r](0.0806,0.2310){50}
\rput[r](0.0806,0.3131){100}
\rput{L}(0.0096,0.2310){actions}

\PST@Border(0.0864,0.3542)
(0.0864,0.1078)
(0.3136,0.1078)
(0.3136,0.3542)
(0.0864,0.3542)

\catcode`@=12
\fi
\endpspicture} 
    \subfloat[2 resources]{% GNUPLOT: LaTeX picture using PSTRICKS macros
% Define new PST objects, if not already defined
\ifx\PSTloaded\undefined
\def\PSTloaded{t}

\catcode`@=11

\newpsobject{PST@Border}{psline}{linewidth=.0015,linestyle=solid}

\catcode`@=12

\fi
\psset{unit=5.0in,xunit=5.0in,yunit=3.0in}
\pspicture(0.000000,0.000000)(0.225000,0.35)
\ifx\nofigs\undefined
\catcode`@=11

\newrgbcolor{PST@COLOR0}{1 1 1}
\newrgbcolor{PST@COLOR1}{0.992 0.992 0.992}
\newrgbcolor{PST@COLOR2}{0.984 0.984 0.984}
\newrgbcolor{PST@COLOR3}{0.976 0.976 0.976}
\newrgbcolor{PST@COLOR4}{0.968 0.968 0.968}
\newrgbcolor{PST@COLOR5}{0.96 0.96 0.96}
\newrgbcolor{PST@COLOR6}{0.952 0.952 0.952}
\newrgbcolor{PST@COLOR7}{0.944 0.944 0.944}
\newrgbcolor{PST@COLOR8}{0.937 0.937 0.937}
\newrgbcolor{PST@COLOR9}{0.929 0.929 0.929}
\newrgbcolor{PST@COLOR10}{0.921 0.921 0.921}
\newrgbcolor{PST@COLOR11}{0.913 0.913 0.913}
\newrgbcolor{PST@COLOR12}{0.905 0.905 0.905}
\newrgbcolor{PST@COLOR13}{0.897 0.897 0.897}
\newrgbcolor{PST@COLOR14}{0.889 0.889 0.889}
\newrgbcolor{PST@COLOR15}{0.881 0.881 0.881}
\newrgbcolor{PST@COLOR16}{0.874 0.874 0.874}
\newrgbcolor{PST@COLOR17}{0.866 0.866 0.866}
\newrgbcolor{PST@COLOR18}{0.858 0.858 0.858}
\newrgbcolor{PST@COLOR19}{0.85 0.85 0.85}
\newrgbcolor{PST@COLOR20}{0.842 0.842 0.842}
\newrgbcolor{PST@COLOR21}{0.834 0.834 0.834}
\newrgbcolor{PST@COLOR22}{0.826 0.826 0.826}
\newrgbcolor{PST@COLOR23}{0.818 0.818 0.818}
\newrgbcolor{PST@COLOR24}{0.811 0.811 0.811}
\newrgbcolor{PST@COLOR25}{0.803 0.803 0.803}
\newrgbcolor{PST@COLOR26}{0.795 0.795 0.795}
\newrgbcolor{PST@COLOR27}{0.787 0.787 0.787}
\newrgbcolor{PST@COLOR28}{0.779 0.779 0.779}
\newrgbcolor{PST@COLOR29}{0.771 0.771 0.771}
\newrgbcolor{PST@COLOR30}{0.763 0.763 0.763}
\newrgbcolor{PST@COLOR31}{0.755 0.755 0.755}
\newrgbcolor{PST@COLOR32}{0.748 0.748 0.748}
\newrgbcolor{PST@COLOR33}{0.74 0.74 0.74}
\newrgbcolor{PST@COLOR34}{0.732 0.732 0.732}
\newrgbcolor{PST@COLOR35}{0.724 0.724 0.724}
\newrgbcolor{PST@COLOR36}{0.716 0.716 0.716}
\newrgbcolor{PST@COLOR37}{0.708 0.708 0.708}
\newrgbcolor{PST@COLOR38}{0.7 0.7 0.7}
\newrgbcolor{PST@COLOR39}{0.692 0.692 0.692}
\newrgbcolor{PST@COLOR40}{0.685 0.685 0.685}
\newrgbcolor{PST@COLOR41}{0.677 0.677 0.677}
\newrgbcolor{PST@COLOR42}{0.669 0.669 0.669}
\newrgbcolor{PST@COLOR43}{0.661 0.661 0.661}
\newrgbcolor{PST@COLOR44}{0.653 0.653 0.653}
\newrgbcolor{PST@COLOR45}{0.645 0.645 0.645}
\newrgbcolor{PST@COLOR46}{0.637 0.637 0.637}
\newrgbcolor{PST@COLOR47}{0.629 0.629 0.629}
\newrgbcolor{PST@COLOR48}{0.622 0.622 0.622}
\newrgbcolor{PST@COLOR49}{0.614 0.614 0.614}
\newrgbcolor{PST@COLOR50}{0.606 0.606 0.606}
\newrgbcolor{PST@COLOR51}{0.598 0.598 0.598}
\newrgbcolor{PST@COLOR52}{0.59 0.59 0.59}
\newrgbcolor{PST@COLOR53}{0.582 0.582 0.582}
\newrgbcolor{PST@COLOR54}{0.574 0.574 0.574}
\newrgbcolor{PST@COLOR55}{0.566 0.566 0.566}
\newrgbcolor{PST@COLOR56}{0.559 0.559 0.559}
\newrgbcolor{PST@COLOR57}{0.551 0.551 0.551}
\newrgbcolor{PST@COLOR58}{0.543 0.543 0.543}
\newrgbcolor{PST@COLOR59}{0.535 0.535 0.535}
\newrgbcolor{PST@COLOR60}{0.527 0.527 0.527}
\newrgbcolor{PST@COLOR61}{0.519 0.519 0.519}
\newrgbcolor{PST@COLOR62}{0.511 0.511 0.511}
\newrgbcolor{PST@COLOR63}{0.503 0.503 0.503}
\newrgbcolor{PST@COLOR64}{0.496 0.496 0.496}
\newrgbcolor{PST@COLOR65}{0.488 0.488 0.488}
\newrgbcolor{PST@COLOR66}{0.48 0.48 0.48}
\newrgbcolor{PST@COLOR67}{0.472 0.472 0.472}
\newrgbcolor{PST@COLOR68}{0.464 0.464 0.464}
\newrgbcolor{PST@COLOR69}{0.456 0.456 0.456}
\newrgbcolor{PST@COLOR70}{0.448 0.448 0.448}
\newrgbcolor{PST@COLOR71}{0.44 0.44 0.44}
\newrgbcolor{PST@COLOR72}{0.433 0.433 0.433}
\newrgbcolor{PST@COLOR73}{0.425 0.425 0.425}
\newrgbcolor{PST@COLOR74}{0.417 0.417 0.417}
\newrgbcolor{PST@COLOR75}{0.409 0.409 0.409}
\newrgbcolor{PST@COLOR76}{0.401 0.401 0.401}
\newrgbcolor{PST@COLOR77}{0.393 0.393 0.393}
\newrgbcolor{PST@COLOR78}{0.385 0.385 0.385}
\newrgbcolor{PST@COLOR79}{0.377 0.377 0.377}
\newrgbcolor{PST@COLOR80}{0.37 0.37 0.37}
\newrgbcolor{PST@COLOR81}{0.362 0.362 0.362}
\newrgbcolor{PST@COLOR82}{0.354 0.354 0.354}
\newrgbcolor{PST@COLOR83}{0.346 0.346 0.346}
\newrgbcolor{PST@COLOR84}{0.338 0.338 0.338}
\newrgbcolor{PST@COLOR85}{0.33 0.33 0.33}
\newrgbcolor{PST@COLOR86}{0.322 0.322 0.322}
\newrgbcolor{PST@COLOR87}{0.314 0.314 0.314}
\newrgbcolor{PST@COLOR88}{0.307 0.307 0.307}
\newrgbcolor{PST@COLOR89}{0.299 0.299 0.299}
\newrgbcolor{PST@COLOR90}{0.291 0.291 0.291}
\newrgbcolor{PST@COLOR91}{0.283 0.283 0.283}
\newrgbcolor{PST@COLOR92}{0.275 0.275 0.275}
\newrgbcolor{PST@COLOR93}{0.267 0.267 0.267}
\newrgbcolor{PST@COLOR94}{0.259 0.259 0.259}
\newrgbcolor{PST@COLOR95}{0.251 0.251 0.251}
\newrgbcolor{PST@COLOR96}{0.244 0.244 0.244}
\newrgbcolor{PST@COLOR97}{0.236 0.236 0.236}
\newrgbcolor{PST@COLOR98}{0.228 0.228 0.228}
\newrgbcolor{PST@COLOR99}{0.22 0.22 0.22}
\newrgbcolor{PST@COLOR100}{0.212 0.212 0.212}
\newrgbcolor{PST@COLOR101}{0.204 0.204 0.204}
\newrgbcolor{PST@COLOR102}{0.196 0.196 0.196}
\newrgbcolor{PST@COLOR103}{0.188 0.188 0.188}
\newrgbcolor{PST@COLOR104}{0.181 0.181 0.181}
\newrgbcolor{PST@COLOR105}{0.173 0.173 0.173}
\newrgbcolor{PST@COLOR106}{0.165 0.165 0.165}
\newrgbcolor{PST@COLOR107}{0.157 0.157 0.157}
\newrgbcolor{PST@COLOR108}{0.149 0.149 0.149}
\newrgbcolor{PST@COLOR109}{0.141 0.141 0.141}
\newrgbcolor{PST@COLOR110}{0.133 0.133 0.133}
\newrgbcolor{PST@COLOR111}{0.125 0.125 0.125}
\newrgbcolor{PST@COLOR112}{0.118 0.118 0.118}
\newrgbcolor{PST@COLOR113}{0.11 0.11 0.11}
\newrgbcolor{PST@COLOR114}{0.102 0.102 0.102}
\newrgbcolor{PST@COLOR115}{0.094 0.094 0.094}
\newrgbcolor{PST@COLOR116}{0.086 0.086 0.086}
\newrgbcolor{PST@COLOR117}{0.078 0.078 0.078}
\newrgbcolor{PST@COLOR118}{0.07 0.07 0.07}
\newrgbcolor{PST@COLOR119}{0.062 0.062 0.062}
\newrgbcolor{PST@COLOR120}{0.055 0.055 0.055}
\newrgbcolor{PST@COLOR121}{0.047 0.047 0.047}
\newrgbcolor{PST@COLOR122}{0.039 0.039 0.039}
\newrgbcolor{PST@COLOR123}{0.031 0.031 0.031}
\newrgbcolor{PST@COLOR124}{0.023 0.023 0.023}
\newrgbcolor{PST@COLOR125}{0.015 0.015 0.015}
\newrgbcolor{PST@COLOR126}{0.007 0.007 0.007}
\newrgbcolor{PST@COLOR127}{0 0 0}

\def\polypmIIId#1{\pspolygon[linestyle=none,fillstyle=solid,fillcolor=PST@COLOR#1]}

\polypmIIId{122} (0.0568,0.19)  (0.0,0.19)  (0.0,0.1078)(0.0568,0.1078)
\polypmIIId{125}  (0.0568,0.272) (0.0,0.272) (0.0,0.19)  (0.0568,0.19)
\polypmIIId{126}  (0.0568,0.3542)(0.0,0.3542)(0.0,0.272) (0.0568,0.272)

\polypmIIId{121} (0.1136,   0.19)  (0.0568,0.19)  (0.0568,0.1078)(0.1136,0.1078)
\polypmIIId{124}  (0.1136,   0.272) (0.0568,0.272) (0.0568,0.19)  (0.1136,0.19)
\polypmIIId{126}  (0.1136,   0.3542)(0.0568,0.3542)(0.0568,0.272) (0.1136,0.272)

\polypmIIId{120}(0.1704,0.19)  (0.1136,   0.19)  (0.1136,   0.1078)(0.1704,0.1078)
\polypmIIId{124} (0.1704,0.272) (0.1136,   0.272) (0.1136,   0.19)  (0.1704,0.19)
\polypmIIId{125}  (0.1704,0.3542)(0.1136,   0.3542)(0.1136,   0.272) (0.1704,0.272)

\polypmIIId{119}(0.2272,0.19)  (0.1704,0.19)  (0.1704,0.1078)(0.2272,0.1078)
\polypmIIId{123}  (0.2272,0.272) (0.1704,0.272) (0.1704,0.19)  (0.2272,0.19)
\polypmIIId{125}  (0.2272,0.3542)(0.1704,0.3542)(0.1704,0.272) (0.2272,0.272)

\rput(0.0284,0.07){3}
\rput(0.0852,0.07){4}
\rput(0.1420,0.07){5}
\rput(0.1988,0.07){6}
\rput(0.1136,0.0070){years}


\PST@Border(0.0,0.3542)
(0.0,0.1078)
(0.2272,0.1078)
(0.2272,0.3542)
(0.0,0.3542)

\catcode`@=12
\fi
\endpspicture}
    \subfloat[4 resources]{% GNUPLOT: LaTeX picture using PSTRICKS macros
% Define new PST objects, if not already defined
\ifx\PSTloaded\undefined
\def\PSTloaded{t}

\catcode`@=11

\newpsobject{PST@Border}{psline}{linewidth=.0015,linestyle=solid}

\catcode`@=12

\fi
\psset{unit=5.0in,xunit=5.0in,yunit=3.0in}
\pspicture(0.000000,0.000000)(0.3136,0.35)
\ifx\nofigs\undefined
\catcode`@=11

\newrgbcolor{PST@COLOR0}{1 1 1}
\newrgbcolor{PST@COLOR1}{0.992 0.992 0.992}
\newrgbcolor{PST@COLOR2}{0.984 0.984 0.984}
\newrgbcolor{PST@COLOR3}{0.976 0.976 0.976}
\newrgbcolor{PST@COLOR4}{0.968 0.968 0.968}
\newrgbcolor{PST@COLOR5}{0.96 0.96 0.96}
\newrgbcolor{PST@COLOR6}{0.952 0.952 0.952}
\newrgbcolor{PST@COLOR7}{0.944 0.944 0.944}
\newrgbcolor{PST@COLOR8}{0.937 0.937 0.937}
\newrgbcolor{PST@COLOR9}{0.929 0.929 0.929}
\newrgbcolor{PST@COLOR10}{0.921 0.921 0.921}
\newrgbcolor{PST@COLOR11}{0.913 0.913 0.913}
\newrgbcolor{PST@COLOR12}{0.905 0.905 0.905}
\newrgbcolor{PST@COLOR13}{0.897 0.897 0.897}
\newrgbcolor{PST@COLOR14}{0.889 0.889 0.889}
\newrgbcolor{PST@COLOR15}{0.881 0.881 0.881}
\newrgbcolor{PST@COLOR16}{0.874 0.874 0.874}
\newrgbcolor{PST@COLOR17}{0.866 0.866 0.866}
\newrgbcolor{PST@COLOR18}{0.858 0.858 0.858}
\newrgbcolor{PST@COLOR19}{0.85 0.85 0.85}
\newrgbcolor{PST@COLOR20}{0.842 0.842 0.842}
\newrgbcolor{PST@COLOR21}{0.834 0.834 0.834}
\newrgbcolor{PST@COLOR22}{0.826 0.826 0.826}
\newrgbcolor{PST@COLOR23}{0.818 0.818 0.818}
\newrgbcolor{PST@COLOR24}{0.811 0.811 0.811}
\newrgbcolor{PST@COLOR25}{0.803 0.803 0.803}
\newrgbcolor{PST@COLOR26}{0.795 0.795 0.795}
\newrgbcolor{PST@COLOR27}{0.787 0.787 0.787}
\newrgbcolor{PST@COLOR28}{0.779 0.779 0.779}
\newrgbcolor{PST@COLOR29}{0.771 0.771 0.771}
\newrgbcolor{PST@COLOR30}{0.763 0.763 0.763}
\newrgbcolor{PST@COLOR31}{0.755 0.755 0.755}
\newrgbcolor{PST@COLOR32}{0.748 0.748 0.748}
\newrgbcolor{PST@COLOR33}{0.74 0.74 0.74}
\newrgbcolor{PST@COLOR34}{0.732 0.732 0.732}
\newrgbcolor{PST@COLOR35}{0.724 0.724 0.724}
\newrgbcolor{PST@COLOR36}{0.716 0.716 0.716}
\newrgbcolor{PST@COLOR37}{0.708 0.708 0.708}
\newrgbcolor{PST@COLOR38}{0.7 0.7 0.7}
\newrgbcolor{PST@COLOR39}{0.692 0.692 0.692}
\newrgbcolor{PST@COLOR40}{0.685 0.685 0.685}
\newrgbcolor{PST@COLOR41}{0.677 0.677 0.677}
\newrgbcolor{PST@COLOR42}{0.669 0.669 0.669}
\newrgbcolor{PST@COLOR43}{0.661 0.661 0.661}
\newrgbcolor{PST@COLOR44}{0.653 0.653 0.653}
\newrgbcolor{PST@COLOR45}{0.645 0.645 0.645}
\newrgbcolor{PST@COLOR46}{0.637 0.637 0.637}
\newrgbcolor{PST@COLOR47}{0.629 0.629 0.629}
\newrgbcolor{PST@COLOR48}{0.622 0.622 0.622}
\newrgbcolor{PST@COLOR49}{0.614 0.614 0.614}
\newrgbcolor{PST@COLOR50}{0.606 0.606 0.606}
\newrgbcolor{PST@COLOR51}{0.598 0.598 0.598}
\newrgbcolor{PST@COLOR52}{0.59 0.59 0.59}
\newrgbcolor{PST@COLOR53}{0.582 0.582 0.582}
\newrgbcolor{PST@COLOR54}{0.574 0.574 0.574}
\newrgbcolor{PST@COLOR55}{0.566 0.566 0.566}
\newrgbcolor{PST@COLOR56}{0.559 0.559 0.559}
\newrgbcolor{PST@COLOR57}{0.551 0.551 0.551}
\newrgbcolor{PST@COLOR58}{0.543 0.543 0.543}
\newrgbcolor{PST@COLOR59}{0.535 0.535 0.535}
\newrgbcolor{PST@COLOR60}{0.527 0.527 0.527}
\newrgbcolor{PST@COLOR61}{0.519 0.519 0.519}
\newrgbcolor{PST@COLOR62}{0.511 0.511 0.511}
\newrgbcolor{PST@COLOR63}{0.503 0.503 0.503}
\newrgbcolor{PST@COLOR64}{0.496 0.496 0.496}
\newrgbcolor{PST@COLOR65}{0.488 0.488 0.488}
\newrgbcolor{PST@COLOR66}{0.48 0.48 0.48}
\newrgbcolor{PST@COLOR67}{0.472 0.472 0.472}
\newrgbcolor{PST@COLOR68}{0.464 0.464 0.464}
\newrgbcolor{PST@COLOR69}{0.456 0.456 0.456}
\newrgbcolor{PST@COLOR70}{0.448 0.448 0.448}
\newrgbcolor{PST@COLOR71}{0.44 0.44 0.44}
\newrgbcolor{PST@COLOR72}{0.433 0.433 0.433}
\newrgbcolor{PST@COLOR73}{0.425 0.425 0.425}
\newrgbcolor{PST@COLOR74}{0.417 0.417 0.417}
\newrgbcolor{PST@COLOR75}{0.409 0.409 0.409}
\newrgbcolor{PST@COLOR76}{0.401 0.401 0.401}
\newrgbcolor{PST@COLOR77}{0.393 0.393 0.393}
\newrgbcolor{PST@COLOR78}{0.385 0.385 0.385}
\newrgbcolor{PST@COLOR79}{0.377 0.377 0.377}
\newrgbcolor{PST@COLOR80}{0.37 0.37 0.37}
\newrgbcolor{PST@COLOR81}{0.362 0.362 0.362}
\newrgbcolor{PST@COLOR82}{0.354 0.354 0.354}
\newrgbcolor{PST@COLOR83}{0.346 0.346 0.346}
\newrgbcolor{PST@COLOR84}{0.338 0.338 0.338}
\newrgbcolor{PST@COLOR85}{0.33 0.33 0.33}
\newrgbcolor{PST@COLOR86}{0.322 0.322 0.322}
\newrgbcolor{PST@COLOR87}{0.314 0.314 0.314}
\newrgbcolor{PST@COLOR88}{0.307 0.307 0.307}
\newrgbcolor{PST@COLOR89}{0.299 0.299 0.299}
\newrgbcolor{PST@COLOR90}{0.291 0.291 0.291}
\newrgbcolor{PST@COLOR91}{0.283 0.283 0.283}
\newrgbcolor{PST@COLOR92}{0.275 0.275 0.275}
\newrgbcolor{PST@COLOR93}{0.267 0.267 0.267}
\newrgbcolor{PST@COLOR94}{0.259 0.259 0.259}
\newrgbcolor{PST@COLOR95}{0.251 0.251 0.251}
\newrgbcolor{PST@COLOR96}{0.244 0.244 0.244}
\newrgbcolor{PST@COLOR97}{0.236 0.236 0.236}
\newrgbcolor{PST@COLOR98}{0.228 0.228 0.228}
\newrgbcolor{PST@COLOR99}{0.22 0.22 0.22}
\newrgbcolor{PST@COLOR100}{0.212 0.212 0.212}
\newrgbcolor{PST@COLOR101}{0.204 0.204 0.204}
\newrgbcolor{PST@COLOR102}{0.196 0.196 0.196}
\newrgbcolor{PST@COLOR103}{0.188 0.188 0.188}
\newrgbcolor{PST@COLOR104}{0.181 0.181 0.181}
\newrgbcolor{PST@COLOR105}{0.173 0.173 0.173}
\newrgbcolor{PST@COLOR106}{0.165 0.165 0.165}
\newrgbcolor{PST@COLOR107}{0.157 0.157 0.157}
\newrgbcolor{PST@COLOR108}{0.149 0.149 0.149}
\newrgbcolor{PST@COLOR109}{0.141 0.141 0.141}
\newrgbcolor{PST@COLOR110}{0.133 0.133 0.133}
\newrgbcolor{PST@COLOR111}{0.125 0.125 0.125}
\newrgbcolor{PST@COLOR112}{0.118 0.118 0.118}
\newrgbcolor{PST@COLOR113}{0.11 0.11 0.11}
\newrgbcolor{PST@COLOR114}{0.102 0.102 0.102}
\newrgbcolor{PST@COLOR115}{0.094 0.094 0.094}
\newrgbcolor{PST@COLOR116}{0.086 0.086 0.086}
\newrgbcolor{PST@COLOR117}{0.078 0.078 0.078}
\newrgbcolor{PST@COLOR118}{0.07 0.07 0.07}
\newrgbcolor{PST@COLOR119}{0.062 0.062 0.062}
\newrgbcolor{PST@COLOR120}{0.055 0.055 0.055}
\newrgbcolor{PST@COLOR121}{0.047 0.047 0.047}
\newrgbcolor{PST@COLOR122}{0.039 0.039 0.039}
\newrgbcolor{PST@COLOR123}{0.031 0.031 0.031}
\newrgbcolor{PST@COLOR124}{0.023 0.023 0.023}
\newrgbcolor{PST@COLOR125}{0.015 0.015 0.015}
\newrgbcolor{PST@COLOR126}{0.007 0.007 0.007}
\newrgbcolor{PST@COLOR127}{0 0 0}

\def\polypmIIId#1{\pspolygon[linestyle=none,fillstyle=solid,fillcolor=PST@COLOR#1]}

\polypmIIId{117} (0.0568,0.19)  (0.0,0.19)  (0.0,0.1078)(0.0568,0.1078)
\polypmIIId{120}  (0.0568,0.272) (0.0,0.272) (0.0,0.19)  (0.0568,0.19)
\polypmIIId{124}  (0.0568,0.3542)(0.0,0.3542)(0.0,0.272) (0.0568,0.272)

\polypmIIId{115} (0.1136,   0.19)  (0.0568,0.19)  (0.0568,0.1078)(0.1136,0.1078)
\polypmIIId{120}  (0.1136,   0.272) (0.0568,0.272) (0.0568,0.19)  (0.1136,0.19)
\polypmIIId{123}  (0.1136,   0.3542)(0.0568,0.3542)(0.0568,0.272) (0.1136,0.272)

\polypmIIId{113}(0.1704,0.19)  (0.1136,   0.19)  (0.1136,   0.1078)(0.1704,0.1078)
\polypmIIId{118} (0.1704,0.272) (0.1136,   0.272) (0.1136,   0.19)  (0.1704,0.19)
\polypmIIId{122}  (0.1704,0.3542)(0.1136,   0.3542)(0.1136,   0.272) (0.1704,0.272)

\polypmIIId{112}(0.2272,0.19)  (0.1704,0.19)  (0.1704,0.1078)(0.2272,0.1078)
\polypmIIId{118}  (0.2272,0.272) (0.1704,0.272) (0.1704,0.19)  (0.2272,0.19)
\polypmIIId{122}  (0.2272,0.3542)(0.1704,0.3542)(0.1704,0.272) (0.2272,0.272)

\rput(0.0284,0.07){3}
\rput(0.0852,0.07){4}
\rput(0.1420,0.07){5}
\rput(0.1988,0.07){6}
\rput(0.1136,0.0070){years}

\PST@Border(0.0,0.3542)
(0.0,0.1078)
(0.2272,0.1078)
(0.2272,0.3542)
(0.0,0.3542)

\polypmIIId{0}(0.2329,0.1078)(0.2442,0.1078)(0.2442,0.1098)(0.2329,0.1098)
\polypmIIId{1}(0.2329,0.1097)(0.2442,0.1097)(0.2442,0.1117)(0.2329,0.1117)
\polypmIIId{2}(0.2329,0.1116)(0.2442,0.1116)(0.2442,0.1136)(0.2329,0.1136)
\polypmIIId{3}(0.2329,0.1135)(0.2442,0.1135)(0.2442,0.1156)(0.2329,0.1156)
\polypmIIId{4}(0.2329,0.1155)(0.2442,0.1155)(0.2442,0.1175)(0.2329,0.1175)
\polypmIIId{5}(0.2329,0.1174)(0.2442,0.1174)(0.2442,0.1194)(0.2329,0.1194)
\polypmIIId{6}(0.2329,0.1193)(0.2442,0.1193)(0.2442,0.1213)(0.2329,0.1213)
\polypmIIId{7}(0.2329,0.1212)(0.2442,0.1212)(0.2442,0.1233)(0.2329,0.1233)
\polypmIIId{8}(0.2329,0.1232)(0.2442,0.1232)(0.2442,0.1252)(0.2329,0.1252)
\polypmIIId{9}(0.2329,0.1251)(0.2442,0.1251)(0.2442,0.1271)(0.2329,0.1271)
\polypmIIId{10}(0.2329,0.127)(0.2442,0.127)(0.2442,0.129)(0.2329,0.129)
\polypmIIId{11}(0.2329,0.1289)(0.2442,0.1289)(0.2442,0.131)(0.2329,0.131)
\polypmIIId{12}(0.2329,0.1309)(0.2442,0.1309)(0.2442,0.1329)(0.2329,0.1329)
\polypmIIId{13}(0.2329,0.1328)(0.2442,0.1328)(0.2442,0.1348)(0.2329,0.1348)
\polypmIIId{14}(0.2329,0.1347)(0.2442,0.1347)(0.2442,0.1367)(0.2329,0.1367)
\polypmIIId{15}(0.2329,0.1366)(0.2442,0.1366)(0.2442,0.1387)(0.2329,0.1387)
\polypmIIId{16}(0.2329,0.1386)(0.2442,0.1386)(0.2442,0.1406)(0.2329,0.1406)
\polypmIIId{17}(0.2329,0.1405)(0.2442,0.1405)(0.2442,0.1425)(0.2329,0.1425)
\polypmIIId{18}(0.2329,0.1424)(0.2442,0.1424)(0.2442,0.1444)(0.2329,0.1444)
\polypmIIId{19}(0.2329,0.1443)(0.2442,0.1443)(0.2442,0.1464)(0.2329,0.1464)
\polypmIIId{20}(0.2329,0.1463)(0.2442,0.1463)(0.2442,0.1483)(0.2329,0.1483)
\polypmIIId{21}(0.2329,0.1482)(0.2442,0.1482)(0.2442,0.1502)(0.2329,0.1502)
\polypmIIId{22}(0.2329,0.1501)(0.2442,0.1501)(0.2442,0.1521)(0.2329,0.1521)
\polypmIIId{23}(0.2329,0.152)(0.2442,0.152)(0.2442,0.1541)(0.2329,0.1541)
\polypmIIId{24}(0.2329,0.154)(0.2442,0.154)(0.2442,0.156)(0.2329,0.156)
\polypmIIId{25}(0.2329,0.1559)(0.2442,0.1559)(0.2442,0.1579)(0.2329,0.1579)
\polypmIIId{26}(0.2329,0.1578)(0.2442,0.1578)(0.2442,0.1598)(0.2329,0.1598)
\polypmIIId{27}(0.2329,0.1597)(0.2442,0.1597)(0.2442,0.1618)(0.2329,0.1618)
\polypmIIId{28}(0.2329,0.1617)(0.2442,0.1617)(0.2442,0.1637)(0.2329,0.1637)
\polypmIIId{29}(0.2329,0.1636)(0.2442,0.1636)(0.2442,0.1656)(0.2329,0.1656)
\polypmIIId{30}(0.2329,0.1655)(0.2442,0.1655)(0.2442,0.1675)(0.2329,0.1675)
\polypmIIId{31}(0.2329,0.1674)(0.2442,0.1674)(0.2442,0.1695)(0.2329,0.1695)
\polypmIIId{32}(0.2329,0.1694)(0.2442,0.1694)(0.2442,0.1714)(0.2329,0.1714)
\polypmIIId{33}(0.2329,0.1713)(0.2442,0.1713)(0.2442,0.1733)(0.2329,0.1733)
\polypmIIId{34}(0.2329,0.1732)(0.2442,0.1732)(0.2442,0.1752)(0.2329,0.1752)
\polypmIIId{35}(0.2329,0.1751)(0.2442,0.1751)(0.2442,0.1772)(0.2329,0.1772)
\polypmIIId{36}(0.2329,0.1771)(0.2442,0.1771)(0.2442,0.1791)(0.2329,0.1791)
\polypmIIId{37}(0.2329,0.179)(0.2442,0.179)(0.2442,0.181)(0.2329,0.181)
\polypmIIId{38}(0.2329,0.1809)(0.2442,0.1809)(0.2442,0.1829)(0.2329,0.1829)
\polypmIIId{39}(0.2329,0.1828)(0.2442,0.1828)(0.2442,0.1849)(0.2329,0.1849)
\polypmIIId{40}(0.2329,0.1848)(0.2442,0.1848)(0.2442,0.1868)(0.2329,0.1868)
\polypmIIId{41}(0.2329,0.1867)(0.2442,0.1867)(0.2442,0.1887)(0.2329,0.1887)
\polypmIIId{42}(0.2329,0.1886)(0.2442,0.1886)(0.2442,0.1906)(0.2329,0.1906)
\polypmIIId{43}(0.2329,0.1905)(0.2442,0.1905)(0.2442,0.1926)(0.2329,0.1926)
\polypmIIId{44}(0.2329,0.1925)(0.2442,0.1925)(0.2442,0.1945)(0.2329,0.1945)
\polypmIIId{45}(0.2329,0.1944)(0.2442,0.1944)(0.2442,0.1964)(0.2329,0.1964)
\polypmIIId{46}(0.2329,0.1963)(0.2442,0.1963)(0.2442,0.1983)(0.2329,0.1983)
\polypmIIId{47}(0.2329,0.1982)(0.2442,0.1982)(0.2442,0.2003)(0.2329,0.2003)
\polypmIIId{48}(0.2329,0.2002)(0.2442,0.2002)(0.2442,0.2022)(0.2329,0.2022)
\polypmIIId{49}(0.2329,0.2021)(0.2442,0.2021)(0.2442,0.2041)(0.2329,0.2041)
\polypmIIId{50}(0.2329,0.204)(0.2442,0.204)(0.2442,0.206)(0.2329,0.206)
\polypmIIId{51}(0.2329,0.2059)(0.2442,0.2059)(0.2442,0.208)(0.2329,0.208)
\polypmIIId{52}(0.2329,0.2079)(0.2442,0.2079)(0.2442,0.2099)(0.2329,0.2099)
\polypmIIId{53}(0.2329,0.2098)(0.2442,0.2098)(0.2442,0.2118)(0.2329,0.2118)
\polypmIIId{54}(0.2329,0.2117)(0.2442,0.2117)(0.2442,0.2137)(0.2329,0.2137)
\polypmIIId{55}(0.2329,0.2136)(0.2442,0.2136)(0.2442,0.2157)(0.2329,0.2157)
\polypmIIId{56}(0.2329,0.2156)(0.2442,0.2156)(0.2442,0.2176)(0.2329,0.2176)
\polypmIIId{57}(0.2329,0.2175)(0.2442,0.2175)(0.2442,0.2195)(0.2329,0.2195)
\polypmIIId{58}(0.2329,0.2194)(0.2442,0.2194)(0.2442,0.2214)(0.2329,0.2214)
\polypmIIId{59}(0.2329,0.2213)(0.2442,0.2213)(0.2442,0.2234)(0.2329,0.2234)
\polypmIIId{60}(0.2329,0.2233)(0.2442,0.2233)(0.2442,0.2253)(0.2329,0.2253)
\polypmIIId{61}(0.2329,0.2252)(0.2442,0.2252)(0.2442,0.2272)(0.2329,0.2272)
\polypmIIId{62}(0.2329,0.2271)(0.2442,0.2271)(0.2442,0.2291)(0.2329,0.2291)
\polypmIIId{63}(0.2329,0.229)(0.2442,0.229)(0.2442,0.2311)(0.2329,0.2311)
\polypmIIId{64}(0.2329,0.231)(0.2442,0.231)(0.2442,0.233)(0.2329,0.233)
\polypmIIId{65}(0.2329,0.2329)(0.2442,0.2329)(0.2442,0.2349)(0.2329,0.2349)
\polypmIIId{66}(0.2329,0.2348)(0.2442,0.2348)(0.2442,0.2368)(0.2329,0.2368)
\polypmIIId{67}(0.2329,0.2367)(0.2442,0.2367)(0.2442,0.2388)(0.2329,0.2388)
\polypmIIId{68}(0.2329,0.2387)(0.2442,0.2387)(0.2442,0.2407)(0.2329,0.2407)
\polypmIIId{69}(0.2329,0.2406)(0.2442,0.2406)(0.2442,0.2426)(0.2329,0.2426)
\polypmIIId{70}(0.2329,0.2425)(0.2442,0.2425)(0.2442,0.2445)(0.2329,0.2445)
\polypmIIId{71}(0.2329,0.2444)(0.2442,0.2444)(0.2442,0.2465)(0.2329,0.2465)
\polypmIIId{72}(0.2329,0.2464)(0.2442,0.2464)(0.2442,0.2484)(0.2329,0.2484)
\polypmIIId{73}(0.2329,0.2483)(0.2442,0.2483)(0.2442,0.2503)(0.2329,0.2503)
\polypmIIId{74}(0.2329,0.2502)(0.2442,0.2502)(0.2442,0.2522)(0.2329,0.2522)
\polypmIIId{75}(0.2329,0.2521)(0.2442,0.2521)(0.2442,0.2542)(0.2329,0.2542)
\polypmIIId{76}(0.2329,0.2541)(0.2442,0.2541)(0.2442,0.2561)(0.2329,0.2561)
\polypmIIId{77}(0.2329,0.256)(0.2442,0.256)(0.2442,0.258)(0.2329,0.258)
\polypmIIId{78}(0.2329,0.2579)(0.2442,0.2579)(0.2442,0.2599)(0.2329,0.2599)
\polypmIIId{79}(0.2329,0.2598)(0.2442,0.2598)(0.2442,0.2619)(0.2329,0.2619)
\polypmIIId{80}(0.2329,0.2618)(0.2442,0.2618)(0.2442,0.2638)(0.2329,0.2638)
\polypmIIId{81}(0.2329,0.2637)(0.2442,0.2637)(0.2442,0.2657)(0.2329,0.2657)
\polypmIIId{82}(0.2329,0.2656)(0.2442,0.2656)(0.2442,0.2676)(0.2329,0.2676)
\polypmIIId{83}(0.2329,0.2675)(0.2442,0.2675)(0.2442,0.2696)(0.2329,0.2696)
\polypmIIId{84}(0.2329,0.2695)(0.2442,0.2695)(0.2442,0.2715)(0.2329,0.2715)
\polypmIIId{85}(0.2329,0.2714)(0.2442,0.2714)(0.2442,0.2734)(0.2329,0.2734)
\polypmIIId{86}(0.2329,0.2733)(0.2442,0.2733)(0.2442,0.2753)(0.2329,0.2753)
\polypmIIId{87}(0.2329,0.2752)(0.2442,0.2752)(0.2442,0.2773)(0.2329,0.2773)
\polypmIIId{88}(0.2329,0.2772)(0.2442,0.2772)(0.2442,0.2792)(0.2329,0.2792)
\polypmIIId{89}(0.2329,0.2791)(0.2442,0.2791)(0.2442,0.2811)(0.2329,0.2811)
\polypmIIId{90}(0.2329,0.281)(0.2442,0.281)(0.2442,0.283)(0.2329,0.283)
\polypmIIId{91}(0.2329,0.2829)(0.2442,0.2829)(0.2442,0.285)(0.2329,0.285)
\polypmIIId{92}(0.2329,0.2849)(0.2442,0.2849)(0.2442,0.2869)(0.2329,0.2869)
\polypmIIId{93}(0.2329,0.2868)(0.2442,0.2868)(0.2442,0.2888)(0.2329,0.2888)
\polypmIIId{94}(0.2329,0.2887)(0.2442,0.2887)(0.2442,0.2907)(0.2329,0.2907)
\polypmIIId{95}(0.2329,0.2906)(0.2442,0.2906)(0.2442,0.2927)(0.2329,0.2927)
\polypmIIId{96}(0.2329,0.2926)(0.2442,0.2926)(0.2442,0.2946)(0.2329,0.2946)
\polypmIIId{97}(0.2329,0.2945)(0.2442,0.2945)(0.2442,0.2965)(0.2329,0.2965)
\polypmIIId{98}(0.2329,0.2964)(0.2442,0.2964)(0.2442,0.2984)(0.2329,0.2984)
\polypmIIId{99}(0.2329,0.2983)(0.2442,0.2983)(0.2442,0.3004)(0.2329,0.3004)
\polypmIIId{100}(0.2329,0.3003)(0.2442,0.3003)(0.2442,0.3023)(0.2329,0.3023)
\polypmIIId{101}(0.2329,0.3022)(0.2442,0.3022)(0.2442,0.3042)(0.2329,0.3042)
\polypmIIId{102}(0.2329,0.3041)(0.2442,0.3041)(0.2442,0.3061)(0.2329,0.3061)
\polypmIIId{103}(0.2329,0.306)(0.2442,0.306)(0.2442,0.3081)(0.2329,0.3081)
\polypmIIId{104}(0.2329,0.308)(0.2442,0.308)(0.2442,0.31)(0.2329,0.31)
\polypmIIId{105}(0.2329,0.3099)(0.2442,0.3099)(0.2442,0.3119)(0.2329,0.3119)
\polypmIIId{106}(0.2329,0.3118)(0.2442,0.3118)(0.2442,0.3138)(0.2329,0.3138)
\polypmIIId{107}(0.2329,0.3137)(0.2442,0.3137)(0.2442,0.3158)(0.2329,0.3158)
\polypmIIId{108}(0.2329,0.3157)(0.2442,0.3157)(0.2442,0.3177)(0.2329,0.3177)
\polypmIIId{109}(0.2329,0.3176)(0.2442,0.3176)(0.2442,0.3196)(0.2329,0.3196)
\polypmIIId{110}(0.2329,0.3195)(0.2442,0.3195)(0.2442,0.3215)(0.2329,0.3215)
\polypmIIId{111}(0.2329,0.3214)(0.2442,0.3214)(0.2442,0.3235)(0.2329,0.3235)
\polypmIIId{112}(0.2329,0.3234)(0.2442,0.3234)(0.2442,0.3254)(0.2329,0.3254)
\polypmIIId{113}(0.2329,0.3253)(0.2442,0.3253)(0.2442,0.3273)(0.2329,0.3273)
\polypmIIId{114}(0.2329,0.3272)(0.2442,0.3272)(0.2442,0.3292)(0.2329,0.3292)
\polypmIIId{115}(0.2329,0.3291)(0.2442,0.3291)(0.2442,0.3312)(0.2329,0.3312)
\polypmIIId{116}(0.2329,0.3311)(0.2442,0.3311)(0.2442,0.3331)(0.2329,0.3331)
\polypmIIId{117}(0.2329,0.333)(0.2442,0.333)(0.2442,0.335)(0.2329,0.335)
\polypmIIId{118}(0.2329,0.3349)(0.2442,0.3349)(0.2442,0.3369)(0.2329,0.3369)
\polypmIIId{119}(0.2329,0.3368)(0.2442,0.3368)(0.2442,0.3389)(0.2329,0.3389)
\polypmIIId{120}(0.2329,0.3388)(0.2442,0.3388)(0.2442,0.3408)(0.2329,0.3408)
\polypmIIId{121}(0.2329,0.3407)(0.2442,0.3407)(0.2442,0.3427)(0.2329,0.3427)
\polypmIIId{122}(0.2329,0.3426)(0.2442,0.3426)(0.2442,0.3446)(0.2329,0.3446)
\polypmIIId{123}(0.2329,0.3445)(0.2442,0.3445)(0.2442,0.3466)(0.2329,0.3466)
\polypmIIId{124}(0.2329,0.3465)(0.2442,0.3465)(0.2442,0.3485)(0.2329,0.3485)
\polypmIIId{125}(0.2329,0.3484)(0.2442,0.3484)(0.2442,0.3504)(0.2329,0.3504)
\polypmIIId{126}(0.2329,0.3503)(0.2442,0.3503)(0.2442,0.3523)(0.2329,0.3523)
\polypmIIId{127}(0.2329,0.3522)(0.2442,0.3522)(0.2442,0.3542)(0.2329,0.3542)

\PST@Border(0.2329,0.1078)
(0.2442,0.1078)
(0.2442,0.3542)
(0.2329,0.3542)
(0.2329,0.1078)


\rput[l](0.2502,0.1301){0.997}
\rput[l](0.2502,0.2048){0.998}
\rput[l](0.2502,0.2795){0.999}
\rput[l](0.2502,0.3542){1}

\catcode`@=12
\fi
\endpspicture}
  \caption{TSLP solution quality on uncorrelated instances ($\alpha = 1.0$).}
  \label{fig:tabusolcomp10}
\end{figure}

Looking at figures~\ref{fig:tabusolcomp00}, \ref{fig:tabusolcomp01} and \ref{fig:tabusolcomp10}, which show the results of the TSLP tests, some impact of
the correlation level on the quality of the obtained solutions can again be seen, as the heatmaps for the instances with some level of correlation
are paler than the ones for the uncorrelated instances. Specially on the instances with weak correlation (figure~\ref{fig:tabusolcomp01}), the algorithm seems
to obtain the worst solutions.

Comparing the results in respect to the number of resources, it can be seen that instances with more resources are harder to solve. That influence
can be seen mainly on figure~\ref{fig:tabusolcomp01}. Once again, an observation of the quality of the solutions related to the amount of actions on the instance
shows that the TSLP also obtained worse solutions on instances with fewer actions. The influence of the number of years in those tests was too weak to be 
considered significant.

Figures~\ref{fig:greedysolcomp00}, \ref{fig:greedysolcomp01} and \ref{fig:greedysolcomp10} present the results of the GALP heuristic.
Once again, it is possible to see a reduction on the quality of the solutions found when the quantity of resources or level of correlation on
the instances are increased. Besides, increasing the number of actions on the instances also enabled GALP to find better solutions, and
the influence of the number of years was also not significant, as observed on the TSLP tests.

\begin{figure}[H]
  \centering
    \subfloat[1 resource]{% GNUPLOT: LaTeX picture using PSTRICKS macros
% Define new PST objects, if not already defined
\ifx\PSTloaded\undefined
\def\PSTloaded{t}

\catcode`@=11

\newpsobject{PST@Border}{psline}{linewidth=.0015,linestyle=solid}

\catcode`@=12

\fi
\psset{unit=5.0in,xunit=5.0in,yunit=3.0in}
\pspicture(0.000000,0.000000)(0.31, 0.35)
\ifx\nofigs\undefined
\catcode`@=11

\newrgbcolor{PST@COLOR0}{1 1 1}
\newrgbcolor{PST@COLOR1}{0.992 0.992 0.992}
\newrgbcolor{PST@COLOR2}{0.984 0.984 0.984}
\newrgbcolor{PST@COLOR3}{0.976 0.976 0.976}
\newrgbcolor{PST@COLOR4}{0.968 0.968 0.968}
\newrgbcolor{PST@COLOR5}{0.96 0.96 0.96}
\newrgbcolor{PST@COLOR6}{0.952 0.952 0.952}
\newrgbcolor{PST@COLOR7}{0.944 0.944 0.944}
\newrgbcolor{PST@COLOR8}{0.937 0.937 0.937}
\newrgbcolor{PST@COLOR9}{0.929 0.929 0.929}
\newrgbcolor{PST@COLOR10}{0.921 0.921 0.921}
\newrgbcolor{PST@COLOR11}{0.913 0.913 0.913}
\newrgbcolor{PST@COLOR12}{0.905 0.905 0.905}
\newrgbcolor{PST@COLOR13}{0.897 0.897 0.897}
\newrgbcolor{PST@COLOR14}{0.889 0.889 0.889}
\newrgbcolor{PST@COLOR15}{0.881 0.881 0.881}
\newrgbcolor{PST@COLOR16}{0.874 0.874 0.874}
\newrgbcolor{PST@COLOR17}{0.866 0.866 0.866}
\newrgbcolor{PST@COLOR18}{0.858 0.858 0.858}
\newrgbcolor{PST@COLOR19}{0.85 0.85 0.85}
\newrgbcolor{PST@COLOR20}{0.842 0.842 0.842}
\newrgbcolor{PST@COLOR21}{0.834 0.834 0.834}
\newrgbcolor{PST@COLOR22}{0.826 0.826 0.826}
\newrgbcolor{PST@COLOR23}{0.818 0.818 0.818}
\newrgbcolor{PST@COLOR24}{0.811 0.811 0.811}
\newrgbcolor{PST@COLOR25}{0.803 0.803 0.803}
\newrgbcolor{PST@COLOR26}{0.795 0.795 0.795}
\newrgbcolor{PST@COLOR27}{0.787 0.787 0.787}
\newrgbcolor{PST@COLOR28}{0.779 0.779 0.779}
\newrgbcolor{PST@COLOR29}{0.771 0.771 0.771}
\newrgbcolor{PST@COLOR30}{0.763 0.763 0.763}
\newrgbcolor{PST@COLOR31}{0.755 0.755 0.755}
\newrgbcolor{PST@COLOR32}{0.748 0.748 0.748}
\newrgbcolor{PST@COLOR33}{0.74 0.74 0.74}
\newrgbcolor{PST@COLOR34}{0.732 0.732 0.732}
\newrgbcolor{PST@COLOR35}{0.724 0.724 0.724}
\newrgbcolor{PST@COLOR36}{0.716 0.716 0.716}
\newrgbcolor{PST@COLOR37}{0.708 0.708 0.708}
\newrgbcolor{PST@COLOR38}{0.7 0.7 0.7}
\newrgbcolor{PST@COLOR39}{0.692 0.692 0.692}
\newrgbcolor{PST@COLOR40}{0.685 0.685 0.685}
\newrgbcolor{PST@COLOR41}{0.677 0.677 0.677}
\newrgbcolor{PST@COLOR42}{0.669 0.669 0.669}
\newrgbcolor{PST@COLOR43}{0.661 0.661 0.661}
\newrgbcolor{PST@COLOR44}{0.653 0.653 0.653}
\newrgbcolor{PST@COLOR45}{0.645 0.645 0.645}
\newrgbcolor{PST@COLOR46}{0.637 0.637 0.637}
\newrgbcolor{PST@COLOR47}{0.629 0.629 0.629}
\newrgbcolor{PST@COLOR48}{0.622 0.622 0.622}
\newrgbcolor{PST@COLOR49}{0.614 0.614 0.614}
\newrgbcolor{PST@COLOR50}{0.606 0.606 0.606}
\newrgbcolor{PST@COLOR51}{0.598 0.598 0.598}
\newrgbcolor{PST@COLOR52}{0.59 0.59 0.59}
\newrgbcolor{PST@COLOR53}{0.582 0.582 0.582}
\newrgbcolor{PST@COLOR54}{0.574 0.574 0.574}
\newrgbcolor{PST@COLOR55}{0.566 0.566 0.566}
\newrgbcolor{PST@COLOR56}{0.559 0.559 0.559}
\newrgbcolor{PST@COLOR57}{0.551 0.551 0.551}
\newrgbcolor{PST@COLOR58}{0.543 0.543 0.543}
\newrgbcolor{PST@COLOR59}{0.535 0.535 0.535}
\newrgbcolor{PST@COLOR60}{0.527 0.527 0.527}
\newrgbcolor{PST@COLOR61}{0.519 0.519 0.519}
\newrgbcolor{PST@COLOR62}{0.511 0.511 0.511}
\newrgbcolor{PST@COLOR63}{0.503 0.503 0.503}
\newrgbcolor{PST@COLOR64}{0.496 0.496 0.496}
\newrgbcolor{PST@COLOR65}{0.488 0.488 0.488}
\newrgbcolor{PST@COLOR66}{0.48 0.48 0.48}
\newrgbcolor{PST@COLOR67}{0.472 0.472 0.472}
\newrgbcolor{PST@COLOR68}{0.464 0.464 0.464}
\newrgbcolor{PST@COLOR69}{0.456 0.456 0.456}
\newrgbcolor{PST@COLOR70}{0.448 0.448 0.448}
\newrgbcolor{PST@COLOR71}{0.44 0.44 0.44}
\newrgbcolor{PST@COLOR72}{0.433 0.433 0.433}
\newrgbcolor{PST@COLOR73}{0.425 0.425 0.425}
\newrgbcolor{PST@COLOR74}{0.417 0.417 0.417}
\newrgbcolor{PST@COLOR75}{0.409 0.409 0.409}
\newrgbcolor{PST@COLOR76}{0.401 0.401 0.401}
\newrgbcolor{PST@COLOR77}{0.393 0.393 0.393}
\newrgbcolor{PST@COLOR78}{0.385 0.385 0.385}
\newrgbcolor{PST@COLOR79}{0.377 0.377 0.377}
\newrgbcolor{PST@COLOR80}{0.37 0.37 0.37}
\newrgbcolor{PST@COLOR81}{0.362 0.362 0.362}
\newrgbcolor{PST@COLOR82}{0.354 0.354 0.354}
\newrgbcolor{PST@COLOR83}{0.346 0.346 0.346}
\newrgbcolor{PST@COLOR84}{0.338 0.338 0.338}
\newrgbcolor{PST@COLOR85}{0.33 0.33 0.33}
\newrgbcolor{PST@COLOR86}{0.322 0.322 0.322}
\newrgbcolor{PST@COLOR87}{0.314 0.314 0.314}
\newrgbcolor{PST@COLOR88}{0.307 0.307 0.307}
\newrgbcolor{PST@COLOR89}{0.299 0.299 0.299}
\newrgbcolor{PST@COLOR90}{0.291 0.291 0.291}
\newrgbcolor{PST@COLOR91}{0.283 0.283 0.283}
\newrgbcolor{PST@COLOR92}{0.275 0.275 0.275}
\newrgbcolor{PST@COLOR93}{0.267 0.267 0.267}
\newrgbcolor{PST@COLOR94}{0.259 0.259 0.259}
\newrgbcolor{PST@COLOR95}{0.251 0.251 0.251}
\newrgbcolor{PST@COLOR96}{0.244 0.244 0.244}
\newrgbcolor{PST@COLOR97}{0.236 0.236 0.236}
\newrgbcolor{PST@COLOR98}{0.228 0.228 0.228}
\newrgbcolor{PST@COLOR99}{0.22 0.22 0.22}
\newrgbcolor{PST@COLOR100}{0.212 0.212 0.212}
\newrgbcolor{PST@COLOR101}{0.204 0.204 0.204}
\newrgbcolor{PST@COLOR102}{0.196 0.196 0.196}
\newrgbcolor{PST@COLOR103}{0.188 0.188 0.188}
\newrgbcolor{PST@COLOR104}{0.181 0.181 0.181}
\newrgbcolor{PST@COLOR105}{0.173 0.173 0.173}
\newrgbcolor{PST@COLOR106}{0.165 0.165 0.165}
\newrgbcolor{PST@COLOR107}{0.157 0.157 0.157}
\newrgbcolor{PST@COLOR108}{0.149 0.149 0.149}
\newrgbcolor{PST@COLOR109}{0.141 0.141 0.141}
\newrgbcolor{PST@COLOR110}{0.133 0.133 0.133}
\newrgbcolor{PST@COLOR111}{0.125 0.125 0.125}
\newrgbcolor{PST@COLOR112}{0.118 0.118 0.118}
\newrgbcolor{PST@COLOR113}{0.11 0.11 0.11}
\newrgbcolor{PST@COLOR114}{0.102 0.102 0.102}
\newrgbcolor{PST@COLOR115}{0.094 0.094 0.094}
\newrgbcolor{PST@COLOR116}{0.086 0.086 0.086}
\newrgbcolor{PST@COLOR117}{0.078 0.078 0.078}
\newrgbcolor{PST@COLOR118}{0.07 0.07 0.07}
\newrgbcolor{PST@COLOR119}{0.062 0.062 0.062}
\newrgbcolor{PST@COLOR120}{0.055 0.055 0.055}
\newrgbcolor{PST@COLOR121}{0.047 0.047 0.047}
\newrgbcolor{PST@COLOR122}{0.039 0.039 0.039}
\newrgbcolor{PST@COLOR123}{0.031 0.031 0.031}
\newrgbcolor{PST@COLOR124}{0.023 0.023 0.023}
\newrgbcolor{PST@COLOR125}{0.015 0.015 0.015}
\newrgbcolor{PST@COLOR126}{0.007 0.007 0.007}
\newrgbcolor{PST@COLOR127}{0 0 0}


\def\polypmIIId#1{\pspolygon[linestyle=none,fillstyle=solid,fillcolor=PST@COLOR#1]}

\polypmIIId{63}(0.1432,0.19)(0.0864,0.19)(0.0864,0.1078)(0.1432,0.1078)
\polypmIIId{97}(0.1432,0.272)(0.0864,0.272)(0.0864,0.19)(0.1432,0.19)
\polypmIIId{114}(0.1432,0.3542)(0.0864,0.3542)(0.0864,0.272)(0.1432,0.272)

\polypmIIId{68}(0.2,0.19)(0.1432,0.19)(0.1432,0.1078)(0.2,0.1078)
\polypmIIId{97}(0.2,0.272)(0.1432,0.272)(0.1432,0.19)(0.2,0.19)
\polypmIIId{115}(0.2,0.3542)(0.1432,0.3542)(0.1432,0.272)(0.2,0.272)

\polypmIIId{69}(0.2568,0.19)(0.2,0.19)(0.2,0.1078)(0.2568,0.1078)
\polypmIIId{99}(0.2568,0.272)(0.2,0.272)(0.2,0.19)(0.2568,0.19)
\polypmIIId{116}(0.2568,0.3542)(0.2,0.3542)(0.2,0.272)(0.2568,0.272)

\polypmIIId{65}(0.3136,0.19)(0.2568,0.19)(0.2568,0.1078)(0.3136,0.1078)
\polypmIIId{102}(0.3136,0.272)(0.2568,0.272)(0.2568,0.19)(0.3136,0.19)
\polypmIIId{117}(0.3136,0.3542)(0.2568,0.3542)(0.2568,0.272)(0.3136,0.272)

\rput(0.1148,0.07){3}
\rput(0.1716,0.07){4}
\rput(0.2284,0.07){5}
\rput(0.2852,0.07){6}
\rput(0.2000,0.0070){years}

\rput[r](0.0806,0.1489){25}
\rput[r](0.0806,0.2310){50}
\rput[r](0.0806,0.3131){100}
\rput{L}(0.0096,0.2310){actions}

\PST@Border(0.0864,0.3542)
(0.0864,0.1078)
(0.3136,0.1078)
(0.3136,0.3542)
(0.0864,0.3542)

\catcode`@=12
\fi
\endpspicture} 
    \subfloat[2 resources]{% GNUPLOT: LaTeX picture using PSTRICKS macros
% Define new PST objects, if not already defined
\ifx\PSTloaded\undefined
\def\PSTloaded{t}

\catcode`@=11

\newpsobject{PST@Border}{psline}{linewidth=.0015,linestyle=solid}

\catcode`@=12

\fi
\psset{unit=5.0in,xunit=5.0in,yunit=3.0in}
\pspicture(0.000000,0.000000)(0.225000,0.35)
\ifx\nofigs\undefined
\catcode`@=11

\newrgbcolor{PST@COLOR0}{1 1 1}
\newrgbcolor{PST@COLOR1}{0.992 0.992 0.992}
\newrgbcolor{PST@COLOR2}{0.984 0.984 0.984}
\newrgbcolor{PST@COLOR3}{0.976 0.976 0.976}
\newrgbcolor{PST@COLOR4}{0.968 0.968 0.968}
\newrgbcolor{PST@COLOR5}{0.96 0.96 0.96}
\newrgbcolor{PST@COLOR6}{0.952 0.952 0.952}
\newrgbcolor{PST@COLOR7}{0.944 0.944 0.944}
\newrgbcolor{PST@COLOR8}{0.937 0.937 0.937}
\newrgbcolor{PST@COLOR9}{0.929 0.929 0.929}
\newrgbcolor{PST@COLOR10}{0.921 0.921 0.921}
\newrgbcolor{PST@COLOR11}{0.913 0.913 0.913}
\newrgbcolor{PST@COLOR12}{0.905 0.905 0.905}
\newrgbcolor{PST@COLOR13}{0.897 0.897 0.897}
\newrgbcolor{PST@COLOR14}{0.889 0.889 0.889}
\newrgbcolor{PST@COLOR15}{0.881 0.881 0.881}
\newrgbcolor{PST@COLOR16}{0.874 0.874 0.874}
\newrgbcolor{PST@COLOR17}{0.866 0.866 0.866}
\newrgbcolor{PST@COLOR18}{0.858 0.858 0.858}
\newrgbcolor{PST@COLOR19}{0.85 0.85 0.85}
\newrgbcolor{PST@COLOR20}{0.842 0.842 0.842}
\newrgbcolor{PST@COLOR21}{0.834 0.834 0.834}
\newrgbcolor{PST@COLOR22}{0.826 0.826 0.826}
\newrgbcolor{PST@COLOR23}{0.818 0.818 0.818}
\newrgbcolor{PST@COLOR24}{0.811 0.811 0.811}
\newrgbcolor{PST@COLOR25}{0.803 0.803 0.803}
\newrgbcolor{PST@COLOR26}{0.795 0.795 0.795}
\newrgbcolor{PST@COLOR27}{0.787 0.787 0.787}
\newrgbcolor{PST@COLOR28}{0.779 0.779 0.779}
\newrgbcolor{PST@COLOR29}{0.771 0.771 0.771}
\newrgbcolor{PST@COLOR30}{0.763 0.763 0.763}
\newrgbcolor{PST@COLOR31}{0.755 0.755 0.755}
\newrgbcolor{PST@COLOR32}{0.748 0.748 0.748}
\newrgbcolor{PST@COLOR33}{0.74 0.74 0.74}
\newrgbcolor{PST@COLOR34}{0.732 0.732 0.732}
\newrgbcolor{PST@COLOR35}{0.724 0.724 0.724}
\newrgbcolor{PST@COLOR36}{0.716 0.716 0.716}
\newrgbcolor{PST@COLOR37}{0.708 0.708 0.708}
\newrgbcolor{PST@COLOR38}{0.7 0.7 0.7}
\newrgbcolor{PST@COLOR39}{0.692 0.692 0.692}
\newrgbcolor{PST@COLOR40}{0.685 0.685 0.685}
\newrgbcolor{PST@COLOR41}{0.677 0.677 0.677}
\newrgbcolor{PST@COLOR42}{0.669 0.669 0.669}
\newrgbcolor{PST@COLOR43}{0.661 0.661 0.661}
\newrgbcolor{PST@COLOR44}{0.653 0.653 0.653}
\newrgbcolor{PST@COLOR45}{0.645 0.645 0.645}
\newrgbcolor{PST@COLOR46}{0.637 0.637 0.637}
\newrgbcolor{PST@COLOR47}{0.629 0.629 0.629}
\newrgbcolor{PST@COLOR48}{0.622 0.622 0.622}
\newrgbcolor{PST@COLOR49}{0.614 0.614 0.614}
\newrgbcolor{PST@COLOR50}{0.606 0.606 0.606}
\newrgbcolor{PST@COLOR51}{0.598 0.598 0.598}
\newrgbcolor{PST@COLOR52}{0.59 0.59 0.59}
\newrgbcolor{PST@COLOR53}{0.582 0.582 0.582}
\newrgbcolor{PST@COLOR54}{0.574 0.574 0.574}
\newrgbcolor{PST@COLOR55}{0.566 0.566 0.566}
\newrgbcolor{PST@COLOR56}{0.559 0.559 0.559}
\newrgbcolor{PST@COLOR57}{0.551 0.551 0.551}
\newrgbcolor{PST@COLOR58}{0.543 0.543 0.543}
\newrgbcolor{PST@COLOR59}{0.535 0.535 0.535}
\newrgbcolor{PST@COLOR60}{0.527 0.527 0.527}
\newrgbcolor{PST@COLOR61}{0.519 0.519 0.519}
\newrgbcolor{PST@COLOR62}{0.511 0.511 0.511}
\newrgbcolor{PST@COLOR63}{0.503 0.503 0.503}
\newrgbcolor{PST@COLOR64}{0.496 0.496 0.496}
\newrgbcolor{PST@COLOR65}{0.488 0.488 0.488}
\newrgbcolor{PST@COLOR66}{0.48 0.48 0.48}
\newrgbcolor{PST@COLOR67}{0.472 0.472 0.472}
\newrgbcolor{PST@COLOR68}{0.464 0.464 0.464}
\newrgbcolor{PST@COLOR69}{0.456 0.456 0.456}
\newrgbcolor{PST@COLOR70}{0.448 0.448 0.448}
\newrgbcolor{PST@COLOR71}{0.44 0.44 0.44}
\newrgbcolor{PST@COLOR72}{0.433 0.433 0.433}
\newrgbcolor{PST@COLOR73}{0.425 0.425 0.425}
\newrgbcolor{PST@COLOR74}{0.417 0.417 0.417}
\newrgbcolor{PST@COLOR75}{0.409 0.409 0.409}
\newrgbcolor{PST@COLOR76}{0.401 0.401 0.401}
\newrgbcolor{PST@COLOR77}{0.393 0.393 0.393}
\newrgbcolor{PST@COLOR78}{0.385 0.385 0.385}
\newrgbcolor{PST@COLOR79}{0.377 0.377 0.377}
\newrgbcolor{PST@COLOR80}{0.37 0.37 0.37}
\newrgbcolor{PST@COLOR81}{0.362 0.362 0.362}
\newrgbcolor{PST@COLOR82}{0.354 0.354 0.354}
\newrgbcolor{PST@COLOR83}{0.346 0.346 0.346}
\newrgbcolor{PST@COLOR84}{0.338 0.338 0.338}
\newrgbcolor{PST@COLOR85}{0.33 0.33 0.33}
\newrgbcolor{PST@COLOR86}{0.322 0.322 0.322}
\newrgbcolor{PST@COLOR87}{0.314 0.314 0.314}
\newrgbcolor{PST@COLOR88}{0.307 0.307 0.307}
\newrgbcolor{PST@COLOR89}{0.299 0.299 0.299}
\newrgbcolor{PST@COLOR90}{0.291 0.291 0.291}
\newrgbcolor{PST@COLOR91}{0.283 0.283 0.283}
\newrgbcolor{PST@COLOR92}{0.275 0.275 0.275}
\newrgbcolor{PST@COLOR93}{0.267 0.267 0.267}
\newrgbcolor{PST@COLOR94}{0.259 0.259 0.259}
\newrgbcolor{PST@COLOR95}{0.251 0.251 0.251}
\newrgbcolor{PST@COLOR96}{0.244 0.244 0.244}
\newrgbcolor{PST@COLOR97}{0.236 0.236 0.236}
\newrgbcolor{PST@COLOR98}{0.228 0.228 0.228}
\newrgbcolor{PST@COLOR99}{0.22 0.22 0.22}
\newrgbcolor{PST@COLOR100}{0.212 0.212 0.212}
\newrgbcolor{PST@COLOR101}{0.204 0.204 0.204}
\newrgbcolor{PST@COLOR102}{0.196 0.196 0.196}
\newrgbcolor{PST@COLOR103}{0.188 0.188 0.188}
\newrgbcolor{PST@COLOR104}{0.181 0.181 0.181}
\newrgbcolor{PST@COLOR105}{0.173 0.173 0.173}
\newrgbcolor{PST@COLOR106}{0.165 0.165 0.165}
\newrgbcolor{PST@COLOR107}{0.157 0.157 0.157}
\newrgbcolor{PST@COLOR108}{0.149 0.149 0.149}
\newrgbcolor{PST@COLOR109}{0.141 0.141 0.141}
\newrgbcolor{PST@COLOR110}{0.133 0.133 0.133}
\newrgbcolor{PST@COLOR111}{0.125 0.125 0.125}
\newrgbcolor{PST@COLOR112}{0.118 0.118 0.118}
\newrgbcolor{PST@COLOR113}{0.11 0.11 0.11}
\newrgbcolor{PST@COLOR114}{0.102 0.102 0.102}
\newrgbcolor{PST@COLOR115}{0.094 0.094 0.094}
\newrgbcolor{PST@COLOR116}{0.086 0.086 0.086}
\newrgbcolor{PST@COLOR117}{0.078 0.078 0.078}
\newrgbcolor{PST@COLOR118}{0.07 0.07 0.07}
\newrgbcolor{PST@COLOR119}{0.062 0.062 0.062}
\newrgbcolor{PST@COLOR120}{0.055 0.055 0.055}
\newrgbcolor{PST@COLOR121}{0.047 0.047 0.047}
\newrgbcolor{PST@COLOR122}{0.039 0.039 0.039}
\newrgbcolor{PST@COLOR123}{0.031 0.031 0.031}
\newrgbcolor{PST@COLOR124}{0.023 0.023 0.023}
\newrgbcolor{PST@COLOR125}{0.015 0.015 0.015}
\newrgbcolor{PST@COLOR126}{0.007 0.007 0.007}
\newrgbcolor{PST@COLOR127}{0 0 0}

\def\polypmIIId#1{\pspolygon[linestyle=none,fillstyle=solid,fillcolor=PST@COLOR#1]}

\polypmIIId{48} (0.0568,0.19)  (0.0,0.19)  (0.0,0.1078)(0.0568,0.1078)
\polypmIIId{86}  (0.0568,0.272) (0.0,0.272) (0.0,0.19)  (0.0568,0.19)
\polypmIIId{111}  (0.0568,0.3542)(0.0,0.3542)(0.0,0.272) (0.0568,0.272)

\polypmIIId{53} (0.1136,   0.19)  (0.0568,0.19)  (0.0568,0.1078)(0.1136,0.1078)
\polypmIIId{93}  (0.1136,   0.272) (0.0568,0.272) (0.0568,0.19)  (0.1136,0.19)
\polypmIIId{111}  (0.1136,   0.3542)(0.0568,0.3542)(0.0568,0.272) (0.1136,0.272)

\polypmIIId{52}(0.1704,0.19)  (0.1136,   0.19)  (0.1136,   0.1078)(0.1704,0.1078)
\polypmIIId{92} (0.1704,0.272) (0.1136,   0.272) (0.1136,   0.19)  (0.1704,0.19)
\polypmIIId{112}  (0.1704,0.3542)(0.1136,   0.3542)(0.1136,   0.272) (0.1704,0.272)

\polypmIIId{50}(0.2272,0.19)  (0.1704,0.19)  (0.1704,0.1078)(0.2272,0.1078)
\polypmIIId{95}  (0.2272,0.272) (0.1704,0.272) (0.1704,0.19)  (0.2272,0.19)
\polypmIIId{112}  (0.2272,0.3542)(0.1704,0.3542)(0.1704,0.272) (0.2272,0.272)

\rput(0.0284,0.07){3}
\rput(0.0852,0.07){4}
\rput(0.1420,0.07){5}
\rput(0.1988,0.07){6}
\rput(0.1136,0.0070){years}


\PST@Border(0.0,0.3542)
(0.0,0.1078)
(0.2272,0.1078)
(0.2272,0.3542)
(0.0,0.3542)

\catcode`@=12
\fi
\endpspicture}
    \subfloat[4 resources]{% GNUPLOT: LaTeX picture using PSTRICKS macros
% Define new PST objects, if not already defined
\ifx\PSTloaded\undefined
\def\PSTloaded{t}

\catcode`@=11

\newpsobject{PST@Border}{psline}{linewidth=.0015,linestyle=solid}

\catcode`@=12

\fi
\psset{unit=5.0in,xunit=5.0in,yunit=3.0in}
\pspicture(0.000000,0.000000)(0.3136,0.35)
\ifx\nofigs\undefined
\catcode`@=11

\newrgbcolor{PST@COLOR0}{1 1 1}
\newrgbcolor{PST@COLOR1}{0.992 0.992 0.992}
\newrgbcolor{PST@COLOR2}{0.984 0.984 0.984}
\newrgbcolor{PST@COLOR3}{0.976 0.976 0.976}
\newrgbcolor{PST@COLOR4}{0.968 0.968 0.968}
\newrgbcolor{PST@COLOR5}{0.96 0.96 0.96}
\newrgbcolor{PST@COLOR6}{0.952 0.952 0.952}
\newrgbcolor{PST@COLOR7}{0.944 0.944 0.944}
\newrgbcolor{PST@COLOR8}{0.937 0.937 0.937}
\newrgbcolor{PST@COLOR9}{0.929 0.929 0.929}
\newrgbcolor{PST@COLOR10}{0.921 0.921 0.921}
\newrgbcolor{PST@COLOR11}{0.913 0.913 0.913}
\newrgbcolor{PST@COLOR12}{0.905 0.905 0.905}
\newrgbcolor{PST@COLOR13}{0.897 0.897 0.897}
\newrgbcolor{PST@COLOR14}{0.889 0.889 0.889}
\newrgbcolor{PST@COLOR15}{0.881 0.881 0.881}
\newrgbcolor{PST@COLOR16}{0.874 0.874 0.874}
\newrgbcolor{PST@COLOR17}{0.866 0.866 0.866}
\newrgbcolor{PST@COLOR18}{0.858 0.858 0.858}
\newrgbcolor{PST@COLOR19}{0.85 0.85 0.85}
\newrgbcolor{PST@COLOR20}{0.842 0.842 0.842}
\newrgbcolor{PST@COLOR21}{0.834 0.834 0.834}
\newrgbcolor{PST@COLOR22}{0.826 0.826 0.826}
\newrgbcolor{PST@COLOR23}{0.818 0.818 0.818}
\newrgbcolor{PST@COLOR24}{0.811 0.811 0.811}
\newrgbcolor{PST@COLOR25}{0.803 0.803 0.803}
\newrgbcolor{PST@COLOR26}{0.795 0.795 0.795}
\newrgbcolor{PST@COLOR27}{0.787 0.787 0.787}
\newrgbcolor{PST@COLOR28}{0.779 0.779 0.779}
\newrgbcolor{PST@COLOR29}{0.771 0.771 0.771}
\newrgbcolor{PST@COLOR30}{0.763 0.763 0.763}
\newrgbcolor{PST@COLOR31}{0.755 0.755 0.755}
\newrgbcolor{PST@COLOR32}{0.748 0.748 0.748}
\newrgbcolor{PST@COLOR33}{0.74 0.74 0.74}
\newrgbcolor{PST@COLOR34}{0.732 0.732 0.732}
\newrgbcolor{PST@COLOR35}{0.724 0.724 0.724}
\newrgbcolor{PST@COLOR36}{0.716 0.716 0.716}
\newrgbcolor{PST@COLOR37}{0.708 0.708 0.708}
\newrgbcolor{PST@COLOR38}{0.7 0.7 0.7}
\newrgbcolor{PST@COLOR39}{0.692 0.692 0.692}
\newrgbcolor{PST@COLOR40}{0.685 0.685 0.685}
\newrgbcolor{PST@COLOR41}{0.677 0.677 0.677}
\newrgbcolor{PST@COLOR42}{0.669 0.669 0.669}
\newrgbcolor{PST@COLOR43}{0.661 0.661 0.661}
\newrgbcolor{PST@COLOR44}{0.653 0.653 0.653}
\newrgbcolor{PST@COLOR45}{0.645 0.645 0.645}
\newrgbcolor{PST@COLOR46}{0.637 0.637 0.637}
\newrgbcolor{PST@COLOR47}{0.629 0.629 0.629}
\newrgbcolor{PST@COLOR48}{0.622 0.622 0.622}
\newrgbcolor{PST@COLOR49}{0.614 0.614 0.614}
\newrgbcolor{PST@COLOR50}{0.606 0.606 0.606}
\newrgbcolor{PST@COLOR51}{0.598 0.598 0.598}
\newrgbcolor{PST@COLOR52}{0.59 0.59 0.59}
\newrgbcolor{PST@COLOR53}{0.582 0.582 0.582}
\newrgbcolor{PST@COLOR54}{0.574 0.574 0.574}
\newrgbcolor{PST@COLOR55}{0.566 0.566 0.566}
\newrgbcolor{PST@COLOR56}{0.559 0.559 0.559}
\newrgbcolor{PST@COLOR57}{0.551 0.551 0.551}
\newrgbcolor{PST@COLOR58}{0.543 0.543 0.543}
\newrgbcolor{PST@COLOR59}{0.535 0.535 0.535}
\newrgbcolor{PST@COLOR60}{0.527 0.527 0.527}
\newrgbcolor{PST@COLOR61}{0.519 0.519 0.519}
\newrgbcolor{PST@COLOR62}{0.511 0.511 0.511}
\newrgbcolor{PST@COLOR63}{0.503 0.503 0.503}
\newrgbcolor{PST@COLOR64}{0.496 0.496 0.496}
\newrgbcolor{PST@COLOR65}{0.488 0.488 0.488}
\newrgbcolor{PST@COLOR66}{0.48 0.48 0.48}
\newrgbcolor{PST@COLOR67}{0.472 0.472 0.472}
\newrgbcolor{PST@COLOR68}{0.464 0.464 0.464}
\newrgbcolor{PST@COLOR69}{0.456 0.456 0.456}
\newrgbcolor{PST@COLOR70}{0.448 0.448 0.448}
\newrgbcolor{PST@COLOR71}{0.44 0.44 0.44}
\newrgbcolor{PST@COLOR72}{0.433 0.433 0.433}
\newrgbcolor{PST@COLOR73}{0.425 0.425 0.425}
\newrgbcolor{PST@COLOR74}{0.417 0.417 0.417}
\newrgbcolor{PST@COLOR75}{0.409 0.409 0.409}
\newrgbcolor{PST@COLOR76}{0.401 0.401 0.401}
\newrgbcolor{PST@COLOR77}{0.393 0.393 0.393}
\newrgbcolor{PST@COLOR78}{0.385 0.385 0.385}
\newrgbcolor{PST@COLOR79}{0.377 0.377 0.377}
\newrgbcolor{PST@COLOR80}{0.37 0.37 0.37}
\newrgbcolor{PST@COLOR81}{0.362 0.362 0.362}
\newrgbcolor{PST@COLOR82}{0.354 0.354 0.354}
\newrgbcolor{PST@COLOR83}{0.346 0.346 0.346}
\newrgbcolor{PST@COLOR84}{0.338 0.338 0.338}
\newrgbcolor{PST@COLOR85}{0.33 0.33 0.33}
\newrgbcolor{PST@COLOR86}{0.322 0.322 0.322}
\newrgbcolor{PST@COLOR87}{0.314 0.314 0.314}
\newrgbcolor{PST@COLOR88}{0.307 0.307 0.307}
\newrgbcolor{PST@COLOR89}{0.299 0.299 0.299}
\newrgbcolor{PST@COLOR90}{0.291 0.291 0.291}
\newrgbcolor{PST@COLOR91}{0.283 0.283 0.283}
\newrgbcolor{PST@COLOR92}{0.275 0.275 0.275}
\newrgbcolor{PST@COLOR93}{0.267 0.267 0.267}
\newrgbcolor{PST@COLOR94}{0.259 0.259 0.259}
\newrgbcolor{PST@COLOR95}{0.251 0.251 0.251}
\newrgbcolor{PST@COLOR96}{0.244 0.244 0.244}
\newrgbcolor{PST@COLOR97}{0.236 0.236 0.236}
\newrgbcolor{PST@COLOR98}{0.228 0.228 0.228}
\newrgbcolor{PST@COLOR99}{0.22 0.22 0.22}
\newrgbcolor{PST@COLOR100}{0.212 0.212 0.212}
\newrgbcolor{PST@COLOR101}{0.204 0.204 0.204}
\newrgbcolor{PST@COLOR102}{0.196 0.196 0.196}
\newrgbcolor{PST@COLOR103}{0.188 0.188 0.188}
\newrgbcolor{PST@COLOR104}{0.181 0.181 0.181}
\newrgbcolor{PST@COLOR105}{0.173 0.173 0.173}
\newrgbcolor{PST@COLOR106}{0.165 0.165 0.165}
\newrgbcolor{PST@COLOR107}{0.157 0.157 0.157}
\newrgbcolor{PST@COLOR108}{0.149 0.149 0.149}
\newrgbcolor{PST@COLOR109}{0.141 0.141 0.141}
\newrgbcolor{PST@COLOR110}{0.133 0.133 0.133}
\newrgbcolor{PST@COLOR111}{0.125 0.125 0.125}
\newrgbcolor{PST@COLOR112}{0.118 0.118 0.118}
\newrgbcolor{PST@COLOR113}{0.11 0.11 0.11}
\newrgbcolor{PST@COLOR114}{0.102 0.102 0.102}
\newrgbcolor{PST@COLOR115}{0.094 0.094 0.094}
\newrgbcolor{PST@COLOR116}{0.086 0.086 0.086}
\newrgbcolor{PST@COLOR117}{0.078 0.078 0.078}
\newrgbcolor{PST@COLOR118}{0.07 0.07 0.07}
\newrgbcolor{PST@COLOR119}{0.062 0.062 0.062}
\newrgbcolor{PST@COLOR120}{0.055 0.055 0.055}
\newrgbcolor{PST@COLOR121}{0.047 0.047 0.047}
\newrgbcolor{PST@COLOR122}{0.039 0.039 0.039}
\newrgbcolor{PST@COLOR123}{0.031 0.031 0.031}
\newrgbcolor{PST@COLOR124}{0.023 0.023 0.023}
\newrgbcolor{PST@COLOR125}{0.015 0.015 0.015}
\newrgbcolor{PST@COLOR126}{0.007 0.007 0.007}
\newrgbcolor{PST@COLOR127}{0 0 0}

\def\polypmIIId#1{\pspolygon[linestyle=none,fillstyle=solid,fillcolor=PST@COLOR#1]}

\polypmIIId{2} (0.0568,0.19)  (0.0,0.19)  (0.0,0.1078)(0.0568,0.1078)
\polypmIIId{70}  (0.0568,0.272) (0.0,0.272) (0.0,0.19)  (0.0568,0.19)
\polypmIIId{100}  (0.0568,0.3542)(0.0,0.3542)(0.0,0.272) (0.0568,0.272)

\polypmIIId{7} (0.1136,   0.19)  (0.0568,0.19)  (0.0568,0.1078)(0.1136,0.1078)
\polypmIIId{68}  (0.1136,   0.272) (0.0568,0.272) (0.0568,0.19)  (0.1136,0.19)
\polypmIIId{101}  (0.1136,   0.3542)(0.0568,0.3542)(0.0568,0.272) (0.1136,0.272)

\polypmIIId{9}(0.1704,0.19)  (0.1136,   0.19)  (0.1136,   0.1078)(0.1704,0.1078)
\polypmIIId{71} (0.1704,0.272) (0.1136,   0.272) (0.1136,   0.19)  (0.1704,0.19)
\polypmIIId{101}  (0.1704,0.3542)(0.1136,   0.3542)(0.1136,   0.272) (0.1704,0.272)

\polypmIIId{9}(0.2272,0.19)  (0.1704,0.19)  (0.1704,0.1078)(0.2272,0.1078)
\polypmIIId{73}  (0.2272,0.272) (0.1704,0.272) (0.1704,0.19)  (0.2272,0.19)
\polypmIIId{101}  (0.2272,0.3542)(0.1704,0.3542)(0.1704,0.272) (0.2272,0.272)

\rput(0.0284,0.07){3}
\rput(0.0852,0.07){4}
\rput(0.1420,0.07){5}
\rput(0.1988,0.07){6}
\rput(0.1136,0.0070){years}

\PST@Border(0.0,0.3542)
(0.0,0.1078)
(0.2272,0.1078)
(0.2272,0.3542)
(0.0,0.3542)

\polypmIIId{0}(0.2329,0.1078)(0.2442,0.1078)(0.2442,0.1098)(0.2329,0.1098)
\polypmIIId{1}(0.2329,0.1097)(0.2442,0.1097)(0.2442,0.1117)(0.2329,0.1117)
\polypmIIId{2}(0.2329,0.1116)(0.2442,0.1116)(0.2442,0.1136)(0.2329,0.1136)
\polypmIIId{3}(0.2329,0.1135)(0.2442,0.1135)(0.2442,0.1156)(0.2329,0.1156)
\polypmIIId{4}(0.2329,0.1155)(0.2442,0.1155)(0.2442,0.1175)(0.2329,0.1175)
\polypmIIId{5}(0.2329,0.1174)(0.2442,0.1174)(0.2442,0.1194)(0.2329,0.1194)
\polypmIIId{6}(0.2329,0.1193)(0.2442,0.1193)(0.2442,0.1213)(0.2329,0.1213)
\polypmIIId{7}(0.2329,0.1212)(0.2442,0.1212)(0.2442,0.1233)(0.2329,0.1233)
\polypmIIId{8}(0.2329,0.1232)(0.2442,0.1232)(0.2442,0.1252)(0.2329,0.1252)
\polypmIIId{9}(0.2329,0.1251)(0.2442,0.1251)(0.2442,0.1271)(0.2329,0.1271)
\polypmIIId{10}(0.2329,0.127)(0.2442,0.127)(0.2442,0.129)(0.2329,0.129)
\polypmIIId{11}(0.2329,0.1289)(0.2442,0.1289)(0.2442,0.131)(0.2329,0.131)
\polypmIIId{12}(0.2329,0.1309)(0.2442,0.1309)(0.2442,0.1329)(0.2329,0.1329)
\polypmIIId{13}(0.2329,0.1328)(0.2442,0.1328)(0.2442,0.1348)(0.2329,0.1348)
\polypmIIId{14}(0.2329,0.1347)(0.2442,0.1347)(0.2442,0.1367)(0.2329,0.1367)
\polypmIIId{15}(0.2329,0.1366)(0.2442,0.1366)(0.2442,0.1387)(0.2329,0.1387)
\polypmIIId{16}(0.2329,0.1386)(0.2442,0.1386)(0.2442,0.1406)(0.2329,0.1406)
\polypmIIId{17}(0.2329,0.1405)(0.2442,0.1405)(0.2442,0.1425)(0.2329,0.1425)
\polypmIIId{18}(0.2329,0.1424)(0.2442,0.1424)(0.2442,0.1444)(0.2329,0.1444)
\polypmIIId{19}(0.2329,0.1443)(0.2442,0.1443)(0.2442,0.1464)(0.2329,0.1464)
\polypmIIId{20}(0.2329,0.1463)(0.2442,0.1463)(0.2442,0.1483)(0.2329,0.1483)
\polypmIIId{21}(0.2329,0.1482)(0.2442,0.1482)(0.2442,0.1502)(0.2329,0.1502)
\polypmIIId{22}(0.2329,0.1501)(0.2442,0.1501)(0.2442,0.1521)(0.2329,0.1521)
\polypmIIId{23}(0.2329,0.152)(0.2442,0.152)(0.2442,0.1541)(0.2329,0.1541)
\polypmIIId{24}(0.2329,0.154)(0.2442,0.154)(0.2442,0.156)(0.2329,0.156)
\polypmIIId{25}(0.2329,0.1559)(0.2442,0.1559)(0.2442,0.1579)(0.2329,0.1579)
\polypmIIId{26}(0.2329,0.1578)(0.2442,0.1578)(0.2442,0.1598)(0.2329,0.1598)
\polypmIIId{27}(0.2329,0.1597)(0.2442,0.1597)(0.2442,0.1618)(0.2329,0.1618)
\polypmIIId{28}(0.2329,0.1617)(0.2442,0.1617)(0.2442,0.1637)(0.2329,0.1637)
\polypmIIId{29}(0.2329,0.1636)(0.2442,0.1636)(0.2442,0.1656)(0.2329,0.1656)
\polypmIIId{30}(0.2329,0.1655)(0.2442,0.1655)(0.2442,0.1675)(0.2329,0.1675)
\polypmIIId{31}(0.2329,0.1674)(0.2442,0.1674)(0.2442,0.1695)(0.2329,0.1695)
\polypmIIId{32}(0.2329,0.1694)(0.2442,0.1694)(0.2442,0.1714)(0.2329,0.1714)
\polypmIIId{33}(0.2329,0.1713)(0.2442,0.1713)(0.2442,0.1733)(0.2329,0.1733)
\polypmIIId{34}(0.2329,0.1732)(0.2442,0.1732)(0.2442,0.1752)(0.2329,0.1752)
\polypmIIId{35}(0.2329,0.1751)(0.2442,0.1751)(0.2442,0.1772)(0.2329,0.1772)
\polypmIIId{36}(0.2329,0.1771)(0.2442,0.1771)(0.2442,0.1791)(0.2329,0.1791)
\polypmIIId{37}(0.2329,0.179)(0.2442,0.179)(0.2442,0.181)(0.2329,0.181)
\polypmIIId{38}(0.2329,0.1809)(0.2442,0.1809)(0.2442,0.1829)(0.2329,0.1829)
\polypmIIId{39}(0.2329,0.1828)(0.2442,0.1828)(0.2442,0.1849)(0.2329,0.1849)
\polypmIIId{40}(0.2329,0.1848)(0.2442,0.1848)(0.2442,0.1868)(0.2329,0.1868)
\polypmIIId{41}(0.2329,0.1867)(0.2442,0.1867)(0.2442,0.1887)(0.2329,0.1887)
\polypmIIId{42}(0.2329,0.1886)(0.2442,0.1886)(0.2442,0.1906)(0.2329,0.1906)
\polypmIIId{43}(0.2329,0.1905)(0.2442,0.1905)(0.2442,0.1926)(0.2329,0.1926)
\polypmIIId{44}(0.2329,0.1925)(0.2442,0.1925)(0.2442,0.1945)(0.2329,0.1945)
\polypmIIId{45}(0.2329,0.1944)(0.2442,0.1944)(0.2442,0.1964)(0.2329,0.1964)
\polypmIIId{46}(0.2329,0.1963)(0.2442,0.1963)(0.2442,0.1983)(0.2329,0.1983)
\polypmIIId{47}(0.2329,0.1982)(0.2442,0.1982)(0.2442,0.2003)(0.2329,0.2003)
\polypmIIId{48}(0.2329,0.2002)(0.2442,0.2002)(0.2442,0.2022)(0.2329,0.2022)
\polypmIIId{49}(0.2329,0.2021)(0.2442,0.2021)(0.2442,0.2041)(0.2329,0.2041)
\polypmIIId{50}(0.2329,0.204)(0.2442,0.204)(0.2442,0.206)(0.2329,0.206)
\polypmIIId{51}(0.2329,0.2059)(0.2442,0.2059)(0.2442,0.208)(0.2329,0.208)
\polypmIIId{52}(0.2329,0.2079)(0.2442,0.2079)(0.2442,0.2099)(0.2329,0.2099)
\polypmIIId{53}(0.2329,0.2098)(0.2442,0.2098)(0.2442,0.2118)(0.2329,0.2118)
\polypmIIId{54}(0.2329,0.2117)(0.2442,0.2117)(0.2442,0.2137)(0.2329,0.2137)
\polypmIIId{55}(0.2329,0.2136)(0.2442,0.2136)(0.2442,0.2157)(0.2329,0.2157)
\polypmIIId{56}(0.2329,0.2156)(0.2442,0.2156)(0.2442,0.2176)(0.2329,0.2176)
\polypmIIId{57}(0.2329,0.2175)(0.2442,0.2175)(0.2442,0.2195)(0.2329,0.2195)
\polypmIIId{58}(0.2329,0.2194)(0.2442,0.2194)(0.2442,0.2214)(0.2329,0.2214)
\polypmIIId{59}(0.2329,0.2213)(0.2442,0.2213)(0.2442,0.2234)(0.2329,0.2234)
\polypmIIId{60}(0.2329,0.2233)(0.2442,0.2233)(0.2442,0.2253)(0.2329,0.2253)
\polypmIIId{61}(0.2329,0.2252)(0.2442,0.2252)(0.2442,0.2272)(0.2329,0.2272)
\polypmIIId{62}(0.2329,0.2271)(0.2442,0.2271)(0.2442,0.2291)(0.2329,0.2291)
\polypmIIId{63}(0.2329,0.229)(0.2442,0.229)(0.2442,0.2311)(0.2329,0.2311)
\polypmIIId{64}(0.2329,0.231)(0.2442,0.231)(0.2442,0.233)(0.2329,0.233)
\polypmIIId{65}(0.2329,0.2329)(0.2442,0.2329)(0.2442,0.2349)(0.2329,0.2349)
\polypmIIId{66}(0.2329,0.2348)(0.2442,0.2348)(0.2442,0.2368)(0.2329,0.2368)
\polypmIIId{67}(0.2329,0.2367)(0.2442,0.2367)(0.2442,0.2388)(0.2329,0.2388)
\polypmIIId{68}(0.2329,0.2387)(0.2442,0.2387)(0.2442,0.2407)(0.2329,0.2407)
\polypmIIId{69}(0.2329,0.2406)(0.2442,0.2406)(0.2442,0.2426)(0.2329,0.2426)
\polypmIIId{70}(0.2329,0.2425)(0.2442,0.2425)(0.2442,0.2445)(0.2329,0.2445)
\polypmIIId{71}(0.2329,0.2444)(0.2442,0.2444)(0.2442,0.2465)(0.2329,0.2465)
\polypmIIId{72}(0.2329,0.2464)(0.2442,0.2464)(0.2442,0.2484)(0.2329,0.2484)
\polypmIIId{73}(0.2329,0.2483)(0.2442,0.2483)(0.2442,0.2503)(0.2329,0.2503)
\polypmIIId{74}(0.2329,0.2502)(0.2442,0.2502)(0.2442,0.2522)(0.2329,0.2522)
\polypmIIId{75}(0.2329,0.2521)(0.2442,0.2521)(0.2442,0.2542)(0.2329,0.2542)
\polypmIIId{76}(0.2329,0.2541)(0.2442,0.2541)(0.2442,0.2561)(0.2329,0.2561)
\polypmIIId{77}(0.2329,0.256)(0.2442,0.256)(0.2442,0.258)(0.2329,0.258)
\polypmIIId{78}(0.2329,0.2579)(0.2442,0.2579)(0.2442,0.2599)(0.2329,0.2599)
\polypmIIId{79}(0.2329,0.2598)(0.2442,0.2598)(0.2442,0.2619)(0.2329,0.2619)
\polypmIIId{80}(0.2329,0.2618)(0.2442,0.2618)(0.2442,0.2638)(0.2329,0.2638)
\polypmIIId{81}(0.2329,0.2637)(0.2442,0.2637)(0.2442,0.2657)(0.2329,0.2657)
\polypmIIId{82}(0.2329,0.2656)(0.2442,0.2656)(0.2442,0.2676)(0.2329,0.2676)
\polypmIIId{83}(0.2329,0.2675)(0.2442,0.2675)(0.2442,0.2696)(0.2329,0.2696)
\polypmIIId{84}(0.2329,0.2695)(0.2442,0.2695)(0.2442,0.2715)(0.2329,0.2715)
\polypmIIId{85}(0.2329,0.2714)(0.2442,0.2714)(0.2442,0.2734)(0.2329,0.2734)
\polypmIIId{86}(0.2329,0.2733)(0.2442,0.2733)(0.2442,0.2753)(0.2329,0.2753)
\polypmIIId{87}(0.2329,0.2752)(0.2442,0.2752)(0.2442,0.2773)(0.2329,0.2773)
\polypmIIId{88}(0.2329,0.2772)(0.2442,0.2772)(0.2442,0.2792)(0.2329,0.2792)
\polypmIIId{89}(0.2329,0.2791)(0.2442,0.2791)(0.2442,0.2811)(0.2329,0.2811)
\polypmIIId{90}(0.2329,0.281)(0.2442,0.281)(0.2442,0.283)(0.2329,0.283)
\polypmIIId{91}(0.2329,0.2829)(0.2442,0.2829)(0.2442,0.285)(0.2329,0.285)
\polypmIIId{92}(0.2329,0.2849)(0.2442,0.2849)(0.2442,0.2869)(0.2329,0.2869)
\polypmIIId{93}(0.2329,0.2868)(0.2442,0.2868)(0.2442,0.2888)(0.2329,0.2888)
\polypmIIId{94}(0.2329,0.2887)(0.2442,0.2887)(0.2442,0.2907)(0.2329,0.2907)
\polypmIIId{95}(0.2329,0.2906)(0.2442,0.2906)(0.2442,0.2927)(0.2329,0.2927)
\polypmIIId{96}(0.2329,0.2926)(0.2442,0.2926)(0.2442,0.2946)(0.2329,0.2946)
\polypmIIId{97}(0.2329,0.2945)(0.2442,0.2945)(0.2442,0.2965)(0.2329,0.2965)
\polypmIIId{98}(0.2329,0.2964)(0.2442,0.2964)(0.2442,0.2984)(0.2329,0.2984)
\polypmIIId{99}(0.2329,0.2983)(0.2442,0.2983)(0.2442,0.3004)(0.2329,0.3004)
\polypmIIId{100}(0.2329,0.3003)(0.2442,0.3003)(0.2442,0.3023)(0.2329,0.3023)
\polypmIIId{101}(0.2329,0.3022)(0.2442,0.3022)(0.2442,0.3042)(0.2329,0.3042)
\polypmIIId{102}(0.2329,0.3041)(0.2442,0.3041)(0.2442,0.3061)(0.2329,0.3061)
\polypmIIId{103}(0.2329,0.306)(0.2442,0.306)(0.2442,0.3081)(0.2329,0.3081)
\polypmIIId{104}(0.2329,0.308)(0.2442,0.308)(0.2442,0.31)(0.2329,0.31)
\polypmIIId{105}(0.2329,0.3099)(0.2442,0.3099)(0.2442,0.3119)(0.2329,0.3119)
\polypmIIId{106}(0.2329,0.3118)(0.2442,0.3118)(0.2442,0.3138)(0.2329,0.3138)
\polypmIIId{107}(0.2329,0.3137)(0.2442,0.3137)(0.2442,0.3158)(0.2329,0.3158)
\polypmIIId{108}(0.2329,0.3157)(0.2442,0.3157)(0.2442,0.3177)(0.2329,0.3177)
\polypmIIId{109}(0.2329,0.3176)(0.2442,0.3176)(0.2442,0.3196)(0.2329,0.3196)
\polypmIIId{110}(0.2329,0.3195)(0.2442,0.3195)(0.2442,0.3215)(0.2329,0.3215)
\polypmIIId{111}(0.2329,0.3214)(0.2442,0.3214)(0.2442,0.3235)(0.2329,0.3235)
\polypmIIId{112}(0.2329,0.3234)(0.2442,0.3234)(0.2442,0.3254)(0.2329,0.3254)
\polypmIIId{113}(0.2329,0.3253)(0.2442,0.3253)(0.2442,0.3273)(0.2329,0.3273)
\polypmIIId{114}(0.2329,0.3272)(0.2442,0.3272)(0.2442,0.3292)(0.2329,0.3292)
\polypmIIId{115}(0.2329,0.3291)(0.2442,0.3291)(0.2442,0.3312)(0.2329,0.3312)
\polypmIIId{116}(0.2329,0.3311)(0.2442,0.3311)(0.2442,0.3331)(0.2329,0.3331)
\polypmIIId{117}(0.2329,0.333)(0.2442,0.333)(0.2442,0.335)(0.2329,0.335)
\polypmIIId{118}(0.2329,0.3349)(0.2442,0.3349)(0.2442,0.3369)(0.2329,0.3369)
\polypmIIId{119}(0.2329,0.3368)(0.2442,0.3368)(0.2442,0.3389)(0.2329,0.3389)
\polypmIIId{120}(0.2329,0.3388)(0.2442,0.3388)(0.2442,0.3408)(0.2329,0.3408)
\polypmIIId{121}(0.2329,0.3407)(0.2442,0.3407)(0.2442,0.3427)(0.2329,0.3427)
\polypmIIId{122}(0.2329,0.3426)(0.2442,0.3426)(0.2442,0.3446)(0.2329,0.3446)
\polypmIIId{123}(0.2329,0.3445)(0.2442,0.3445)(0.2442,0.3466)(0.2329,0.3466)
\polypmIIId{124}(0.2329,0.3465)(0.2442,0.3465)(0.2442,0.3485)(0.2329,0.3485)
\polypmIIId{125}(0.2329,0.3484)(0.2442,0.3484)(0.2442,0.3504)(0.2329,0.3504)
\polypmIIId{126}(0.2329,0.3503)(0.2442,0.3503)(0.2442,0.3523)(0.2329,0.3523)
\polypmIIId{127}(0.2329,0.3522)(0.2442,0.3522)(0.2442,0.3542)(0.2329,0.3542)

\PST@Border(0.2329,0.1078)
(0.2442,0.1078)
(0.2442,0.3542)
(0.2329,0.3542)
(0.2329,0.1078)


\rput[l](0.2502,0.1301){0.997}
\rput[l](0.2502,0.2048){0.998}
\rput[l](0.2502,0.2795){0.999}
\rput[l](0.2502,0.3542){1}

\catcode`@=12
\fi
\endpspicture} 
  \caption{GALP solution quality on strongly correlated instances ($\alpha = 0.0$).}
  \label{fig:greedysolcomp00}
\end{figure}

\begin{figure}[H]
  \centering
    \subfloat[1 resource]{% GNUPLOT: LaTeX picture using PSTRICKS macros
% Define new PST objects, if not already defined
\ifx\PSTloaded\undefined
\def\PSTloaded{t}

\catcode`@=11

\newpsobject{PST@Border}{psline}{linewidth=.0015,linestyle=solid}

\catcode`@=12

\fi
\psset{unit=5.0in,xunit=5.0in,yunit=3.0in}
\pspicture(0.000000,0.000000)(0.31, 0.35)
\ifx\nofigs\undefined
\catcode`@=11

\newrgbcolor{PST@COLOR0}{1 1 1}
\newrgbcolor{PST@COLOR1}{0.992 0.992 0.992}
\newrgbcolor{PST@COLOR2}{0.984 0.984 0.984}
\newrgbcolor{PST@COLOR3}{0.976 0.976 0.976}
\newrgbcolor{PST@COLOR4}{0.968 0.968 0.968}
\newrgbcolor{PST@COLOR5}{0.96 0.96 0.96}
\newrgbcolor{PST@COLOR6}{0.952 0.952 0.952}
\newrgbcolor{PST@COLOR7}{0.944 0.944 0.944}
\newrgbcolor{PST@COLOR8}{0.937 0.937 0.937}
\newrgbcolor{PST@COLOR9}{0.929 0.929 0.929}
\newrgbcolor{PST@COLOR10}{0.921 0.921 0.921}
\newrgbcolor{PST@COLOR11}{0.913 0.913 0.913}
\newrgbcolor{PST@COLOR12}{0.905 0.905 0.905}
\newrgbcolor{PST@COLOR13}{0.897 0.897 0.897}
\newrgbcolor{PST@COLOR14}{0.889 0.889 0.889}
\newrgbcolor{PST@COLOR15}{0.881 0.881 0.881}
\newrgbcolor{PST@COLOR16}{0.874 0.874 0.874}
\newrgbcolor{PST@COLOR17}{0.866 0.866 0.866}
\newrgbcolor{PST@COLOR18}{0.858 0.858 0.858}
\newrgbcolor{PST@COLOR19}{0.85 0.85 0.85}
\newrgbcolor{PST@COLOR20}{0.842 0.842 0.842}
\newrgbcolor{PST@COLOR21}{0.834 0.834 0.834}
\newrgbcolor{PST@COLOR22}{0.826 0.826 0.826}
\newrgbcolor{PST@COLOR23}{0.818 0.818 0.818}
\newrgbcolor{PST@COLOR24}{0.811 0.811 0.811}
\newrgbcolor{PST@COLOR25}{0.803 0.803 0.803}
\newrgbcolor{PST@COLOR26}{0.795 0.795 0.795}
\newrgbcolor{PST@COLOR27}{0.787 0.787 0.787}
\newrgbcolor{PST@COLOR28}{0.779 0.779 0.779}
\newrgbcolor{PST@COLOR29}{0.771 0.771 0.771}
\newrgbcolor{PST@COLOR30}{0.763 0.763 0.763}
\newrgbcolor{PST@COLOR31}{0.755 0.755 0.755}
\newrgbcolor{PST@COLOR32}{0.748 0.748 0.748}
\newrgbcolor{PST@COLOR33}{0.74 0.74 0.74}
\newrgbcolor{PST@COLOR34}{0.732 0.732 0.732}
\newrgbcolor{PST@COLOR35}{0.724 0.724 0.724}
\newrgbcolor{PST@COLOR36}{0.716 0.716 0.716}
\newrgbcolor{PST@COLOR37}{0.708 0.708 0.708}
\newrgbcolor{PST@COLOR38}{0.7 0.7 0.7}
\newrgbcolor{PST@COLOR39}{0.692 0.692 0.692}
\newrgbcolor{PST@COLOR40}{0.685 0.685 0.685}
\newrgbcolor{PST@COLOR41}{0.677 0.677 0.677}
\newrgbcolor{PST@COLOR42}{0.669 0.669 0.669}
\newrgbcolor{PST@COLOR43}{0.661 0.661 0.661}
\newrgbcolor{PST@COLOR44}{0.653 0.653 0.653}
\newrgbcolor{PST@COLOR45}{0.645 0.645 0.645}
\newrgbcolor{PST@COLOR46}{0.637 0.637 0.637}
\newrgbcolor{PST@COLOR47}{0.629 0.629 0.629}
\newrgbcolor{PST@COLOR48}{0.622 0.622 0.622}
\newrgbcolor{PST@COLOR49}{0.614 0.614 0.614}
\newrgbcolor{PST@COLOR50}{0.606 0.606 0.606}
\newrgbcolor{PST@COLOR51}{0.598 0.598 0.598}
\newrgbcolor{PST@COLOR52}{0.59 0.59 0.59}
\newrgbcolor{PST@COLOR53}{0.582 0.582 0.582}
\newrgbcolor{PST@COLOR54}{0.574 0.574 0.574}
\newrgbcolor{PST@COLOR55}{0.566 0.566 0.566}
\newrgbcolor{PST@COLOR56}{0.559 0.559 0.559}
\newrgbcolor{PST@COLOR57}{0.551 0.551 0.551}
\newrgbcolor{PST@COLOR58}{0.543 0.543 0.543}
\newrgbcolor{PST@COLOR59}{0.535 0.535 0.535}
\newrgbcolor{PST@COLOR60}{0.527 0.527 0.527}
\newrgbcolor{PST@COLOR61}{0.519 0.519 0.519}
\newrgbcolor{PST@COLOR62}{0.511 0.511 0.511}
\newrgbcolor{PST@COLOR63}{0.503 0.503 0.503}
\newrgbcolor{PST@COLOR64}{0.496 0.496 0.496}
\newrgbcolor{PST@COLOR65}{0.488 0.488 0.488}
\newrgbcolor{PST@COLOR66}{0.48 0.48 0.48}
\newrgbcolor{PST@COLOR67}{0.472 0.472 0.472}
\newrgbcolor{PST@COLOR68}{0.464 0.464 0.464}
\newrgbcolor{PST@COLOR69}{0.456 0.456 0.456}
\newrgbcolor{PST@COLOR70}{0.448 0.448 0.448}
\newrgbcolor{PST@COLOR71}{0.44 0.44 0.44}
\newrgbcolor{PST@COLOR72}{0.433 0.433 0.433}
\newrgbcolor{PST@COLOR73}{0.425 0.425 0.425}
\newrgbcolor{PST@COLOR74}{0.417 0.417 0.417}
\newrgbcolor{PST@COLOR75}{0.409 0.409 0.409}
\newrgbcolor{PST@COLOR76}{0.401 0.401 0.401}
\newrgbcolor{PST@COLOR77}{0.393 0.393 0.393}
\newrgbcolor{PST@COLOR78}{0.385 0.385 0.385}
\newrgbcolor{PST@COLOR79}{0.377 0.377 0.377}
\newrgbcolor{PST@COLOR80}{0.37 0.37 0.37}
\newrgbcolor{PST@COLOR81}{0.362 0.362 0.362}
\newrgbcolor{PST@COLOR82}{0.354 0.354 0.354}
\newrgbcolor{PST@COLOR83}{0.346 0.346 0.346}
\newrgbcolor{PST@COLOR84}{0.338 0.338 0.338}
\newrgbcolor{PST@COLOR85}{0.33 0.33 0.33}
\newrgbcolor{PST@COLOR86}{0.322 0.322 0.322}
\newrgbcolor{PST@COLOR87}{0.314 0.314 0.314}
\newrgbcolor{PST@COLOR88}{0.307 0.307 0.307}
\newrgbcolor{PST@COLOR89}{0.299 0.299 0.299}
\newrgbcolor{PST@COLOR90}{0.291 0.291 0.291}
\newrgbcolor{PST@COLOR91}{0.283 0.283 0.283}
\newrgbcolor{PST@COLOR92}{0.275 0.275 0.275}
\newrgbcolor{PST@COLOR93}{0.267 0.267 0.267}
\newrgbcolor{PST@COLOR94}{0.259 0.259 0.259}
\newrgbcolor{PST@COLOR95}{0.251 0.251 0.251}
\newrgbcolor{PST@COLOR96}{0.244 0.244 0.244}
\newrgbcolor{PST@COLOR97}{0.236 0.236 0.236}
\newrgbcolor{PST@COLOR98}{0.228 0.228 0.228}
\newrgbcolor{PST@COLOR99}{0.22 0.22 0.22}
\newrgbcolor{PST@COLOR100}{0.212 0.212 0.212}
\newrgbcolor{PST@COLOR101}{0.204 0.204 0.204}
\newrgbcolor{PST@COLOR102}{0.196 0.196 0.196}
\newrgbcolor{PST@COLOR103}{0.188 0.188 0.188}
\newrgbcolor{PST@COLOR104}{0.181 0.181 0.181}
\newrgbcolor{PST@COLOR105}{0.173 0.173 0.173}
\newrgbcolor{PST@COLOR106}{0.165 0.165 0.165}
\newrgbcolor{PST@COLOR107}{0.157 0.157 0.157}
\newrgbcolor{PST@COLOR108}{0.149 0.149 0.149}
\newrgbcolor{PST@COLOR109}{0.141 0.141 0.141}
\newrgbcolor{PST@COLOR110}{0.133 0.133 0.133}
\newrgbcolor{PST@COLOR111}{0.125 0.125 0.125}
\newrgbcolor{PST@COLOR112}{0.118 0.118 0.118}
\newrgbcolor{PST@COLOR113}{0.11 0.11 0.11}
\newrgbcolor{PST@COLOR114}{0.102 0.102 0.102}
\newrgbcolor{PST@COLOR115}{0.094 0.094 0.094}
\newrgbcolor{PST@COLOR116}{0.086 0.086 0.086}
\newrgbcolor{PST@COLOR117}{0.078 0.078 0.078}
\newrgbcolor{PST@COLOR118}{0.07 0.07 0.07}
\newrgbcolor{PST@COLOR119}{0.062 0.062 0.062}
\newrgbcolor{PST@COLOR120}{0.055 0.055 0.055}
\newrgbcolor{PST@COLOR121}{0.047 0.047 0.047}
\newrgbcolor{PST@COLOR122}{0.039 0.039 0.039}
\newrgbcolor{PST@COLOR123}{0.031 0.031 0.031}
\newrgbcolor{PST@COLOR124}{0.023 0.023 0.023}
\newrgbcolor{PST@COLOR125}{0.015 0.015 0.015}
\newrgbcolor{PST@COLOR126}{0.007 0.007 0.007}
\newrgbcolor{PST@COLOR127}{0 0 0}


\def\polypmIIId#1{\pspolygon[linestyle=none,fillstyle=solid,fillcolor=PST@COLOR#1]}

\polypmIIId{74}(0.1432,0.19)(0.0864,0.19)(0.0864,0.1078)(0.1432,0.1078)
\polypmIIId{103}(0.1432,0.272)(0.0864,0.272)(0.0864,0.19)(0.1432,0.19)
\polypmIIId{117}(0.1432,0.3542)(0.0864,0.3542)(0.0864,0.272)(0.1432,0.272)

\polypmIIId{71}(0.2,0.19)(0.1432,0.19)(0.1432,0.1078)(0.2,0.1078)
\polypmIIId{102}(0.2,0.272)(0.1432,0.272)(0.1432,0.19)(0.2,0.19)
\polypmIIId{118}(0.2,0.3542)(0.1432,0.3542)(0.1432,0.272)(0.2,0.272)

\polypmIIId{70}(0.2568,0.19)(0.2,0.19)(0.2,0.1078)(0.2568,0.1078)
\polypmIIId{104}(0.2568,0.272)(0.2,0.272)(0.2,0.19)(0.2568,0.19)
\polypmIIId{118}(0.2568,0.3542)(0.2,0.3542)(0.2,0.272)(0.2568,0.272)

\polypmIIId{73}(0.3136,0.19)(0.2568,0.19)(0.2568,0.1078)(0.3136,0.1078)
\polypmIIId{105}(0.3136,0.272)(0.2568,0.272)(0.2568,0.19)(0.3136,0.19)
\polypmIIId{118}(0.3136,0.3542)(0.2568,0.3542)(0.2568,0.272)(0.3136,0.272)


\rput(0.1148,0.07){3}
\rput(0.1716,0.07){4}
\rput(0.2284,0.07){5}
\rput(0.2852,0.07){6}
\rput(0.2000,0.0070){years}

\rput[r](0.0806,0.1489){25}
\rput[r](0.0806,0.2310){50}
\rput[r](0.0806,0.3131){100}
\rput{L}(0.0096,0.2310){actions}

\PST@Border(0.0864,0.3542)
(0.0864,0.1078)
(0.3136,0.1078)
(0.3136,0.3542)
(0.0864,0.3542)

\catcode`@=12
\fi
\endpspicture} 
    \subfloat[2 resources]{% GNUPLOT: LaTeX picture using PSTRICKS macros
% Define new PST objects, if not already defined
\ifx\PSTloaded\undefined
\def\PSTloaded{t}

\catcode`@=11

\newpsobject{PST@Border}{psline}{linewidth=.0015,linestyle=solid}

\catcode`@=12

\fi
\psset{unit=5.0in,xunit=5.0in,yunit=3.0in}
\pspicture(0.000000,0.000000)(0.225000,0.35)
\ifx\nofigs\undefined
\catcode`@=11

\newrgbcolor{PST@COLOR0}{1 1 1}
\newrgbcolor{PST@COLOR1}{0.992 0.992 0.992}
\newrgbcolor{PST@COLOR2}{0.984 0.984 0.984}
\newrgbcolor{PST@COLOR3}{0.976 0.976 0.976}
\newrgbcolor{PST@COLOR4}{0.968 0.968 0.968}
\newrgbcolor{PST@COLOR5}{0.96 0.96 0.96}
\newrgbcolor{PST@COLOR6}{0.952 0.952 0.952}
\newrgbcolor{PST@COLOR7}{0.944 0.944 0.944}
\newrgbcolor{PST@COLOR8}{0.937 0.937 0.937}
\newrgbcolor{PST@COLOR9}{0.929 0.929 0.929}
\newrgbcolor{PST@COLOR10}{0.921 0.921 0.921}
\newrgbcolor{PST@COLOR11}{0.913 0.913 0.913}
\newrgbcolor{PST@COLOR12}{0.905 0.905 0.905}
\newrgbcolor{PST@COLOR13}{0.897 0.897 0.897}
\newrgbcolor{PST@COLOR14}{0.889 0.889 0.889}
\newrgbcolor{PST@COLOR15}{0.881 0.881 0.881}
\newrgbcolor{PST@COLOR16}{0.874 0.874 0.874}
\newrgbcolor{PST@COLOR17}{0.866 0.866 0.866}
\newrgbcolor{PST@COLOR18}{0.858 0.858 0.858}
\newrgbcolor{PST@COLOR19}{0.85 0.85 0.85}
\newrgbcolor{PST@COLOR20}{0.842 0.842 0.842}
\newrgbcolor{PST@COLOR21}{0.834 0.834 0.834}
\newrgbcolor{PST@COLOR22}{0.826 0.826 0.826}
\newrgbcolor{PST@COLOR23}{0.818 0.818 0.818}
\newrgbcolor{PST@COLOR24}{0.811 0.811 0.811}
\newrgbcolor{PST@COLOR25}{0.803 0.803 0.803}
\newrgbcolor{PST@COLOR26}{0.795 0.795 0.795}
\newrgbcolor{PST@COLOR27}{0.787 0.787 0.787}
\newrgbcolor{PST@COLOR28}{0.779 0.779 0.779}
\newrgbcolor{PST@COLOR29}{0.771 0.771 0.771}
\newrgbcolor{PST@COLOR30}{0.763 0.763 0.763}
\newrgbcolor{PST@COLOR31}{0.755 0.755 0.755}
\newrgbcolor{PST@COLOR32}{0.748 0.748 0.748}
\newrgbcolor{PST@COLOR33}{0.74 0.74 0.74}
\newrgbcolor{PST@COLOR34}{0.732 0.732 0.732}
\newrgbcolor{PST@COLOR35}{0.724 0.724 0.724}
\newrgbcolor{PST@COLOR36}{0.716 0.716 0.716}
\newrgbcolor{PST@COLOR37}{0.708 0.708 0.708}
\newrgbcolor{PST@COLOR38}{0.7 0.7 0.7}
\newrgbcolor{PST@COLOR39}{0.692 0.692 0.692}
\newrgbcolor{PST@COLOR40}{0.685 0.685 0.685}
\newrgbcolor{PST@COLOR41}{0.677 0.677 0.677}
\newrgbcolor{PST@COLOR42}{0.669 0.669 0.669}
\newrgbcolor{PST@COLOR43}{0.661 0.661 0.661}
\newrgbcolor{PST@COLOR44}{0.653 0.653 0.653}
\newrgbcolor{PST@COLOR45}{0.645 0.645 0.645}
\newrgbcolor{PST@COLOR46}{0.637 0.637 0.637}
\newrgbcolor{PST@COLOR47}{0.629 0.629 0.629}
\newrgbcolor{PST@COLOR48}{0.622 0.622 0.622}
\newrgbcolor{PST@COLOR49}{0.614 0.614 0.614}
\newrgbcolor{PST@COLOR50}{0.606 0.606 0.606}
\newrgbcolor{PST@COLOR51}{0.598 0.598 0.598}
\newrgbcolor{PST@COLOR52}{0.59 0.59 0.59}
\newrgbcolor{PST@COLOR53}{0.582 0.582 0.582}
\newrgbcolor{PST@COLOR54}{0.574 0.574 0.574}
\newrgbcolor{PST@COLOR55}{0.566 0.566 0.566}
\newrgbcolor{PST@COLOR56}{0.559 0.559 0.559}
\newrgbcolor{PST@COLOR57}{0.551 0.551 0.551}
\newrgbcolor{PST@COLOR58}{0.543 0.543 0.543}
\newrgbcolor{PST@COLOR59}{0.535 0.535 0.535}
\newrgbcolor{PST@COLOR60}{0.527 0.527 0.527}
\newrgbcolor{PST@COLOR61}{0.519 0.519 0.519}
\newrgbcolor{PST@COLOR62}{0.511 0.511 0.511}
\newrgbcolor{PST@COLOR63}{0.503 0.503 0.503}
\newrgbcolor{PST@COLOR64}{0.496 0.496 0.496}
\newrgbcolor{PST@COLOR65}{0.488 0.488 0.488}
\newrgbcolor{PST@COLOR66}{0.48 0.48 0.48}
\newrgbcolor{PST@COLOR67}{0.472 0.472 0.472}
\newrgbcolor{PST@COLOR68}{0.464 0.464 0.464}
\newrgbcolor{PST@COLOR69}{0.456 0.456 0.456}
\newrgbcolor{PST@COLOR70}{0.448 0.448 0.448}
\newrgbcolor{PST@COLOR71}{0.44 0.44 0.44}
\newrgbcolor{PST@COLOR72}{0.433 0.433 0.433}
\newrgbcolor{PST@COLOR73}{0.425 0.425 0.425}
\newrgbcolor{PST@COLOR74}{0.417 0.417 0.417}
\newrgbcolor{PST@COLOR75}{0.409 0.409 0.409}
\newrgbcolor{PST@COLOR76}{0.401 0.401 0.401}
\newrgbcolor{PST@COLOR77}{0.393 0.393 0.393}
\newrgbcolor{PST@COLOR78}{0.385 0.385 0.385}
\newrgbcolor{PST@COLOR79}{0.377 0.377 0.377}
\newrgbcolor{PST@COLOR80}{0.37 0.37 0.37}
\newrgbcolor{PST@COLOR81}{0.362 0.362 0.362}
\newrgbcolor{PST@COLOR82}{0.354 0.354 0.354}
\newrgbcolor{PST@COLOR83}{0.346 0.346 0.346}
\newrgbcolor{PST@COLOR84}{0.338 0.338 0.338}
\newrgbcolor{PST@COLOR85}{0.33 0.33 0.33}
\newrgbcolor{PST@COLOR86}{0.322 0.322 0.322}
\newrgbcolor{PST@COLOR87}{0.314 0.314 0.314}
\newrgbcolor{PST@COLOR88}{0.307 0.307 0.307}
\newrgbcolor{PST@COLOR89}{0.299 0.299 0.299}
\newrgbcolor{PST@COLOR90}{0.291 0.291 0.291}
\newrgbcolor{PST@COLOR91}{0.283 0.283 0.283}
\newrgbcolor{PST@COLOR92}{0.275 0.275 0.275}
\newrgbcolor{PST@COLOR93}{0.267 0.267 0.267}
\newrgbcolor{PST@COLOR94}{0.259 0.259 0.259}
\newrgbcolor{PST@COLOR95}{0.251 0.251 0.251}
\newrgbcolor{PST@COLOR96}{0.244 0.244 0.244}
\newrgbcolor{PST@COLOR97}{0.236 0.236 0.236}
\newrgbcolor{PST@COLOR98}{0.228 0.228 0.228}
\newrgbcolor{PST@COLOR99}{0.22 0.22 0.22}
\newrgbcolor{PST@COLOR100}{0.212 0.212 0.212}
\newrgbcolor{PST@COLOR101}{0.204 0.204 0.204}
\newrgbcolor{PST@COLOR102}{0.196 0.196 0.196}
\newrgbcolor{PST@COLOR103}{0.188 0.188 0.188}
\newrgbcolor{PST@COLOR104}{0.181 0.181 0.181}
\newrgbcolor{PST@COLOR105}{0.173 0.173 0.173}
\newrgbcolor{PST@COLOR106}{0.165 0.165 0.165}
\newrgbcolor{PST@COLOR107}{0.157 0.157 0.157}
\newrgbcolor{PST@COLOR108}{0.149 0.149 0.149}
\newrgbcolor{PST@COLOR109}{0.141 0.141 0.141}
\newrgbcolor{PST@COLOR110}{0.133 0.133 0.133}
\newrgbcolor{PST@COLOR111}{0.125 0.125 0.125}
\newrgbcolor{PST@COLOR112}{0.118 0.118 0.118}
\newrgbcolor{PST@COLOR113}{0.11 0.11 0.11}
\newrgbcolor{PST@COLOR114}{0.102 0.102 0.102}
\newrgbcolor{PST@COLOR115}{0.094 0.094 0.094}
\newrgbcolor{PST@COLOR116}{0.086 0.086 0.086}
\newrgbcolor{PST@COLOR117}{0.078 0.078 0.078}
\newrgbcolor{PST@COLOR118}{0.07 0.07 0.07}
\newrgbcolor{PST@COLOR119}{0.062 0.062 0.062}
\newrgbcolor{PST@COLOR120}{0.055 0.055 0.055}
\newrgbcolor{PST@COLOR121}{0.047 0.047 0.047}
\newrgbcolor{PST@COLOR122}{0.039 0.039 0.039}
\newrgbcolor{PST@COLOR123}{0.031 0.031 0.031}
\newrgbcolor{PST@COLOR124}{0.023 0.023 0.023}
\newrgbcolor{PST@COLOR125}{0.015 0.015 0.015}
\newrgbcolor{PST@COLOR126}{0.007 0.007 0.007}
\newrgbcolor{PST@COLOR127}{0 0 0}

\def\polypmIIId#1{\pspolygon[linestyle=none,fillstyle=solid,fillcolor=PST@COLOR#1]}

\polypmIIId{56} (0.0568,0.19)  (0.0,0.19)  (0.0,0.1078)(0.0568,0.1078)
\polypmIIId{92}  (0.0568,0.272) (0.0,0.272) (0.0,0.19)  (0.0568,0.19)
\polypmIIId{113}  (0.0568,0.3542)(0.0,0.3542)(0.0,0.272) (0.0568,0.272)

\polypmIIId{58} (0.1136,   0.19)  (0.0568,0.19)  (0.0568,0.1078)(0.1136,0.1078)
\polypmIIId{90}  (0.1136,   0.272) (0.0568,0.272) (0.0568,0.19)  (0.1136,0.19)
\polypmIIId{113}  (0.1136,   0.3542)(0.0568,0.3542)(0.0568,0.272) (0.1136,0.272)

\polypmIIId{57}(0.1704,0.19)  (0.1136,   0.19)  (0.1136,   0.1078)(0.1704,0.1078)
\polypmIIId{92} (0.1704,0.272) (0.1136,   0.272) (0.1136,   0.19)  (0.1704,0.19)
\polypmIIId{112}  (0.1704,0.3542)(0.1136,   0.3542)(0.1136,   0.272) (0.1704,0.272)

\polypmIIId{59}(0.2272,0.19)  (0.1704,0.19)  (0.1704,0.1078)(0.2272,0.1078)
\polypmIIId{96}  (0.2272,0.272) (0.1704,0.272) (0.1704,0.19)  (0.2272,0.19)
\polypmIIId{113}  (0.2272,0.3542)(0.1704,0.3542)(0.1704,0.272) (0.2272,0.272)

\rput(0.0284,0.07){3}
\rput(0.0852,0.07){4}
\rput(0.1420,0.07){5}
\rput(0.1988,0.07){6}
\rput(0.1136,0.0070){years}


\PST@Border(0.0,0.3542)
(0.0,0.1078)
(0.2272,0.1078)
(0.2272,0.3542)
(0.0,0.3542)

\catcode`@=12
\fi
\endpspicture}
    \subfloat[4 resources]{% GNUPLOT: LaTeX picture using PSTRICKS macros
% Define new PST objects, if not already defined
\ifx\PSTloaded\undefined
\def\PSTloaded{t}

\catcode`@=11

\newpsobject{PST@Border}{psline}{linewidth=.0015,linestyle=solid}

\catcode`@=12

\fi
\psset{unit=5.0in,xunit=5.0in,yunit=3.0in}
\pspicture(0.000000,0.000000)(0.3136,0.35)
\ifx\nofigs\undefined
\catcode`@=11

\newrgbcolor{PST@COLOR0}{1 1 1}
\newrgbcolor{PST@COLOR1}{0.992 0.992 0.992}
\newrgbcolor{PST@COLOR2}{0.984 0.984 0.984}
\newrgbcolor{PST@COLOR3}{0.976 0.976 0.976}
\newrgbcolor{PST@COLOR4}{0.968 0.968 0.968}
\newrgbcolor{PST@COLOR5}{0.96 0.96 0.96}
\newrgbcolor{PST@COLOR6}{0.952 0.952 0.952}
\newrgbcolor{PST@COLOR7}{0.944 0.944 0.944}
\newrgbcolor{PST@COLOR8}{0.937 0.937 0.937}
\newrgbcolor{PST@COLOR9}{0.929 0.929 0.929}
\newrgbcolor{PST@COLOR10}{0.921 0.921 0.921}
\newrgbcolor{PST@COLOR11}{0.913 0.913 0.913}
\newrgbcolor{PST@COLOR12}{0.905 0.905 0.905}
\newrgbcolor{PST@COLOR13}{0.897 0.897 0.897}
\newrgbcolor{PST@COLOR14}{0.889 0.889 0.889}
\newrgbcolor{PST@COLOR15}{0.881 0.881 0.881}
\newrgbcolor{PST@COLOR16}{0.874 0.874 0.874}
\newrgbcolor{PST@COLOR17}{0.866 0.866 0.866}
\newrgbcolor{PST@COLOR18}{0.858 0.858 0.858}
\newrgbcolor{PST@COLOR19}{0.85 0.85 0.85}
\newrgbcolor{PST@COLOR20}{0.842 0.842 0.842}
\newrgbcolor{PST@COLOR21}{0.834 0.834 0.834}
\newrgbcolor{PST@COLOR22}{0.826 0.826 0.826}
\newrgbcolor{PST@COLOR23}{0.818 0.818 0.818}
\newrgbcolor{PST@COLOR24}{0.811 0.811 0.811}
\newrgbcolor{PST@COLOR25}{0.803 0.803 0.803}
\newrgbcolor{PST@COLOR26}{0.795 0.795 0.795}
\newrgbcolor{PST@COLOR27}{0.787 0.787 0.787}
\newrgbcolor{PST@COLOR28}{0.779 0.779 0.779}
\newrgbcolor{PST@COLOR29}{0.771 0.771 0.771}
\newrgbcolor{PST@COLOR30}{0.763 0.763 0.763}
\newrgbcolor{PST@COLOR31}{0.755 0.755 0.755}
\newrgbcolor{PST@COLOR32}{0.748 0.748 0.748}
\newrgbcolor{PST@COLOR33}{0.74 0.74 0.74}
\newrgbcolor{PST@COLOR34}{0.732 0.732 0.732}
\newrgbcolor{PST@COLOR35}{0.724 0.724 0.724}
\newrgbcolor{PST@COLOR36}{0.716 0.716 0.716}
\newrgbcolor{PST@COLOR37}{0.708 0.708 0.708}
\newrgbcolor{PST@COLOR38}{0.7 0.7 0.7}
\newrgbcolor{PST@COLOR39}{0.692 0.692 0.692}
\newrgbcolor{PST@COLOR40}{0.685 0.685 0.685}
\newrgbcolor{PST@COLOR41}{0.677 0.677 0.677}
\newrgbcolor{PST@COLOR42}{0.669 0.669 0.669}
\newrgbcolor{PST@COLOR43}{0.661 0.661 0.661}
\newrgbcolor{PST@COLOR44}{0.653 0.653 0.653}
\newrgbcolor{PST@COLOR45}{0.645 0.645 0.645}
\newrgbcolor{PST@COLOR46}{0.637 0.637 0.637}
\newrgbcolor{PST@COLOR47}{0.629 0.629 0.629}
\newrgbcolor{PST@COLOR48}{0.622 0.622 0.622}
\newrgbcolor{PST@COLOR49}{0.614 0.614 0.614}
\newrgbcolor{PST@COLOR50}{0.606 0.606 0.606}
\newrgbcolor{PST@COLOR51}{0.598 0.598 0.598}
\newrgbcolor{PST@COLOR52}{0.59 0.59 0.59}
\newrgbcolor{PST@COLOR53}{0.582 0.582 0.582}
\newrgbcolor{PST@COLOR54}{0.574 0.574 0.574}
\newrgbcolor{PST@COLOR55}{0.566 0.566 0.566}
\newrgbcolor{PST@COLOR56}{0.559 0.559 0.559}
\newrgbcolor{PST@COLOR57}{0.551 0.551 0.551}
\newrgbcolor{PST@COLOR58}{0.543 0.543 0.543}
\newrgbcolor{PST@COLOR59}{0.535 0.535 0.535}
\newrgbcolor{PST@COLOR60}{0.527 0.527 0.527}
\newrgbcolor{PST@COLOR61}{0.519 0.519 0.519}
\newrgbcolor{PST@COLOR62}{0.511 0.511 0.511}
\newrgbcolor{PST@COLOR63}{0.503 0.503 0.503}
\newrgbcolor{PST@COLOR64}{0.496 0.496 0.496}
\newrgbcolor{PST@COLOR65}{0.488 0.488 0.488}
\newrgbcolor{PST@COLOR66}{0.48 0.48 0.48}
\newrgbcolor{PST@COLOR67}{0.472 0.472 0.472}
\newrgbcolor{PST@COLOR68}{0.464 0.464 0.464}
\newrgbcolor{PST@COLOR69}{0.456 0.456 0.456}
\newrgbcolor{PST@COLOR70}{0.448 0.448 0.448}
\newrgbcolor{PST@COLOR71}{0.44 0.44 0.44}
\newrgbcolor{PST@COLOR72}{0.433 0.433 0.433}
\newrgbcolor{PST@COLOR73}{0.425 0.425 0.425}
\newrgbcolor{PST@COLOR74}{0.417 0.417 0.417}
\newrgbcolor{PST@COLOR75}{0.409 0.409 0.409}
\newrgbcolor{PST@COLOR76}{0.401 0.401 0.401}
\newrgbcolor{PST@COLOR77}{0.393 0.393 0.393}
\newrgbcolor{PST@COLOR78}{0.385 0.385 0.385}
\newrgbcolor{PST@COLOR79}{0.377 0.377 0.377}
\newrgbcolor{PST@COLOR80}{0.37 0.37 0.37}
\newrgbcolor{PST@COLOR81}{0.362 0.362 0.362}
\newrgbcolor{PST@COLOR82}{0.354 0.354 0.354}
\newrgbcolor{PST@COLOR83}{0.346 0.346 0.346}
\newrgbcolor{PST@COLOR84}{0.338 0.338 0.338}
\newrgbcolor{PST@COLOR85}{0.33 0.33 0.33}
\newrgbcolor{PST@COLOR86}{0.322 0.322 0.322}
\newrgbcolor{PST@COLOR87}{0.314 0.314 0.314}
\newrgbcolor{PST@COLOR88}{0.307 0.307 0.307}
\newrgbcolor{PST@COLOR89}{0.299 0.299 0.299}
\newrgbcolor{PST@COLOR90}{0.291 0.291 0.291}
\newrgbcolor{PST@COLOR91}{0.283 0.283 0.283}
\newrgbcolor{PST@COLOR92}{0.275 0.275 0.275}
\newrgbcolor{PST@COLOR93}{0.267 0.267 0.267}
\newrgbcolor{PST@COLOR94}{0.259 0.259 0.259}
\newrgbcolor{PST@COLOR95}{0.251 0.251 0.251}
\newrgbcolor{PST@COLOR96}{0.244 0.244 0.244}
\newrgbcolor{PST@COLOR97}{0.236 0.236 0.236}
\newrgbcolor{PST@COLOR98}{0.228 0.228 0.228}
\newrgbcolor{PST@COLOR99}{0.22 0.22 0.22}
\newrgbcolor{PST@COLOR100}{0.212 0.212 0.212}
\newrgbcolor{PST@COLOR101}{0.204 0.204 0.204}
\newrgbcolor{PST@COLOR102}{0.196 0.196 0.196}
\newrgbcolor{PST@COLOR103}{0.188 0.188 0.188}
\newrgbcolor{PST@COLOR104}{0.181 0.181 0.181}
\newrgbcolor{PST@COLOR105}{0.173 0.173 0.173}
\newrgbcolor{PST@COLOR106}{0.165 0.165 0.165}
\newrgbcolor{PST@COLOR107}{0.157 0.157 0.157}
\newrgbcolor{PST@COLOR108}{0.149 0.149 0.149}
\newrgbcolor{PST@COLOR109}{0.141 0.141 0.141}
\newrgbcolor{PST@COLOR110}{0.133 0.133 0.133}
\newrgbcolor{PST@COLOR111}{0.125 0.125 0.125}
\newrgbcolor{PST@COLOR112}{0.118 0.118 0.118}
\newrgbcolor{PST@COLOR113}{0.11 0.11 0.11}
\newrgbcolor{PST@COLOR114}{0.102 0.102 0.102}
\newrgbcolor{PST@COLOR115}{0.094 0.094 0.094}
\newrgbcolor{PST@COLOR116}{0.086 0.086 0.086}
\newrgbcolor{PST@COLOR117}{0.078 0.078 0.078}
\newrgbcolor{PST@COLOR118}{0.07 0.07 0.07}
\newrgbcolor{PST@COLOR119}{0.062 0.062 0.062}
\newrgbcolor{PST@COLOR120}{0.055 0.055 0.055}
\newrgbcolor{PST@COLOR121}{0.047 0.047 0.047}
\newrgbcolor{PST@COLOR122}{0.039 0.039 0.039}
\newrgbcolor{PST@COLOR123}{0.031 0.031 0.031}
\newrgbcolor{PST@COLOR124}{0.023 0.023 0.023}
\newrgbcolor{PST@COLOR125}{0.015 0.015 0.015}
\newrgbcolor{PST@COLOR126}{0.007 0.007 0.007}
\newrgbcolor{PST@COLOR127}{0 0 0}

\def\polypmIIId#1{\pspolygon[linestyle=none,fillstyle=solid,fillcolor=PST@COLOR#1]}

\polypmIIId{13} (0.0568,0.19)  (0.0,0.19)  (0.0,0.1078)(0.0568,0.1078)
\polypmIIId{75}  (0.0568,0.272) (0.0,0.272) (0.0,0.19)  (0.0568,0.19)
\polypmIIId{98}  (0.0568,0.3542)(0.0,0.3542)(0.0,0.272) (0.0568,0.272)

\polypmIIId{13} (0.1136,   0.19)  (0.0568,0.19)  (0.0568,0.1078)(0.1136,0.1078)
\polypmIIId{72}  (0.1136,   0.272) (0.0568,0.272) (0.0568,0.19)  (0.1136,0.19)
\polypmIIId{100}  (0.1136,   0.3542)(0.0568,0.3542)(0.0568,0.272) (0.1136,0.272)

\polypmIIId{21}(0.1704,0.19)  (0.1136,   0.19)  (0.1136,   0.1078)(0.1704,0.1078)
\polypmIIId{72} (0.1704,0.272) (0.1136,   0.272) (0.1136,   0.19)  (0.1704,0.19)
\polypmIIId{99}  (0.1704,0.3542)(0.1136,   0.3542)(0.1136,   0.272) (0.1704,0.272)

\polypmIIId{19}(0.2272,0.19)  (0.1704,0.19)  (0.1704,0.1078)(0.2272,0.1078)
\polypmIIId{76}  (0.2272,0.272) (0.1704,0.272) (0.1704,0.19)  (0.2272,0.19)
\polypmIIId{101}  (0.2272,0.3542)(0.1704,0.3542)(0.1704,0.272) (0.2272,0.272)

\rput(0.0284,0.07){3}
\rput(0.0852,0.07){4}
\rput(0.1420,0.07){5}
\rput(0.1988,0.07){6}
\rput(0.1136,0.0070){years}

\PST@Border(0.0,0.3542)
(0.0,0.1078)
(0.2272,0.1078)
(0.2272,0.3542)
(0.0,0.3542)

\polypmIIId{0}(0.2329,0.1078)(0.2442,0.1078)(0.2442,0.1098)(0.2329,0.1098)
\polypmIIId{1}(0.2329,0.1097)(0.2442,0.1097)(0.2442,0.1117)(0.2329,0.1117)
\polypmIIId{2}(0.2329,0.1116)(0.2442,0.1116)(0.2442,0.1136)(0.2329,0.1136)
\polypmIIId{3}(0.2329,0.1135)(0.2442,0.1135)(0.2442,0.1156)(0.2329,0.1156)
\polypmIIId{4}(0.2329,0.1155)(0.2442,0.1155)(0.2442,0.1175)(0.2329,0.1175)
\polypmIIId{5}(0.2329,0.1174)(0.2442,0.1174)(0.2442,0.1194)(0.2329,0.1194)
\polypmIIId{6}(0.2329,0.1193)(0.2442,0.1193)(0.2442,0.1213)(0.2329,0.1213)
\polypmIIId{7}(0.2329,0.1212)(0.2442,0.1212)(0.2442,0.1233)(0.2329,0.1233)
\polypmIIId{8}(0.2329,0.1232)(0.2442,0.1232)(0.2442,0.1252)(0.2329,0.1252)
\polypmIIId{9}(0.2329,0.1251)(0.2442,0.1251)(0.2442,0.1271)(0.2329,0.1271)
\polypmIIId{10}(0.2329,0.127)(0.2442,0.127)(0.2442,0.129)(0.2329,0.129)
\polypmIIId{11}(0.2329,0.1289)(0.2442,0.1289)(0.2442,0.131)(0.2329,0.131)
\polypmIIId{12}(0.2329,0.1309)(0.2442,0.1309)(0.2442,0.1329)(0.2329,0.1329)
\polypmIIId{13}(0.2329,0.1328)(0.2442,0.1328)(0.2442,0.1348)(0.2329,0.1348)
\polypmIIId{14}(0.2329,0.1347)(0.2442,0.1347)(0.2442,0.1367)(0.2329,0.1367)
\polypmIIId{15}(0.2329,0.1366)(0.2442,0.1366)(0.2442,0.1387)(0.2329,0.1387)
\polypmIIId{16}(0.2329,0.1386)(0.2442,0.1386)(0.2442,0.1406)(0.2329,0.1406)
\polypmIIId{17}(0.2329,0.1405)(0.2442,0.1405)(0.2442,0.1425)(0.2329,0.1425)
\polypmIIId{18}(0.2329,0.1424)(0.2442,0.1424)(0.2442,0.1444)(0.2329,0.1444)
\polypmIIId{19}(0.2329,0.1443)(0.2442,0.1443)(0.2442,0.1464)(0.2329,0.1464)
\polypmIIId{20}(0.2329,0.1463)(0.2442,0.1463)(0.2442,0.1483)(0.2329,0.1483)
\polypmIIId{21}(0.2329,0.1482)(0.2442,0.1482)(0.2442,0.1502)(0.2329,0.1502)
\polypmIIId{22}(0.2329,0.1501)(0.2442,0.1501)(0.2442,0.1521)(0.2329,0.1521)
\polypmIIId{23}(0.2329,0.152)(0.2442,0.152)(0.2442,0.1541)(0.2329,0.1541)
\polypmIIId{24}(0.2329,0.154)(0.2442,0.154)(0.2442,0.156)(0.2329,0.156)
\polypmIIId{25}(0.2329,0.1559)(0.2442,0.1559)(0.2442,0.1579)(0.2329,0.1579)
\polypmIIId{26}(0.2329,0.1578)(0.2442,0.1578)(0.2442,0.1598)(0.2329,0.1598)
\polypmIIId{27}(0.2329,0.1597)(0.2442,0.1597)(0.2442,0.1618)(0.2329,0.1618)
\polypmIIId{28}(0.2329,0.1617)(0.2442,0.1617)(0.2442,0.1637)(0.2329,0.1637)
\polypmIIId{29}(0.2329,0.1636)(0.2442,0.1636)(0.2442,0.1656)(0.2329,0.1656)
\polypmIIId{30}(0.2329,0.1655)(0.2442,0.1655)(0.2442,0.1675)(0.2329,0.1675)
\polypmIIId{31}(0.2329,0.1674)(0.2442,0.1674)(0.2442,0.1695)(0.2329,0.1695)
\polypmIIId{32}(0.2329,0.1694)(0.2442,0.1694)(0.2442,0.1714)(0.2329,0.1714)
\polypmIIId{33}(0.2329,0.1713)(0.2442,0.1713)(0.2442,0.1733)(0.2329,0.1733)
\polypmIIId{34}(0.2329,0.1732)(0.2442,0.1732)(0.2442,0.1752)(0.2329,0.1752)
\polypmIIId{35}(0.2329,0.1751)(0.2442,0.1751)(0.2442,0.1772)(0.2329,0.1772)
\polypmIIId{36}(0.2329,0.1771)(0.2442,0.1771)(0.2442,0.1791)(0.2329,0.1791)
\polypmIIId{37}(0.2329,0.179)(0.2442,0.179)(0.2442,0.181)(0.2329,0.181)
\polypmIIId{38}(0.2329,0.1809)(0.2442,0.1809)(0.2442,0.1829)(0.2329,0.1829)
\polypmIIId{39}(0.2329,0.1828)(0.2442,0.1828)(0.2442,0.1849)(0.2329,0.1849)
\polypmIIId{40}(0.2329,0.1848)(0.2442,0.1848)(0.2442,0.1868)(0.2329,0.1868)
\polypmIIId{41}(0.2329,0.1867)(0.2442,0.1867)(0.2442,0.1887)(0.2329,0.1887)
\polypmIIId{42}(0.2329,0.1886)(0.2442,0.1886)(0.2442,0.1906)(0.2329,0.1906)
\polypmIIId{43}(0.2329,0.1905)(0.2442,0.1905)(0.2442,0.1926)(0.2329,0.1926)
\polypmIIId{44}(0.2329,0.1925)(0.2442,0.1925)(0.2442,0.1945)(0.2329,0.1945)
\polypmIIId{45}(0.2329,0.1944)(0.2442,0.1944)(0.2442,0.1964)(0.2329,0.1964)
\polypmIIId{46}(0.2329,0.1963)(0.2442,0.1963)(0.2442,0.1983)(0.2329,0.1983)
\polypmIIId{47}(0.2329,0.1982)(0.2442,0.1982)(0.2442,0.2003)(0.2329,0.2003)
\polypmIIId{48}(0.2329,0.2002)(0.2442,0.2002)(0.2442,0.2022)(0.2329,0.2022)
\polypmIIId{49}(0.2329,0.2021)(0.2442,0.2021)(0.2442,0.2041)(0.2329,0.2041)
\polypmIIId{50}(0.2329,0.204)(0.2442,0.204)(0.2442,0.206)(0.2329,0.206)
\polypmIIId{51}(0.2329,0.2059)(0.2442,0.2059)(0.2442,0.208)(0.2329,0.208)
\polypmIIId{52}(0.2329,0.2079)(0.2442,0.2079)(0.2442,0.2099)(0.2329,0.2099)
\polypmIIId{53}(0.2329,0.2098)(0.2442,0.2098)(0.2442,0.2118)(0.2329,0.2118)
\polypmIIId{54}(0.2329,0.2117)(0.2442,0.2117)(0.2442,0.2137)(0.2329,0.2137)
\polypmIIId{55}(0.2329,0.2136)(0.2442,0.2136)(0.2442,0.2157)(0.2329,0.2157)
\polypmIIId{56}(0.2329,0.2156)(0.2442,0.2156)(0.2442,0.2176)(0.2329,0.2176)
\polypmIIId{57}(0.2329,0.2175)(0.2442,0.2175)(0.2442,0.2195)(0.2329,0.2195)
\polypmIIId{58}(0.2329,0.2194)(0.2442,0.2194)(0.2442,0.2214)(0.2329,0.2214)
\polypmIIId{59}(0.2329,0.2213)(0.2442,0.2213)(0.2442,0.2234)(0.2329,0.2234)
\polypmIIId{60}(0.2329,0.2233)(0.2442,0.2233)(0.2442,0.2253)(0.2329,0.2253)
\polypmIIId{61}(0.2329,0.2252)(0.2442,0.2252)(0.2442,0.2272)(0.2329,0.2272)
\polypmIIId{62}(0.2329,0.2271)(0.2442,0.2271)(0.2442,0.2291)(0.2329,0.2291)
\polypmIIId{63}(0.2329,0.229)(0.2442,0.229)(0.2442,0.2311)(0.2329,0.2311)
\polypmIIId{64}(0.2329,0.231)(0.2442,0.231)(0.2442,0.233)(0.2329,0.233)
\polypmIIId{65}(0.2329,0.2329)(0.2442,0.2329)(0.2442,0.2349)(0.2329,0.2349)
\polypmIIId{66}(0.2329,0.2348)(0.2442,0.2348)(0.2442,0.2368)(0.2329,0.2368)
\polypmIIId{67}(0.2329,0.2367)(0.2442,0.2367)(0.2442,0.2388)(0.2329,0.2388)
\polypmIIId{68}(0.2329,0.2387)(0.2442,0.2387)(0.2442,0.2407)(0.2329,0.2407)
\polypmIIId{69}(0.2329,0.2406)(0.2442,0.2406)(0.2442,0.2426)(0.2329,0.2426)
\polypmIIId{70}(0.2329,0.2425)(0.2442,0.2425)(0.2442,0.2445)(0.2329,0.2445)
\polypmIIId{71}(0.2329,0.2444)(0.2442,0.2444)(0.2442,0.2465)(0.2329,0.2465)
\polypmIIId{72}(0.2329,0.2464)(0.2442,0.2464)(0.2442,0.2484)(0.2329,0.2484)
\polypmIIId{73}(0.2329,0.2483)(0.2442,0.2483)(0.2442,0.2503)(0.2329,0.2503)
\polypmIIId{74}(0.2329,0.2502)(0.2442,0.2502)(0.2442,0.2522)(0.2329,0.2522)
\polypmIIId{75}(0.2329,0.2521)(0.2442,0.2521)(0.2442,0.2542)(0.2329,0.2542)
\polypmIIId{76}(0.2329,0.2541)(0.2442,0.2541)(0.2442,0.2561)(0.2329,0.2561)
\polypmIIId{77}(0.2329,0.256)(0.2442,0.256)(0.2442,0.258)(0.2329,0.258)
\polypmIIId{78}(0.2329,0.2579)(0.2442,0.2579)(0.2442,0.2599)(0.2329,0.2599)
\polypmIIId{79}(0.2329,0.2598)(0.2442,0.2598)(0.2442,0.2619)(0.2329,0.2619)
\polypmIIId{80}(0.2329,0.2618)(0.2442,0.2618)(0.2442,0.2638)(0.2329,0.2638)
\polypmIIId{81}(0.2329,0.2637)(0.2442,0.2637)(0.2442,0.2657)(0.2329,0.2657)
\polypmIIId{82}(0.2329,0.2656)(0.2442,0.2656)(0.2442,0.2676)(0.2329,0.2676)
\polypmIIId{83}(0.2329,0.2675)(0.2442,0.2675)(0.2442,0.2696)(0.2329,0.2696)
\polypmIIId{84}(0.2329,0.2695)(0.2442,0.2695)(0.2442,0.2715)(0.2329,0.2715)
\polypmIIId{85}(0.2329,0.2714)(0.2442,0.2714)(0.2442,0.2734)(0.2329,0.2734)
\polypmIIId{86}(0.2329,0.2733)(0.2442,0.2733)(0.2442,0.2753)(0.2329,0.2753)
\polypmIIId{87}(0.2329,0.2752)(0.2442,0.2752)(0.2442,0.2773)(0.2329,0.2773)
\polypmIIId{88}(0.2329,0.2772)(0.2442,0.2772)(0.2442,0.2792)(0.2329,0.2792)
\polypmIIId{89}(0.2329,0.2791)(0.2442,0.2791)(0.2442,0.2811)(0.2329,0.2811)
\polypmIIId{90}(0.2329,0.281)(0.2442,0.281)(0.2442,0.283)(0.2329,0.283)
\polypmIIId{91}(0.2329,0.2829)(0.2442,0.2829)(0.2442,0.285)(0.2329,0.285)
\polypmIIId{92}(0.2329,0.2849)(0.2442,0.2849)(0.2442,0.2869)(0.2329,0.2869)
\polypmIIId{93}(0.2329,0.2868)(0.2442,0.2868)(0.2442,0.2888)(0.2329,0.2888)
\polypmIIId{94}(0.2329,0.2887)(0.2442,0.2887)(0.2442,0.2907)(0.2329,0.2907)
\polypmIIId{95}(0.2329,0.2906)(0.2442,0.2906)(0.2442,0.2927)(0.2329,0.2927)
\polypmIIId{96}(0.2329,0.2926)(0.2442,0.2926)(0.2442,0.2946)(0.2329,0.2946)
\polypmIIId{97}(0.2329,0.2945)(0.2442,0.2945)(0.2442,0.2965)(0.2329,0.2965)
\polypmIIId{98}(0.2329,0.2964)(0.2442,0.2964)(0.2442,0.2984)(0.2329,0.2984)
\polypmIIId{99}(0.2329,0.2983)(0.2442,0.2983)(0.2442,0.3004)(0.2329,0.3004)
\polypmIIId{100}(0.2329,0.3003)(0.2442,0.3003)(0.2442,0.3023)(0.2329,0.3023)
\polypmIIId{101}(0.2329,0.3022)(0.2442,0.3022)(0.2442,0.3042)(0.2329,0.3042)
\polypmIIId{102}(0.2329,0.3041)(0.2442,0.3041)(0.2442,0.3061)(0.2329,0.3061)
\polypmIIId{103}(0.2329,0.306)(0.2442,0.306)(0.2442,0.3081)(0.2329,0.3081)
\polypmIIId{104}(0.2329,0.308)(0.2442,0.308)(0.2442,0.31)(0.2329,0.31)
\polypmIIId{105}(0.2329,0.3099)(0.2442,0.3099)(0.2442,0.3119)(0.2329,0.3119)
\polypmIIId{106}(0.2329,0.3118)(0.2442,0.3118)(0.2442,0.3138)(0.2329,0.3138)
\polypmIIId{107}(0.2329,0.3137)(0.2442,0.3137)(0.2442,0.3158)(0.2329,0.3158)
\polypmIIId{108}(0.2329,0.3157)(0.2442,0.3157)(0.2442,0.3177)(0.2329,0.3177)
\polypmIIId{109}(0.2329,0.3176)(0.2442,0.3176)(0.2442,0.3196)(0.2329,0.3196)
\polypmIIId{110}(0.2329,0.3195)(0.2442,0.3195)(0.2442,0.3215)(0.2329,0.3215)
\polypmIIId{111}(0.2329,0.3214)(0.2442,0.3214)(0.2442,0.3235)(0.2329,0.3235)
\polypmIIId{112}(0.2329,0.3234)(0.2442,0.3234)(0.2442,0.3254)(0.2329,0.3254)
\polypmIIId{113}(0.2329,0.3253)(0.2442,0.3253)(0.2442,0.3273)(0.2329,0.3273)
\polypmIIId{114}(0.2329,0.3272)(0.2442,0.3272)(0.2442,0.3292)(0.2329,0.3292)
\polypmIIId{115}(0.2329,0.3291)(0.2442,0.3291)(0.2442,0.3312)(0.2329,0.3312)
\polypmIIId{116}(0.2329,0.3311)(0.2442,0.3311)(0.2442,0.3331)(0.2329,0.3331)
\polypmIIId{117}(0.2329,0.333)(0.2442,0.333)(0.2442,0.335)(0.2329,0.335)
\polypmIIId{118}(0.2329,0.3349)(0.2442,0.3349)(0.2442,0.3369)(0.2329,0.3369)
\polypmIIId{119}(0.2329,0.3368)(0.2442,0.3368)(0.2442,0.3389)(0.2329,0.3389)
\polypmIIId{120}(0.2329,0.3388)(0.2442,0.3388)(0.2442,0.3408)(0.2329,0.3408)
\polypmIIId{121}(0.2329,0.3407)(0.2442,0.3407)(0.2442,0.3427)(0.2329,0.3427)
\polypmIIId{122}(0.2329,0.3426)(0.2442,0.3426)(0.2442,0.3446)(0.2329,0.3446)
\polypmIIId{123}(0.2329,0.3445)(0.2442,0.3445)(0.2442,0.3466)(0.2329,0.3466)
\polypmIIId{124}(0.2329,0.3465)(0.2442,0.3465)(0.2442,0.3485)(0.2329,0.3485)
\polypmIIId{125}(0.2329,0.3484)(0.2442,0.3484)(0.2442,0.3504)(0.2329,0.3504)
\polypmIIId{126}(0.2329,0.3503)(0.2442,0.3503)(0.2442,0.3523)(0.2329,0.3523)
\polypmIIId{127}(0.2329,0.3522)(0.2442,0.3522)(0.2442,0.3542)(0.2329,0.3542)

\PST@Border(0.2329,0.1078)
(0.2442,0.1078)
(0.2442,0.3542)
(0.2329,0.3542)
(0.2329,0.1078)


\rput[l](0.2502,0.1301){0.997}
\rput[l](0.2502,0.2048){0.998}
\rput[l](0.2502,0.2795){0.999}
\rput[l](0.2502,0.3542){1}

\catcode`@=12
\fi
\endpspicture}
  \caption{GALP solution quality on weakly correlated instances ($\alpha = 0.1$).}
  \label{fig:greedysolcomp01}
\end{figure}

\begin{figure}[H]
  \centering
    \subfloat[1 resource]{% GNUPLOT: LaTeX picture using PSTRICKS macros
% Define new PST objects, if not already defined
\ifx\PSTloaded\undefined
\def\PSTloaded{t}

\catcode`@=11

\newpsobject{PST@Border}{psline}{linewidth=.0015,linestyle=solid}

\catcode`@=12

\fi
\psset{unit=5.0in,xunit=5.0in,yunit=3.0in}
\pspicture(0.000000,0.000000)(0.31, 0.35)
\ifx\nofigs\undefined
\catcode`@=11

\newrgbcolor{PST@COLOR0}{1 1 1}
\newrgbcolor{PST@COLOR1}{0.992 0.992 0.992}
\newrgbcolor{PST@COLOR2}{0.984 0.984 0.984}
\newrgbcolor{PST@COLOR3}{0.976 0.976 0.976}
\newrgbcolor{PST@COLOR4}{0.968 0.968 0.968}
\newrgbcolor{PST@COLOR5}{0.96 0.96 0.96}
\newrgbcolor{PST@COLOR6}{0.952 0.952 0.952}
\newrgbcolor{PST@COLOR7}{0.944 0.944 0.944}
\newrgbcolor{PST@COLOR8}{0.937 0.937 0.937}
\newrgbcolor{PST@COLOR9}{0.929 0.929 0.929}
\newrgbcolor{PST@COLOR10}{0.921 0.921 0.921}
\newrgbcolor{PST@COLOR11}{0.913 0.913 0.913}
\newrgbcolor{PST@COLOR12}{0.905 0.905 0.905}
\newrgbcolor{PST@COLOR13}{0.897 0.897 0.897}
\newrgbcolor{PST@COLOR14}{0.889 0.889 0.889}
\newrgbcolor{PST@COLOR15}{0.881 0.881 0.881}
\newrgbcolor{PST@COLOR16}{0.874 0.874 0.874}
\newrgbcolor{PST@COLOR17}{0.866 0.866 0.866}
\newrgbcolor{PST@COLOR18}{0.858 0.858 0.858}
\newrgbcolor{PST@COLOR19}{0.85 0.85 0.85}
\newrgbcolor{PST@COLOR20}{0.842 0.842 0.842}
\newrgbcolor{PST@COLOR21}{0.834 0.834 0.834}
\newrgbcolor{PST@COLOR22}{0.826 0.826 0.826}
\newrgbcolor{PST@COLOR23}{0.818 0.818 0.818}
\newrgbcolor{PST@COLOR24}{0.811 0.811 0.811}
\newrgbcolor{PST@COLOR25}{0.803 0.803 0.803}
\newrgbcolor{PST@COLOR26}{0.795 0.795 0.795}
\newrgbcolor{PST@COLOR27}{0.787 0.787 0.787}
\newrgbcolor{PST@COLOR28}{0.779 0.779 0.779}
\newrgbcolor{PST@COLOR29}{0.771 0.771 0.771}
\newrgbcolor{PST@COLOR30}{0.763 0.763 0.763}
\newrgbcolor{PST@COLOR31}{0.755 0.755 0.755}
\newrgbcolor{PST@COLOR32}{0.748 0.748 0.748}
\newrgbcolor{PST@COLOR33}{0.74 0.74 0.74}
\newrgbcolor{PST@COLOR34}{0.732 0.732 0.732}
\newrgbcolor{PST@COLOR35}{0.724 0.724 0.724}
\newrgbcolor{PST@COLOR36}{0.716 0.716 0.716}
\newrgbcolor{PST@COLOR37}{0.708 0.708 0.708}
\newrgbcolor{PST@COLOR38}{0.7 0.7 0.7}
\newrgbcolor{PST@COLOR39}{0.692 0.692 0.692}
\newrgbcolor{PST@COLOR40}{0.685 0.685 0.685}
\newrgbcolor{PST@COLOR41}{0.677 0.677 0.677}
\newrgbcolor{PST@COLOR42}{0.669 0.669 0.669}
\newrgbcolor{PST@COLOR43}{0.661 0.661 0.661}
\newrgbcolor{PST@COLOR44}{0.653 0.653 0.653}
\newrgbcolor{PST@COLOR45}{0.645 0.645 0.645}
\newrgbcolor{PST@COLOR46}{0.637 0.637 0.637}
\newrgbcolor{PST@COLOR47}{0.629 0.629 0.629}
\newrgbcolor{PST@COLOR48}{0.622 0.622 0.622}
\newrgbcolor{PST@COLOR49}{0.614 0.614 0.614}
\newrgbcolor{PST@COLOR50}{0.606 0.606 0.606}
\newrgbcolor{PST@COLOR51}{0.598 0.598 0.598}
\newrgbcolor{PST@COLOR52}{0.59 0.59 0.59}
\newrgbcolor{PST@COLOR53}{0.582 0.582 0.582}
\newrgbcolor{PST@COLOR54}{0.574 0.574 0.574}
\newrgbcolor{PST@COLOR55}{0.566 0.566 0.566}
\newrgbcolor{PST@COLOR56}{0.559 0.559 0.559}
\newrgbcolor{PST@COLOR57}{0.551 0.551 0.551}
\newrgbcolor{PST@COLOR58}{0.543 0.543 0.543}
\newrgbcolor{PST@COLOR59}{0.535 0.535 0.535}
\newrgbcolor{PST@COLOR60}{0.527 0.527 0.527}
\newrgbcolor{PST@COLOR61}{0.519 0.519 0.519}
\newrgbcolor{PST@COLOR62}{0.511 0.511 0.511}
\newrgbcolor{PST@COLOR63}{0.503 0.503 0.503}
\newrgbcolor{PST@COLOR64}{0.496 0.496 0.496}
\newrgbcolor{PST@COLOR65}{0.488 0.488 0.488}
\newrgbcolor{PST@COLOR66}{0.48 0.48 0.48}
\newrgbcolor{PST@COLOR67}{0.472 0.472 0.472}
\newrgbcolor{PST@COLOR68}{0.464 0.464 0.464}
\newrgbcolor{PST@COLOR69}{0.456 0.456 0.456}
\newrgbcolor{PST@COLOR70}{0.448 0.448 0.448}
\newrgbcolor{PST@COLOR71}{0.44 0.44 0.44}
\newrgbcolor{PST@COLOR72}{0.433 0.433 0.433}
\newrgbcolor{PST@COLOR73}{0.425 0.425 0.425}
\newrgbcolor{PST@COLOR74}{0.417 0.417 0.417}
\newrgbcolor{PST@COLOR75}{0.409 0.409 0.409}
\newrgbcolor{PST@COLOR76}{0.401 0.401 0.401}
\newrgbcolor{PST@COLOR77}{0.393 0.393 0.393}
\newrgbcolor{PST@COLOR78}{0.385 0.385 0.385}
\newrgbcolor{PST@COLOR79}{0.377 0.377 0.377}
\newrgbcolor{PST@COLOR80}{0.37 0.37 0.37}
\newrgbcolor{PST@COLOR81}{0.362 0.362 0.362}
\newrgbcolor{PST@COLOR82}{0.354 0.354 0.354}
\newrgbcolor{PST@COLOR83}{0.346 0.346 0.346}
\newrgbcolor{PST@COLOR84}{0.338 0.338 0.338}
\newrgbcolor{PST@COLOR85}{0.33 0.33 0.33}
\newrgbcolor{PST@COLOR86}{0.322 0.322 0.322}
\newrgbcolor{PST@COLOR87}{0.314 0.314 0.314}
\newrgbcolor{PST@COLOR88}{0.307 0.307 0.307}
\newrgbcolor{PST@COLOR89}{0.299 0.299 0.299}
\newrgbcolor{PST@COLOR90}{0.291 0.291 0.291}
\newrgbcolor{PST@COLOR91}{0.283 0.283 0.283}
\newrgbcolor{PST@COLOR92}{0.275 0.275 0.275}
\newrgbcolor{PST@COLOR93}{0.267 0.267 0.267}
\newrgbcolor{PST@COLOR94}{0.259 0.259 0.259}
\newrgbcolor{PST@COLOR95}{0.251 0.251 0.251}
\newrgbcolor{PST@COLOR96}{0.244 0.244 0.244}
\newrgbcolor{PST@COLOR97}{0.236 0.236 0.236}
\newrgbcolor{PST@COLOR98}{0.228 0.228 0.228}
\newrgbcolor{PST@COLOR99}{0.22 0.22 0.22}
\newrgbcolor{PST@COLOR100}{0.212 0.212 0.212}
\newrgbcolor{PST@COLOR101}{0.204 0.204 0.204}
\newrgbcolor{PST@COLOR102}{0.196 0.196 0.196}
\newrgbcolor{PST@COLOR103}{0.188 0.188 0.188}
\newrgbcolor{PST@COLOR104}{0.181 0.181 0.181}
\newrgbcolor{PST@COLOR105}{0.173 0.173 0.173}
\newrgbcolor{PST@COLOR106}{0.165 0.165 0.165}
\newrgbcolor{PST@COLOR107}{0.157 0.157 0.157}
\newrgbcolor{PST@COLOR108}{0.149 0.149 0.149}
\newrgbcolor{PST@COLOR109}{0.141 0.141 0.141}
\newrgbcolor{PST@COLOR110}{0.133 0.133 0.133}
\newrgbcolor{PST@COLOR111}{0.125 0.125 0.125}
\newrgbcolor{PST@COLOR112}{0.118 0.118 0.118}
\newrgbcolor{PST@COLOR113}{0.11 0.11 0.11}
\newrgbcolor{PST@COLOR114}{0.102 0.102 0.102}
\newrgbcolor{PST@COLOR115}{0.094 0.094 0.094}
\newrgbcolor{PST@COLOR116}{0.086 0.086 0.086}
\newrgbcolor{PST@COLOR117}{0.078 0.078 0.078}
\newrgbcolor{PST@COLOR118}{0.07 0.07 0.07}
\newrgbcolor{PST@COLOR119}{0.062 0.062 0.062}
\newrgbcolor{PST@COLOR120}{0.055 0.055 0.055}
\newrgbcolor{PST@COLOR121}{0.047 0.047 0.047}
\newrgbcolor{PST@COLOR122}{0.039 0.039 0.039}
\newrgbcolor{PST@COLOR123}{0.031 0.031 0.031}
\newrgbcolor{PST@COLOR124}{0.023 0.023 0.023}
\newrgbcolor{PST@COLOR125}{0.015 0.015 0.015}
\newrgbcolor{PST@COLOR126}{0.007 0.007 0.007}
\newrgbcolor{PST@COLOR127}{0 0 0}


\def\polypmIIId#1{\pspolygon[linestyle=none,fillstyle=solid,fillcolor=PST@COLOR#1]}

\polypmIIId{116}(0.1432,0.19)(0.0864,0.19)(0.0864,0.1078)(0.1432,0.1078)
\polypmIIId{122}(0.1432,0.272)(0.0864,0.272)(0.0864,0.19)(0.1432,0.19)
\polypmIIId{125}(0.1432,0.3542)(0.0864,0.3542)(0.0864,0.272)(0.1432,0.272)

\polypmIIId{118}(0.2,0.19)(0.1432,0.19)(0.1432,0.1078)(0.2,0.1078)
\polypmIIId{122}(0.2,0.272)(0.1432,0.272)(0.1432,0.19)(0.2,0.19)
\polypmIIId{125}(0.2,0.3542)(0.1432,0.3542)(0.1432,0.272)(0.2,0.272)

\polypmIIId{119}(0.2568,0.19)(0.2,0.19)(0.2,0.1078)(0.2568,0.1078)
\polypmIIId{124}(0.2568,0.272)(0.2,0.272)(0.2,0.19)(0.2568,0.19)
\polypmIIId{125}(0.2568,0.3542)(0.2,0.3542)(0.2,0.272)(0.2568,0.272)

\polypmIIId{120}(0.3136,0.19)(0.2568,0.19)(0.2568,0.1078)(0.3136,0.1078)
\polypmIIId{125}(0.3136,0.272)(0.2568,0.272)(0.2568,0.19)(0.3136,0.19)
\polypmIIId{125}(0.3136,0.3542)(0.2568,0.3542)(0.2568,0.272)(0.3136,0.272)

\rput(0.1148,0.07){3}
\rput(0.1716,0.07){4}
\rput(0.2284,0.07){5}
\rput(0.2852,0.07){6}
\rput(0.2000,0.0070){years}

\rput[r](0.0806,0.1489){25}
\rput[r](0.0806,0.2310){50}
\rput[r](0.0806,0.3131){100}
\rput{L}(0.0096,0.2310){actions}

\PST@Border(0.0864,0.3542)
(0.0864,0.1078)
(0.3136,0.1078)
(0.3136,0.3542)
(0.0864,0.3542)

\catcode`@=12
\fi
\endpspicture} 
    \subfloat[2 resources]{% GNUPLOT: LaTeX picture using PSTRICKS macros
% Define new PST objects, if not already defined
\ifx\PSTloaded\undefined
\def\PSTloaded{t}

\catcode`@=11

\newpsobject{PST@Border}{psline}{linewidth=.0015,linestyle=solid}

\catcode`@=12

\fi
\psset{unit=5.0in,xunit=5.0in,yunit=3.0in}
\pspicture(0.000000,0.000000)(0.225000,0.35)
\ifx\nofigs\undefined
\catcode`@=11

\newrgbcolor{PST@COLOR0}{1 1 1}
\newrgbcolor{PST@COLOR1}{0.992 0.992 0.992}
\newrgbcolor{PST@COLOR2}{0.984 0.984 0.984}
\newrgbcolor{PST@COLOR3}{0.976 0.976 0.976}
\newrgbcolor{PST@COLOR4}{0.968 0.968 0.968}
\newrgbcolor{PST@COLOR5}{0.96 0.96 0.96}
\newrgbcolor{PST@COLOR6}{0.952 0.952 0.952}
\newrgbcolor{PST@COLOR7}{0.944 0.944 0.944}
\newrgbcolor{PST@COLOR8}{0.937 0.937 0.937}
\newrgbcolor{PST@COLOR9}{0.929 0.929 0.929}
\newrgbcolor{PST@COLOR10}{0.921 0.921 0.921}
\newrgbcolor{PST@COLOR11}{0.913 0.913 0.913}
\newrgbcolor{PST@COLOR12}{0.905 0.905 0.905}
\newrgbcolor{PST@COLOR13}{0.897 0.897 0.897}
\newrgbcolor{PST@COLOR14}{0.889 0.889 0.889}
\newrgbcolor{PST@COLOR15}{0.881 0.881 0.881}
\newrgbcolor{PST@COLOR16}{0.874 0.874 0.874}
\newrgbcolor{PST@COLOR17}{0.866 0.866 0.866}
\newrgbcolor{PST@COLOR18}{0.858 0.858 0.858}
\newrgbcolor{PST@COLOR19}{0.85 0.85 0.85}
\newrgbcolor{PST@COLOR20}{0.842 0.842 0.842}
\newrgbcolor{PST@COLOR21}{0.834 0.834 0.834}
\newrgbcolor{PST@COLOR22}{0.826 0.826 0.826}
\newrgbcolor{PST@COLOR23}{0.818 0.818 0.818}
\newrgbcolor{PST@COLOR24}{0.811 0.811 0.811}
\newrgbcolor{PST@COLOR25}{0.803 0.803 0.803}
\newrgbcolor{PST@COLOR26}{0.795 0.795 0.795}
\newrgbcolor{PST@COLOR27}{0.787 0.787 0.787}
\newrgbcolor{PST@COLOR28}{0.779 0.779 0.779}
\newrgbcolor{PST@COLOR29}{0.771 0.771 0.771}
\newrgbcolor{PST@COLOR30}{0.763 0.763 0.763}
\newrgbcolor{PST@COLOR31}{0.755 0.755 0.755}
\newrgbcolor{PST@COLOR32}{0.748 0.748 0.748}
\newrgbcolor{PST@COLOR33}{0.74 0.74 0.74}
\newrgbcolor{PST@COLOR34}{0.732 0.732 0.732}
\newrgbcolor{PST@COLOR35}{0.724 0.724 0.724}
\newrgbcolor{PST@COLOR36}{0.716 0.716 0.716}
\newrgbcolor{PST@COLOR37}{0.708 0.708 0.708}
\newrgbcolor{PST@COLOR38}{0.7 0.7 0.7}
\newrgbcolor{PST@COLOR39}{0.692 0.692 0.692}
\newrgbcolor{PST@COLOR40}{0.685 0.685 0.685}
\newrgbcolor{PST@COLOR41}{0.677 0.677 0.677}
\newrgbcolor{PST@COLOR42}{0.669 0.669 0.669}
\newrgbcolor{PST@COLOR43}{0.661 0.661 0.661}
\newrgbcolor{PST@COLOR44}{0.653 0.653 0.653}
\newrgbcolor{PST@COLOR45}{0.645 0.645 0.645}
\newrgbcolor{PST@COLOR46}{0.637 0.637 0.637}
\newrgbcolor{PST@COLOR47}{0.629 0.629 0.629}
\newrgbcolor{PST@COLOR48}{0.622 0.622 0.622}
\newrgbcolor{PST@COLOR49}{0.614 0.614 0.614}
\newrgbcolor{PST@COLOR50}{0.606 0.606 0.606}
\newrgbcolor{PST@COLOR51}{0.598 0.598 0.598}
\newrgbcolor{PST@COLOR52}{0.59 0.59 0.59}
\newrgbcolor{PST@COLOR53}{0.582 0.582 0.582}
\newrgbcolor{PST@COLOR54}{0.574 0.574 0.574}
\newrgbcolor{PST@COLOR55}{0.566 0.566 0.566}
\newrgbcolor{PST@COLOR56}{0.559 0.559 0.559}
\newrgbcolor{PST@COLOR57}{0.551 0.551 0.551}
\newrgbcolor{PST@COLOR58}{0.543 0.543 0.543}
\newrgbcolor{PST@COLOR59}{0.535 0.535 0.535}
\newrgbcolor{PST@COLOR60}{0.527 0.527 0.527}
\newrgbcolor{PST@COLOR61}{0.519 0.519 0.519}
\newrgbcolor{PST@COLOR62}{0.511 0.511 0.511}
\newrgbcolor{PST@COLOR63}{0.503 0.503 0.503}
\newrgbcolor{PST@COLOR64}{0.496 0.496 0.496}
\newrgbcolor{PST@COLOR65}{0.488 0.488 0.488}
\newrgbcolor{PST@COLOR66}{0.48 0.48 0.48}
\newrgbcolor{PST@COLOR67}{0.472 0.472 0.472}
\newrgbcolor{PST@COLOR68}{0.464 0.464 0.464}
\newrgbcolor{PST@COLOR69}{0.456 0.456 0.456}
\newrgbcolor{PST@COLOR70}{0.448 0.448 0.448}
\newrgbcolor{PST@COLOR71}{0.44 0.44 0.44}
\newrgbcolor{PST@COLOR72}{0.433 0.433 0.433}
\newrgbcolor{PST@COLOR73}{0.425 0.425 0.425}
\newrgbcolor{PST@COLOR74}{0.417 0.417 0.417}
\newrgbcolor{PST@COLOR75}{0.409 0.409 0.409}
\newrgbcolor{PST@COLOR76}{0.401 0.401 0.401}
\newrgbcolor{PST@COLOR77}{0.393 0.393 0.393}
\newrgbcolor{PST@COLOR78}{0.385 0.385 0.385}
\newrgbcolor{PST@COLOR79}{0.377 0.377 0.377}
\newrgbcolor{PST@COLOR80}{0.37 0.37 0.37}
\newrgbcolor{PST@COLOR81}{0.362 0.362 0.362}
\newrgbcolor{PST@COLOR82}{0.354 0.354 0.354}
\newrgbcolor{PST@COLOR83}{0.346 0.346 0.346}
\newrgbcolor{PST@COLOR84}{0.338 0.338 0.338}
\newrgbcolor{PST@COLOR85}{0.33 0.33 0.33}
\newrgbcolor{PST@COLOR86}{0.322 0.322 0.322}
\newrgbcolor{PST@COLOR87}{0.314 0.314 0.314}
\newrgbcolor{PST@COLOR88}{0.307 0.307 0.307}
\newrgbcolor{PST@COLOR89}{0.299 0.299 0.299}
\newrgbcolor{PST@COLOR90}{0.291 0.291 0.291}
\newrgbcolor{PST@COLOR91}{0.283 0.283 0.283}
\newrgbcolor{PST@COLOR92}{0.275 0.275 0.275}
\newrgbcolor{PST@COLOR93}{0.267 0.267 0.267}
\newrgbcolor{PST@COLOR94}{0.259 0.259 0.259}
\newrgbcolor{PST@COLOR95}{0.251 0.251 0.251}
\newrgbcolor{PST@COLOR96}{0.244 0.244 0.244}
\newrgbcolor{PST@COLOR97}{0.236 0.236 0.236}
\newrgbcolor{PST@COLOR98}{0.228 0.228 0.228}
\newrgbcolor{PST@COLOR99}{0.22 0.22 0.22}
\newrgbcolor{PST@COLOR100}{0.212 0.212 0.212}
\newrgbcolor{PST@COLOR101}{0.204 0.204 0.204}
\newrgbcolor{PST@COLOR102}{0.196 0.196 0.196}
\newrgbcolor{PST@COLOR103}{0.188 0.188 0.188}
\newrgbcolor{PST@COLOR104}{0.181 0.181 0.181}
\newrgbcolor{PST@COLOR105}{0.173 0.173 0.173}
\newrgbcolor{PST@COLOR106}{0.165 0.165 0.165}
\newrgbcolor{PST@COLOR107}{0.157 0.157 0.157}
\newrgbcolor{PST@COLOR108}{0.149 0.149 0.149}
\newrgbcolor{PST@COLOR109}{0.141 0.141 0.141}
\newrgbcolor{PST@COLOR110}{0.133 0.133 0.133}
\newrgbcolor{PST@COLOR111}{0.125 0.125 0.125}
\newrgbcolor{PST@COLOR112}{0.118 0.118 0.118}
\newrgbcolor{PST@COLOR113}{0.11 0.11 0.11}
\newrgbcolor{PST@COLOR114}{0.102 0.102 0.102}
\newrgbcolor{PST@COLOR115}{0.094 0.094 0.094}
\newrgbcolor{PST@COLOR116}{0.086 0.086 0.086}
\newrgbcolor{PST@COLOR117}{0.078 0.078 0.078}
\newrgbcolor{PST@COLOR118}{0.07 0.07 0.07}
\newrgbcolor{PST@COLOR119}{0.062 0.062 0.062}
\newrgbcolor{PST@COLOR120}{0.055 0.055 0.055}
\newrgbcolor{PST@COLOR121}{0.047 0.047 0.047}
\newrgbcolor{PST@COLOR122}{0.039 0.039 0.039}
\newrgbcolor{PST@COLOR123}{0.031 0.031 0.031}
\newrgbcolor{PST@COLOR124}{0.023 0.023 0.023}
\newrgbcolor{PST@COLOR125}{0.015 0.015 0.015}
\newrgbcolor{PST@COLOR126}{0.007 0.007 0.007}
\newrgbcolor{PST@COLOR127}{0 0 0}

\def\polypmIIId#1{\pspolygon[linestyle=none,fillstyle=solid,fillcolor=PST@COLOR#1]}

\polypmIIId{112} (0.0568,0.19)  (0.0,0.19)  (0.0,0.1078)(0.0568,0.1078)
\polypmIIId{121}  (0.0568,0.272) (0.0,0.272) (0.0,0.19)  (0.0568,0.19)
\polypmIIId{123}  (0.0568,0.3542)(0.0,0.3542)(0.0,0.272) (0.0568,0.272)

\polypmIIId{109} (0.1136,   0.19)  (0.0568,0.19)  (0.0568,0.1078)(0.1136,0.1078)
\polypmIIId{120}  (0.1136,   0.272) (0.0568,0.272) (0.0568,0.19)  (0.1136,0.19)
\polypmIIId{123}  (0.1136,   0.3542)(0.0568,0.3542)(0.0568,0.272) (0.1136,0.272)

\polypmIIId{111}(0.1704,0.19)  (0.1136,   0.19)  (0.1136,   0.1078)(0.1704,0.1078)
\polypmIIId{120} (0.1704,0.272) (0.1136,   0.272) (0.1136,   0.19)  (0.1704,0.19)
\polypmIIId{123}  (0.1704,0.3542)(0.1136,   0.3542)(0.1136,   0.272) (0.1704,0.272)

\polypmIIId{110}(0.2272,0.19)  (0.1704,0.19)  (0.1704,0.1078)(0.2272,0.1078)
\polypmIIId{120}  (0.2272,0.272) (0.1704,0.272) (0.1704,0.19)  (0.2272,0.19)
\polypmIIId{123}  (0.2272,0.3542)(0.1704,0.3542)(0.1704,0.272) (0.2272,0.272)

\rput(0.0284,0.07){3}
\rput(0.0852,0.07){4}
\rput(0.1420,0.07){5}
\rput(0.1988,0.07){6}
\rput(0.1136,0.0070){years}


\PST@Border(0.0,0.3542)
(0.0,0.1078)
(0.2272,0.1078)
(0.2272,0.3542)
(0.0,0.3542)

\catcode`@=12
\fi
\endpspicture}
    \subfloat[4 resources]{% GNUPLOT: LaTeX picture using PSTRICKS macros
% Define new PST objects, if not already defined
\ifx\PSTloaded\undefined
\def\PSTloaded{t}

\catcode`@=11

\newpsobject{PST@Border}{psline}{linewidth=.0015,linestyle=solid}

\catcode`@=12

\fi
\psset{unit=5.0in,xunit=5.0in,yunit=3.0in}
\pspicture(0.000000,0.000000)(0.3136,0.35)
\ifx\nofigs\undefined
\catcode`@=11

\newrgbcolor{PST@COLOR0}{1 1 1}
\newrgbcolor{PST@COLOR1}{0.992 0.992 0.992}
\newrgbcolor{PST@COLOR2}{0.984 0.984 0.984}
\newrgbcolor{PST@COLOR3}{0.976 0.976 0.976}
\newrgbcolor{PST@COLOR4}{0.968 0.968 0.968}
\newrgbcolor{PST@COLOR5}{0.96 0.96 0.96}
\newrgbcolor{PST@COLOR6}{0.952 0.952 0.952}
\newrgbcolor{PST@COLOR7}{0.944 0.944 0.944}
\newrgbcolor{PST@COLOR8}{0.937 0.937 0.937}
\newrgbcolor{PST@COLOR9}{0.929 0.929 0.929}
\newrgbcolor{PST@COLOR10}{0.921 0.921 0.921}
\newrgbcolor{PST@COLOR11}{0.913 0.913 0.913}
\newrgbcolor{PST@COLOR12}{0.905 0.905 0.905}
\newrgbcolor{PST@COLOR13}{0.897 0.897 0.897}
\newrgbcolor{PST@COLOR14}{0.889 0.889 0.889}
\newrgbcolor{PST@COLOR15}{0.881 0.881 0.881}
\newrgbcolor{PST@COLOR16}{0.874 0.874 0.874}
\newrgbcolor{PST@COLOR17}{0.866 0.866 0.866}
\newrgbcolor{PST@COLOR18}{0.858 0.858 0.858}
\newrgbcolor{PST@COLOR19}{0.85 0.85 0.85}
\newrgbcolor{PST@COLOR20}{0.842 0.842 0.842}
\newrgbcolor{PST@COLOR21}{0.834 0.834 0.834}
\newrgbcolor{PST@COLOR22}{0.826 0.826 0.826}
\newrgbcolor{PST@COLOR23}{0.818 0.818 0.818}
\newrgbcolor{PST@COLOR24}{0.811 0.811 0.811}
\newrgbcolor{PST@COLOR25}{0.803 0.803 0.803}
\newrgbcolor{PST@COLOR26}{0.795 0.795 0.795}
\newrgbcolor{PST@COLOR27}{0.787 0.787 0.787}
\newrgbcolor{PST@COLOR28}{0.779 0.779 0.779}
\newrgbcolor{PST@COLOR29}{0.771 0.771 0.771}
\newrgbcolor{PST@COLOR30}{0.763 0.763 0.763}
\newrgbcolor{PST@COLOR31}{0.755 0.755 0.755}
\newrgbcolor{PST@COLOR32}{0.748 0.748 0.748}
\newrgbcolor{PST@COLOR33}{0.74 0.74 0.74}
\newrgbcolor{PST@COLOR34}{0.732 0.732 0.732}
\newrgbcolor{PST@COLOR35}{0.724 0.724 0.724}
\newrgbcolor{PST@COLOR36}{0.716 0.716 0.716}
\newrgbcolor{PST@COLOR37}{0.708 0.708 0.708}
\newrgbcolor{PST@COLOR38}{0.7 0.7 0.7}
\newrgbcolor{PST@COLOR39}{0.692 0.692 0.692}
\newrgbcolor{PST@COLOR40}{0.685 0.685 0.685}
\newrgbcolor{PST@COLOR41}{0.677 0.677 0.677}
\newrgbcolor{PST@COLOR42}{0.669 0.669 0.669}
\newrgbcolor{PST@COLOR43}{0.661 0.661 0.661}
\newrgbcolor{PST@COLOR44}{0.653 0.653 0.653}
\newrgbcolor{PST@COLOR45}{0.645 0.645 0.645}
\newrgbcolor{PST@COLOR46}{0.637 0.637 0.637}
\newrgbcolor{PST@COLOR47}{0.629 0.629 0.629}
\newrgbcolor{PST@COLOR48}{0.622 0.622 0.622}
\newrgbcolor{PST@COLOR49}{0.614 0.614 0.614}
\newrgbcolor{PST@COLOR50}{0.606 0.606 0.606}
\newrgbcolor{PST@COLOR51}{0.598 0.598 0.598}
\newrgbcolor{PST@COLOR52}{0.59 0.59 0.59}
\newrgbcolor{PST@COLOR53}{0.582 0.582 0.582}
\newrgbcolor{PST@COLOR54}{0.574 0.574 0.574}
\newrgbcolor{PST@COLOR55}{0.566 0.566 0.566}
\newrgbcolor{PST@COLOR56}{0.559 0.559 0.559}
\newrgbcolor{PST@COLOR57}{0.551 0.551 0.551}
\newrgbcolor{PST@COLOR58}{0.543 0.543 0.543}
\newrgbcolor{PST@COLOR59}{0.535 0.535 0.535}
\newrgbcolor{PST@COLOR60}{0.527 0.527 0.527}
\newrgbcolor{PST@COLOR61}{0.519 0.519 0.519}
\newrgbcolor{PST@COLOR62}{0.511 0.511 0.511}
\newrgbcolor{PST@COLOR63}{0.503 0.503 0.503}
\newrgbcolor{PST@COLOR64}{0.496 0.496 0.496}
\newrgbcolor{PST@COLOR65}{0.488 0.488 0.488}
\newrgbcolor{PST@COLOR66}{0.48 0.48 0.48}
\newrgbcolor{PST@COLOR67}{0.472 0.472 0.472}
\newrgbcolor{PST@COLOR68}{0.464 0.464 0.464}
\newrgbcolor{PST@COLOR69}{0.456 0.456 0.456}
\newrgbcolor{PST@COLOR70}{0.448 0.448 0.448}
\newrgbcolor{PST@COLOR71}{0.44 0.44 0.44}
\newrgbcolor{PST@COLOR72}{0.433 0.433 0.433}
\newrgbcolor{PST@COLOR73}{0.425 0.425 0.425}
\newrgbcolor{PST@COLOR74}{0.417 0.417 0.417}
\newrgbcolor{PST@COLOR75}{0.409 0.409 0.409}
\newrgbcolor{PST@COLOR76}{0.401 0.401 0.401}
\newrgbcolor{PST@COLOR77}{0.393 0.393 0.393}
\newrgbcolor{PST@COLOR78}{0.385 0.385 0.385}
\newrgbcolor{PST@COLOR79}{0.377 0.377 0.377}
\newrgbcolor{PST@COLOR80}{0.37 0.37 0.37}
\newrgbcolor{PST@COLOR81}{0.362 0.362 0.362}
\newrgbcolor{PST@COLOR82}{0.354 0.354 0.354}
\newrgbcolor{PST@COLOR83}{0.346 0.346 0.346}
\newrgbcolor{PST@COLOR84}{0.338 0.338 0.338}
\newrgbcolor{PST@COLOR85}{0.33 0.33 0.33}
\newrgbcolor{PST@COLOR86}{0.322 0.322 0.322}
\newrgbcolor{PST@COLOR87}{0.314 0.314 0.314}
\newrgbcolor{PST@COLOR88}{0.307 0.307 0.307}
\newrgbcolor{PST@COLOR89}{0.299 0.299 0.299}
\newrgbcolor{PST@COLOR90}{0.291 0.291 0.291}
\newrgbcolor{PST@COLOR91}{0.283 0.283 0.283}
\newrgbcolor{PST@COLOR92}{0.275 0.275 0.275}
\newrgbcolor{PST@COLOR93}{0.267 0.267 0.267}
\newrgbcolor{PST@COLOR94}{0.259 0.259 0.259}
\newrgbcolor{PST@COLOR95}{0.251 0.251 0.251}
\newrgbcolor{PST@COLOR96}{0.244 0.244 0.244}
\newrgbcolor{PST@COLOR97}{0.236 0.236 0.236}
\newrgbcolor{PST@COLOR98}{0.228 0.228 0.228}
\newrgbcolor{PST@COLOR99}{0.22 0.22 0.22}
\newrgbcolor{PST@COLOR100}{0.212 0.212 0.212}
\newrgbcolor{PST@COLOR101}{0.204 0.204 0.204}
\newrgbcolor{PST@COLOR102}{0.196 0.196 0.196}
\newrgbcolor{PST@COLOR103}{0.188 0.188 0.188}
\newrgbcolor{PST@COLOR104}{0.181 0.181 0.181}
\newrgbcolor{PST@COLOR105}{0.173 0.173 0.173}
\newrgbcolor{PST@COLOR106}{0.165 0.165 0.165}
\newrgbcolor{PST@COLOR107}{0.157 0.157 0.157}
\newrgbcolor{PST@COLOR108}{0.149 0.149 0.149}
\newrgbcolor{PST@COLOR109}{0.141 0.141 0.141}
\newrgbcolor{PST@COLOR110}{0.133 0.133 0.133}
\newrgbcolor{PST@COLOR111}{0.125 0.125 0.125}
\newrgbcolor{PST@COLOR112}{0.118 0.118 0.118}
\newrgbcolor{PST@COLOR113}{0.11 0.11 0.11}
\newrgbcolor{PST@COLOR114}{0.102 0.102 0.102}
\newrgbcolor{PST@COLOR115}{0.094 0.094 0.094}
\newrgbcolor{PST@COLOR116}{0.086 0.086 0.086}
\newrgbcolor{PST@COLOR117}{0.078 0.078 0.078}
\newrgbcolor{PST@COLOR118}{0.07 0.07 0.07}
\newrgbcolor{PST@COLOR119}{0.062 0.062 0.062}
\newrgbcolor{PST@COLOR120}{0.055 0.055 0.055}
\newrgbcolor{PST@COLOR121}{0.047 0.047 0.047}
\newrgbcolor{PST@COLOR122}{0.039 0.039 0.039}
\newrgbcolor{PST@COLOR123}{0.031 0.031 0.031}
\newrgbcolor{PST@COLOR124}{0.023 0.023 0.023}
\newrgbcolor{PST@COLOR125}{0.015 0.015 0.015}
\newrgbcolor{PST@COLOR126}{0.007 0.007 0.007}
\newrgbcolor{PST@COLOR127}{0 0 0}

\def\polypmIIId#1{\pspolygon[linestyle=none,fillstyle=solid,fillcolor=PST@COLOR#1]}

\polypmIIId{99} (0.0568,0.19)  (0.0,0.19)  (0.0,0.1078)(0.0568,0.1078)
\polypmIIId{111}  (0.0568,0.272) (0.0,0.272) (0.0,0.19)  (0.0568,0.19)
\polypmIIId{121}  (0.0568,0.3542)(0.0,0.3542)(0.0,0.272) (0.0568,0.272)

\polypmIIId{96} (0.1136,   0.19)  (0.0568,0.19)  (0.0568,0.1078)(0.1136,0.1078)
\polypmIIId{115}  (0.1136,   0.272) (0.0568,0.272) (0.0568,0.19)  (0.1136,0.19)
\polypmIIId{120}  (0.1136,   0.3542)(0.0568,0.3542)(0.0568,0.272) (0.1136,0.272)

\polypmIIId{99}(0.1704,0.19)  (0.1136,   0.19)  (0.1136,   0.1078)(0.1704,0.1078)
\polypmIIId{113} (0.1704,0.272) (0.1136,   0.272) (0.1136,   0.19)  (0.1704,0.19)
\polypmIIId{121}  (0.1704,0.3542)(0.1136,   0.3542)(0.1136,   0.272) (0.1704,0.272)

\polypmIIId{100}(0.2272,0.19)  (0.1704,0.19)  (0.1704,0.1078)(0.2272,0.1078)
\polypmIIId{113}  (0.2272,0.272) (0.1704,0.272) (0.1704,0.19)  (0.2272,0.19)
\polypmIIId{121}  (0.2272,0.3542)(0.1704,0.3542)(0.1704,0.272) (0.2272,0.272)

\rput(0.0284,0.07){3}
\rput(0.0852,0.07){4}
\rput(0.1420,0.07){5}
\rput(0.1988,0.07){6}
\rput(0.1136,0.0070){years}

\PST@Border(0.0,0.3542)
(0.0,0.1078)
(0.2272,0.1078)
(0.2272,0.3542)
(0.0,0.3542)

\polypmIIId{0}(0.2329,0.1078)(0.2442,0.1078)(0.2442,0.1098)(0.2329,0.1098)
\polypmIIId{1}(0.2329,0.1097)(0.2442,0.1097)(0.2442,0.1117)(0.2329,0.1117)
\polypmIIId{2}(0.2329,0.1116)(0.2442,0.1116)(0.2442,0.1136)(0.2329,0.1136)
\polypmIIId{3}(0.2329,0.1135)(0.2442,0.1135)(0.2442,0.1156)(0.2329,0.1156)
\polypmIIId{4}(0.2329,0.1155)(0.2442,0.1155)(0.2442,0.1175)(0.2329,0.1175)
\polypmIIId{5}(0.2329,0.1174)(0.2442,0.1174)(0.2442,0.1194)(0.2329,0.1194)
\polypmIIId{6}(0.2329,0.1193)(0.2442,0.1193)(0.2442,0.1213)(0.2329,0.1213)
\polypmIIId{7}(0.2329,0.1212)(0.2442,0.1212)(0.2442,0.1233)(0.2329,0.1233)
\polypmIIId{8}(0.2329,0.1232)(0.2442,0.1232)(0.2442,0.1252)(0.2329,0.1252)
\polypmIIId{9}(0.2329,0.1251)(0.2442,0.1251)(0.2442,0.1271)(0.2329,0.1271)
\polypmIIId{10}(0.2329,0.127)(0.2442,0.127)(0.2442,0.129)(0.2329,0.129)
\polypmIIId{11}(0.2329,0.1289)(0.2442,0.1289)(0.2442,0.131)(0.2329,0.131)
\polypmIIId{12}(0.2329,0.1309)(0.2442,0.1309)(0.2442,0.1329)(0.2329,0.1329)
\polypmIIId{13}(0.2329,0.1328)(0.2442,0.1328)(0.2442,0.1348)(0.2329,0.1348)
\polypmIIId{14}(0.2329,0.1347)(0.2442,0.1347)(0.2442,0.1367)(0.2329,0.1367)
\polypmIIId{15}(0.2329,0.1366)(0.2442,0.1366)(0.2442,0.1387)(0.2329,0.1387)
\polypmIIId{16}(0.2329,0.1386)(0.2442,0.1386)(0.2442,0.1406)(0.2329,0.1406)
\polypmIIId{17}(0.2329,0.1405)(0.2442,0.1405)(0.2442,0.1425)(0.2329,0.1425)
\polypmIIId{18}(0.2329,0.1424)(0.2442,0.1424)(0.2442,0.1444)(0.2329,0.1444)
\polypmIIId{19}(0.2329,0.1443)(0.2442,0.1443)(0.2442,0.1464)(0.2329,0.1464)
\polypmIIId{20}(0.2329,0.1463)(0.2442,0.1463)(0.2442,0.1483)(0.2329,0.1483)
\polypmIIId{21}(0.2329,0.1482)(0.2442,0.1482)(0.2442,0.1502)(0.2329,0.1502)
\polypmIIId{22}(0.2329,0.1501)(0.2442,0.1501)(0.2442,0.1521)(0.2329,0.1521)
\polypmIIId{23}(0.2329,0.152)(0.2442,0.152)(0.2442,0.1541)(0.2329,0.1541)
\polypmIIId{24}(0.2329,0.154)(0.2442,0.154)(0.2442,0.156)(0.2329,0.156)
\polypmIIId{25}(0.2329,0.1559)(0.2442,0.1559)(0.2442,0.1579)(0.2329,0.1579)
\polypmIIId{26}(0.2329,0.1578)(0.2442,0.1578)(0.2442,0.1598)(0.2329,0.1598)
\polypmIIId{27}(0.2329,0.1597)(0.2442,0.1597)(0.2442,0.1618)(0.2329,0.1618)
\polypmIIId{28}(0.2329,0.1617)(0.2442,0.1617)(0.2442,0.1637)(0.2329,0.1637)
\polypmIIId{29}(0.2329,0.1636)(0.2442,0.1636)(0.2442,0.1656)(0.2329,0.1656)
\polypmIIId{30}(0.2329,0.1655)(0.2442,0.1655)(0.2442,0.1675)(0.2329,0.1675)
\polypmIIId{31}(0.2329,0.1674)(0.2442,0.1674)(0.2442,0.1695)(0.2329,0.1695)
\polypmIIId{32}(0.2329,0.1694)(0.2442,0.1694)(0.2442,0.1714)(0.2329,0.1714)
\polypmIIId{33}(0.2329,0.1713)(0.2442,0.1713)(0.2442,0.1733)(0.2329,0.1733)
\polypmIIId{34}(0.2329,0.1732)(0.2442,0.1732)(0.2442,0.1752)(0.2329,0.1752)
\polypmIIId{35}(0.2329,0.1751)(0.2442,0.1751)(0.2442,0.1772)(0.2329,0.1772)
\polypmIIId{36}(0.2329,0.1771)(0.2442,0.1771)(0.2442,0.1791)(0.2329,0.1791)
\polypmIIId{37}(0.2329,0.179)(0.2442,0.179)(0.2442,0.181)(0.2329,0.181)
\polypmIIId{38}(0.2329,0.1809)(0.2442,0.1809)(0.2442,0.1829)(0.2329,0.1829)
\polypmIIId{39}(0.2329,0.1828)(0.2442,0.1828)(0.2442,0.1849)(0.2329,0.1849)
\polypmIIId{40}(0.2329,0.1848)(0.2442,0.1848)(0.2442,0.1868)(0.2329,0.1868)
\polypmIIId{41}(0.2329,0.1867)(0.2442,0.1867)(0.2442,0.1887)(0.2329,0.1887)
\polypmIIId{42}(0.2329,0.1886)(0.2442,0.1886)(0.2442,0.1906)(0.2329,0.1906)
\polypmIIId{43}(0.2329,0.1905)(0.2442,0.1905)(0.2442,0.1926)(0.2329,0.1926)
\polypmIIId{44}(0.2329,0.1925)(0.2442,0.1925)(0.2442,0.1945)(0.2329,0.1945)
\polypmIIId{45}(0.2329,0.1944)(0.2442,0.1944)(0.2442,0.1964)(0.2329,0.1964)
\polypmIIId{46}(0.2329,0.1963)(0.2442,0.1963)(0.2442,0.1983)(0.2329,0.1983)
\polypmIIId{47}(0.2329,0.1982)(0.2442,0.1982)(0.2442,0.2003)(0.2329,0.2003)
\polypmIIId{48}(0.2329,0.2002)(0.2442,0.2002)(0.2442,0.2022)(0.2329,0.2022)
\polypmIIId{49}(0.2329,0.2021)(0.2442,0.2021)(0.2442,0.2041)(0.2329,0.2041)
\polypmIIId{50}(0.2329,0.204)(0.2442,0.204)(0.2442,0.206)(0.2329,0.206)
\polypmIIId{51}(0.2329,0.2059)(0.2442,0.2059)(0.2442,0.208)(0.2329,0.208)
\polypmIIId{52}(0.2329,0.2079)(0.2442,0.2079)(0.2442,0.2099)(0.2329,0.2099)
\polypmIIId{53}(0.2329,0.2098)(0.2442,0.2098)(0.2442,0.2118)(0.2329,0.2118)
\polypmIIId{54}(0.2329,0.2117)(0.2442,0.2117)(0.2442,0.2137)(0.2329,0.2137)
\polypmIIId{55}(0.2329,0.2136)(0.2442,0.2136)(0.2442,0.2157)(0.2329,0.2157)
\polypmIIId{56}(0.2329,0.2156)(0.2442,0.2156)(0.2442,0.2176)(0.2329,0.2176)
\polypmIIId{57}(0.2329,0.2175)(0.2442,0.2175)(0.2442,0.2195)(0.2329,0.2195)
\polypmIIId{58}(0.2329,0.2194)(0.2442,0.2194)(0.2442,0.2214)(0.2329,0.2214)
\polypmIIId{59}(0.2329,0.2213)(0.2442,0.2213)(0.2442,0.2234)(0.2329,0.2234)
\polypmIIId{60}(0.2329,0.2233)(0.2442,0.2233)(0.2442,0.2253)(0.2329,0.2253)
\polypmIIId{61}(0.2329,0.2252)(0.2442,0.2252)(0.2442,0.2272)(0.2329,0.2272)
\polypmIIId{62}(0.2329,0.2271)(0.2442,0.2271)(0.2442,0.2291)(0.2329,0.2291)
\polypmIIId{63}(0.2329,0.229)(0.2442,0.229)(0.2442,0.2311)(0.2329,0.2311)
\polypmIIId{64}(0.2329,0.231)(0.2442,0.231)(0.2442,0.233)(0.2329,0.233)
\polypmIIId{65}(0.2329,0.2329)(0.2442,0.2329)(0.2442,0.2349)(0.2329,0.2349)
\polypmIIId{66}(0.2329,0.2348)(0.2442,0.2348)(0.2442,0.2368)(0.2329,0.2368)
\polypmIIId{67}(0.2329,0.2367)(0.2442,0.2367)(0.2442,0.2388)(0.2329,0.2388)
\polypmIIId{68}(0.2329,0.2387)(0.2442,0.2387)(0.2442,0.2407)(0.2329,0.2407)
\polypmIIId{69}(0.2329,0.2406)(0.2442,0.2406)(0.2442,0.2426)(0.2329,0.2426)
\polypmIIId{70}(0.2329,0.2425)(0.2442,0.2425)(0.2442,0.2445)(0.2329,0.2445)
\polypmIIId{71}(0.2329,0.2444)(0.2442,0.2444)(0.2442,0.2465)(0.2329,0.2465)
\polypmIIId{72}(0.2329,0.2464)(0.2442,0.2464)(0.2442,0.2484)(0.2329,0.2484)
\polypmIIId{73}(0.2329,0.2483)(0.2442,0.2483)(0.2442,0.2503)(0.2329,0.2503)
\polypmIIId{74}(0.2329,0.2502)(0.2442,0.2502)(0.2442,0.2522)(0.2329,0.2522)
\polypmIIId{75}(0.2329,0.2521)(0.2442,0.2521)(0.2442,0.2542)(0.2329,0.2542)
\polypmIIId{76}(0.2329,0.2541)(0.2442,0.2541)(0.2442,0.2561)(0.2329,0.2561)
\polypmIIId{77}(0.2329,0.256)(0.2442,0.256)(0.2442,0.258)(0.2329,0.258)
\polypmIIId{78}(0.2329,0.2579)(0.2442,0.2579)(0.2442,0.2599)(0.2329,0.2599)
\polypmIIId{79}(0.2329,0.2598)(0.2442,0.2598)(0.2442,0.2619)(0.2329,0.2619)
\polypmIIId{80}(0.2329,0.2618)(0.2442,0.2618)(0.2442,0.2638)(0.2329,0.2638)
\polypmIIId{81}(0.2329,0.2637)(0.2442,0.2637)(0.2442,0.2657)(0.2329,0.2657)
\polypmIIId{82}(0.2329,0.2656)(0.2442,0.2656)(0.2442,0.2676)(0.2329,0.2676)
\polypmIIId{83}(0.2329,0.2675)(0.2442,0.2675)(0.2442,0.2696)(0.2329,0.2696)
\polypmIIId{84}(0.2329,0.2695)(0.2442,0.2695)(0.2442,0.2715)(0.2329,0.2715)
\polypmIIId{85}(0.2329,0.2714)(0.2442,0.2714)(0.2442,0.2734)(0.2329,0.2734)
\polypmIIId{86}(0.2329,0.2733)(0.2442,0.2733)(0.2442,0.2753)(0.2329,0.2753)
\polypmIIId{87}(0.2329,0.2752)(0.2442,0.2752)(0.2442,0.2773)(0.2329,0.2773)
\polypmIIId{88}(0.2329,0.2772)(0.2442,0.2772)(0.2442,0.2792)(0.2329,0.2792)
\polypmIIId{89}(0.2329,0.2791)(0.2442,0.2791)(0.2442,0.2811)(0.2329,0.2811)
\polypmIIId{90}(0.2329,0.281)(0.2442,0.281)(0.2442,0.283)(0.2329,0.283)
\polypmIIId{91}(0.2329,0.2829)(0.2442,0.2829)(0.2442,0.285)(0.2329,0.285)
\polypmIIId{92}(0.2329,0.2849)(0.2442,0.2849)(0.2442,0.2869)(0.2329,0.2869)
\polypmIIId{93}(0.2329,0.2868)(0.2442,0.2868)(0.2442,0.2888)(0.2329,0.2888)
\polypmIIId{94}(0.2329,0.2887)(0.2442,0.2887)(0.2442,0.2907)(0.2329,0.2907)
\polypmIIId{95}(0.2329,0.2906)(0.2442,0.2906)(0.2442,0.2927)(0.2329,0.2927)
\polypmIIId{96}(0.2329,0.2926)(0.2442,0.2926)(0.2442,0.2946)(0.2329,0.2946)
\polypmIIId{97}(0.2329,0.2945)(0.2442,0.2945)(0.2442,0.2965)(0.2329,0.2965)
\polypmIIId{98}(0.2329,0.2964)(0.2442,0.2964)(0.2442,0.2984)(0.2329,0.2984)
\polypmIIId{99}(0.2329,0.2983)(0.2442,0.2983)(0.2442,0.3004)(0.2329,0.3004)
\polypmIIId{100}(0.2329,0.3003)(0.2442,0.3003)(0.2442,0.3023)(0.2329,0.3023)
\polypmIIId{101}(0.2329,0.3022)(0.2442,0.3022)(0.2442,0.3042)(0.2329,0.3042)
\polypmIIId{102}(0.2329,0.3041)(0.2442,0.3041)(0.2442,0.3061)(0.2329,0.3061)
\polypmIIId{103}(0.2329,0.306)(0.2442,0.306)(0.2442,0.3081)(0.2329,0.3081)
\polypmIIId{104}(0.2329,0.308)(0.2442,0.308)(0.2442,0.31)(0.2329,0.31)
\polypmIIId{105}(0.2329,0.3099)(0.2442,0.3099)(0.2442,0.3119)(0.2329,0.3119)
\polypmIIId{106}(0.2329,0.3118)(0.2442,0.3118)(0.2442,0.3138)(0.2329,0.3138)
\polypmIIId{107}(0.2329,0.3137)(0.2442,0.3137)(0.2442,0.3158)(0.2329,0.3158)
\polypmIIId{108}(0.2329,0.3157)(0.2442,0.3157)(0.2442,0.3177)(0.2329,0.3177)
\polypmIIId{109}(0.2329,0.3176)(0.2442,0.3176)(0.2442,0.3196)(0.2329,0.3196)
\polypmIIId{110}(0.2329,0.3195)(0.2442,0.3195)(0.2442,0.3215)(0.2329,0.3215)
\polypmIIId{111}(0.2329,0.3214)(0.2442,0.3214)(0.2442,0.3235)(0.2329,0.3235)
\polypmIIId{112}(0.2329,0.3234)(0.2442,0.3234)(0.2442,0.3254)(0.2329,0.3254)
\polypmIIId{113}(0.2329,0.3253)(0.2442,0.3253)(0.2442,0.3273)(0.2329,0.3273)
\polypmIIId{114}(0.2329,0.3272)(0.2442,0.3272)(0.2442,0.3292)(0.2329,0.3292)
\polypmIIId{115}(0.2329,0.3291)(0.2442,0.3291)(0.2442,0.3312)(0.2329,0.3312)
\polypmIIId{116}(0.2329,0.3311)(0.2442,0.3311)(0.2442,0.3331)(0.2329,0.3331)
\polypmIIId{117}(0.2329,0.333)(0.2442,0.333)(0.2442,0.335)(0.2329,0.335)
\polypmIIId{118}(0.2329,0.3349)(0.2442,0.3349)(0.2442,0.3369)(0.2329,0.3369)
\polypmIIId{119}(0.2329,0.3368)(0.2442,0.3368)(0.2442,0.3389)(0.2329,0.3389)
\polypmIIId{120}(0.2329,0.3388)(0.2442,0.3388)(0.2442,0.3408)(0.2329,0.3408)
\polypmIIId{121}(0.2329,0.3407)(0.2442,0.3407)(0.2442,0.3427)(0.2329,0.3427)
\polypmIIId{122}(0.2329,0.3426)(0.2442,0.3426)(0.2442,0.3446)(0.2329,0.3446)
\polypmIIId{123}(0.2329,0.3445)(0.2442,0.3445)(0.2442,0.3466)(0.2329,0.3466)
\polypmIIId{124}(0.2329,0.3465)(0.2442,0.3465)(0.2442,0.3485)(0.2329,0.3485)
\polypmIIId{125}(0.2329,0.3484)(0.2442,0.3484)(0.2442,0.3504)(0.2329,0.3504)
\polypmIIId{126}(0.2329,0.3503)(0.2442,0.3503)(0.2442,0.3523)(0.2329,0.3523)
\polypmIIId{127}(0.2329,0.3522)(0.2442,0.3522)(0.2442,0.3542)(0.2329,0.3542)

\PST@Border(0.2329,0.1078)
(0.2442,0.1078)
(0.2442,0.3542)
(0.2329,0.3542)
(0.2329,0.1078)


\rput[l](0.2502,0.1301){0.997}
\rput[l](0.2502,0.2048){0.998}
\rput[l](0.2502,0.2795){0.999}
\rput[l](0.2502,0.3542){1}

\catcode`@=12
\fi
\endpspicture}
  \caption{GALP solution quality on uncorrelated instances ($\alpha = 1.0$).}
  \label{fig:greedysolcomp10}
\end{figure}

Observing the figures a comparison between the two heuristics can also be made.
By comparing figures~\ref{fig:tabusolcomp00}, \ref{fig:tabusolcomp01} and \ref{fig:tabusolcomp10} to 
figures~\ref{fig:greedysolcomp00}, \ref{fig:greedysolcomp01} and \ref{fig:greedysolcomp10}, the first group 
of figures show a darker tone than the second. It indicates that on average the TSLP was able to find better
solutions than the GALP on the tested instances. All the experiments results are available under request to the authors.

Even considering that the heuristics starting point was already a good solution, since it was on average
99.652\% of the best known solution, the heuristics were able to improve the solutions. On average, 
the solutions found by the GALP were 99.912\% of the best known solutions, and the TSLP obtained 
even better results, with it's solutions being 99.967\% of the best known ones, on average.

Regarding the execution time, the TSLP had a maximum execution time of 2 minutes, with an average of 30 seconds.
GALP had a negligible running time, below 1 second for all instances.
