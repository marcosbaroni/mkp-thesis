\section{Conclusion}
\label{sec:conclu}
The main objectives on this work were (1) to understand and formally model the EDCO's loss reduction problem, (2) test the hardness
of the created model and (3) solve the model with exact and heuristic approaches.

The electricity loss reduction problem was modelled as a generalization of the famous Knapsack Problem. So the first goal was met
with the creation of a mathematical model combining characteristics of the Bounded Knapsack Problem, the Multidimensional Knapsack
Problem, the Multiple Knapsack Problem and the Partially-Odered Knapsack Problem. While reaching this first goal, two 
generalizations of the Knapsack Problem were created, the POMMBKP and the POMMBKPDSA, the first being a generalization of
all those Knapsack Problem variations and the second being a generalization of the first. No work concerning any of these two new 
Knapsack Problem versions was found on the literature, and we believe that, due to their generality, they may be used to model 
several real world problems with greater details.

Due to the lack of real world instances for the EDCO's problem (the POMMBKPDSA), another contribution was made in the form of an 
artificial instance generator and a set of benchmark instances for the problem, available under request to the authors. As a way to achieve objective (2)
those bechmark instances were solved with the CPLEX solver. These experiments demonstrated that with the increase of some dimensions
of the instances, the exact solver is no longer able to prove the optimal solution within a limited time. It was shown that this 
incarnation of the Knapsack Problem, even if not exactly as predicted by literature, is also sensitive to the instances correlation level.

To achieve objective (3), besides the CPLEX solver two heuristics were proposed, the first based on a greedy approach (GALP) and the second based 
on the metaheuristic Tabu Search (TSLP). The heuristics were used to solve the benchmark instances and their solutions were compared to the best
known ones. Both heuristics where able to find good quality solution, quite close to the best known. Particularly, the TSLP obtained better results
concerning the quality of the solutions found, but at a greater time cost.

However it is worthy to note that the best solutions were still found by the CPLEX solver. Besides, even when the CPLEX's solving 
process was interrupted before the end, on the worst cases the solution was at most at 0.35\% of the optimal solution. Being the method which
found the best solutions, even in the cases it was interrupted, the CPLEX is probably the best option 
to solve instances of the problem with the dimensions expected to be found in practice.
