\section{Experimental Data}
\label{sec:exp_data}

In order to test the hardness of the POMMBKPDSA and assess the quality of the solutions obtained by the methods 
proposed on the previous section, a large number of instances of the problem are needed to make useful 
conclusions. Unfortunatelly, our local EDCO wasn't able to provide that number of instances, so we had 
to develop an instance generator to create a sufficient amount of artificial instances to run the tests. 
The generator creates instances according to the formulation of the POMMBKPDSA, with parameters that can be 
adjusted to build instances of different sizes and characteristics. It receives as parameters:
\begin{itemize}
  \item $Y$, the number of years (knapsacks);
  \item $A$, the number of actions (items);
  \item $R$, the number of budgets or resources (dimensions);
  \item $\alpha$, the level of correlation between each action's cost and profit.
\end{itemize}
After those input parameters, the instance generation procedure follows three steps.
At first the actions characteristics are generated (recovery curve, costs, markets and electricity value)
using uniform distributions with variations similar to those found in real context.
The second step is to compute yearly goals and yearly budgets,
based on total sums of the characteristics of the actions.
The last step generates some random dependencies between actions.
This is done avoiding cyclical dependencies, as any action involved in a cyclical 
dependency would never be chosen.

%The algorithm for the instance generator is composed of three parts: the actions generation,
%the budgets and goals generation and the dependencies generation. When the actions are 
%being generated, the first characteristic created is the elecricity value ($v_a$), followed
%by the yearly ($u_{a,y}$) and global ($m_a$) markets. This part also generates the cost of the
%action on each budget ($c_{a,r}$) and the action's electricity recovery curve ($e_{j,k}$). The 
%algorithm's second part covers the creation of the yearly goals ($g_y$) and the yearly budgets
%for each resource ($o_{y,r}$). The last part generates some random dependencies between actions.
%The code for the generator is presented in algorithm~\ref{alg:gen}.

% Explicar \alpha, a variação de anos/ações/recursos,

%Firstly, lines~\ref{alg:gen:line:actionbegin} to~\ref{alg:gen:line:actionend} show the creation of the $A$ actions.
%The electricity value is created at line~\ref{alg:gen:line:actionvalue} as a real value between
%1.0 and 2.0. The yearly market is generated at line~\ref{alg:gen:line:market} as a random integer between 0 e 50 
%and the global market is created at line~\ref{alg:gen:line:uc} as $90\%$ of the yearly markets sum.
%The global market is created a bit below the yearly markets sum because if it is equal or greater than that sum, the 
%global market restriction becomes useless, as it will never be reached. At line~\ref{alg:gen:line:cost}, 
%the action's costs over each resource are created. Lines~\ref{alg:gen:line:vectorbegin} to~\ref{alg:gen:line:vectorend} build a 
%random array of size $Y$ where the sum of all components is 1. This array is used to create the actions loss reduction curve at 
%line~\ref{alg:gen:line:rec}.

%The total electricity loss avoided by the action (the electricity recovered) is created as a function of two components: the sum of the action's
%cost plus a fixed value and a random number. The correlation parameter ($\alpha$) is used to dose how much each component will be used on the curve's 
%creation. When $\alpha$ is zero, only the first component is utilized and the amount of electricity recovered by the action, and consequently its profit,
%is strongly correlated to the action's total cost, in other words, costlier actions recover more electricity. On the other side, when $\alpha$ equals
%one, there is no correlation between an action's cost and it's profit. The total electricity loss is then divided by the number of years to generate the 
%action's loss reduction curve, using the normalized array created before.

% \begin{figure}
\begin{algorithmic}[1]
\REQUIRE $Y, A, R, \alpha$
\ENSURE Generated instance
\STATE $r \gets 0.10$
\FOR{ $a=1$ to $A$}
  \label{alg:gen:line:actionbegin}
  \STATE $v_a \gets RandomReal(1.0, 2.0)$
    \label{alg:gen:line:actionvalue}
  \FOR{ $y=1$ to $Y$}
    \STATE $u_{a,y} \gets RandomInteger(0, 50)$
      \label{alg:gen:line:market}
  \ENDFOR
  \STATE $m_{a} \gets 0.9 * Sum(u_{a})$
    \label{alg:gen:line:uc}
  \FOR{$r=1$ to $R$}
    \STATE $c_{a,r} \gets RandomReal(1.0, 100.0)$
      \label{alg:gen:line:cost}
  \ENDFOR
  \FOR{$y=1$ to $Y$}
    \label{alg:gen:line:vectorbegin}
    \STATE $W_{y} \gets RandomReal(0.0, 1.0)$
  \ENDFOR
  \FOR{$y=1$ to $Y$}
    \STATE $W_{y} \gets W_{y}/Sum(W_y)$
  \ENDFOR
    \label{alg:gen:line:vectorend}
  \FOR{$y=1$ to $Y$}
    \STATE $e_{a,y} \gets ((1 - \alpha) \times ((2 \times Sum(c_{a}) + 10)/v_a) + (\alpha \times RandomReal(1.0, 100.0) \times R)) \times W_{y} \times (1 + r)^y$
      \label{alg:gen:line:rec}
  \ENDFOR
\ENDFOR
  \label{alg:gen:line:actionend}
\FOR{$y=1$ to $Y$}
  \label{alg:gen:line:budgbegin}
  \STATE $aux \gets 0.5 * Sum(e_{y})/Y$
  \STATE $g_{y} \gets RandomReal(0.95*aux, 1.05*aux)$
  \FOR{$r=1$ to $R$}
    \STATE $aux \gets 0.5 * Sum(c_{y,r})/Y$
    \STATE $o_{y,r} \gets RandomReal(0.95*aux, 1.05*aux)$
  \ENDFOR
\ENDFOR
  \label{alg:gen:line:budgend}
\FOR{$i=1$ to $A/10$}
  \STATE GenerateDependency()
    \label{alg:gen:line:deps}
\ENDFOR
\end{algorithmic}
\caption{Instance Generator}
\label{alg:gen}
\end{figure}


%At the second part, between lines~\ref{alg:gen:line:budgbegin} and~\ref{alg:gen:line:budgend}, the loss reduction goals and the budgets for
%each resource are created, for each year. Yearly goals are created as half the sum of all electricity recovered by all actions, divided 
%by the number of years($\pm5\%$). This approach ensures solutions won't have too few actions, and won't be composed of all possible 
%actions allocations. Bugdets are created analogously, but one budget per resource per year is created.

%Line~\ref{alg:gen:line:deps} creates the dependencies between actions. For each execution of the loop, function GenerateDependency() 
%choses two random actions and creates a dependency between them, avoiding cyclical dependencies, as any action involved in a cyclical 
%dependency would never be chosen.
