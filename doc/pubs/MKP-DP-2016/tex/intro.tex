- Falar do MKP. \\
- Falar do algoritmo de programação dinamica. \\
--- (Dizer que para o algoritmo de programação dinâmica tem comportamento polinomial se fixada as dimenões) \\
- Falar da estrutura de dados e da proposta de acelerar o algoritmo. \\

The multidimensional knapsack problem (MKP) is a strongly NP-hard combinatorial
optimization problem which can be viewed as a resource allocation problem and
defined as follows:

\begin{align}
  \text{maximize} & \sum_{j=1}^n p_j x_j \\
  \text{subject to} & \sum_{j=1}^n w_{ij} x_j \leqslant c_i \quad i \in \{1, \ldots, m\}\\
   & x_j \in \{0, 1\}, \quad j \in \{1, \ldots, n\}.
\end{align}

The problem can be interpreted as a set of $n$ items with profits $p_j$
and a set of $m$ resources with capacities $c_i$.
Each item $j$ consumes an amount $w_{ij}$ from each resource $i$, if selected.
The objective is to select a subset of items with maximum total profit,
not exceeding the defined resource capacities.
The decision variable $x_j$ indicates if $j$-th item is selected.
It is considered an integer programming problem (IP) since its variables $x_i$
are restricted to be integers.

The multidimensional knapsack problem can be applied on budget planning 
scenarios and project selections~\cite{mcmillan1973resource},
cutting stock problems~\cite{Gilmore-Gomory-1966}, loading problems~\cite{Shih-1979},
allocation of processors and databases in distributed computer programs~\cite{Gavish-Pirckul-1982}.

The problem is a generalization of the well-known knapsack problem (KP) in which
$m = 1$.
However it is a NP-hard problem significantly harder to solve in practice than the KP.
Due its simple definition but challenging difficulty of solving, the MKP is often used to
to verify the efficiency of novel metaheuristics.

