\documentclass[10pt, conference, compsocconf]{IEEEtran}

\usepackage{amsfonts}
\usepackage{amssymb,amsmath}
\usepackage{hyperref}
\usepackage{algorithm}
\usepackage{algpseudocode}
\usepackage{graphicx}
\usepackage[skip=2pt,font=footnotesize]{caption}

\begin{document}

\title{A shuffled frog leaping algorithm for
the multidimensional knapsack problem}

\author{\IEEEauthorblockN{Marcos Daniel Valad\~ao Baroni}
\IEEEauthorblockA{ Departamento de Inform\'atica\\
Universidade Federal do Esp\'irito Santo\\
Vit\'oria, Esp\'irito Santo, Brazil\\
Email: mbaroni@ninfa.inf.ufes.br }
\and
\IEEEauthorblockN{Fl\'avio Miguel Varej\~ao}
\IEEEauthorblockA{ Departamento de Inform\'atica\\
Universidade Federal do Esp\'irito Santo\\
Vit\'oria, Esp\'irito Santo, Brazil\\
Email: fvarejao@ninfa.inf.ufes.br }
}

\maketitle

\begin{abstract}
%\boldmath
The abstract goes here.
\end{abstract}

\section{Introduction}
\label{sec:intro}

The Multidimensional Knapsack Problem (MKP) is a strongly NP-hard combinatorial
optimization problem which can be viewed as a resource allocation problem and
defined as follows:

\begin{align*}
  \text{maximize} & \sum_{j=1}^n p_j x_j \\
  \text{subject to} & \sum_{j=1}^n w_{ij} x_j \leqslant c_i \quad i \in \{1, \ldots, m\}\\
   & x_j \in \{0, 1\}, \quad j \in \{1, \ldots, n\}.
\end{align*}

% Define the MKP
The problem can be interpreted as a set of $n$ itens with profits $p_j$
and a set of $m$ resources with capacities $c_i$.
Each item $j$ consumes an amount $w_{ij}$ from each resource $i$, if selected.
The objective is to select a subset of items with maximum total profit,
not exceeding the defined resource capacities.
The decision variable $x_j$ indicates if $j$-th item is selected.

The multidimensional knapsack problem can be applied on budget planning 
scenarios, subset project selections, cutting stock problems, task scheduling,
allocation of processors and databases in distributed computer programs.
The problem is a generalization of the well-known knapsack problem (KP) in which
$m = 1$.

The MKP is a NP-Hard problem significantly harder to solve in practice than the KP.
Despite the existence of a fully polynomial approximation scheme (FPAS) for the KP,
finding a FPAS for the MKP is NP-hard for $m \geqslant 2$~\cite{magazine1984note}.
Due its simple definition but challenging difficulty the MKP is often used to
to verify the efficiency of novel metaheuristics.

%A metaheuristic is a set of concepts that can be used to define heuristic methods
%that can be applied to a wide set of different problems.
%In other words, a metaheuristic can be seen as a general algorithmic framework which can be applied to
%different optimization problems with relatively few modifications to make them adapted to a specific problem.”

In this paper we address the application of a metaheuristic called shuffled
frog leaping algorithm (SFLA) to the multidimensional knapsack problem.
The SFLA is a metaheuristic proposed by Eusuff and Lansey~\cite{eusuff2003optimization, eusuff2006shuffled}
which combines concepts from two other widely used metaheuristics:
The shuffled complex evolution algorithm (SCE) and the 
Particle Swarm Optimization (PSO), providing a robust heuristic which has been
successfully applied to several optimization problems~\cite{bhattacharjee2014shuffled,
horng2014construction, xu2013effective, fang2012effective, luo2014improved}.
%knapsack problem~\cite{bhattacharjee2014shuffled}, construction of support vector
%machine~\cite{horng2014construction}, scheduling problems~\cite{xu2013effective, fang2012effective},
%vehicle routing~\cite{luo2014improved}.

The reminder of the paper is organized as follows:
Section~\ref{sec:sfla} presents the shuffled frog leaping algorithm.
Section~\ref{sec:sfla-mkp} proposes the application of SFLA for the multidimensional
knapsack problem.
Section~\ref{sec:exp} comprises several computational experiments.
In section~\ref{sec:conc} we make our concluding remarks about the experimental
results.

\section{The shuffled frog leaping}
\label{sec:sfla}

The SFLA is a metaheuristic to solve discrete and combinatorial problems
based on the memetics of living beings and recalls the behavior of a
group of frogs searching for the location that has the maximum amount of available food.
In the following subsections we present the concepts of SCE and PSO to finally
present the shuffled frog leaping algorithm.

\subsection{The shuffled complex evolution}

\subsection{The particle swarm optimization}
\subsection{The shuffled frog leaping algorithm}
% introduzir o SFL
% memeplex... PSO + Complex Evolution
% apresentar o algoritmo
%\cite{Shih-1979}

\section{A SFLA for the MKP}
\label{sec:sfla-mkp}

\section{Computational experiments}
\label{sec:exp}
% Instancias (tightness)
% parametros
% Gráficos:
%   - qualidade
%   - gráfico exemplificando o progresso durante as iterações.
% Futuro: testar cuzamentos de tipos diferentes e com mais individuos.

\section{Conclusions and future remarks}
\label{sec:conc}
% Conseguiu alcançar soluções "near optimal"

%\bibliographystyle{abbrv}
\bibliographystyle{IEEEtran}
\bibliography{../../refs}

%\printbibliography

\end{document}

