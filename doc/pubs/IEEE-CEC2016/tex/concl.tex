
In this paper we addressed the development of a hybrid heuristic for the MKP
implementing a population based algorithm called shuffled complex evolution
for solving the multidimensional knapsack problem assisted by an efficient
variable fixing procedure.
Its performance was verified through several computational experiments.

The SCE algorithm, which combines the ideas of a controlled random search with
the concepts of competitive evolution, proved to be able to achieve fast
convergence ratio, finding good quality near optimal solutions, demanding small
amount of computational time.

The application of the core concept, through a variable fixing procedure for MKP,
proved to be efficient to reduce the size of the problems which provided fast
execution time, producing higher quality solutions.

\scecore algorithm presented faster convergence speed, achieving higher
quality solutions in all cases, achieving at least $99.02\%$ of best known, in less than $2$ seconds
for every instance.
The variable fixing procedure also brought robustness for the method, as the quality
of the solution found increased in case of larger instances.

The hybrid heuristic developed in this paper is very well suitable if it is necessary
to compute a good solution in a small processing time.
The hybrid heuristic could achieved $99.61\%$ on average of quality of the best known solution for
the 270 Chu-Beasley instances and $99.46\%$ on average for the Glover-Kochenberger instances.

Future works include the investigation of different crossing procedures 
and the use of local search in the process of evolving complexes.
Besides we want to investigate the possibility of using the efficiency measure
concept directly on metaheuristics.

