Two main tests was considered:
(a) using the set of problems defined by Chu and Beasley~\cite{Chu-Beasley-1998}
and (b) a set composed by 11 instances provided by Glover and Kochenberger in
\cite{glover1996critical}.

Columns {\bf n} and {\bf m} indicate the size of each instance,
{\bf time} column shows the average execution time (lower is better),
{\bf quality} column shows the average ratio of the solution found and
the best known solution from literature, variance shown in parentheses.
\\[2pt]
\begin{minipage}[c]{0.4\linewidth}
  \begin{center}
      {\bf Table I}: Performance on Chu-Beasley problems. \\
    \begin{tabular}{|r|r|rr|rr|} \cline{3-6}
  \multicolumn{2}{c|}{} &
    \multicolumn{2}{c|}{\bf time (s)} &
    \multicolumn{2}{c|}{\bf quality (\%)} \\ \hline
  \textbf{n}   &
    \textbf{m}  &
    \textbf{SCE} &
    \textbf{\scecore} &
    {\bf SCE} &
    {\bf \scecore}  \\ \hline
100
  &  5 & 1.31\fvar{0.03} & 0.17\fvar{0.00} & 97.60\fvar{0.56} & 99.83\fvar{0.02} \\ \hline
  & 10 & 1.43\fvar{0.04} & 0.26\fvar{0.00} & 96.96\fvar{0.99} & 99.75\fvar{0.04} \\ \hline
  & 30 & 1.75\fvar{0.08} & 1.01\fvar{0.04} & 96.66\fvar{0.66} & 98.89\fvar{0.11} \\ \hline
250
  &  5& 2.87\fvar{0.09} & 0.69\fvar{0.01} & 94.98\fvar{0.33} & 99.92\fvar{0.00} \\ \hline
  & 10 & 3.08\fvar{0.09} & 0.83\fvar{0.01} & 94.95\fvar{0.35} & 99.75\fvar{0.00} \\ \hline
  & 30 & 3.82\fvar{0.14} & 1.45\fvar{0.05} & 94.65\fvar{0.39} & 98.89\fvar{0.04} \\ \hline
500 
  &  5 & 5.74\fvar{0.14} & 1.23\fvar{0.01} & 93.73\fvar{0.28} & 99.86\fvar{0.00} \\ \hline
  & 10 & 5.85\fvar{0.34} & 1.33\fvar{0.03} & 93.65\fvar{0.25} & 99.71\fvar{0.00} \\ \hline
  & 30 & 6.17\fvar{1.12} & 1.84\fvar{0.19} & 93.30\fvar{0.34} & 99.28\fvar{0.01} \\ \hline
\end{tabular}

  \end{center}
  \vfill
\end{minipage}
\begin{minipage}[c]{0.6\linewidth}
  \begin{center}
      {\bf Table II}: Performance on Glover-Kochenberger problems.  \\
    \begin{tabular}{rrr|cc|rr} \hline
  \multirow{2}{*}{\textbf{\#}} &
  \multirow{2}{*}{\textbf{n}} &
  \multirow{2}{*}{\textbf{m}} &
    \multicolumn{2}{c|}{\textbf{time}(s)} &
    \multicolumn{2}{c}{\textbf{quality}(\%)} \\
  &
    &
    &
    \textbf{SCE} &
    \textbf{SCEcr} &
    \textbf{SCE} &
    \textbf{SCEcr} \\ \hline
  01   &  100 &  15 &   1.47\fvar{0.00} & \textbf{0.08}\fvar{0.0} & 97.66\fvar{0.03} & \textbf{99.24}\fvar{0.02} \\ \hline
    02 &  100 &  25 &   1.61\fvar{0.00} & \textbf{0.09}\fvar{0.0} & 97.94\fvar{0.04} & \textbf{98.94}\fvar{0.09} \\ \hline
    03 &  150 &  25 &   2.51\fvar{0.01} & \textbf{0.09}\fvar{0.0} & 97.22\fvar{0.04} & \textbf{99.09}\fvar{0.02} \\ \hline
    04 &  150 &  50 &   3.56\fvar{0.03} & \textbf{0.09}\fvar{0.0} & 97.40\fvar{0.04} & \textbf{98.52}\fvar{0.02} \\ \hline
    05 &  200 &  25 &   3.55\fvar{0.01} & \textbf{0.09}\fvar{0.0} & 96.88\fvar{0.03} & \textbf{99.28}\fvar{0.01} \\ \hline
    06 &  200 &  50 &   4.81\fvar{0.09} & \textbf{0.10}\fvar{0.0} & 97.68\fvar{0.02} & \textbf{98.90}\fvar{0.03} \\ \hline
    07 &  500 &  25 &   7.30\fvar{0.09} & \textbf{0.10}\fvar{0.0} & 97.12\fvar{0.01} & \textbf{99.54}\fvar{0.00} \\ \hline
    08 &  500 &  50 &  12.20\fvar{0.47} & \textbf{0.11}\fvar{0.0} & 97.27\fvar{0.01} & \textbf{99.33}\fvar{0.01} \\ \hline
    09 & 1500 &  25 &  24.61\fvar{1.73} & \textbf{0.12}\fvar{0.0} & 95.40\fvar{0.01} & \textbf{98.22}\fvar{0.00} \\ \hline
    10 & 1500 &  50 &  33.79\fvar{2.44} & \textbf{0.13}\fvar{0.0} & 97.50\fvar{0.00} & \textbf{99.64}\fvar{0.00} \\ \hline
    11 & 2500 & 100 & 121.28\fvar{194.74} & \textbf{0.15}\fvar{0.0} & 97.95\fvar{0.00} & \textbf{99.70}\fvar{0.00} \\ \hline
\end{tabular}

  \end{center}
\end{minipage}
\\

It can be noticed that \scecore achieved high quality solutions, at least $98.22\%$
of best known solution, spending small amount of processing time.
