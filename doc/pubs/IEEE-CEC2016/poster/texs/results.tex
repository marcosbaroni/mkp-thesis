Two main tests was considered:
(a) using the well-known set of problems defined by Chu and Beasley~\cite{Chu-Beasley-1998}
and (b) a large set of randomly generated instances using uniform distribution.
The number of constraints $m$ varies among $5$, $10$ and $30$, and the number
of variables $n$ varies among $100$, $250$ and $500$.
\\[2mm]
\begin{minipage}[c]{0.4\linewidth}
  \begin{center}
    %\begin{tabular}{rrr|cc|cc} \hline
  \multirow{2}{*}{\bf n} &
  \multirow{2}{*}{\bf m} &
  \multirow{2}{*}{\textbf{$\alpha$}} &
    \multicolumn{2}{c|}{\textbf{time} (s)} &
    \multicolumn{2}{c}{\textbf{quality}(\%)} \\
  &
    &
    &
    \textbf{SCE} &
    \textbf{\scecore} &
    \textbf{SCE} &
    {\bf \scecore}  \\ \hline
100   &  5 & 0.25 & 1.22\fvar{0.04} & \textbf{0.17}\fvar{0.00} & 96.51\fvar{0.92} & \textbf{99.73}\fvar{0.04} \\
        &    & 0.50 & 1.34\fvar{0.02} & \textbf{0.18}\fvar{0.00} & 97.42\fvar{0.55} & \textbf{99.86}\fvar{0.01} \\
        &    & 0.75 & 1.37\fvar{0.03} & \textbf{0.17}\fvar{0.00} & 98.87\fvar{0.20} & \textbf{99.91}\fvar{0.00} \\ \cline{2-7}
        & 10 & 0.25 & 1.32\fvar{0.04} & \textbf{0.25}\fvar{0.00} & 95.68\fvar{1.28} & \textbf{99.53}\fvar{0.09} \\
        &    & 0.50 & 1.51\fvar{0.04} & \textbf{0.25}\fvar{0.00} & 96.65\fvar{0.49} & \textbf{99.76}\fvar{0.03} \\
        &    & 0.75 & 1.46\fvar{0.04} & \textbf{0.27}\fvar{0.00} & 98.54\fvar{0.19} & \textbf{99.96}\fvar{0.00} \\ \cline{2-7}
        & 30 & 0.25 & 1.74\fvar{0.06} & \textbf{1.20}\fvar{0.03} & 95.38\fvar{1.01} & \textbf{97.96}\fvar{0.22} \\
        &    & 0.50 & 1.79\fvar{0.08} & \textbf{0.89}\fvar{0.06} & 96.41\fvar{0.63} & \textbf{99.18}\fvar{0.06} \\
        &    & 0.75 & 1.72\fvar{0.09} & \textbf{0.95}\fvar{0.04} & 98.18\fvar{0.33} & \textbf{99.52}\fvar{0.04} \\ \hline
\multirow{2}{*}{250} & 5 & 0.25 & 2.87\fvar{0.07} & \textbf{0.69}\fvar{0.01} & 93.22\fvar{0.64} & \textbf{99.86}\fvar{0.00} \\
        &    & 0.50 & 2.82\fvar{0.11} & \textbf{0.70}\fvar{0.01} & 94.88\fvar{0.21} & \textbf{99.94}\fvar{0.00} \\
        &    & 0.75 & 2.93\fvar{0.08} & \textbf{0.69}\fvar{0.01} & 97.57\fvar{0.10} & \textbf{99.96}\fvar{0.00} \\ \cline{2-7}
        & 10 & 0.25 & 3.08\fvar{0.09} & \textbf{0.87}\fvar{0.01} & 93.14\fvar{0.67} & \textbf{99.58}\fvar{0.01} \\
        &    & 0.50 & 3.03\fvar{0.09} & \textbf{0.79}\fvar{0.02} & 94.55\fvar{0.26} & \textbf{99.79}\fvar{0.00} \\
        &    & 0.75 & 3.12\fvar{0.09} & \textbf{0.84}\fvar{0.01} & 97.16\fvar{0.13} & \textbf{99.88}\fvar{0.00} \\ \cline{2-7}
        & 30 & 0.25 & 3.74\fvar{0.12} & \textbf{1.52}\fvar{0.04} & 93.10\fvar{0.74} & \textbf{98.42}\fvar{0.08} \\
        &    & 0.50 & 3.74\fvar{0.16} & \textbf{1.36}\fvar{0.06} & 94.20\fvar{0.30} & \textbf{99.33}\fvar{0.02} \\
        &    & 0.75 & 3.99\fvar{0.13} & \textbf{1.48}\fvar{0.04} & 96.64\fvar{0.14} & \textbf{99.59}\fvar{0.01} \\ \hline
\multirow{2}{*}{500} & 5 & 0.25 & 5.62\fvar{0.10} & \textbf{1.25}\fvar{0.02} & 91.37\fvar{0.50} & \textbf{99.77}\fvar{0.00} \\
        &    & 0.50 & 5.72\fvar{0.19} & \textbf{1.24}\fvar{0.01} & 93.39\fvar{0.27} & \textbf{99.88}\fvar{0.00} \\
        &    & 0.75 & 5.88\fvar{0.14} & \textbf{1.20}\fvar{0.02} & 96.42\fvar{0.06} & \textbf{99.92}\fvar{0.00} \\ \cline{2-7}
        & 10 & 0.25 & 5.97\fvar{0.17} & \textbf{1.41}\fvar{0.02} & 91.62\fvar{0.50} & \textbf{99.51}\fvar{0.01} \\
        &    & 0.50 & 6.11\fvar{0.23} & \textbf{1.36}\fvar{0.03} & 93.09\fvar{0.20} & \textbf{99.77}\fvar{0.00} \\
        &    & 0.75 & 5.47\fvar{0.61} & \textbf{1.21}\fvar{0.03} & 96.24\fvar{0.06} & \textbf{99.84}\fvar{0.00} \\ \cline{2-7}
        & 30 & 0.25 & 6.20\fvar{1.14} & \textbf{1.96}\fvar{0.22} & 91.37\fvar{0.82} & \textbf{98.76}\fvar{0.02} \\
        &    & 0.50 & 6.26\fvar{1.07} & \textbf{1.82}\fvar{0.14} & 92.56\fvar{0.13} & \textbf{99.42}\fvar{0.01} \\
        &    & 0.75 & 6.05\fvar{1.16} & \textbf{1.73}\fvar{0.20} & 95.97\fvar{0.06} & \textbf{99.67}\fvar{0.00} \\ \hline
\end{tabular}

    \renewcommand{\arraystretch}{1.5}%
\fontsize{8.5pt}{1em}\selectfont 
\begin{center}
\begin{tabular}{|r|r|r|rr|} \hline
\textbf{n}   & \textbf{m}  & \textbf{$\alpha$} & \textbf{SCE t (s)} & \textbf{gap (\%)} \\ \hline
100 & 5 & 0.25 & 0.79 & 96.5 \\
    &   & 0.5 & 0.81 & 97.4 \\
    &   & 0.75 & 0.83 & 98.9 \\ \cline{2-5}
    & 10 & 0.25 & 0.75 & 95.7 \\
    &    & 0.5 & 0.93 & 96.7 \\
    &    & 0.75 & 0.89 & 98.5 \\ \cline{2-5}
    & 30 & 0.25 & 1.01 & 95.4 \\
    &    & 0.5 & 1.07 & 96.4 \\
    &    & 0.75 & 0.99 & 98.2 \\ \cline{2-5}
    & \multicolumn{3}{r}{\textbf{average gap}}  & $\bf 97.1$  \\ \hline \hline
\textbf{n}   & \textbf{m}  & \textbf{$\alpha$} & \textbf{SCE t (s)} & \textbf{gap (\%)} \\ \hline
250 & 5 & 0.25 & 1.72 & 93.2 \\
    &   & 0.5 & 1.75 & 94.9 \\
    &   & 0.75 & 1.78 & 97.6 \\ \cline{2-5}
    & 10 & 0.25 & 1.84 & 93.1 \\
    &    & 0.5 & 1.84 & 94.6 \\
    &    & 0.75 & 1.81 & 97.2 \\ \cline{2-5}
    & 30 & 0.25 & 2.21 & 93.2 \\
    &    & 0.5 & 2.21 & 94.2 \\
    &    & 0.75 & 2.31 & 96.6 \\ \cline{2-5}
    & \multicolumn{3}{r}{\textbf{average gap}}  & $\bf 95.0$  \\ \hline \hline
\textbf{n}   & \textbf{m}  & \textbf{$\alpha$} & \textbf{SCE t (s)} & \textbf{gap (\%)} \\ \hline
500 & 5 & 0.25 & 3.16 & 91.4 \\
    &   & 0.5 & 3.18 & 93.4 \\
    &   & 0.75 & 3.34 & 96.4 \\ \cline{2-5}
    & 10 & 0.25 & 3.39 & 91.7 \\
    &    & 0.5 & 3.37 & 93.1 \\
    &    & 0.75 & 3.44 & 96.2 \\ \cline{2-5}
    & 30 & 0.25 & 3.83 & 91.4 \\
    &    & 0.5 & 3.90 & 92.6 \\
    &    & 0.75 & 3.99 & 96.0 \\ \cline{2-5}
    & \multicolumn{3}{r}{\textbf{average gap}}  & $\bf 93.6$  \\ \hline
\end{tabular}
\end{center}

        \\[2mm]
        \scecore~performance on Chu-Beasley problems.
  \end{center}
\end{minipage}
\begin{minipage}[c]{0.6\linewidth}
  \begin{center}
    \begin{tabular}{rrr|cc|rr} \hline
  \multirow{2}{*}{\textbf{\#}} &
  \multirow{2}{*}{\textbf{n}} &
  \multirow{2}{*}{\textbf{m}} &
    \multicolumn{2}{c|}{\textbf{time}(s)} &
    \multicolumn{2}{c}{\textbf{quality}(\%)} \\
  &
    &
    &
    \textbf{SCE} &
    \textbf{SCEcr} &
    \textbf{SCE} &
    \textbf{SCEcr} \\ \hline
  01   &  100 &  15 &   1.47\fvar{0.00} & \textbf{0.08}\fvar{0.0} & 97.66\fvar{0.03} & \textbf{99.24}\fvar{0.02} \\ \hline
    02 &  100 &  25 &   1.61\fvar{0.00} & \textbf{0.09}\fvar{0.0} & 97.94\fvar{0.04} & \textbf{98.94}\fvar{0.09} \\ \hline
    03 &  150 &  25 &   2.51\fvar{0.01} & \textbf{0.09}\fvar{0.0} & 97.22\fvar{0.04} & \textbf{99.09}\fvar{0.02} \\ \hline
    04 &  150 &  50 &   3.56\fvar{0.03} & \textbf{0.09}\fvar{0.0} & 97.40\fvar{0.04} & \textbf{98.52}\fvar{0.02} \\ \hline
    05 &  200 &  25 &   3.55\fvar{0.01} & \textbf{0.09}\fvar{0.0} & 96.88\fvar{0.03} & \textbf{99.28}\fvar{0.01} \\ \hline
    06 &  200 &  50 &   4.81\fvar{0.09} & \textbf{0.10}\fvar{0.0} & 97.68\fvar{0.02} & \textbf{98.90}\fvar{0.03} \\ \hline
    07 &  500 &  25 &   7.30\fvar{0.09} & \textbf{0.10}\fvar{0.0} & 97.12\fvar{0.01} & \textbf{99.54}\fvar{0.00} \\ \hline
    08 &  500 &  50 &  12.20\fvar{0.47} & \textbf{0.11}\fvar{0.0} & 97.27\fvar{0.01} & \textbf{99.33}\fvar{0.01} \\ \hline
    09 & 1500 &  25 &  24.61\fvar{1.73} & \textbf{0.12}\fvar{0.0} & 95.40\fvar{0.01} & \textbf{98.22}\fvar{0.00} \\ \hline
    10 & 1500 &  50 &  33.79\fvar{2.44} & \textbf{0.13}\fvar{0.0} & 97.50\fvar{0.00} & \textbf{99.64}\fvar{0.00} \\ \hline
    11 & 2500 & 100 & 121.28\fvar{194.74} & \textbf{0.15}\fvar{0.0} & 97.95\fvar{0.00} & \textbf{99.70}\fvar{0.00} \\ \hline
\end{tabular}

        \\[2mm]
        Random generated problems.
        \scecore~performance on Glover-Kochenberger problems.
  \end{center}
\end{minipage}
\\[2mm]
For the set of random instances all best known solution was found by the solver
SCIP running for at least 10 minutes.
SCIP is an open-source integer programming solver which
implements the branch-and-cut algorithm.
