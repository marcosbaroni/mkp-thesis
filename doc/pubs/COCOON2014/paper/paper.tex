\documentclass{llncs}
\usepackage{llncsdoc}

\usepackage[citestyle=authoryear]{biblatex}

\addbibresource{ref.bib}

\usepackage{amsfonts}
\usepackage{amssymb}
\usepackage{amsmath}
\usepackage{algorithm}
\usepackage{algpseudocode}
\usepackage{todonotes}
\usepackage{hyperref}

\title{\it Papers title}

\author{
Marcos Daniel Valad\~ao Baroni,
Jo\~ao Carlos,
Fl\'avio Miguel Varej\~ao,
Diego Luchi,\\
}

\institute{
N\'ucleo de Infer\^encia em Algoritmos (NINFA) - Departamento de Inform\'atica\\
Universidade Federal do Esp\'irito Santo\\
Vit\'oria, Esp\'irito Santo, Brasil
}


\begin{document}

\maketitle

\begin{abstract}
Electricity fraud corresponds to a major source of loss of Electricity Distribution Companies (EDCO's) in developing countries.
To attenuate this, EDCO's have at their disposal several loss reduction actions.
These actions must be performed to reach a given energy loss reduction goal, established
by the regulatory agency.
If this goal is not reached, the profit margin of the company is reduced.
This work defines a formal model for the action allocation task 
and proposes an exact mathematic programming method for solving small instances of
the problem.
%We compare the techniques considering running time and solution quality.
\end{abstract}

\section{Introduction}
\label{sec:introduction}
Electricity loss due to fraud in developing countries is a major issue among
their electricity distribution companies (EDCOs).
According to technical reports from a brazilian EDCO, about 13\% of overall
distributed electricity is lost due to fraud, wheras only \%7 is lost due to
inherent physical phenomena from electricity transmission~\cite{ANEEL2012} .
For the EDCOs this excessive non-technical loss represents a profit reduction
and ultimately an impact on the quality of the distribution service, which
motivates even more non-technical losses by consumers.

%In principle, EDCO's could increase the charge of consumers to payoff the loss of electricity due to non-technical sources.
%However, the Brazilian regulatory agency of electricity (ANEEL) dictates that only a given amount of non-technical loss
%may be covered by increasing the charges of non-fraudulent clients. 
%This electricity threshold is known as \textit{non-technical loss reducing goal}.
%The electricity loss above the goal cannot be paid by consumers and results in 
%profit reduction to the EDCO's.

%In practice, the existence of a non-technical loss reducing goal means that EDCO's are forbidden to charge clients more than a fixed value per Kilowatt/hour to
%attenuate losses due to non-technical sources.
%The non-technical loss reducing goal (and indirectly the maximum Kilowatt/hour charge) is established by ANEEL by studying the EDCO's 
%profile and the historical non-technical losses behaviour of similar regions of the country.
%Additionally, the EDCO's are required to \textit{reduce} the Kilowatt/hour electricity charge if they manage to mitigate
%electricity loss more than the established non-technical loss reducing goal. 
%This means that there is no gain to the EDCO's if they reduce the non-technical losses more than the established goal.

%Usually, the non-technical loss reducing goal is given for the following three years,
%defining a \textit{non-technical loss reducing curve},
%so that EDCO's may plan their \textit{non-technical loss reducing actions} for the
%following years.
%If EDCO's don't meet the imposed non-technical loss reduction curve in a given year, their profit in that year
%is reduced by the corresponding value of the electricity under the goal. For instance, if the regulatory
%agency establishes that the non-technical loss reduction goal is $40\,GW$, $50\,GW$, $60\,GW$, and the EDCO achieves $30\,GW$, $40\,GW$, $60\,GW$;
%$20\,GW$ won't be sold to consumers and will ultimately represent a profit loss.
%However, if the EDCO's reduces more electricity loss than the established goal in a year,
%they are required to reduce their gains to match the profit that they would have had in that year if they were exactly
%on the non-technical loss reduction goal.

%This scenario introduces a new problem to the EDCO's: given an investment portfolio comprised of 
%several non-technical loss reducing actions, a non-technical loss reduction curve and a yearly budget, which is the best possible 
%allocation of actions that maximizes the return of the investment? That is, the 
%allocation with the greatest profit or, in other words, the best \textit{investment strategy}.
%For instance, if the EDCO chooses to do nothing, it would not be able to charge for some portion of the
%distributed electricity,
%since this usually costs more than performing the non-technical loss reducing actions, doing nothing would not
%be the optimal course of action. On the other hand, reduce the loss to a greater extent than the
%imposed reduction curve is also not the optimal decision, since the over-reduced electricity loss does not
%increase the overall EDCO profit.

%The current methodology used by EDCO's is to manually select the fraud reducing actions in
%a heuristic trial-and-error fashion. As we shall present in this work, choosing the best non-technical loss reducing actions is
%a complex combinatorial problem, thus the current \textit{de facto} procedure hardly finds to optimum solution of the problem.
%Since reducing technical losses is an important activity of EDCOS's, this motivates the use of computational techniques to 
%find the best, or at least good, solutions to our problem.


\subsection{Organization}
%The reminder of the paper is organized as follows: Section~\ref{sec:defi} defines the
%problem of choosing non-technical loss reducing actions more formally.
%Section~\ref{sec:relat} is comprised of an investigation of related works in the literature 
%and a explanation of the inherent complexity of our problem.
%Section~\ref{sec:alter} introduces our methods for solving the defined problem. 
%In section~\ref{sec:exp} we present our experimental evaluation. Finally, in section~\ref{sec:con} 
%we make our concluding remarks about the experimental results and lay ground 
%for future work.

                 


\section{Problem Definition}
\label{sec:definition}
%To apply optimization algorithms in our problem we first need to define it formally. Lets assume
%that we must maximize the \textit{net present value} (the present investment return considering 
%inflation) for a reduction plan of $M$ years, given:

%\begin{itemize}
%    \item a yearly budget for a set of $L$ resources $o_{il}$, $1 \le i \le M$, $1 \le l \le L$;
%    
%    \item a fraud reduction goal $g_i$, $1 \le i \le M$ that represents the
%    amount of electricity loss that must be reduced;
%
%    \item an \textit{internal rate of return} $r$, that represents the yearly depreciation of the investment
%    (constant for all investments and years). 
%
%\end{itemize}

%We also assume that there are several actions to choose from a portfolio of actions of size $N$.
%Each action $j$, $1 \le j \le N$, contained in the portfolio has several parameters: 
%\begin{itemize}
%    \item the electricity value $v_j$, that represents the value of the unit of electricity for each action $j$ in portfolio,
%    i.e., the value of each Kilowatt of electricity;
%    \item $m_j$, the maximum number of times that action $j$ may be performed, we
%    shall refer to it as \textit{the market} of the action;
%    \item $u_{j,i}$ the maximum number of times that action $j$ may be performed in the $i$-th year;
%    \item $c_{j,l}$, the $l$-th cost associated to each execution of action $j$;
%    \item $e_{j,k}$, the amount of lost electricity that the action $j$ will reduce after 
%    $k$ years it was executed, when taken once;
%    \item a set $D_j \subset \mathbb{N^*}$ that represents which actions must be performed prior action $j$.
%    For each action $j$ the action $D_{j,d}$ ($1 \le d \le |D_j|$) must be executed a number of times defined by
%    \item the parameter $Q_{j,d} \in \mathbb{R^+}$.
%\end{itemize}

%Our objective is to find a set of values for variables $x_{j,i}$, $\forall i,j$, $x_{j,i} \in \mathbb{N} $, 
%the number of times that the action $j$ will be performed in the $i$-th year. We wish to choose a combination
%of values that maximizes the net present value and is under the fraud reduction goal for all years. 

%The constraints of the problem are the yearly budget restriction,
%\begin{equation}
%    \sum_{j=1}^{N} x_{j, i} \cdot c_{j,l} \le o_{i,l} \, \forall i, l,
%	\label{eq:budget}
%\end{equation}
%the market restriction,
%\begin{equation}
%     \sum_{i=1}^{M} x_{j, i} \le m_j \, \forall j,
%	\label{eq:market}
%\end{equation}
%the maximum number of times the action $x_{j, i}$ may be performed in $i$-th year,
%\begin{equation}
%     x_{j, i} \le u_{j, i} \, \forall j, i,
%	\label{eq:maxacts}
%\end{equation}
%the goal restriction,
%\begin{equation}
%    \label{eq:goal}
%    \sum_{j=1}^{N} \sum_{k=1}^{M}R_{i,j,k}(\bar{x}) \leq g_i \, \forall i, \\
%\end{equation}
%$R_{i,j,k}$ being the fraud reduction in the $k$-th year by the action $j$ taken on $i$-th year, defined as
%\begin{equation}
%    \label{eq:rec}
%    R_{i,j,k}(\bar{x}) = x_{j, i} \cdot e_{j, k - i + 1} \, \forall k \geq i,
%\end{equation}
%and the dependency restriction for all actions and for all years,
%\begin{equation}
%    \label{eq:dependency}
%    \forall j,k \quad \sum_{i=1}^{k} x_{d, i} \ge x_{j, d} \cdot Q_{j, d} \, \forall d \in D_j.
%\end{equation}

%To introduce the objective function we must define some auxiliary concepts:
%The yearly total cost, $C_i$,
%\begin{equation}
%\label{eq:cost}
%C_{i}(\bar{x}) =  \sum_{j=1}^{N} \sum_{l=1}^{L} x_{j, i} \cdot c_{j,l} \, \forall i,
%\end{equation}
%the gained profit in $i$-th year, due to elimination of the fraud, $V_i$,
%\begin{equation}
%    V_{i}(\bar{x}) = \sum_{j=1}^{N} \sum_{k=1}^{M} R_{k, j, i}(\bar{x}) \cdot v_j \, \forall i,
%\end{equation}
%by definition $V_i - C_i$ is the total cash flow in $i$-th year. 

%Finally, we may define the objective function as:
%\begin{equation}
%    \label{eq:objective}
%    max(O(\bar{x})) = max\left(\sum_{i=1}^{M} \frac{V_i(\bar{x}) - C_i(\bar{x})}{(1+r)^i}\right).
%\end{equation}

%The objective function represents the net present value, that is,
%the sum of the yearly cash flows, $V_i - C_i$, adjusted by the internal rate of return for all years.

%\subsection{Realistic Version}

%To adapt the previously defined problem to the current state of our local EDCO, we simplify it in several  ways:

%\begin{itemize}
%%terminar de editar
%\item We consider that the yearly budgets are insufficient to surpass the goal for any given year;
%%because this is usually the case in real world scenarios. %This consideration
%%renders the equation~\ref{eq:objective} imprecise. 

%\item We consider all actions on portfolio have an positive cash flow;

%\item We ignore the dependencies among actions, that is, the set $D_j$ is empty
%for all actions; %This is necessary to simplify the analysis we perform in this work.

%\item We assume that both the vectors that define the costs of the $j$-th  action $c_j$
%and the $i$-th budget $o_i$ have size 1;%, that is, actions use resources of only
%one kind.

%\item We assume that the electricity  value $v_j$ is constant for all actions;

%\item We assume that the value of $\sum_{i=0}^{N} u_{j,i} \leq m_j, \forall j$.
%%Meaning that there is no yearly limit to the amount of times actions may be
%%performed.

%\end{itemize}

%The simplifications may be altered accordingly to the current
%state of the distributer. Other companies may eliminate or consider other restrictions,
%and as far as we know, the model is general enough to model for the
%majority of possible scenarios.

%\todo[inline]{Terminar seção.}






\section{Related Work}
\label{sec:related}
\section{Related Work} 
\label{sec:relat}

The exact formulation of the problem has shown to be quite particular and
no work addressing a similar problem was found in literature.
If the depreciation of the investments is not considered ($r=0$),
the problem may  be viewed as a \textit{Partially-Ordered Multidimensional
Multiple Knapsack Problem} (POMMKP). Which is, as far as we know, a
novel version of the classical Knapsack Problem.

Additionally, If we consider just the cost $c_{j,l}$ of each action (weight), the
reduction of lost electricity  $e_{j,k}$ of each action (profit) and the yearly budget (capacity)
on a single year situation ($M=1$) the problem could be classified as a simple
\textit{Knapsack Problem}~\cite{pisinger1995}.
If we now add the dependency restriction, the problem could be considered
a \textit{Partially-Ordered Knapsack Problem}~\cite{pok2002}.
If we consider multiple years ($M > 1$) instead of the dependency restriction, it would characterize
a \textit{Multiple Knapsack Problem}, with each year with its own capacity.
Considering multiple resources ($M=1, L > 1$) and no dependency, the problem can be formulated as a
\textit{Multidimensional Knapsack Problem}.

All the problems mentioned above are hard to solve optimally, these problems are known 
as $\mathcal{NP}$-hard problems and are no know polynomial time algorithms to solve, unless $\mathcal{P} = \mathcal{NP}$.
For the classical knapsack problem there is a FPTAS (\textit{Fully Polynomial Time Approximation Scheme}),
while for the others variants mentioned above are hard to aproximate and there are only PTAS 
(\textit{Polynomial Time Approximation Scheme}) to solve them with a certain degree of 
approximation of the optimal solution, ~\cite{pok2002, puchinger2006core, dawande2000approximation}. 

The POMMKP can be considered as a generalization of the above mentioned problems,
thus we may conclude that POMMKP is at least as difficult as any of them.
Since our problem is a generalization of the POMMKP, it is as difficult as the POMMKP.
For this reason there is no algorithm that solves our problem in polynomial time (considering
$\mathcal{P} \ne \mathcal{NP}$).
Given the difficulty of approximating the optimal solution of this problem through exact algorithms, heuristics methods are necessary 
for solving larger instances of the problem.


\section{Problem Solving Methods}
\label{sec:solving}
%To solve our allocation problem we have developed three methodologies: 1) an exact algorithm
%using integer programming, 2) a heuristic Gradient Ascent algorithm using the LP-relaxed version of the integer
%programming model as a starting solution (GALP) and 3) a Tabu Search
%algorithm that also uses the LP solution as a starting point (TSLP). In the following subsections we present each approach and comment on their particularities.
%%The implementation of all methos is available at 

%%To solve our allocation problem we have developed three methodologies: an exact approach
%%and two heuristic algorithms.
%%The exact approach considered the realistic version of the MIP formulation, previously defined on Section~\ref{sec:defi}.
%%The first heuristic algorithm applies a gradient ascent method on an initial solution from a LP-relaxed model of the MIP formulation.
%
%%The second heuristic algorithm takes this same initial 
%
%%As duas baordagens heuristicas sao inicializadas com uma solucao inicial 
%%As outras duas abordagens heuristicas sao inicializadas com a solucao otima truncated do LP-relaxation of the original formulation.
%
%%using integer programming and two heuristic algorithms using the LP-relaxed version of MIP
%%with an a-posteriori heuristic allocation scheme.
%%algorithm. In the following subsections we present each approach and comment on their particularities.

\subsection{Exact Mathematical Programming Approach}
%The exact approach considers the realistic version of the 
%MIP formulation, previously defined in Section~\ref{sec:defi}.
%To solve this formulation, we have used a branch-and-bound method~\cite{lawler1966branch}
%included in the GLPK solver \cite{GLPK}.

\subsection{Gradient Ascent using the LP solution (GALP)}
%%However this method is known to get stuck in local minima.
%We use a simple heuristic approach, namely Gradient Ascent (GA), using
%the truncated solution of the relaxed version of the linear problem (considering $x_{j,i} \in [0,u_{j,i}]$) as a initial search point.
%Since the relaxed version of the linear problem is easily solvable by the simplex 
%algorithm ~\cite{dantzig1955generalized} and the Gradient Ascent algorithm has 
%fast convergence characteristics, this approach is very efficient, 
%converging in less than 10 iterations in most of our test instances. 
%However because it is a greedy heuristic, it has the flaw of getting stuck in local minima,
%because it never accepts a solution if it is worse than the current one.
%Algorithm \ref{alg:ga} depicts the procedure.

%We have compared the solution quality randomly starting the $GA$ algorithm in different locations of the
%search space and found that in our experiments it never found a better solution
%than using the truncated Linear Problem solution as a start.

%%\begin{figure}
%\begin{algorithm}[H]
%\begin{algorithmic}[1]
%\Function{Gradient Ascent}{Initial solution $S_{ini}$}
%\State $S_{best} \gets S_{ini}$
%\State $HasImproved \gets True$
%\While{$HasImproved$}
%    \State $HasImproved \gets False$
%    \For{\label{alg:neigh}Each $Neig$ in $Neighborhood(V_{best})$}
%        \If{$Evaluate(Neig) > Evaluate(S_{ini})$}
%            \State $S_{best} \gets Neig$
%            \State $HasImproved \gets True$
%        \EndIf
%    \EndFor
%\EndWhile
%\State \Return $S_{best}$
%\EndFunction
%\end{algorithmic}
%\caption{Gradient Ascent Algorithm}
%\label{alg:ga}
%\end{algorithm}
%%\end{figure}

%Line \ref{alg:neigh} of the algorithm uses the function $Neighborhood(Solution)$, that
%returns a set containing all possible solutions generated from $Solution$ by adding a
%single action to the current solution.

%
\subsection{Tabu Search using the LP solution (TSLP)}

%The second applied technique was the Tabu Search algorithm ~\cite{glover1997tabu}, again using the Linear Programming solution as the initial candidate solution. In~\cite{puchinger2010}, the authors claim that Tabu Search initialized with LP solution is the best known method to solve Multidimensional Knapsack Problem. We expect that the Tabu Search algorithm to be more prone to scape
%local minima, since it can accept solutions worst than the current one. 
%The pseudocode is depicted in the algorithm~\ref{alg:ts} ~\cite{brownlee2011clever}.

%We have also compared the solution quality if we randomly start the $TS$ algorithm in the
%search space and found that in our experiments it never found a better solution
%than using the truncated Linear Problem solution as a start.

%%\begin{figure}
%\begin{algorithm}[H]
%\begin{algorithmic}[1]
%\Function{Tabu Search}{Initial solution $S_{ini}$}
%\State $S \gets S_{ini}$
%\State $S_{best} \gets S$
%\State $TabuList \gets \emptyset$
%\While{$StoppingCondition$}
%    \State $CandList \gets \emptyset$
%    \For{\label{alg:tabuneig}Each $S_{cand}$ in $TabuNeighborhood(S)$}
%        \If{$featureDiff(S_{cand},S) \notin TabuList$}
%            \State $CandList \gets CandList + S_{cand}$
%        \EndIf
%    \EndFor
%    \State $S \gets BestCand(CandList)$
%    \If{$Fitness(S) > Fitness(S_{best})$}
%        \State $TabuList \gets TabuList + featureDiff(S, S_{best})$
%        \State $S_{best} \gets S$
%        \While{$size(TabuList) > maxSize$}
%            \State $RemoveOldest(TabuList)$
%        \EndWhile
%    \EndIf
%\EndWhile
%\State \Return $S_{best}$
%\EndFunction
%\end{algorithmic}
%\caption{Tabu Search Algorithm}
%\label{alg:ts}
%\end{algorithm}
%%\end{figure}

%The for loop on Line \ref{alg:tabuneig} uses the function $TabuNeighborhood(Solution)$,
%that returns a neighborhood of  solutions generated from $Solution$.
%In our case, we chose to return in this list the same neighborhood of GALP
%concatenated with a randomly generated set of size $n_{tabu}$ that differ from the base solution
%by at most three actions. For instance, if $Solution = \{10, 12, 30, 10\}$, a possible return of
%$TabuNeighborhood(Solution)$ could be $\{\{12, 11, 30, 10\},\{10,12,29,10\},\{9,11,30,10\}\}$.

%%We opted to use the actual solution of the problem as a tabu element instead of a movement.
%%Albeit this is not the standard approach, we have found it to be more effective, converging
%%faster and yielding statistical equivalent results in comparison to the standard approach
%%in our problem.

 
 


\section{Experiments}
\label{sec:experiments}
- Base de dados utilizaca \\
- Parametros dos algoritmos \\
- Análise dos resultados (comparação)


\section{Conclusion and Future Works}
\label{sec:conclusion}

%This works presents a modeling of a relevant problem to Electricity Distribution Companies in developing countries.
%The modeling yields a challenging optimization problem. We propose a exact technique to solve easier instances
%of the problem and two heuristics to tackle the more difficult instances.

%We conclude that our incarnation  of the classical Knapsack is sensible to the correlation between weight and cost, as
%predicted by the literature on the problem.

%We have tested two heuristic algorithm, namely Tabu Searh using the Linear Problem solution as an initial search point
%(TSLP) and Gradient Ascent using the Linear Problem solution as an initial search point (GALP), both achieving good solutions and
%execution times. In particular, the TSLP algorithm was statically better than the GALP algorithm
%in our test instances, with bigger running times.

%Future work includes the investigation an algorithm for the complete version of the problem and
%a deeper study of what makes some instances take more time than others despite 
%having the same amount of years, actions and $\alpha$ and being randomly generated
%in a similar way.

%\vspace{1cm}


\end{document}

