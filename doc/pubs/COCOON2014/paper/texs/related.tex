%The exact formulation of the problem has shown to be quite particular and
%no work addressing a similar problem was found in literature.
%If the depreciation of the investments is not considered ($r=0$),
%the problem may  be viewed as a \textit{Partially-Ordered Multidimensional
%Multiple Knapsack Problem} (POMMKP). Which is, as far as we know, a
%novel version of the classical Knapsack Problem.

%Additionally, If we consider just the cost $c_{j,l}$ of each action (weight), the
%reduction of lost electricity  $e_{j,k}$ of each action (profit) and the yearly budget (capacity)
%on a single year situation ($M=1$) the problem could be classified as a simple
%\textit{Knapsack Problem}~\cite{pisinger1995}.
%If we now add the dependency restriction, the problem could be considered
%a \textit{Partially-Ordered Knapsack Problem}~\cite{pok2002}.
%If we consider multiple years ($M > 1$) instead of the dependency restriction, it would characterize
%a \textit{Multiple Knapsack Problem}, with each year with its own capacity.
%Considering multiple resources ($M=1, L > 1$) and no dependency, the problem can be formulated as a
%\textit{Multidimensional Knapsack Problem}.

%All the problems mentioned above are hard to solve optimally, these problems are known 
%as $\mathcal{NP}$-hard problems and are no know polynomial time algorithms to solve, unless $\mathcal{P} = \mathcal{NP}$.
%For the classical knapsack problem there is a FPTAS (\textit{Fully Polynomial Time Approximation Scheme}),
%while for the others variants mentioned above are hard to aproximate and there are only PTAS 
%(\textit{Polynomial Time Approximation Scheme}) to solve them with a certain degree of 
%approximation of the optimal solution, ~\cite{pok2002, puchinger2006core, dawande2000approximation}. 

%The POMMKP can be considered as a generalization of the above mentioned problems,
%thus we may conclude that POMMKP is at least as difficult as any of them.
%Since our problem is a generalization of the POMMKP, it is as difficult as the POMMKP.
%For this reason there is no algorithm that solves our problem in polynomial time (considering
%$\mathcal{P} \ne \mathcal{NP}$).
%Given the difficulty of approximating the optimal solution of this problem through exact algorithms, heuristics methods are necessary 
%for solving larger instances of the problem.

