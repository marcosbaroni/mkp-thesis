To apply optimization algorithms in our problem we first need to define it formally. Lets assume
that we must maximize the \textit{net present value} (the present investment return considering 
inflation) for a reduction plan of $M$ years, given:

\begin{itemize}
    \item a yearly budget for a set of $L$ resources $o_{il}$, $1 \le i \le M$, $1 \le l \le L$;
    
    \item a fraud reduction goal $g_i$, $1 \le i \le M$ that represents the
    amount of electricity loss that must be reduced;

    \item an \textit{internal rate of return} $r$, that represents the yearly depreciation of the investment
    (constant for all investments and years). 

\end{itemize}

We also assume that there are several actions to choose from a portfolio of actions of size $N$.
Each action $j$, $1 \le j \le N$, contained in the portfolio has several parameters: 
\begin{itemize}
    \item the electricity value $v_j$, that represents the value of the unit of electricity for each action $j$ in portfolio,
    i.e., the value of each Kilowatt of electricity;
    \item $m_j$, the maximum number of times that action $j$ may be performed, we
    shall refer to it as \textit{the market} of the action;
    \item $u_{j,i}$ the maximum number of times that action $j$ may be performed in the $i$-th year;
    \item $c_{j,l}$, the $l$-th cost associated to each execution of action $j$;
    \item $e_{j,k}$, the amount of lost electricity that the action $j$ will reduce after 
    $k$ years it was executed, when taken once;
    \item a set $D_j \subset \mathbb{N^*}$ that represents which actions must be performed prior action $j$.
    For each action $j$ the action $D_{j,d}$ ($1 \le d \le |D_j|$) must be executed a number of times defined by
    \item the parameter $Q_{j,d} \in \mathbb{R^+}$.
\end{itemize}

Our objective is to find a set of values for variables $x_{j,i}$, $\forall i,j$, $x_{j,i} \in \mathbb{N} $, 
the number of times that the action $j$ will be performed in the $i$-th year. We wish to choose a combination
of values that maximizes the net present value and is under the fraud reduction goal for all years. 

The constraints of the problem are the yearly budget restriction,
\begin{equation}
    \sum_{j=1}^{N} x_{j, i} \cdot c_{j,l} \le o_{i,l} \, \forall i, l,
	\label{eq:budget}
\end{equation}
the market restriction,
\begin{equation}
     \sum_{i=1}^{M} x_{j, i} \le m_j \, \forall j,
	\label{eq:market}
\end{equation}
the maximum number of times the action $x_{j, i}$ may be performed in $i$-th year,
\begin{equation}
     x_{j, i} \le u_{j, i} \, \forall j, i,
	\label{eq:maxacts}
\end{equation}
the goal restriction,
\begin{equation}
    \label{eq:goal}
    \sum_{j=1}^{N} \sum_{k=1}^{M}R_{i,j,k}(\bar{x}) \leq g_i \, \forall i, \\
\end{equation}
$R_{i,j,k}$ being the fraud reduction in the $k$-th year by the action $j$ taken on $i$-th year, defined as
\begin{equation}
    \label{eq:rec}
    R_{i,j,k}(\bar{x}) = x_{j, i} \cdot e_{j, k - i + 1} \, \forall k \geq i,
\end{equation}
and the dependency restriction for all actions and for all years,
\begin{equation}
    \label{eq:dependency}
    \forall j,k \quad \sum_{i=1}^{k} x_{d, i} \ge x_{j, d} \cdot Q_{j, d} \, \forall d \in D_j.
\end{equation}

To introduce the objective function we must define some auxiliary concepts:
The yearly total cost, $C_i$,
\begin{equation}
\label{eq:cost}
C_{i}(\bar{x}) =  \sum_{j=1}^{N} \sum_{l=1}^{L} x_{j, i} \cdot c_{j,l} \, \forall i,
\end{equation}
the gained profit in $i$-th year, due to elimination of the fraud, $V_i$,
\begin{equation}
    V_{i}(\bar{x}) = \sum_{j=1}^{N} \sum_{k=1}^{M} R_{k, j, i}(\bar{x}) \cdot v_j \, \forall i,
\end{equation}
by definition $V_i - C_i$ is the total cash flow in $i$-th year. 

Finally, we may define the objective function as:
\begin{equation}
    \label{eq:objective}
    max(O(\bar{x})) = max\left(\sum_{i=1}^{M} \frac{V_i(\bar{x}) - C_i(\bar{x})}{(1+r)^i}\right).
\end{equation}

The objective function represents the net present value, that is,
the sum of the yearly cash flows, $V_i - C_i$, adjusted by the internal rate of return for all years.

\subsection{Realistic Version}

To adapt the previously defined problem to the current state of our local EDCO, we simplify it in several  ways:

\begin{itemize}
%terminar de editar
\item We consider that the yearly budgets are insufficient to surpass the goal for any given year;
%because this is usually the case in real world scenarios. %This consideration
%renders the equation~\ref{eq:objective} imprecise. 

\item We consider all actions on portfolio have an positive cash flow;

\item We ignore the dependencies among actions, that is, the set $D_j$ is empty
for all actions; %This is necessary to simplify the analysis we perform in this work.

\item We assume that both the vectors that define the costs of the $j$-th  action $c_j$
and the $i$-th budget $o_i$ have size 1;%, that is, actions use resources of only
%one kind.

\item We assume that the electricity  value $v_j$ is constant for all actions;

\item We assume that the value of $\sum_{i=0}^{N} u_{j,i} \leq m_j, \forall j$.
%Meaning that there is no yearly limit to the amount of times actions may be
%performed.

\end{itemize}

The simplifications may be altered accordingly to the current
state of the distributer. Other companies may eliminate or consider other restrictions,
and as far as we know, the model is general enough to model for the
majority of possible scenarios.

%\todo[inline]{Terminar seção.}




