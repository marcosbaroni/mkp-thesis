Electricity loss due to fraud in developing countries is a major issue among
their electricity distribution companies (EDCOs).
According to technical reports from a brazilian EDCO, about 13\% of overall
distributed electricity is lost due to fraud, wheras only \%7 is lost due to
inherent physical phenomena from electricity transmission~\cite{ANEEL2012} .
For the EDCOs this excessive non-technical loss represents a profit reduction
and ultimately an impact on the quality of the distribution service, which
motivates even more non-technical losses by consumers.

%In principle, EDCO's could increase the charge of consumers to payoff the loss of electricity due to non-technical sources.
%However, the Brazilian regulatory agency of electricity (ANEEL) dictates that only a given amount of non-technical loss
%may be covered by increasing the charges of non-fraudulent clients. 
%This electricity threshold is known as \textit{non-technical loss reducing goal}.
%The electricity loss above the goal cannot be paid by consumers and results in 
%profit reduction to the EDCO's.

%In practice, the existence of a non-technical loss reducing goal means that EDCO's are forbidden to charge clients more than a fixed value per Kilowatt/hour to
%attenuate losses due to non-technical sources.
%The non-technical loss reducing goal (and indirectly the maximum Kilowatt/hour charge) is established by ANEEL by studying the EDCO's 
%profile and the historical non-technical losses behaviour of similar regions of the country.
%Additionally, the EDCO's are required to \textit{reduce} the Kilowatt/hour electricity charge if they manage to mitigate
%electricity loss more than the established non-technical loss reducing goal. 
%This means that there is no gain to the EDCO's if they reduce the non-technical losses more than the established goal.

%Usually, the non-technical loss reducing goal is given for the following three years,
%defining a \textit{non-technical loss reducing curve},
%so that EDCO's may plan their \textit{non-technical loss reducing actions} for the
%following years.
%If EDCO's don't meet the imposed non-technical loss reduction curve in a given year, their profit in that year
%is reduced by the corresponding value of the electricity under the goal. For instance, if the regulatory
%agency establishes that the non-technical loss reduction goal is $40\,GW$, $50\,GW$, $60\,GW$, and the EDCO achieves $30\,GW$, $40\,GW$, $60\,GW$;
%$20\,GW$ won't be sold to consumers and will ultimately represent a profit loss.
%However, if the EDCO's reduces more electricity loss than the established goal in a year,
%they are required to reduce their gains to match the profit that they would have had in that year if they were exactly
%on the non-technical loss reduction goal.

%This scenario introduces a new problem to the EDCO's: given an investment portfolio comprised of 
%several non-technical loss reducing actions, a non-technical loss reduction curve and a yearly budget, which is the best possible 
%allocation of actions that maximizes the return of the investment? That is, the 
%allocation with the greatest profit or, in other words, the best \textit{investment strategy}.
%For instance, if the EDCO chooses to do nothing, it would not be able to charge for some portion of the
%distributed electricity,
%since this usually costs more than performing the non-technical loss reducing actions, doing nothing would not
%be the optimal course of action. On the other hand, reduce the loss to a greater extent than the
%imposed reduction curve is also not the optimal decision, since the over-reduced electricity loss does not
%increase the overall EDCO profit.

%The current methodology used by EDCO's is to manually select the fraud reducing actions in
%a heuristic trial-and-error fashion. As we shall present in this work, choosing the best non-technical loss reducing actions is
%a complex combinatorial problem, thus the current \textit{de facto} procedure hardly finds to optimum solution of the problem.
%Since reducing technical losses is an important activity of EDCOS's, this motivates the use of computational techniques to 
%find the best, or at least good, solutions to our problem.


\subsection{Organization}
%The reminder of the paper is organized as follows: Section~\ref{sec:defi} defines the
%problem of choosing non-technical loss reducing actions more formally.
%Section~\ref{sec:relat} is comprised of an investigation of related works in the literature 
%and a explanation of the inherent complexity of our problem.
%Section~\ref{sec:alter} introduces our methods for solving the defined problem. 
%In section~\ref{sec:exp} we present our experimental evaluation. Finally, in section~\ref{sec:con} 
%we make our concluding remarks about the experimental results and lay ground 
%for future work.

                 
