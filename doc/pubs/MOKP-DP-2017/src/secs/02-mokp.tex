% Breve definicao de multiobjective opt
A general multiobjective optimization problem can be described as a vector
function $f$ that maps a tuple of $n$ parameters (decision variables) to a tuple
of $\np$ objectives.
Formally:
\begin{align*}
  \text{min/max} ~ \sol{y} &= f(\sol{x}) = 
    \big(f_1(\sol{x})
    ,f_2(\sol{x})
    ,\ldots
    ,f_{\np}(\sol{x})\big) \\
  \text{subject to} ~ \sol{x} & = (x_1, x_2, \ldots, x_n) \in X
\end{align*}
where $\sol{x}$ is the \emph{decision vector}, $X$ denotes the set
of feasible solutions, and $\sol{y}$ is the \emph{objective vector} where
each objective has to be minimized (or maximized).

Considering two decision vectors $\sol{a}, \sol{b} \in X$, $\sol{a}$ is said to
\emph{dominate} $\sol{b}$ if, and only if $\sol{a}$ is at least as good as $\sol{b}$
in all objectives and better than $\sol{b}$ in at least one objective.
For shortening we will say that $\sol{a}$ dominates $\sol{b}$ by saying $\dominates{a}{b}$.
Formally:
\begin{displaymath}
    \dominates{a}{b} = \left\{
      \begin{array}{l}
          \forall i \in \{1, 2, \ldots, \np\}: f_i(\sol{a}) \geq f_i(\sol{b}) ~\text{and}\\
          \exists j \in \{1, 2, \ldots, \np\}: f_j(\sol{a}) > f_j(\sol{b})
  \end{array} \right.
\end{displaymath}
This relation is based on the dominance relation proposed by Weingartner and
Ness for the multidimensional knapsack problem~\cite{weingartner1967methods}.

A feasible solution $\sol{a} \in X$ is called \emph{efficient} %or \emph{non-dominated}
if its not dominated by any other feasible solution.
%if there is no other feasible solution $\sol{b} \in X$ such that $\sol{b}$ dominates $\sol{a}$.
The set of all efficient solutions of a multiobjective optimization problem is
known as \emph{Pareto optimal}.
Solving a multiobjective problem consists in giving its Pareto optimal set.

An instance of a multiobjective knapsack problem (MOKP) with $\np$
objectives consists of an integer capacity $W > 0$ and $n$ items.
Each item $i$ has a positive weight $w_i$ and non negative integer
profits $p_i^1, p_i^2, \ldots, p_{i}^{\np}$.
Each profit $p_i^k$ represents the contribution of the $i$-th item for $k$-th objective.
A solution is represented by a set $\sol{s} \subseteq \{1, \ldots, n\}$
containing the indexes of the itens included in the solution.
A solution is feasible if the total weight included in the knapsack does
not exceed its capacity.
Formally the definition of the problem is:
\begin{align*}
    \text{max   } & f(\sol{s}) = 
      \big(f_1(\sol{s}) ,f_2(\sol{s}) ,\ldots ,f_{\np}(\sol{s})\big) \\
    \text{subject to   } & w(\sol{s}) < W \\
    & \sol{s} \subseteq I_n\\
    \text{where} \phantom{mmmmm} \\
    I_n &= \{1, \ldots, n\}\\
    f_j(\sol{s}) &= \sum_{i \in \sol{s}} p^j_i\\
    w(\sol{s}) &= \sum_{i \in \sol{s}} w_i
\end{align*}

The MOKP is considered a \nphard{} problem since it is a generalization
of the well-known 0$-$1 knapsack problem, in which $\np = 1$.
It is quite difficult to determine the Pareto optimal set for the MOKP,
especially for high dimension instances, in which the solution it self
(Pareto optimal set) tends to grow exponentially.
Even for the bi-objective case, small problems may prove intractable.
For this reason we are interested in developing efficient methods for
handling large solution sets, which may bring tractability to previously
intractable instances.

% Definições, Propriedados e teoremas para o MOKP: dominancia
