% Breve resumo sobre métodos de programação dynamica
% Explicação overview do método para o MOKP
% - Process overview of DP for MKP
% - Dom. relations
% - Application of dom. rel.

\missing{Paragrafo de introducao da secao, justificando toda a explicacao que segue..}

This dynamic programming algorithm is based on the classical Nemhauser-Ullmann algorithm proposed
in \cite{nemhauser1969discrete}, which consists of a sequential process of $n$ stages.
At each stage a new set of solution is constructed using a concept of domination
proposed by Weingartner and Ness~\cite{weingartner1967methods} for knapsack problems.
At the the $n$-th stage the Pareto optimal set obtained.

Some property of the MOKP will be explored
three filter as optimizations for the trying
to reduce the number of solutions
handled on mid stages of the algorithm.
We will introduce the dynamic programming (DP) algorithm for the MOKP proposed in \cite{bazgan2009}.

\subsection{Definitions}
Before we will present some definitions...

\subsection{The basic DP algorithm and partial dominance filter}

%\subsection{Preliminaries}
A solution $\sol{x}$ is called \emph{deficient} if it has available space
to fit one or more item, i.e.\,, $\weight{x} + min\{ w_i : i \notin \sol{x} \} \leq W$.
Considering two solution $\sol{x}, \sol{y} \subseteq I_n$ we say
$\sol{x}$ \emph{partially dominates} $\sol{y}$ if it dominates $\sol{y}$ and
does not weight more than $\sol{y}$.
For shortening we will say that $\sol{x}$ partially dominates $\sol{y}$ by saying $\dominatesp{x}{y}$.
Formally:
\begin{displaymath}
    \dominatesp{x}{y} = \left\{
      \begin{array}{l}
          \dominates{x}{y} \quad \text{and}\\
          w(\sol{x}) \leq w(\sol{y})
  \end{array} \right.
\end{displaymath}

It's pseudocode is presented in Algorithm~\ref{alg:nemull}.
\begin{algorithm}
  \caption{Basic dynamic programming algorithm for MOKP}
  \label{alg:nemull}
  \Kw{$\bsym{p}, \bsym{w}, W$}
\Begin{
  \SetAlgoLined
  $S_0 = \{(0, \ldots, 0)\}$\;
  \For{$k \gets 1, n$}{
    $S_k \gets S_{k-1} \cup \big\{(s^1 + p_k^1, \ldots, s^\np + p_k^\np, s^{\np+1} + w_k)\;$\
    $\phantom{S_k \gets S_{k-1} \cup \;} \big| \; s^{\np+1} + w_k \leq W, \: s \in S_{k-1} \big\}$\;
  }
  $P = \{ s \in S_n \:|\: \nexists (a \in S_n \;|\; a \dom s)\}$\;
  \textbf{return} $P$\;
}

\end{algorithm}
At the $k$-th stage the algorithm receives a set $S^{k-1}$ of solutions and
generates the set $S^k$ of solutions that correspond
to subsets containing exclusively the first $k$ items, i.e.\,,
$\forall\sol{x} \in S^k, \sol{x} \subseteq \{1, \ldots, k\}$.

This is done by expanding $S^{k-1}$ by adding a copy of each solution with the
inclusion of $k$ item (line 4).
We will refer as \emph{partial solutions} all the solutions handled by 
stages prior to $n$-th stage.

The clever part of the algorithm is that it uses the concept of partial dominance
to filter solutions that will not lead to efficient solutions (line 5).
Considering two partial solutions $\sol{x}, \sol{y} \in S^k$, if
$\sol{x}$ is partially dominated by $\sol{y}$ then we may discard $\sol{x}$ since all
solutions generated from $\sol{x}$ will be dominated by those generated from $\sol{y}$.

\subsection{Avoiding deficient solutions}

The first optimization that can be made on Algorithm~\ref{alg:nemull} is
avoiding the generation of deficient solution.
As noted above on line 4, at $k$-th stage all previous solution is copied to the
new solution set without adding $k$-th item.
However coping a solution that has a lot o space left may lead to deficient
solutions.

Considering the $k$-th stage, if a partial solution $\sol{x} \in S^{k-1}$ has enough
space to fit all remaining items, i.e.\,, $\weight{x} + \sum_{i=k}^n w_i \leq W$,
$\sol{x}$ may be discarded and only $\sol{x} \cup \{k\}$ keeped, once
keeping $\sol{x}$ will certainly lead to deficient solutions.

\subsection{Removing unpromissing solutions}

Another optimization that can be applied on later stages is
filtering unpromissing solutions by computing upper bounds for its objectives
functions and comparing it with the set of available lower bounds.
An upper(lower) bound of a solution is an upper(lower) limit each objective
value can achieve given its remaining capacity and the remaining items.
If the upper bound of a solution is dominated by an existing lower bound,
that solution can be discarded since it will not generate an efficient solution.

A lower bound of a solution can be easily computed...

Formally:
\begin{displaymath}
\end{displaymath}
\begin{align*}
    lb(\sol{x}) &= \big(lb_1(\sol{x}),\ldots, lb_\np(\sol{x}) \big) \\
    \text{where} \phantom{mmmmm} \\
    lb_i(\sol{x}) &= ...
\end{align*}

\subsection{Item order}

\begin{algorithm}
  \caption{Bazgan's DP algorithm for the MOKP}
  \label{alg:bazgan}
  \begin{algorithmic}[1]
    \Function{BazDP}{$\bsym{p}, \bsym{w}, W$}
    \State $S^0 = [\emptyset ]$
    \For{$k \gets 1, n$}
      \State $S^{k-1} = [\sol{s_1}, \ldots, \sol{s_p}]$
      \State $w_{left} \gets (w_{k} + \ldots + w_n)$
      \State $i \gets min\{i \;| \; \weight{s_i} \geq w_{left}\}$
        \Comment{Avoiding deficient}
      \While{ $\weight{s^k_i} < w_{left}$ }
        \State $i \gets i + 1$
      \EndWhile
      \State $1$
    \EndFor
  %\State $P = \{\sol{x} \:|\: \nexists \sol{a} \in S^n: \dom{a}{x} \}$
  \State \Return $P$
  \EndFunction
\end{algorithmic}

\end{algorithm}


% Basic simple Dynamic Programming
% Nemhauser-Ullmann algorithm

