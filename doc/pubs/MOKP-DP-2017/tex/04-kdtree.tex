% Histórico da kdtree: 
%  - primeira proposta. 
%  - Objetivo. Eficientcia
%  - Teste de colisão, pertinencia e amplitude
The \kdtree{} is a type of binary search tree for indexing multidimenstional
data with simple construction and low space usage.
Despite its simplicity it efficiently supports operations like nearest
neighbour search and range search~\cite{bentley1975} and is widely used on
spacial geometry algorithms~\cite{preparata2012computational, guttman1984r}, 
clustering algorithms~\cite{kanungo2002efficient, indyk1998approximate}
and ~\cite{owens2007survey}

% Utilizacao:
%  - objetos espaciais 2D/3D):
%  - renderizacao

% Vantagens sobre outras estruturas (quad-tree, etc):
%  - menor sensibilidade a distribuicao tendensiosa.
Its advantages.

% Discussao sobre eficiencia
%  - máximo de dimensoes ( |n| >> 2^dim )
Efficiency notes.

% Estrutura e Operacoes:
%  - estrutura de indexacão
%  - Insercao
%  - range-search
Its operations...

% Utilizacao no Algoritmo
% Possibilidade de utilizacao no processo de range seaarch, ao buscar uma
% solucao dominante.
Use on the algorithm.

% Explicacão da indexacão das solucoes e da funcao de range.
Indexing the solutions and range operations.

% Mencionar que esta proposta de indexacao viabiliza a aplicacão do algoritmo em problemas
% com maiores dimensões.
Tends to increase the feasibility on problems with higher dimensions.
