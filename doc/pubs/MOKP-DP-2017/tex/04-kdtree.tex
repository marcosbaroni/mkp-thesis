% Histórico da kdtree: 
%  - primeira proposta. 
%  - Objetivo. Eficientcia
%  - Teste de colisão, pertinencia e amplitude

% Utilizacao:
%  - objetos espaciais 2D/3D):
%  - renderizacao

% Vantagens sobre outras estruturas (quad-tree, etc):
%  - menor sensibilidade a distribuicao tendensiosa.

% Discussao sobre eficiencia
%  - máximo de dimensoes ( |n| >> 2^dim )

% Estrutura e Operacoes:
%  - estrutura de indexacão
%  - Insercao
%  - range-search

% Utilizacao no Algoritmo
% Possibilidade de utilizacao no processo de range seaarch, ao buscar uma
% solucao dominante.

% Explicacão da indexacão das solucoes e da funcao de range.

% Mencionar que esta proposta de indexacao viabiliza a aplicacão do algoritmo em problemas
% com maiores dimensões.

