
The multidimensional knapsack problem (MKP) is a strongly NP-hard combinatorial
optimization problem which can be viewed as a resource allocation problem and
defined as follows:

\begin{align*}
  \text{maximize} & \sum_{j=1}^n p_j x_j \\
  \text{subject to} & \sum_{j=1}^n w_{ij} x_j \leqslant c_i \quad i \in \{1, \ldots, m\}\\
   & x_j \in \{0, 1\}, \quad j \in \{1, \ldots, n\}.
\end{align*}

% Define the MKP
The problem can be interpreted as a set of $n$ items with profits $p_j$
and a set of $m$ resources with capacities $c_i$.
Each item $j$ consumes an amount $w_{ij}$ from each resource $i$, if selected.
The objective is to select a subset of items with maximum total profit,
not exceeding the defined resource capacities.
The decision variable $x_j$ indicates if $j$-th item is selected.

The multidimensional knapsack problem can be applied on budget planning 
scenarios and project selections~\cite{mcmillan1973resource},
cutting stock problems~\cite{Gilmore-Gomory-1966}, loading problems~\cite{Shih-1979},
allocation of processors and databases in distributed computer programs~\cite{Gavish-Pirckul-1982}.

The problem is a generalization of the well-known knapsack problem (KP) in which
$m = 1$.
However it is a NP-hard problem significantly harder to solve in practice than the KP.
Due its simple definition but challenging difficulty of solving, the MKP is often used to
to verify the efficiency of novel metaheuristics.

In this paper we address the application of a metaheuristic called
shuffled complex evolution (SCE) to the multidimensional knapsack problem.
The SCE is a metaheuristic, proposed by Duan in \cite{duan1992effective},
which combines the ideas of a controlled random search with the concepts
of competitive evolution and shuffling.
The SCE algorithm has been successfully used to solve several problems
like flow shop scheduling~\cite{zhao2014shuffled} and project management~\cite{elbeltagi2007modified}.

The reminder of the paper is organized as follows:
Section~\ref{sec:sce} presents the shuffled complex evolution algorithm
and proposes its application on the multidimensional knapsack problem.
Section~\ref{sec:exp} comprises several computational experiments.
In section~\ref{sec:conc} we make our concluding remarks about the experimental
results.


