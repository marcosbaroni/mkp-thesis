\documentclass{article}

\usepackage[portuguese]{babel}
\usepackage[utf8]{inputenc}

\usepackage{amsfonts}
\usepackage{amssymb,amsmath}
\usepackage{hyperref}
\usepackage{algorithm}
\usepackage{algpseudocode}


% - Arquivo de proposta de tese
% - Versão mais atualizada da descrição do problema
% - Descobertas que temos feito nos últimos tempos.
% - Descrição das atividades que já realizamos
% - Dúvidas sobre o problema
%   - Se é um problema que possue caminho factível para tese de doutorado

\begin{document}

%\begin{abstract}
%\end{abstract}

%\section{Introdução}

A motivação para o estudo do \emph{Problema de Investimentos em Ações para
Redução de Perdas Técnicas} surgiu através de um projeto de pesquisa realizado
junto a EDP-Escelsa.
No contexto deste projeto, a EDP deveria cumprir uma estipulação da ANEEL~\footnote{
Agência Nacional de Energia Elétrica} alcançando uma meta de perda anual
estipulada pela ANEEL.

Um dos objetivos do projeto era auxiliar a EDP na
decisão de um plano plurianual de combate à perda de energia elétrica através
seleção de \emph{ações de combate a perda de energia} dentre um portfólio de
ações sob a restrição de orçamentos anuais.

%~\footnote{é considerada energia perdida aquela que não é faturada.}

Nós estamos interpretando o problema como uma variação (generalização) do
\emph{Problema da Mochila}.
Nesta formulação as ações de combate a perda seriam os \emph{itens} a serem
dispostos nos anos \textit{mochilas} ()



% O que é o problema resumidamente.
% - Origem: problema de recuperação de perdas da escelsa
% - Nossa visão do problema: Generalização do problema de amochila
% - Motivação inicial para estudo do problema: Aplicação real do problema
%     da mochila.
%    - Tese: Aplicação de meta-heurística ao problema
% - Nossas descobertas:
%   - O problema não está tão dificil como se esperava

\end{document}

