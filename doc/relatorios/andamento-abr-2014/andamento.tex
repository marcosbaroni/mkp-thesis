\documentclass{article}

\usepackage[portuguese]{babel}
\usepackage[utf8]{inputenc}

\usepackage{amsfonts}
\usepackage{amssymb,amsmath}
\usepackage{hyperref}
\usepackage{algorithm}
\usepackage{algpseudocode}


% - Arquivo de proposta de tese
% - Versão mais atualizada da descrição do problema
% - Descobertas que temos feito nos últimos tempos.
% - Descrição das atividades que já realizamos
% - Dúvidas sobre o problema
%   - Se é um problema que possue caminho factível para tese de doutorado

\begin{document}

%\begin{abstract}
%\end{abstract}

%\section{Introdução}

A motivação para o estudo do \emph{problema de investimentos em ações para
redução de perdas técnicas} surgiu durante um projeto de pesquisa realizado
junto a EDP-Escelsa.
No contexto do projeto a EDP deveria cumprir uma estipulação da ANEEL~\footnote{
Agência Nacional de Energia Elétrica} de combate a perda de energia atingindo
metas anuais de redução perda de energia.

Na tentativa de atingir essas metas, a EDP montagem de um plano plurianual de
\emph{ações de combate a perda de energia}, retiradas de um portfolio de ações.

Um dos objetivos do projeto era auxiliar a EDP na decisão de um plano plurianual
de combate à perda de energia
elétrica através seleção destas ações e da disposição destas ações nos devidos
anos do planejamento.

Em anexo (\texttt{cocoon2014.pdf}) está o último trabalho que escrevemos sobre o problema.
Na primeira parte da Seção 1 está uma introdução e motivação do problema e na
primeira parte da Seção 2 está a definição formal do problema.

%Nós temos interpretando o problema como uma generalização do
%\emph{problema da mochila} em que as ações de combate a perda seriam os
%\emph{itens}, os anos seriam as \textit{mochilas} (Multi-mochilas).
%Estas ações consomem recursos de dois fundos diferentes: CAPEX (\textit{capital expenditure})
%e OPEX (\textit{operational expenditure}).
%cujos orçamentos seriam as \emph{capacidades} (Multi-dimensional).

Em resumo, temos interpretando o problema como uma generalização do
{\bf problema da mochila} com as seguintes características:
\begin{itemize}
  \item {\bf bounded:} as ações podem ser executadas um número limitado de vezes;
  \item {\bf multi-knapsack:} as ações devem selecionadas e alocadas em um número
    variado de anos, cada um possuindo orçamentos independentes;
  \item {\bf multidimensional:} cada ação consome recursos CAPEX
    (\textit{capital expenditure}), OPEX (\textit{operational expenditure})
	ou de ambos;
  \item {\bf partially-ordered:} pode existir casos em que alocação de uma
    ação $A$ em um determinado ano depende da alocação de uma ação $B$ no ano
	anterior;
\end{itemize}

Para ``complicar'' ainda a modelagem como um problema da mochila, uma ação
também recupera energia nos anos posteriores a sua execução.
Isto faz com que a ação ajude a alcançar a meta de outros anos mas sem gastar
de seus orçamentos.



% O que é o problema resumidamente.
% - Origem: problema de recuperação de perdas da escelsa
% - Nossa visão do problema: Generalização do problema de amochila
% - Motivação inicial para estudo do problema: Aplicação real do problema
%     da mochila.
%    - Tese: Aplicação de meta-heurística ao problema
% - Nossas descobertas:
%   - O problema não está tão dificil como se esperava

\end{document}

