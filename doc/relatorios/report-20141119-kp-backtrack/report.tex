\documentclass{article}

\usepackage{amsmath}
\usepackage{amssymb}
\usepackage[linesnumbered,lined,ruled]{algorithm2e}

\newtheorem{mydef}{Definition}
\let\emph\textbf

\title{Report...}
\author{Marcos Daniel Baroni}

%%%%%%%%%%%%%%%%%%%%%%%%%%%%%%%%%%%%%%%%%%%%%%%%%%%%%%%%%%%%%%%%%%%%%%%%%%%%%%%%
% Objetivo: Apresentar resumidamente os resultados dos testes que fiz sobre
%  tempo esperado do backtrack para o KP;
%  - Comentar sobre artigos:
%    - [1] "On the avrg diff between the solutions to LP and IP KPs"; (comentário, qualidade de solução)
%    - [2] "On finding the Exact Solution of 0-1 KP";
%    - [3] "Random KP in Expected PTime";
%  (- Lembrar que KP:
%     1. é NP-Completo para coeficientes ilimitados;
%     2. admite FPTAS em número de itens;)
%  - Apresentar o algoritmo backtrack para o KP;
%  - Apresentar algoritmo de Programação Dinâmica 
%%%%%%%%%%%%%%%%%%%%%%%%%%%%%%%%%%%%%%%%%%%%%%%%%%%%%%%%%%%%%%%%%%%%%%%%%%%%%%%%

%%%%%%%%%%%%%%%%%%%%%%%%%%%%%%%%%%%%%%%%%%%%%%%%%%%%%%%%%%%%%%%%%%%%%%%%%%%%%%%%
% Seção 1: O KP (FPTAS, NP-Completeness, polinomial expectance, )
%    Definição do problema/relaxação LP;
%    Algoritmo Greedy para a relaxação e diferença entre soluções [1];
% Seção 2: O algoritmo de backtrack ()
%    Apresentação 
% Seção 3: O algoritmo de Prog. Dinamica de Beier
% Seção 4: Resultados
%    *Curva1: Nós abertos até encontrar a melhor (polinomial);
%    *Curva2: Nós abertos até provar a melhor (conjectura: 1.8^(x-3) );
%    [2] Citar probabilidade de encontrar a melhor solução;
%%%%%%%%%%%%%%%%%%%%%%%%%%%%%%%%%%%%%%%%%%%%%%%%%%%%%%%%%%%%%%%%%%%%%%%%%%%%%%%%

\begin{document}

\maketitle



\begin{align*}
  \text{maximize} & \sum_{j=1}^n p_j x_j \\
  \text{subject to} & \sum_{j=1}^n w_j x_j \leq b \\
   & x_j \in \{0, 1\}, \quad j \in \{1, \ldots, n\}.
\end{align*}

\section{}

\section{The Backtracking Approach}
\cite{lueker1998average}
\cite{horowitz1978fundamentals}

\begin{displaymath}
  \frac{p_1}{w_1} \geqslant
  \frac{p_2}{w_2} \geqslant
  \ldots \geqslant
  \frac{p_n}{w_n}
\end{displaymath}

\newpage

\section{The Nemhauser-Ullmann Algorithm}

A brute force method to solve the knapsack problem is to enumerate all possible subsets over the $n$ items.
In order to reduce the search space, a domination concept is used which is usually attributed to Weingartner and Ness~\cite{weingartner1967methods}.
\begin{mydef}[Domination]
A subset $S \in [n]$ with weight $w(S) = \sum_{i \in S} w_i$  and profit $p(S) = \sum_{i \in S} p_i$
\emph{dominates} another subset $T \subseteq [n]$ if $w(S) \leqslant w(T)$ and $p(S) \geqslant p(T)$.
\end{mydef}

For simplicity assume that no two subsets have the same profit.
Then no subset dominated by another subset can be an optimal solution to the knapsack problem, regardless of the specified knapsack capacity.
Consequently, it suffices to consider those sets that are not dominated by any other set.

Using this concept Nemhauser and Ullmann~\cite{nemhauser1969discrete} introduce an elegant
algorithm computing a list of all dominating sets in an iterative manner:

\begin{algorithm}[H]
 \SetKwInOut{Input}{Input}
 \SetKwInOut{Output}{Output}
 \Input{a KP instance}
 \Output{list $S(n)$ of all dominating sets}
 $S(0) \leftarrow \emptyset $\;
 \For{$i\leftarrow 1$ \KwTo $n$}{
   $S'(i) \leftarrow S(i-1)\; \cup \big\{ s \cup \{i\} \,\big|\, s \in S(i-1)\big\}$\;
   $S(i) \leftarrow \big\{ dominates\big(s, S'(i)\big) \,\big|\, s \in S'(i) \big\}$\;
 }
 \caption{The Nemhauser-Ullmann Algorithm}
\end{algorithm}

Let $S(i)$ be the sequence of dominating subsets over the items $1, \ldots, i$.
%The sets in $S(i)$ are assumed to be listed in increasing order of their weights.
Given $S(i-1)$, the sequence $S(i)$ can be computed as follows:
First duplicate all subsets in $S(i-1)$ and then add item $i$ to each of the duplicated sets (line 3).
Now we compute $S(i)$ by removing the sets dominated by any other set (line 4).
The $dominates(s, S)$ procedure checks if subset $s$ dominates all others subsets in $S$.
The result is the ordered sequence $S(i)$ of dominating sets over the items $1, \ldots, i$.

Beier and V{\"o}cking \cite{beier2003random}...

\bibliographystyle{abbrv}
\bibliography{../../refs}

\end{document}

