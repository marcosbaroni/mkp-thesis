\documentclass{article}

\usepackage{amsmath}

\author{Marcos Daniel Baroni}

\begin{document}

% Algorithms to introduce
% - EXACT: Branch-and-cut (CPLEX)
% - DPROG: Bertsimas 2002
% - PROBL: Dyer 1989
% - HEURI: (Fleszar 2009?)
%
% Instances:
% - OR-Library
% - Uniform[0-1]
% - Transformation from other NP-Complete problems (???)
%
% Research questions (what i hope to find out):
% - Which instances are hard/easy for each algorithm?
%   - Why are they hard?
% - Which instances are hard for all algorithms?
%   - Why are they hard for all?
% - Instances from other NP-Complete problems are hard?
%   - What happens with their transition phases.

\begin{abstract}
This article contains a draft...as a backbone for an article over the
computational investigation of the hardness the Multidimensional Knapsack
Problem (MKP) as well as the on performance of algorithms for the solution of
instances.
\end{abstract}

\section{Introduction}

% Definição do problema e aplicações
The 0-1 Multidimensional Knapsack Problem (MKP) is a generalization of the Knapsack
Problem where an item expends more than a single resource type.
A MKP having $n$ itens and $m$ dimensions can be defined as follows.

\begin{align*}
  \text{maximize} & \sum_{j=1}^n p_j x_j \\
  \text{subject to} & \sum_{j=1}^n c_{ij} x_j, \quad i = 1, \ldots, m, \\
   & x_j \in {0, 1}, \quad j = 1, \ldots, n.
\end{align*}

The problem can be applied on budget planning scenarios, subset project
selections, cutting stock problems, task scheduling and allocation of processors
and databases in distributed computer programs.
The special case where $m = 1$ is called Knapsack Problem (KP).

In recent years, due its simple definition, wide application and challenging
difficulty, MKP turned out to be a quite addressed problem for experiments
with metaheuristics.

% prova da NP-complitude. Prova da não existência de um FPTAS.

% Literatura existente (used for metaheuristics testing, methods, 
%    benchmark data base)

\section{Algorithms}
% - EXACT: Branch-and-cut (CPLEX)
% - DPROG: Bertsimas 2002
% - PROBL: Dyer 1989
% - HEURI: (Fleszar 2009?)

\section{Instances}
% - OR-Library
% - Uniform[0-1]
% - Transformation from other NP-Complete problems (???)

\section{Research Questions and Expectations}
% - Which instances are hard/easy for each algorithm?
%   - Why are they hard?
% - Which instances are hard for all algorithms?
%   - Why are they hard for all?
% - Instances from other NP-Complete problems are hard?
%   - What happens with their transition phases.
%
% (número limitado de restrições o problema é mais fácil)
% (coeficientes limitados o problema é mais fácil?)

\end{document}

