\documentclass{article}

\usepackage{amsmath}

\author{Marcos Daniel Baroni}

\begin{document}

% Algorithms to introduce
% - EXACT: Boussier 2010 / Branch-and-cut (CPLEX)
% - DPROG: Bertsimas 2002
% - PROBL: Dyer 1989
% - HEURI: Fleszar 2009
%
% Instances:
% - OR-Library
% - Uniform[0-1]
% - Transformation from other NP-Complete problems (???)
%
% Research questions (what i hope to find out):
% - Which instances are hard/easy for each algorithm?
%   - Why are they hard?
% - Which instances are hard for all algorithms?
%   - Why are they hard for all?
% - Instances from other NP-Complete problems are hard?
%   - What happens with their transition phases.

\begin{abstract}
This article contains a backbone for an article over the computational
investigation of the hardness the Multidimensional Knapsack
Problem (MKP) as well as the on performance of algorithms for the solution of
instances.
\end{abstract}

\section{Introduction}
\label{intro}

% Organização das seções:
%  - Seção 2 : Últimos algoritmos propostos
%  - Seção 3 :
%    - comentário sobre instâncias
%    - criação de instâncias difíceis
%    - as instâncias existentes e resultados até então.
%  - Seção 4 : Dúvidas e de pesquisa

% Definição do problema e aplicações
The 0-1 Multidimensional Knapsack Problem (MKP) is a generalization of the Knapsack
Problem where an item expends more than a single resource type.
A MKP having $n$ itens and $m$ dimensions can be defined as follows.

\begin{align*}
  \text{maximize} & \sum_{j=1}^n p_j x_j \\
  \text{subject to} & \sum_{j=1}^n c_{ij} x_j \leq b_i \quad i \in \{1, \ldots, m\}\\
   & x_j \in \{0, 1\}, \quad j \in \{1, \ldots, n\}.
\end{align*}

The problem can be applied on budget planning scenarios, subset project
selections, cutting stock problems, task scheduling and allocation of processors
and databases in distributed computer programs.
The problem is a generalization of the well-known Knapsack Problem (KP) in which
$ m = 1$.

Several contributions have been made addressing exact, heuristic, approximation
and probabilistic approaches for the MKP.
The purpose of the work is to evaluate the performance the main approaches
presented on literature over different instances of the problem and
iddentify those instances which are hard to solve by all the approaches.

Section \ref{algs} briefly addresses each one of the main approaches for
solving MKP, Section \ref{insts} discusses about the instances of MKP used on
literature and Section \ref{questions} presents some research questions 
we hope to answer with the article.

\section{The main approaches for solving MKP}
\label{algs}
Due its simple definition and challenging difficulty, the MKP
turned a quite addressed problem for experiments with metaheuristics in recent
years, although few of them have exhibit competitive performance compared to
comercial MIP solvers (at least for the instances dealed by literature).
Among the heuristics reporting competitive performance compared to MIP solvers
the newest ones are the one proposed by Fleszar and Hindi \cite{fleszar2009fast}
and one proposed by Hanafi and Wilbaut \cite{hanafi2011improved}.

According to literature MKP does not allow a FPTAS (unless $P=NP$) but a PTAS is
allowed and one was proposed by Frieze and Clarke \cite{frieze1984approximation}.
Dyer and Frieze \cite{dyer1989probabilistic} proposed a probabilistic which,
given a $\epsilon > 0$ answers computes the optimal solution for the problem
with probability at least $1 - \epsilon$ with polynomial time.

At the present moment the most powerful exact method for solving the MKP seems
to be the multi-level search strategy proposed by Boussier \cite{boussier2010multi}
which, until now, is the only approach which have found the optimal solution for
some popular instances with 500 itens and 10 contraints.

\section{Instances}
\label{insts}

The most popular instances for the MKP are those from the OR-LIBRARY
and they are used as a reference for state-of-art methods.
The generation of thoses instances are generaly guided by two parameters: the
tightness $\delta$ of the knapsacks and the profit-weight correlation of the
items.

The tightness of a knapsack rules the relative capacity of the knapsack.
Given a tightness $\delta \in [0,1]$ the capacity $b_i$ is setted to
$b_i = \delta \sum_{i=1}^n x_i$.  % STOPPED HERE, talking about the impact of
% tightness on the hardness of the problem...%

% - OR-Library \cite{Chu-Beasley-1998} (\cite{freville1994efficient})
% - Uniform[0-1] (Dyer 1989) \cite{dyer1989probabilistic}
% - Transformation from other NP-Complete problems \cite{magazine1984note}, ...
% - large dimensions (K.n = m?)
% - Sparse weights

\section{Research Questions and Expectations}
\label{questions}
% - Which instances are hard/easy for each algorithm?
%   - Why are they hard?
% - Which instances are hard for all algorithms?
%   - Why are they hard for all?
% - Instances from other NP-Complete problems are hard?
%   - What happens with their transition phases.
%
% (número limitado de restrições o problema é mais fácil)
% (coeficientes limitados o problema é mais fácil?)

\bibliographystyle{abbrv}
\bibliography{../../refs}

\end{document}

