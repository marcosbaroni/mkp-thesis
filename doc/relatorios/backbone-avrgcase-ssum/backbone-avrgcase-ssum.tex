\documentclass{article}

\usepackage{amsmath}

\title{Estimating the efficiency of backtrack procedures for nonnegative binary programming problems}
\author{Marcos Daniel Baroni}

% 1: Introduction (and Motivation)
%    - Empirical observation of fast extact solving for major instances of KP;
%    - Motivate research on, respectively: Partition, SSum, KP and MKP;
%    - Existent probabilistic analysis for Partition/SSum, KP and MKP;
%    - Similar work on exact methods for others combinatorial problems;
%      - Cite "some linear behavior on backtrack algorithms has been observed
%        for KP, althougth it seems to be hard to prove."
%      - \ref{datri1982probabilistic}: Probabilistic analysis of the subset-sum problem
%      - \ref{martello1984worst}: Worst-case analysis of greedy algorithms for the subset-sum problem
%      - \ref{lai1993worst}: Worst-case analysis of greedy algorithms for the unbounded knapsack, subset-sum and partition problems
%
% 2: The backtrack algorithm (Branch-and-Bound)
%    - Justify the use of backtrack, and not Dyn. Prog.:
%      - Scaling of rational coeficients yields high order integer coefficients;
%      - Transformation of ohter NP-Complete probs to produce high orders itens;
%    - Main (not to say, only) exact approach for exact solution for
%      combinatorial problems;
%    - Possible approaches with Relaxation (LP) and Cuts.
%
% 3: The Analysis and Research Questions
%    - Expected behavior of backtrack algorithm on Partition problems:
%    - Expected number of solutions for the problem;
%    - Probability of i-th item integrate solution
%      - Need to fix a variable, because of simetry?

\begin{document}

\maketitle

% \begin{abstract}
% This article contains brief research proposal on the expected processing
% time of backtrack procedures for some nonnegative integer
% programming problems known to be NP-complete.
% \end{abstract}

%\section{Introduction}

%    - Empirical observation of fast extact solving for major instances of KP;
%    - Motivate research on, respectively: Partition, SSum, KP and MKP;
%    - Existent probabilistic analysis for Partition/SSum, KP and MKP;
%    - Similar work on exact methods for others combinatorial problems;
%      - Cite "some linear behavior on backtrack algorithms has been observed
%        for KP, althougth it seems to be hard to prove."
%      - \cite{datri1982probabilistic}: Probabilistic analysis of the subset-sum problem
%      - \cite{martello1984worst}: Worst-case analysis of greedy algorithms for the subset-sum problem
%      - \cite{lai1993worst}: Worst-case analysis of greedy algorithms for the unbounded knapsack, subset-sum and partition problems

During the last decades much effort has been devoted to the search of
efficient algorithms to solve some well-known NP-hard problems
Those efforts have led to substantial improvement on heuristics and
approximate approaches, but no efficient algorithm has been found yet and
exponential time algorithms is still needed for exact solution.

Despite the exponential time behavior expected for worst cases, empirical
observations has reported efficient time on average for backtrack algorithms for
several problems \cite{cheeseman1991really, wilf1984backtrack,
posa1976hamiltonian, johnson1984np11, purdom1983search}.
An intersting problem presenting this intriguing behavior (and moreover
generalizes other several NP-complete problems) is the multidimensional knapsack
problem (MKP).

The MKP may be defined as follows:
\begin{align*}
  \text{maximize} & \sum_{j=1}^n p_j x_j \\
  \text{subject to} & \sum_{j=1}^n w_{ij} x_j \leq b_i \quad i \in \{1, \ldots, m\}\\
   & x_j \in \{0, 1\}, \quad j \in \{1, \ldots, n\}.
\end{align*}

The MKP can be considered as a generalization of other three well-known
NP-Comple\-te problems~\cite{garey1979}:
\begin{itemize}
  \item KNAPSACK PROBLEM (KP): special case of MKP where
$m = 1$;
  \item SUBSET SUM PROBLEM (SSP): special case of KP where
$p_j = w_j$;
  \item PARTITION PROBLEM (PP): special case of SSP where $b = \lfloor \frac{1}{2}\sum_{i=1}^n w_i \rfloor$.
\end{itemize}
All four problems above are NP-complete problems and they will be considered in
this work as Standard Nonnegative Binary Programing Problems (SNBPPs).
Efficient average time on backtrack algorithms has been observed in practice for
those problems but few related theoretical foundation is known.

Dynamic programming is an exact approach which, at first, sounds most attractive
for SNBPPs since it promises polynomial time for instances with
coefficients bounded by a constant. However:
\begin{enumerate}
  \item[(a)] all theoretical results proving NP-Completeness for SNBPPs pre\-sents
  instances with exponential coefficients;
  \item[(b)] eventually small instances having large coefficients
  (scaled instances from real problems with rational coefficients, for
  example) may appear for which a backtrack algorithm is more suitable;
  \item[(c)] backtrack algorithms are used for others combinatorial
  problems.
\end{enumerate}
For those reasons the analysis of backtrack procedures seems to be of great
relevance.

The main goal of the work is to investigate the expected efficiency of backtrack
procedures for SNBPPs.
This will be done by considering at first simpler versions of backtrack algorithms.
Afterwards more efficient (hence more complex) versions can be analyzed on a
step-by-step manner.

The paper by Donald Knuth \cite{knuth1975estimating} seems to give helpful
directions on how to perform the analysis.

\bibliographystyle{abbrv}
\bibliography{../../refs}

\end{document}

