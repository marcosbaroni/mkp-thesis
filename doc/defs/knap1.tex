\section{Versão 1}

Versão original do problema, inspirado no problema da Escelsa de combate a perdas.

\topico{Conjuntos}
\begin{itemize}
  \item \variavel{N} Nº de ações {\scriptsize $(1 \leq i \leq N) $}
  \item \variavel{Y} Nº de anos  {\scriptsize $(1 \leq j \leq Y) $}
  \item \variavel{R} Nº de recursos {\scriptsize $(1 \leq l \leq R)$}
\end{itemize}

\topico{Parâmetros}
\subtopico{Globais}
\begin{itemize}
  \item \variavel{r} Taxa interna de retorno periódico (juros);
\end{itemize}

\subtopico{Anuais}
\begin{itemize}
  \item \variavel{g^j} Meta anual de redução de perda;
    \decorator{1 \leq j \leq Y}
  \item \variavel{p_l^j} Orçamento anual;
    \decorator{1 \leq l \leq R, \quad 1 \leq j \leq Y}
\end{itemize}

\subtopico{das Ações}
\begin{itemize}
  \item \variavel{m_i} Mercado global;
    \decorator{1 \leq i \leq N}
  \item \variavel{u_i^j} Mercado Anual;
    \decorator{1 \leq i \leq N, \quad 1 \leq j \leq Y}
  \item \variavel{c_{il}} Custo da ação;
    \decorator{1 \leq i \leq N, \quad 1 \leq l \leq R}
  \item \variavel{v_i} Valor da energia;
    \decorator{1 \leq i \leq N}
  \item \variavel{e_i^j} Recuperação realizada pela ação $i$ no $j$-ésimo ano
    após sua execução;
    \decorator{1 \leq i \leq N, \quad 0 \leq j \leq Y-1}
  \item \variavel{D_{it}} Quantidade total de vezes que a ação $t$ precisa ter
    sido feita para que tenha-se um total de $1$ ação $i$.
    \decorator{1 \leq i \leq N, \quad 1 \leq t \leq N}
\end{itemize}

\topico{Variáveis}
\begin{itemize}
  \item \variavel{x_i^j} Número de vezes que a ação $i$ é executada no período $k$;
    \decorator{1 \leq i \leq N, \quad 1 \leq j \leq Y}
\end{itemize}

\topico{Restrições}
\begin{restricoes}
    \restricao
	  {Meta de recuperação anual\footnote{Dúvida: a recuperação ficar muito abaixo da meta não é um problema?}}
	  { \sum_{i = 1}^N \sum_{k \in P_j}}
	  { Rec_i^k(\overline{x})}
	  { \leq }
	  { }
	  { g^j }
	  { \begin{tabular}{l}
	  	  $j = 1, \ldots, Y$
		\end{tabular} }
    \restricao
      {Orçamento Anual}
      { \sum_{i = 1}^N \sum_{k \in P_j} }
      { x_i^k . c_{il} }
	  { \leq }
	  { }
	  { p_l^j }
	  { \begin{tabular}{l}
	      $ j = 1, \ldots, Y $ \\
		  $ l = 1, \ldots, R $ 
		\end{tabular} }
	\restricao
	  {Mercado Global}
      { \sum_{k = 1}^P }
      { x_i^k }
	  { \leq }
	  { }
	  { m_i }
	  { \begin{tabular}{l}
	    $ i = 1, \ldots, N $
		\end{tabular} }
	\restricao
	  {Marcado Anual}
      { \sum_{k \in P_j} }
      { x_i^k }
	  { \leq }
	  { }
	  { u_i^j }
	  { \begin{tabular}{l}
	      $ i = 1, \ldots, N $ \\
		  $ j = 1, \ldots, Y $
		\end{tabular} }
	\restricao
	  {Dependência entre as Ações}
      { \sum_{k' = 1}^k }
      { D_{it} . x_i^{k'} }
	  { \leq }
	  { \sum_{k' = 1 }^{k-1} }
	  { x_t^{k'} }
	  { \begin{tabular}{l}
	      $ i, t=1, \ldots, N$ \\
		  $ k=2, \ldots, Y$
		\end{tabular} }
\end{restricoes}

\topico{Função Objetivo}
\begin{align}
  \nonumber
	  Max \quad Z(\overline{x}) = &
	    \sum_{i=1}^N
	    \sum_{j=1}^{Y}
	    \frac
		  {\big( Rec_i^j(\overline{x}).v_i - Cost_i^j(\overline{x}) \big)}
	      {(1+r)^j}
		+ \\ & + \nonumber
	    \underbrace{
	    \sum_{i=1}^N
	    \sum_{j=1}^{Y}
	    \frac
		  {{Rec'}_i^j(\overline{x}).v_i}
	      {(1+r)^{(j+P.Y)}}
		}_{\text{Lucro pós-planejamento}}
  \label{eq:budget}
\end{align}

