\documentclass{article}

\usepackage[portuguese]{babel}
\usepackage[utf8]{inputenc}

\usepackage[cm]{fullpage} % to use more space from page

\usepackage{longtable} % to use multi-page tables
  % setting longtables alignment
  \setlength\LTleft\parindent
  \setlength\LTright\fill

\usepackage{changepage} % to temporarily adjust page layout

\usepackage[fleqn]{amsmath}
\usepackage{amsthm}
\usepackage{amssymb}
\usepackage{bm} % for bold math text (\bm)

\usepackage{multicol}
\setlength{\columnseprule}{0.1pt}

\usepackage{graphicx}

\usepackage{fullpage}

\allowdisplaybreaks

%\renewcommand{\rmdefault}{ppl}

\newcommand{\myrule}{
  \rule{\textwidth}{0.1pt}
  \vspace{6pt}
}

\newcommand{\topico}[1]{
  \vspace{20pt}
  {\Large \bf #1 }
}

\newcommand{\subtopico}[1]{
  {\large \bf \underline{#1}}
  \\
}

\newcommand{\variavel}[1]{
  { \Large $ \bm{ #1 } \, $}
}

\newcommand{\decorator}[1]{
  {\\ \scriptsize \hspace*{12pt} $ \bm{ #1 } $ }
}

%%%    LISTAGEM DAS RESTRIÇÔES   %%%

% Ambiente para listagem de restrições

\newenvironment{restricoes}
  { \begin{longtable}{lrcclll}}
  {\end{longtable} }

\newcommand{\restricao}[7]{
    \multicolumn{2}{l}{ $\bullet$ #1} & & \\ \nopagebreak
    & ${ \displaystyle \bm{#2} }$
    & ${ \bm{#3} }$
    & ${ \bm{#4} }$
    & ${ \displaystyle \bm{#5} }$
    & ${ \bm{#6} }$
	& \begin{tabular}{l}
	  #7
    \end{tabular}
	\\ \hspace{30pt}
}


% Ambiente para listagem de equações
\newenvironment{equacoes}
	{ \begin{longtable}{lrcll} }
	{ \end{longtable} }

% Definição de uma equação
\newcommand{\equacao}[4]{
    \multicolumn{5}{l}{\text{ \parbox{250pt}{$\bullet$ #1} }} \\   % Descrição
    \phantom{aaaaaa} & $ \displaystyle \bm{#2} $                              % Definição
    & $ = $                 
	& $\displaystyle \bm{#3} $                           % expressão
	&
	\begin{tabular}{l}
	  #4                                 % Repetições
	\end{tabular}
	\vspace{12pt}
	\\
}

\title{The Partially Ordered Multidimensional Multi-Knapsack Problem \\ {Definição} }
\author{Marcos Daniel V. Baroni}

\begin{document}

\maketitle

\topico{Conjuntos}

\begin{itemize}
  \item \variavel{N} Nº de Ações {\scriptsize $(1 < i < N) $}
  \item \variavel{Y} Nº de Anos  {\scriptsize $(1 < j < Y) $}
  \item \variavel{P} Nº de Períodos por ano {\scriptsize $(1 < k < P)$}
    \begin{itemize}
	  \item[$\bullet$]{\small $\bm{P_{j} = \{P.(j-1)+1, \ldots, P.j\}}$ \\ Períodos referentes ao ano $j$;} 
	\end{itemize}
  \item \variavel{R} Nº de Recursos {\scriptsize $(1 < l < R)$}
\end{itemize}

\topico{Parâmetros}

\begin{multicols}{2}

\subtopico{Globais}
\begin{itemize}
  \item \variavel{r} Taxa interna de retorno periódico (juros);
\end{itemize}

\subtopico{Anuais}
\begin{itemize}
  \item \variavel{g^j} Meta anual de redução de perda;
    \decorator{1 \leq j \leq Y}
  \item \variavel{o_l} Orçamento global;
    \decorator{1 \leq l \leq R}
  \item \variavel{p_l^j} Orçamento anual;
    \decorator{1 \leq l \leq R, \quad 1 \leq j \leq Y}
  \item \variavel{s_l^{k}} Orçamento periódico;
    \decorator{1 \leq l \leq R, \quad \quad 1 \leq k \leq P.Y}
\end{itemize}

\vfill \columnbreak

\subtopico{das Ações}
\begin{itemize}
  \item \variavel{m_i} Mercado Global;
    \decorator{1 \leq i \leq N}
  \item \variavel{u_i^j} Mercado anual;
    \decorator{1 \leq i \leq N, \quad 1 \leq j \leq Y}
  \item \variavel{z_i^k} Mercado periódico;
    \decorator{1 \leq i \leq N, \quad 1 \leq k \leq P.Y}
  \item \variavel{c_{il}} Custo da ação;
    \decorator{1 \leq i \leq N, \quad 1 \leq l \leq R}
  \item \variavel{v_i} Valor da energia;
    \decorator{1 \leq i \leq N}
  \item \variavel{e_i^k} Recuperação realizada pela ação $i$ no $k$-ésimo período
    após sua execução;
    \decorator{1 \leq i \leq N, \quad 0 \leq k \leq P.Y-1}
  \item \variavel{D_{it}} Quantidade de vezes que a ação $t$ precisa ser
  	feita para que seja possível a execução de 1 ação $i$.
    \decorator{1 \leq i \leq N, \quad 1 \leq t \leq N}
\end{itemize}

\end{multicols}

\pagebreak

\topico{Variáveis}
\begin{itemize}
  \item \variavel{x_i^k} Número de vezes que a ação $i$ é executada no período $k$;
    \decorator{1 \leq i \leq N, \quad 1 \leq k \leq P.Y}
\end{itemize}

\topico{Equações}

\begin{equacoes}
    \equacao
	  {Recuperação de energia para o período $k$ causada pelas as ações $i$ de todos os períodos.}
	  {Rec_{i}^{k}(\overline{x})}
	  {\sum_{\substack{k' = k-P.Y+1 \\ k' \geq 1}}^{k} x_i^{k'} . e_i^{(k-k'+1)}}
	  { $i \in \{1, \ldots, N\}$ \\ $k \in \{1, \ldots, P\}$ }
	\equacao
	  {Lucro originado pela energia recuperada no período $k$.}
	  {Prof^k(\overline{x})}
	  {\sum_{i=1}^N Rec_{i}^k(\overline{x}) . v_i}
	  { $k \in \{1, \ldots, P\}$}
	\equacao
	  {Custo total de todas as ações executadas no período $k$.}
	  {Cost^k(\overline{x})}
	  {\sum_{i=1}^N \sum_{l=1}^R x_i^k . c_{il}}
	  { $k \in \{1, \ldots, P\}$}
\end{equacoes}

\topico{Restrições}

\begin{restricoes}
    \restricao
	  {Meta de Recuperação Anual\footnote{Dúvida: a recuperação ficar muito abaixo da meta não é um problema?}}
	  { \sum_{i = 1}^N \sum_{k \in P_j}}
	  { Rec_i^k(\overline{x})}
	  { \leq }
	  { }
	  { g^j }
	  { $ j = 1, \ldots, Y $ }
	\\
    \restricao
	  {Orçamento Global}
	  { \sum_{i = 1}^N \sum_{k = 1}^P }
	  { x_i^k . c_{il}}
	  { \leq }
	  { }
	  { o_l }
	  { $ l = 1, \ldots, R $ }
	\\
    \restricao
      {Orçamento Anual}
      { \sum_{i = 1}^N \sum_{k \in P_j} }
      { x_i^k . c_{il} }
	  { \leq }
	  { }
	  { p_l^j }
	  { $ j = 1, \ldots, Y $ \\ $ l = 1, \ldots, R$ }
    \\
	\restricao
	  {Orçamento periódico}
      { \sum_{i = 1}^N }
      { x_i^k . c_{il} }
	  { \leq }
	  { }
	  { s_l^{k} }
	  { $ k = 1, \ldots, P $ \\ $ l = 1, \ldots, R $ }
	\\
	\restricao
	  {Market Global}
      { \sum_{k = 1}^P }
      { x_i^k }
	  { \leq }
	  { }
	  { m_i }
	  { $ i = 1, \ldots, N $ }
	\\
	\restricao
	  {Market Anual}
      { \sum_{k \in P_j} }
      { x_i^k }
	  { \leq }
	  { }
	  { u_i^j }
	  { $ i = 1, \ldots, N $ \\ $ j = 1, \ldots, Y $ }
	\\
	\restricao
	  {Market periódico}
      { }
      { x_i^k }
	  { \leq }
	  { }
	  { z_i^k }
	  { $ i = 1, \ldots, N $ \\ $ k = 1, \ldots, P$ }
	\\
	\restricao
	  {Dependência entre as Ações}
      { \sum_{k' = 1}^k }
      { D_{it} . x_i^{k'} }
	  { \leq }
	  { \sum_{k' = 1 }^{k-1} }
	  { x_t^{k'} }
	  { $i,t=1,\ldots,N$ \\ $k=2,\ldots,P$ }
\end{restricoes}

\topico{Função Objetivo}
\begin{equation}
  \nonumber
    \bm{
	  Max \big(Z(\overline{x})\big) =
	    \sum_{k=1}^P
	    \frac
		  {\big( Prof^k(\overline{x}) - Cost^k(\overline{x}) \big)}
	      {(1+r)^k}
	}
  \label{eq:budget}
\end{equation}

\vfill

\end{document}

