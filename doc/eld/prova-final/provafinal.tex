\documentclass[12pt]{article}
\usepackage[utf8]{inputenc}
\usepackage[cm]{fullpage}
\usepackage{amssymb}
\usepackage{multicol}
\usepackage{graphicx}

\newcommand{\exerc}[3]{ \vspace*{25pt} {$\mathbf{#1)}$} #2 \hfill {\it #3} }
\newcommand{\exitem}[2]{ \texttt{\bf #1)} #2 \\ }
\newcommand*\xor{\mathbin{\oplus}}
\renewcommand{\neg}[1]{ 
  \mkern 1.5mu\overline{\mkern-1.5mu#1\mkern-1.5mu}\mkern 1.5mu
}

\newenvironment{exitems}[1]{
\\
\hspace*{30pt}
\begin{minipage}{0.8\textwidth}
\begin{multicols}{#1} 
}{
\end{multicols}
\end{minipage}
}

\newenvironment{exitemss}[1]{
\\
\hspace*{30pt}
\begin{minipage}{0.8\textwidth}
#1
}{
\end{minipage}
}

\begin{document}

\pagenumbering{gobble}

\begin{center}
{\Large \bf Elementos de Lógica Digital - 2015/2}
\end{center}
{\large \bf Prova Final}
{\bf Professor:} Marcos Daniel Baroni
{\bf Data:} 10/12/2015

% 1. Conversão sistemas numéricos (+complemento de 2)
% 2. Algebra de boole (Simplificação)
% 3. Contador assíncrono
% 4. Projeto de contador síncrono
% 5. Circuito interno de RAM (Expansão?)
% 6. Atributos de uma memória RAM
% 7. Curva do flip-flop JK (??)
% 8. Tabela verdade de somador

\end{document}

