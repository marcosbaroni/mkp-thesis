\documentclass[12pt]{article}
\usepackage[utf8]{inputenc}
\usepackage[cm]{fullpage}
\usepackage{amssymb}
\usepackage{multicol}
\usepackage{graphicx}

\newcommand{\exerc}[3]{ \vspace*{25pt} {$\mathbf{#1)}$} #2 \hfill {\it #3} }
\newcommand{\exitem}[2]{ \texttt{\bf #1)} #2 \\ }
\newcommand*\xor{\mathbin{\oplus}}
\renewcommand{\neg}[1]{ 
  \mkern 1.5mu\overline{\mkern-1.5mu#1\mkern-1.5mu}\mkern 1.5mu
}

\newenvironment{exitems}[1]{
\\
\hspace*{30pt}
\begin{minipage}{0.8\textwidth}
\begin{multicols}{#1} 
}{
\end{multicols}
\end{minipage}
}

\newenvironment{exitemss}[1]{
\\
\hspace*{30pt}
\begin{minipage}{0.8\textwidth}
#1
}{
\end{minipage}
}

\begin{document}

\pagenumbering{gobble}

\begin{center}
{\Large \bf Elementos de Lógica Digital - 2015/2}
\end{center}
{\large \bf Prova Final}
{\bf Professor:} Marcos Daniel Baroni
{\bf Data:} 10/12/2015

\exerc{1}{Simplifique as expressão abaixo utilizando álgebra de Boole.}{(2.5 pontos)}
\\ \hspace*{3em} \exitem{a}{ $S = \neg{A}B + C\neg{D}$ }
   \hspace*{3em} \exitem{b}{ $S = \neg{A}B + C\neg{D}$ }

\exerc{2}{Implemente a tabela verdade de um somador completo.}{(2.5 pontos)}
\begin{multicols}{2}
  \begin{tabular}{|c|c|c||c|c|}
    \hline
    $A$ & $B$ & $T_e$ & $S$ & $T_s$ \\ \hline
    $0$ & $0$ & $0$ & & \\ \hline
    $0$ & $0$ & $1$ & & \\ \hline
    $0$ & $1$ & $0$ & & \\ \hline
    $0$ & $1$ & $1$ & & \\ \hline
    $1$ & $0$ & $0$ & & \\ \hline
    $1$ & $0$ & $1$ & & \\ \hline
    $1$ & $1$ & $0$ & & \\ \hline
    $1$ & $1$ & $1$ & & \\ \hline
  \end{tabular}
\end{multicols}

\exerc{3}{Projete um contador síncrono que execute a sequência abaixo.}{(2.5 pontos)}

\exerc{4}{Utilizando blocos RAM 64x4, esquematize uma RAM 64x8.}{(2.5 pontos)}
\\ \exitem{a}{.}
   \exitem{b}{
\begin{tabular}{|l|c|c|}
 \cline{2-2}
 \multicolumn{1}{c|}{} & {\bf RAM 64x8} \\ \hline
 Capacidade total (em bits) & \phantom{aaaaaaaaaaaaaa} \\ \hline
 Largura de palavra de dados & \\ \hline
 Largura da barra de endereços & \\ \hline
 Palavra de endereço inicial & \\ \hline
 Palavra de endereço final & \\ \hline
\end{tabular}
}

% 1. Álgebra de boole (Simplificação)
%  1.1.
%  1.2.
% 2. Tabela verdade de somador
% 3. Projeto de contador síncrono (utilizando flip-flop JK)
% 4. RAM:
%  4.1. Expansão de uma memória RAM
%  4.2. Atributos (capacidade total, end. inicial e final)

\end{document}

