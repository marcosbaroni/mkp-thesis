\documentclass{article}
\usepackage[utf8]{inputenc}
\usepackage[cm]{fullpage}
\usepackage{amssymb}
\usepackage{multicol}
\usepackage{graphicx}
\usepackage{multicol}

\renewcommand{\emph}[1]{\textbf{#1}}

%\newcommand{\mycell}[3]{ #1 & #2 & \parbox{12cm}{#3} \\ \hline }
\newcommand{\mycell}[3]{

  #1 & #2 & \parbox{6cm}{
    \vspace{.1\baselineskip}
    #3
    \vspace{.1\baselineskip}
  } \\ \hline
}
\newcommand{\titulo}[1]{{\bf #1}}

\newcommand{\itspc}{
  \vspace{-8pt}
  \setlength\itemsep{-2pt}
}

\begin{document}

\pagenumbering{gobble}

\begin{center}
{\Large \bf Segundo Trabalho de Elementos de Lógica Digital - 2015/2}
\\
10/09/2015
\vspace{5mm}
\end{center}

\noindent
\titulo{Professor:} Marcos Daniel Baroni $<$\texttt{marcos.baroni@aluno.ufes.br}$>$

\noindent
\titulo{Data de entrega:} 12 de novembro de 2015

\noindent
\titulo{Regras:}
\begin{enumerate}
  \itspc
  \item O trabalho será feito em dupla;
  \item \underline{Não será tolerado plágio.} Trabalhos copiados serão penalizados com nota zero.
\end{enumerate}

\noindent
\titulo{Ferramenta para simulação:} Logisim (\texttt{http://www.cburch.com/logisim/})

\noindent
\titulo{Material a ser entregue:}
\begin{enumerate}
  \itspc
  \item{Arquivo de simulação}:
  \begin{itemize}
    \itspc
    \item{Enviar por email para \texttt{marcos.baroni@aluno.ufes.br} com arquivo em anexo;}
	\item{O título do email deve estar no formato \texttt{"ELD:TRAB2:nome1:nome2"} ({\it Ex.: \texttt{"ELD:TRAB2:alanturing:donaldknuth"}});}
	\item{Apenas {\bf um arquivo} (\texttt{.circ}) será entregue, contendo os circuitos simulados. Os devidos circuitos devem estar separados em blocos, conforme enunciado.
		Trabalhos fora de padrão serão penalizados.}
  \end{itemize}
  \item{Relatório:}
  \begin{itemize}
    \itspc
	\item{O relatório poderá em papel ou digital (formato PDF) enviado por email juntamente com o arquivo das simulações. Relatórios em outros formatos (.doc, .xls, etc) não serão aceitos.}
	\item{Podem estar manuscritos, desde que estejam claras e organizadas. \underline{A clareza e a organização serão avaliadas};}
    \item{As resoluções devem conter explicações dos passos realizados.}
  \end{itemize}
\end{enumerate}

\noindent
\titulo{\large Simulação 1: Conversão de dado paralelo-série} \\
Projetas os seguintes circuitos:
\begin{enumerate}
	\item{Conversor paralelo-série de 4 bits:}
	\begin{itemize}
		\item{Deve possuir uma entrada "Enable" para escrita do dado em paralelo;}
	\end{itemize}
	\item{Conversor série-paralelo:}
	\item{Simulação1:}
	\begin{itemize}
		\item{}
	\end{itemize}
\end{enumerate}

\vspace{3mm}
\noindent
\titulo{\large Simulação 2: Contador de sequência qualquer} \\
Projetar um contador síncrono que conte a sequência:
\\

\end{document}

