\documentclass{article}
\usepackage[utf8]{inputenc}
\usepackage[cm]{fullpage}
\usepackage{amssymb}
\usepackage{multicol}

\renewcommand{\emph}[1]{\textbf{#1}}

%\newcommand{\mycell}[3]{ #1 & #2 & \parbox{12cm}{#3} \\ \hline }
\newcommand{\mycell}[3]{

  #1 & #2 & \parbox{6cm}{
    \vspace{.1\baselineskip}
    #3
    \vspace{.1\baselineskip}
  } \\ \hline
}
\newcommand{\titulo}[1]{{\bf #1}}


\begin{document}

\pagenumbering{gobble}

\begin{center}
{\LARGE \bf Elementos de Lógica Digital - 2015/2}
\vspace{5mm}
\end{center}

\noindent
\titulo{Professor:} Marcos Daniel Baroni $<$\texttt{marcos.baroni@aluno.ufes.br}$>$

\noindent
\titulo{Carga horária semestral:} 45 horas

\noindent
\titulo{Horário:} Quintas, de 15h as 18h

\noindent
\titulo{Local:} Sala 108, CT 7

\noindent
\titulo{Ementa:}
\begin{itemize}
  \setlength\itemsep{-3pt}
  \item Sistemas de numeração
  \item Funções Lógicas e Portas Lógicas
  \item Simplificação de expressões lógicas (mapa de Karnaugh)
  \item Álgebra de Boole
  \item Circuitos combinacionais e sequenciais
  \item Simplificação de circuitos lógicos
  \item Circuitos aritmeticos (somadores, subtratores)
  \item Flip-flops
  \item Contadores
  \item Multiplexadores e demultiplexadores
  \item Memória
\end{itemize}

\noindent
\titulo{Critério de Avaliação:}
Duas provas parciais ($P$) e dois trabalhos ($T$).
A média parcial ($MP$) é calculada por:\\
\noindent \hspace*{12pt} $ MP = 0.7*P + 0.3 *T$. \\
\noindent A média final será:\\
\noindent \hspace*{12pt} $MF = MP$, \hspace{40pt} se $MP \geqslant 7.0$ \\
\noindent \hspace*{12pt} $MF = (MP+PF)/2$, \hspace{4pt} se $MP < 7.0$ \\
\noindent Onde $PF$ é a noda da prova final.\\
\noindent \hspace*{12pt} $MF \geqslant 5.0 \rightarrow$ \textrm{Aprovado} \\
\noindent \hspace*{12pt} $MF < 5.0 \rightarrow$ \textrm{Reprovado}
\\

\noindent
\titulo{Calendário:}
\vspace{-3pt}
\begin{multicols}{2}
\begin{tabular}{|r|c|l|}
\hline
\mycell{\textbf{Dia}}{\#}{\textbf{Tópico}}
\mycell{06/08}{1}{Introdução, sistemas de numeração, aritmética binária }
\mycell{13/08}{2}{Funções e portas lógicas, Expressões booleanas, circuitos e tabelas verdade, Álgebra de Boole}
\mycell{20/08}{3}{Simplificação de expressões booleanas, Diagramas de Beitch-Karnaugh}
\mycell{27/08}{4}{Circuitos combinacionais}
\mycell{03/09}{5}{Circuitos combinacionais, códigos, codificadores, decodificadores, decodificador de 7 segmentos, circuitos aritméticos}
\mycell{10/09}{6}{\emph{Enunciado 1º trabalho} \\ flip-flops (RS básico, RS com clock, JK, JK com clock/preset/clear)}
\mycell{17/09}{7}{Circuitos Sequenciais, flip-flop (JK mestre-escravo, tipo T, tipo D)}
\end{tabular}

\begin{tabular}{|r|c|l|}
\hline
\mycell{\textbf{Dia}}{\#}{\textbf{Tópico}}
\mycell{24/09}{8}{Exercícios e revisão para 1ª prova}
\mycell{01/10}{9}{\emph{1ª Prova}}
\mycell{08/10}{10}{\emph{Entrega da 1ª Prova} \\ Registrados de deslocamento, conversores paralelo-série, série-paralelo}
\mycell{15/10}{11}{Contadores assíncronos (pulso, década, etc)}
\mycell{22/10}{12}{\emph{Enunciado 2º trabalho} \\ Contadores síncronos}
\mycell{29/10}{13}{Máquinas de estado, multiplex/demultiplex, memórias}
\mycell{05/11}{14}{Memórias, revisão para 2ª prova}
\mycell{12/11}{15}{\emph{2ª prova}}
\mycell{19/11}{}{}
\mycell{26/11}{}{}
\mycell{03/12}{}{}
\mycell{10/12}{}{\emph{Prova Final}}
\end{tabular}
\end{multicols}

\noindent
\titulo{Material bibliográfico:}
\begin{itemize}
  \setlength\itemsep{1pt}
  \item IDOETA, I.V.;CAPUANO, F.G. Elementos de Eletrônica Digital, 27 ed. São Paulo: Ática, 1998.
  \item TOCCI, Ronald J. Sistemas Digitais. 5 Edição. Rio de Janeiro: Prentice Hall do Brasil, 1994.
  \item TANENBAUM, Andrew S. Organização Estruturada de Computadores. 3a Edição. Rio de Janeiro: Livros Técnicos e Científicos, 1990.
  \item STOKHEIN, Roger L. Princípios Digitais, 3a ed. São Paulo: Makron Books, 1996.
\end{itemize}

\end{document}

