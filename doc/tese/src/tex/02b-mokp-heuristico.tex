\section{Evolução estocástica por complexos}

O Algoritmo de Evolução estocástica por complexos (EEC),
\emph{Shuffled Complex Evolution} em inglês,
é um algoritmo evolutivo populacional proposto por Duan~\cite{duan1992effective}
e tem sido utilizado com sucesso em problemas de escalonamento
\cite{zhao2015shuffled}, seleção de projetos~\cite{elbeltagi2007modified},
problema da mochila $0-1$~\cite{bhattacharjee2014shuffled} e
problema da mochila multi-dimensional~\cite{baroni2015shuffled,baroni2016shuffled}.

O EEC é inspirado na evolução natural que ocorre de forma simultânea em
comunidades independentes.
O algoritmo trabalha com uma população particionada em $N$ comunidades,
ou complexos, cada uma contendo $M$ indivíduos.
Inicialmente a população de $N*M$ indivíduos é tomada aleatoriamente do espaço
de soluções viáveis.
Após esta inicialização a população é ordenada em ordem descrecente de aptidão
e o melhor global é identificado.
Toda a população é então particionada em $N$ complexos, cada um contendo $M$ indivíduos.
Neste processo de....