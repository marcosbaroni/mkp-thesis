
Várias abordagens exatas foram propostas pela literatura para encontrar o \paretoset{}
para o MOKP.
Primeiramente pode-se mencionar um trabalho teórico~\cite{klamroth2000dynamic},
sem resultados experimentais, onde cinco modelagens diferentes de programação dinâmica
são apresentadas e discutidas.

Um algoritmo de \emph{duas fases} é proposto em~\cite{visee1998two} para a solução de casos bi-objetivos.
A primeira fase do algoritmo utiliza uma versão escalar do problema, definida através da agregação das funções objetivo.
Esta versão é então considerada para gerar um subconjunto do \paretoset{}.
A segunda fase do algoritmo utiliza um algoritmo de \emph{branch and bound}
para encontrar o restante do \paretoset{}.
Diferentes estratégias de buscas e cortes para o algoritmo branch and bound são discutidas,
cujas performances são comparadas através de experimentos computacionais.

Um algoritmo baseado em um método de rotulação foi proposto em \cite{captivo2003solving}
no qual o MOKP é primeiramente transformado em um problema de caminho mínimo bi-objetivo
sem ciclos e então solucionado utilizando um método de
rotulação inspirado pelo algoritmo proposto em~\cite{martins1999labeling},
porém, explorando características específicas do problema transformado.

Atualmente o algoritmo exato que possui relatos de melhor performance é o algoritmo de
programação dinâmica desenvolvido por Bazgan, Hugot e Vanderpooten em~\cite{bazgan2009}.
O algoritmo é baseado em um clássico algoritmo de programação dinâmica para o problema da mochila
proposto por Nemhauser e Ullmann em~\cite{nemhauser1969discrete}, cujo desenvolvimento
foi inspirado nas observações de dominância entre soluções, apresentadas por Weignartner e Ness em~\cite{weingartner1967methods}.
O algoritmo de Bazgan, porém, aplica três dominâncias desenvolvidas especificamente para o caso multiobjetivo,
permitindo o descarte de uma grande quantidade de estados ao longo das iterações do algoritmo.

Uma proposta de melhoria para o algoritmo de Bazgan foi apresentada
por Figueira, Simões e Vanderpooten em~\cite{figueira2013algorithmic} para o caso bi-objetivo.
O trabalho introduz novas definições de limites superiores e inferiores de valores de
função objetivo, o que possibilita descartar uma quantidade ainda maior de estados
intermediário, resultando em
$20\%$ de melhoria no tempo computacional médio para instâncias bi-objetivo, segundo experimentos computacionais apresentados.

Recentemente cinco técnicas algorítmicas de compressão de dados para o caso bi-objetivo foram
apresentadas em~\cite{correia2018} para o algoritmo de Bazgan
como propostas de melhoria quanto a utilização de memória.
Cada técnica propõe uma modelagem diferente na representação dos estados
gerados pelo algoritmo.
Testes computacionais são apresentados, comparando as cinco técnicas
quanto a relação entre uso de memória e esforço computacional.

Para investigar a eficiência do método de indexação multi-dimensional para
soluções do MOKP no contexto exato, o algoritmo de Bazgan será utilizado.
%especialmente para os casos 3-objetivos.
A seguir será introduzido primeiramente o algoritmo de Nemhauser e Ullmann,
que é a base para o desenvolvimento do algoritmo de Bazgan.
Posteriormente serão descritas as três relações de dominância utilizadas no algoritmo de Bazgan, tal como introduzidas em \cite{bazgan2009}, 
para então ser descrito o próprio algoritmo de Bazgan.

O algoritmo de Nemhauser e Ullmann foi primeiramente proposto para resolver o problema
da mochila $0-1$. Porém, sua concepção é genérica e, com poucas adaptações,
pode ser facilmente aplicada a diversas variações do problema da mochila.
O algoritmo mantém durante a execução um conjunto de estados, cada um representando uma solução
viável para o problema.
A execução do algoritmo é definida por $n$ iterações, sendo $n$ o número total de itens da instância.
A cada iteração um novo item é considerado como item candidato a fazer parte da solução.
Novas soluções contendo este novo item são então geradas e adicionadas
ao conjunto de estados.
Ao final dos $n$ passos as soluções não eficientes são descartadas, restando o conjunto Pareto.
O Algoritmo~\ref{alg:dp1} descreve o algoritmo de Nemhauser e Ullmann para o MOKP.

\begin{algorithm}
  \footnotesize
  \caption{O algoritmo de Nemhauser e Ullmann para o \mokp.}
  \label{alg:dp1}
  \Kw{$\bsym{p}, \bsym{w}, W$}
\Begin{
  \SetAlgoLined
  $S^0 = \big\{\emptyset\big\}$\;
  \For{$k \gets 1, n$}{
    $S_*^k = S^{k-1} \cup \{\sol{x} \cup k \:|\: \sol{x} \in S^{k-1}\}$\;
    TODO...\;
  }
  $P = \{\sol{x} \:|\: \nexists \sol{a} \in S^n: \dom{a}{x} \;|\; \weight{x} \leq W \}$\;
  \textbf{return} $P$\;
}
\end{algorithm}

O Algoritmo~\ref{alg:dp1} inicia definindo o conjunto inicial de estados $S_0$
contendo apenas a solução vazia (linha 2).
A cada $k$-ésima iteração é gerado o conjunto de estados $S_k$ (linhas 3-5).
Um estado $s_k = (s_k^1, \ldots, s^\np_k, s^{\np+1}_k) \in S_k$ representa uma solução
que tem valor $s^i_k$ como $i$-ésimo objetivo ($i = 1, \ldots, \np$)
e $s^{\np+1}_k$ de peso.
Cada um dos estados $s_{k-1} \in S_{k-1}$ é incluído em $S_k$, além de uma cópia tendo
o item $k$ na solução, desde que esta seja viável (linha 4).
Assim sendo, o conjunto $S_k$ é formado por todas as soluções viáveis compostas de itens exclusivamente
pertencentes aos $k$ primeiros itens ($k = 1, \ldots, n$).
Ao final dos $n$ passos são descartadas as soluções não eficientes (linha 6) e retornado
o \paretoset{} (linha 7).

Pode-se observar que a linha 4 do algoritmo tem o potencial de dobrar o tamanho do conjunto $S_k$
a cada iteração, uma vez que, para cada solução de $S_{k-1}$ são adicionadas duas cópias em $S_k$,
induzindo um crescimento exponencial do conjunto de estados.
De fato o Algoritmo~\ref{alg:dp1} pode vir a desnecessariamente manter algumas soluções
não promissoras ao longo do algoritmo, ao descartar apenas as soluções inviáveis durante as iterações.
Um exemplo são a própria solução vazia e outras soluções ineficientes, que são mantidas
por todas as iterações, sendo descartadas apenas na linha 6.

O algoritmo Bazgan propõe reduzir este efeito através da
utilização de algumas relações de dominância, capazes de descartar uma
quantidade considerável de estados mantidos pelos conjuntos $S_k$.
Este descarte resulta em ganho de performance, uma vez que o algoritmo trabalhará com
conjuntos $S_k$ menores.

A ordem na qual os itens são considerados é também uma questão crucial na implementação
do algoritmo, pois a inserção de alguns itens com determinadas características
pode fazer com que o conjunto de soluções mantidos pelo algoritmo cresça desde o início,
o que não é o desejado.
Sabe-se, por exemplo, que no caso do problema da mochila tradicional (escalar),
para se obter uma boa solução, os itens devem ser geralmente considerados em
ordem decrescente segundo a proporção
$p_i/w_i$~\cite{ehrgott2013multicriteria, martello1990knapsack},
porém, para o caso multiobjetivo não existe uma ordem natural semelhante.

Para tanto são introduzidas duas ordenações $\ord^{sum}$ e $\ord^{max}$
derivadas da agregação das ordens $\ord^j$ inferidas pelas proporções
$p_i^j/w_i$ para cada objetivo $(j = 1, \ldots, \np)$.
Considere $r_i^l$ o rank (ou seja, a posição) do item $l$ na ordenação $\ord^j$.
$\ord^{sum}$ denota uma ordenação segundo valores crescentes da soma dos
ranks dos itens nas $\np$ ordenações $(i = 1, \ldots, \np)$.
$\ord^{max}$ denota uma ordenação segundo valores crescentes de máximo ou
piores ranks de itens nas $\np$ ordenações $\ord^j (j = 1, \ldots, \np)$,
onde o pior rank do item $l$ nas $\np$ ordenações $\ord^j (j = 1, \ldots, \np)$
é calculado como $max_{i=1}^{\np}\{r_l^i\} + \frac{1}{\np n} \sum_{i = 1}^{\np} r_l^i$
para distinguir itens com o mesmo rank máximo.
Sempre que necessário será usada a notação $\ord_{rev}$ para se referir à ordem
reversa de uma ordenação $\ord$.

Segundo os testes computacionais relatados em \cite{bazgan2009},
a utilização da ordenação $\ord^{max}$ reduz em, ao menos, $34\%$ o tempo computacional
total demandado pelo algoritmo, chegando a $99\%$ em algumas instâncias, quando
comparada a uma ordenação aleatória.
Por este motivo, sempre que necessário, a ordenação $\ord^{max}$ será
utilizada neste trabalho para se definir a preferência de inserção dos itens na
composição de uma solução.

A seguir, serão apresentadas algumas definições relativas
a o contexto de programação dinâmica e conjunto de estados,
importantes no desenvolvimento e aplicação das relações de dominâncias.

%\subsection{Ordem de consideração dos itens}

\begin{mydef}[Extensão, Restrição e Complemento]
Considere o Algoritmo~\ref{alg:dp1} e qualquer estado $s_k \in S_K (k < n)$.
Um \emph{complemento} de $s_k$ é qualquer subconjunto  $J \subseteq \{k+1, \ldots, n\}$
tal que $s_k^{\np+1} + \sum_{j \in J} w_j \leq W$.
Assume-se que qualquer estado $s_n \in S_n$ admite o conjunto vazio como único complemento.
Um estado $s_n \in S_n$ é uma \emph{extensão} de $s_k \in s_k (k \leq n )$ se, e somente se,
existe um complemento $I$ de $s_k$ tal que $s_n^j = s_k^j + \sum_{i \in I} p_i^j$
para $j = 1, \ldots, \np $
e $s_n^{\np+1} = s_k^{\np+1} + \sum_{i \in I} w_i$.
O conjunto de extensões de $s_k$ é denotado por $Ext(s_k) (k \leq n)$.
Um estado $s_k \in S_k (k \leq n)$ é uma \emph{restrição} do estado $s_n \in S_n$
se, e somente se, $s_n$ é uma extensão de $s_k$.
\end{mydef}

\begin{mydef}[Conjunto cobertura e Conjunto independente]
  Um conjunto $B \subseteq A$ é um \emph{conjunto cobertura} (ou somente \emph{cobertura})
  de $A$ com respeito à relação $\succsim$ se, e somente se, para todo
  $a \in A \backslash B$ existe $b \in B$ tal que $b \succsim a$.
  Um conjunto $B$ é um \emph{conjunto independente} com respeito a
  uma relação $\succsim$ se, e somente se, para todo $b, b' \in B, b \neq b', \neg ( b \succsim b')$.
\end{mydef}

Na seção seguinte serão apresentadas a definição de relação de dominância e a forma com que as relações
de dominância são aplicadas ao Algoritmo~\ref{alg:dp1}.

\subsection{Aplicação de Relações de Dominância}
\label{sec:domrel}

\begin{mydef}[Relação de dominância entre soluções]
Considerando o Algoritmo~\ref{alg:dp1},
uma relação $\rel_k$ sobre os elementos de $S_k, k = 1,\ldots, n$, é uma \emph{relação de dominância}
se, e somente se, para todo $s_k, s_{k'} \in S_k$ vale
\begin{equation}
  s_k \rel_k s_{\til{k}} \Rightarrow \forall s_{\til{n}} \in Ext(s_{\til{n}}),
    \exists s_n \in Ext(s_k), s_n \dom s_{\til{n}}
  \label{eq:domrel}
\end{equation}
\label{def:domrel}
\end{mydef}
Considerando um passo $k$ do algoritmo, uma relação de dominância $\rel_k$
e duas soluções $s_k, s_{k'} \in S_k$,
da Definição~\ref{def:domrel}, conclui-se que se $s_k \rel_k s_{k'}$, então
todas as soluções \emph{geradas} por $s_{k'}$ nos passos seguintes serão
dominadas por soluções geradas por $s_k$.
Por este motivo, a solução $s_{k'}$ pode ser descartada do conjunto $S_k$,
mesmo estas sendo viáveis.
Vale notar que se $\rel_k^i, \; i = 1, \ldots, \nrel\; (\nrel \geq 1)$ são relações de dominância
então $\rel_k = \bigcup^{\nrel}_{i = 1} \rel_k^i$ é também uma relação de dominância.

%Apesar das relações de dominância não serem transitivas por definição,
%costumam ser transitivas por construção, como é o caso
%das três relações de dominância utilizadas neste trabalho.
%Vale notar que se $\rel_k^i, \; i = 1, \ldots, \nrel\; (\nrel \geq 1)$ são relações de dominância
%então $\rel_k = \bigcup^{\nrel}_{i = 1} \rel_k^i$ é também uma relação
%de dominância, geralmente não transitiva mesmo se $\rel_k^i, \; i = 1, \ldots, \nrel$
%forem transitivas.

Para se ter uma implementação eficiente do algoritmo de programação dinâmica,
é recomendável utilizar múltiplas relações de dominância
$\rel_k^1, \ldots, \rel_k^\nrel \; (\nrel \geq 1)$ a cada execução da $k$-ésima iteração
$k = 1, \ldots, n$, uma vez que cada relação $\rel_k^i$ irá descartar um conjunto diferente de
estados ao explorar características específicas do problema.

O Algoritmo~\ref{alg:baz1new} apresenta uma adaptação do Algoritmo~\ref{alg:dp1}
para utilizar múltiplas relações de dominância.
O algoritmo inicia definindo o conjunto inicial de estados $C_0$ contendo apenas a
solução vazia (linha 2).
Em seguida são executadas $n$ iterações onde, na $k$-ésima iteração, o $k$-ésimo item
é considerado (linhas 3 a 7).
Ao iniciar a $k$-ésima iteração, cada um dos estados $s_{k-1} \in C_{k-1}$ é incluído em $C_k^0$,
além de uma cópia tendo o item $k$ na solução, desde que esta seja viável (linha 4).
Em seguida as $p$ relações de dominância são aplicadas de forma sucessiva (linha 5 a 6).
Para cada $i$-ésima relação de dominância, o conjunto $C_k^i$ é definido pelos
estados pertencentes à $C_k^{i-1}$ que não são dominados por nenhum outro estado,
segundo a relação de dominância $D^i_k$ (linha 6).
Após a aplicação das $p$ relações de dominância, o conjunto $C_k$ é definido
como sendo o resultado da aplicação da última relação de dominância.
Assim como o conjunto $S_k$ do Algoritmo~\ref{alg:dp1}, o conjunto $C_k$ é formado por todas as soluções
viáveis compostas de exclusivamente dos primeiros $k$ itens ($k = 1, \ldots, n$),
porém, o conjunto $C_k$ é um conjunto reduzido, um a vez que cada $i$-ésima iteração
teve por objetivo eliminar as soluções dominadas segundo a relação $D^i_k$.
Concluídas as $n$ iterações, o algoritmo retorna o conjunto $C_n$
que caracteriza o \paretoset{}, uma vez que todas as soluções dominadas foram descartadas.


\missingf{Que tal aumentar a descricao desse algoritmo tal como fez no algoritmo 1? Acho que facilita o leitor e dá mais massa ao seu trabalho

\resp Feito.
}

\begin{algorithm}
  \footnotesize
  \caption{Programação dinâmica utilizando múltiplas relações de dominância.}
  \label{alg:baz1new}
  \Begin{
  $C_0 \gets \{(0, \ldots, 0)\}$\;
  \For{$k \gets 1, n$}{
    $C_k^0 \gets C_{k-1} \cup \big\{(s^1 + p_k^1, \ldots, s^\np + p_k^\np, s^{\np+1} + w_k)
      \; | \; s^{\np+1} + w_k \leq W, s \in C_{k-1} \big\}$\;
    \For{$i \gets 1, \nrel$}{
      $C_k^i \gets \big\{s \in C_k^{i-1} \big| \nexists(s' \in C_k^{i-1})[s' \relk^i s]\big\}$\;
    }
    $C_k \gets C_k^\nrel$\;
  }
  \textbf{return} $C_n$\;
}
\end{algorithm}

%% TODO:
%%\begin{myprop}
%%  Para quaisquer relações de dominância $\rel_k^1, \ldots, \rel_k^\nrel (\nrel \geq 1)$
%%  o Algoritmo~\ref{alg:baz1new} retorna $C_n$ conjunto cobertura de $S_k$ 
%%\end{myprop}

%%\begin{proof}
%%  Trivial.
%%\end{proof}

%\begin{algorithm}
%  \footnotesize
%  \caption{Programação dinâmica utilizando múltiplas relações de dominância.}
%  \label{alg:baz1}
%  \begin{algorithmic}[1] % The number tells where the line numbering should start
    \Function{DP}{}
    \State $C^0 \gets \{(0, \ldots, 0)\}$;
    \For{$k \gets 1, n$}
      \State $C_k^0 \gets C_{k-1} \cup \big\{(s_{k-1}^1 + p_k^1, \ldots, s_{k-1}^\np + p_k^\np, s_{k-1}^{\np+1} + w_k)
        \; | \; s_{k-1}^{\np+1} + w_k \leq W, s_{k-1} \in C_{k-1} \big\};$
      \For{$i \gets 1, \nrel$}
        \State {Determinar um conjunto $C_k^i$ cobertura do conjunto $C_{k-1}^i$
          com respeito a $\relk^i$};
      \EndFor
      \State {$C_k \gets C_k^\nrel$};
    \EndFor
  \State \Return $C_n$;
  \EndFunction
\end{algorithmic}

%\end{algorithm}
%A proposição seguinte caracteriza o conjunto $C_k$ obtido ao final de cada iteração.
%Este resultado será útil posteriormente na prova do Teorema~\ref{teo:coverset},
%cujo enunciado estabelece condições que garantem que o Algoritmo~\ref{alg:baz1},
%apesar de reduzir o tamanho dos conjuntos de estados mantidos durante a execução do algoritmo,
%gera o conjunto de vetores objetivos não dominados, referentes às soluções eficientes que compõe
%o \paretoset{}.
%
%\begin{myprop}
%  Para quaisquer relações de dominância $\rel_k^1, \ldots, \rel_k^\nrel (\nrel \geq 1)$
%  sobre $S_k$, o conjunto $C_k$ obtido pelo Algoritmo~\ref{alg:baz1}
%  em cada iteração é um conjunto cobertura de $C_k^0$ com respeito a
%  $\rel_k = \bigcup_{i=1}^\nrel \rel_k^i \; (k = 1, \ldots, n)$.
%  \label{prop:coverset}
%\end{myprop}

%\begin{proof}
%Considere que $s_k \in C_k^0\backslash C_k^\nrel$ foi removido quando
%selecionado um conjunto cobertura na iteração da linha 5.
%Seja $i_1 \in \{1, \ldots, \nrel \}$ a iteração da linha 5, tal que
%$s_k \in C_k^{i_1 - 1}\backslash C_k^{i_1}$.
%Uma vez que $C_k^{i_1}$ é um conjunto cobertura de $C_k^{i_1-1}$ com respeito
%a $\rel^{i_1}_{\til{k}}$, existe $s_{\til{k}}^{(1)} \in C_k^{i_1}$ tal que $ s_{\til{k}}^{(1)} %\rel_k^{i_1} {s_k}$.
%Se $s_{\til{k}}^{(1)} \in C_k^m$ então as propriedades de cobertura são válidas,
%uma vez que $\rel_k^{i_1} \subseteq \rel_k$.
%Caso contrário, existe uma iteração $i_2 > i_1$, correspondente à iteração
%da linha 5 tal que $s_{\til{k}}^{(1)} \in C_k^{i_2-1}\backslash C_k^{i_2}$.
%Como anteriormente,  estabelece-se que existe  $s_{\til{k}}^{(2)} \in C_k^{i_2}$
%tal que $s_{\til{k}}^{(2)} \rel_k^{i_2} s_{\til{k}}^{(1)}$.
%Uma vez que $\rel_k^{i_2} \subseteq \rel_k$ tem-se que
%$s_{\til{k}}^{(2)} \rel_k s_{\til{k}}^{(1)} \rel_k s_k$ e por transitividade de $\rel_k$
%garante-se que $s_{\til{k}}^{(2)} \rel_k s_k$.
%Pela repetição deste processo pode-se garantir que a existência de um estado
%$s_{\til{k}} \in C_k^m$, tal que $s_{\til{k}} \rel_k s_k$.
%\qedhere
%\end{proof}

%\begin{myteo}
%  Para qualquer família de relações de dominância
%  $\rel_k^i (i = 1, \ldots, \nrel; k = 1, \ldots, n)$,
%  o Algoritmo~\ref{alg:baz1} retorna $C_n$ que é um conjunto cobertura de $S_n$
%  com respeito a $\dom$.
%  Além disso, se na iteração $n$ se utilizar ao menos uma relação
%  $\rel_n^i = \dom$ e garantir-se que o conjunto cobertura $C_n^i$
%  é também independente com respeito a $\rel_n^i$ então $C_n$ representa
%  o conjunto de vetores objetivos não dominados.
%  \label{teo:coverset}
%\end{myteo}

%\begin{proof}
%  Considere $s_{\til{n}} \in S_n\backslash C_n$. Todas as restrições
%  foram removidas ao se reter o conjunto cobertura com respeito a
%  $\rel_k = \cup_{i=1}^m \rel_k^i$ durante as iterações $k \leq n$.
%  Seja $k_1$ a iteração mais alta em que $C_{k_1}^0$ ainda
%  contém restrições de $s_{\til{n}}$, as quais serão removidas
%  aplicando uma das relações $\rel_{k_1}^i (i = 1, \ldots, \nrel)$.
%  Considere alguma destas restrições, denotada por $s_{\til{k_1}}^{(n)}$.
%  Uma vez que $s_{\til{k_1}}^{n} \in C^0_{k_1}\backslash C_{k_1}$ sabe-se,
%  pela Proposição~\ref{prop:coverset}, que existe $s_{k_1} \in C_{k_1}$
%  tal que $s_{k_1} \rel_{k_1} s_{\til{k_1}}^{(n)}$.
%  Pela Equação~\ref{eq:domrel}, uma vez que $\rel_{k}$ é uma relação de
%  dominância, tem-se que todas as extensões de $s_{k_1}^{(n)}$, e
%  particularmente para $s_{\til{n}}$, existe $s_{n_1} \in Ext(s_{k_1})$
%  tal que $s_{n_1} \dom s_{\til{n}}$.
%  Se $s_{n_1} \in C_n$, então a propriedade de cobertura é válida.
%  Caso contrário, existe uma iteração $k_2 > k_1$, correspondente
%  a iteração mais alta em que $C_0^{k_2}$ ainda contém restrições de $s_{n_1}$,
%  que serão removidas aplicando-se uma das relações $\rel_{k_2}^i (i = 1, \ldots, \nrel)$.
%  Considere uma destas restrições, denotada por $s_{k_2}^{(n_1)}$.
%  Como anteriormente, estabelecemos a existência de um estado $s_{k_2} \in C_{k_2}$
%  tal que existe $s_{n_2} \in Ext(s_{k_2})$ tal que $s_{n_2} \dom s_{n_1}$.
%  A transitividade de $\dom$ garante que $s_{n_2} \dom s_{\til{n}}$.
%  Pela repetição deste processo, estabelece-se a existência de um estado
%  $s_{k_2} \in C_{k_2}$ tal que existe $s_{n_2} \in Ext(s_{k_2})$
%  tal que $s_n \dom s_{\til{n}}$.

%  %Além disso, selecionando-se um conjunto $C_n^i$ que é conjunto independente
%  %com relação a $\rel_n^i = \dom$, esta propriedade se mantém válida para $C_n^\nrel$
%  %o qual é subconjunto de $C_n^i$.
%  %Dessa forma $C_n$, que corresponde ao conjunto eficiente reduzido,
%  %representa o conjunto de vetores objetivos não dominados.
%  \qedhere
%\end{proof}

%O teorema anterior apenas requer que um dos $n . \nrel$ conjuntos coberturas
%seja independente com respeito à sua relação de dominância.
%Mesmo se todos os conjuntos $C_k^i$ sejam qualquer conjunto cobertura, a eficiência
%prática do Algoritmo~\ref{alg:baz1} induz a geração de conjuntos de tamanho mínimo.
%Isto pode ser facilmente alcançado quando as relações de dominância $\rel_k^i$
%são transitivas, através da seleção de conjuntos $C_k^i$ que são independentes
%com respeito a $\rel_k^i$ (linha 6 do Algoritmo~\ref{alg:baz1}).

%\subsection{Geração de conjuntos cobertura e independente}

%Apresenta-se no Algoritmo~\ref{alg:baz2} uma implementação detalhada de como
%produzir $C_k^i$, um conjunto cobertura e independente de $C_k^{i-1}$ com respeito a relação
%transitiva $\rel_k^i$ como proposta de implementação do procedimento da 
%linha 6 do Algoritmo~\ref{alg:baz1}.
%A proposição seguinte assegura a corretude do algoritmo.

%\begin{algorithm}
%  \caption{Computação do conjunto $C_k^i$ cobertura e independente do conjunto
%  $C_k^{i-1}$ com respeito a relação transitiva $\rel_k^i$.}
%  \label{alg:baz2}
%  \Kw{$C_k^{i-1} = \{ s_{k(1)}, \ldots, s_{k(r)}\}$}
\Begin{
  $C_k^i \gets \{s_{k(1)}\}$\;
  \For{$h \gets 2, r$}{
    $C_k^i = \{s_{\til{k}(1)}, \ldots, s_{\til{k}(l_h)}\}$\;
    $dominated \gets false$\;
    $dominates \gets false$\;
    $j \gets 1$\;
    \While{$ j \leq l_h$ \textbf{and} $\neg dominated$ \textbf{and} $\neg dominates$}{
      \uIf{$s_{\til{k}(j)} \rel_k^i s_{k(h)}$}{
        $dominated \gets true$\;
      }
      \ElseIf{$s_{k(h)} \rel_k^i s_{\til{k}(j)}$}{
        $C_k^i \gets C_k^i\backslash\{s_{\til{k}(j)}\}$\;
        $dominates \gets true$\;
      }
      $j \gets j + 1$\;
    }
    \If{$\neg dominated$}{
      \While{$j \leq l_h$}{
        \If{$s_{k(h)} \rel_k^i s_{\til{k}(j)}$}{
          $C_k^i \gets C_k^i \backslash \{s_{\til{k}(j)}\}$\;
        }
        $j \gets j + 1;$
      }
      $C_k^i \gets C_k^i \cup \{s_{k(h)}\};$
    }
  }
  \textbf{return} $C_k$
}
%\end{algorithm}

%\begin{myprop}
%  Para qualquer relação de dominância $\rel_k^i$ sobre $S_k$, o
%  Algoritmo~\ref{alg:baz2} retorna $C_k^i$ um conjunto cobertura e independente
%  de $C_k^{i-1}$ com respeito a $\rel_k^i (j = 1, \ldots, n; i = 1, \ldots, \nrel)$.
%\end{myprop}

%\begin{proof}
%  Claramente $C_k^i$ é independente com relação a $\rel_k^i$, uma vez que se
%  insere o estado $s_k$ ao conjunto $C_k^i$ na linha 20
%  apenas se não for dominado por nenhum outro estado em $C_k^i$ (linha 9)
%  e todos os estados dominados por $s_k$ tenham sido removidos de $C_k^i$
%  (linha 11 e 17).

%  Pode-se observar agora que $C_k^i$ é um conjunto cobertura de $C_k^{i-1}$
%  com respeito a $\rel_k^i$.
%  Considere $s_{\til{k}} \in C_k^{i-1} \backslash C_k^i$.
%  Isto ocorre tanto porque não passa no teste da linha 9 quanto porque
%  foi removido na linha 11 ou 17.
%  Isto é devido respectivamente a um estado $s_{k*}$ já existente em $C_k^i$
%  ou a ser incluído em $C_k^i$ (na linha 20) tal que $s_{k*}\rel_k^i s_{\til{k}}$.
%  Pode ser que $s_{k*}$ seja removido de $C_k^i$ em uma iteração posterior
%  ao loop (linhas 11 e 17) se existir um novo estado $s^{\hat k} \in C_k^{i-1}$ a ser
%  incluído em $C_k^i$, tal que $s_{\hat k} \rel_k^i s_{k*}$.
%  Entretanto, a transitividade de $\rel_k^i$ garante a existência, ao fim da
%  iteração $k$, de um estado $s_k \in C_k^i$ tal que $s_k \rel_k^i s_{\til{k}}$.
%  \qedhere
%\end{proof}

%O Algoritmo~\ref{alg:baz2} pode ser aprimorado uma vez que geralmente é possível
%gerar estados de $C_k^{i-1} = \{s_{k(1)}, \ldots, s_{k(r)}\}$ conforme
%uma \emph{ordem de preservação de dominância} para $\rel_k^i$ de forma que
%todo $l < j (1 \leq l, j \leq r)$ ou se tem $s_{k(l)} \rel_k^i s_{k(j)}$
%ou $\neg (s_{k(j)} \rel_k^i s_{k(l)})$.
%A proposição seguinte provê a condição necessária e suficiente para estabelecer
%a existência da ordem de preservação de dominância para uma relação de dominância.

%\begin{myprop}
%  Seja $D_k$ uma relação de dominância sobre $S_k$.
%  Existe uma ordem de preservação de dominância para $D_k$ se, e somente se,
%  $\rel_k$ não admite ciclos em sua parte assimétrica.
%  \label{prop:domorder}
%\end{myprop}

%\begin{proof}
%  \noindent \\ $\Rightarrow$ A existência de um ciclo na parte assimétrica de $D^k$
%  implicaria na existência de dois estados consecutivos $s_{k(j)}$ e $s_{k(l)}$
%  neste ciclo sendo $j > l$, o que é uma contradição.\\
%  $\Leftarrow$ Qualquer ordem topológica baseada na parte assimétrica de $D_k$
%  é uma ordem de preservação de dominância para $D_k$.
%  \qedhere
%\end{proof}

%Na Seção seguinte é apresentado um exemplo de ordem de preservação de dominância.
%Se os estados em $C_k^{i-1}$ são gerados conforme uma ordem de preservação de
%dominância para $D_k^i$, a linha 11 e o laço das linhas 16-19 do Algoritmo~\ref{alg:baz2}
%podem ser omitidos.



%\missing{Dizer que o artigo original conclui através de testes qual é a melhor
%ordenação para ser utilizada nas iterações do algoritmo.}

%\subsection{Relações de dominância}
\label{sec:reldom}

\subsection{As Relações de Dominância Utilizadas}

A seguir serão apresentadas as três relações de dominância utilizadas no algoritmo.
Cada relação de dominância explora uma consideração específica.
Por isto recomenda-se utilizar relações de dominância que sejam complementares.
Além disso, para se escolher uma relação de dominância é necessário considerar
sua potencial capacidade de descarte de estados diante do custo computacional requerido.
As primeiras duas relações são razoavelmente triviais de se estabelecer, sendo
a última ainda considerada mesmo um tanto mais complexa, devido a sua
complementaridade diante das duas primeiras.

A primeira relação de dominância tem o objetivo de evitar a geração de soluções ineficientes
através do descarte de soluções relativamente \emph{vazias}, tipicamente geradas nas iterações iniciais do algoritmo.
A segunda relação de dominância é uma generalização para o MOKP da clássica relação de dominância
proposta por Weignartner e Ness e utilizada no Algoritmo~\ref{alg:dp1}.
A terceira relação de dominância visa descartar soluções consideradas não promissoras, i.e.,
que não têm potencial de aumento de função objetivo suficiente para superar o de outras soluções já
existentes.

\subsubsection{A Relação $D^r$}

A primeira relação de dominância se estabelece segundo a seguinte observação.
Quando a capacidade residual de uma solução associada a um estado $s_k$
da iteração $k$ é maior ou igual à soma dos pesos dos itens restantes,
i.e. itens $k+1, \ldots, n$, o único complemento de $s_k$ que pode resultar
em uma solução eficiente é o complemento máximo $I = \{k+1, \ldots, n\}$.
Portanto, neste contexto, não se faz necessário gerar as extensões de $s_k$
que não contenham todos os itens restantes.
Define-se então a relação de dominância $\relI$ sobre $S^k$ para
$k = 1, \ldots, n$ como:
\begin{displaymath}
  \forall s_k, s_{\til{k}} \in S_k, \enskip
  s_k \relI s_{\til{k}}
    \Leftrightarrow
    \begin{cases}
      s_{\til{k}} \in S_{k-1}, \\
      s_k = (s_{\til{k}}^1 + p_k^1, \ldots, s_{\til{k}}^\np + p_k^\np, s_{\til{k}}^{\np+1} + w_k), \; \text{e} \\
      s_{\til{k}}^{\np+1} \leq W - \sum_{i=k}^n w_i
    \end{cases}
\end{displaymath}

A seguinte proposição enuncia que a relação $D_k^r$ é de fato uma relação de dominância.

\begin{myprop}[Relação $D_k^r$]
  \noindent
  $D_k^r$ é uma relação de dominância
  %\begin{enumerate}
  %  \item[(a)] $D_k^r$ é uma relação de dominância
  %  \item[(b)] $D_k^r$ é transitiva
  %  \item[(c)] $D_k^r$ admite ordem de preservação de dominância
  %\end{enumerate}
\end{myprop}

\begin{proof}
  \noindent
  %\begin{enumerate}
    %\item[(a)]{
      Considere dois estados $s_k$ e $s_{\til{k}}$ tal que $s_k \relI s_{\til{k}}$.
      Uma vez que $s_k \relI s_{\til{k}}$ temos que $s_k \dom s_{\til{k}}$ e consequentemente
      $s_k^{\np+1} = s_{\til{k}^{\np+1}} + w_k \leq W - \sum_{i=k+1}^n w_i$.
      Sendo assim, qualquer subconjunto $I \subseteq \{k+1, \ldots, n\}$ é um complemento
      de $s_{\til{k}}$ e $s_k$.
      Portanto, para todo estado $s_{\til{n}} \in Ext(s_{\til{k}})$,
      existe $s_n \in Ext(s_k)$, baseado no mesmo complemento que $s_{\til{n}}$,
      tal que $s_n \dom s_{\til{n}}$.
      Isto estabelece que $\relI$ satisfaz a Equação~\ref{eq:domrel} da
      Definição~\ref{def:domrel}.
    %}
    %\item[(b)]{Trivial.}
    %\item[(c)]{Pela Proposição~\ref{prop:domorder}, uma vez que $\relI$ é transitiva.}
    \qedhere
  %\end{enumerate}
\end{proof}

Esta relação de dominância é um tanto fraca, uma vez que em cada $k$-ésima
iteração ela se aplica apenas entre um estado que não contém o $k$-ésimo item
e sua extensão, que contém o $k$-ésimo item.
Apesar disso, é uma relação de dominância extremamente fácil de ser verificada,
uma vez que, ao ser o valor de $W - \sum_{i=k}^n$ computado, a relação $\relI$
requer apenas uma comparação para ser estabelecida entre dois estados.

\subsubsection{A Relação $\relII$}
%\subsection{A relação $123$}
A relação de dominância seguinte é a generalização para o caso multiobjetivo
da relação de dominância utilizada no Algoritmo~\ref{alg:dp1}.
Esta segunda relação de dominância é definida sobre $s_k$ para $k = 1, \ldots, n$ por:
\begin{displaymath}
  \forall s_k, s_{\til{k}} \in S_k, \; s_k \relII s_{\til{k}}
    \Leftrightarrow
    \begin{cases}
      s_k \dom s_{\til{k}} \; & \text{e} \\
      s_k^{\np+1} \leq s_{\til{k}}^{\np+1} \; & \text{se} \; k < n
    \end{cases}
\end{displaymath}
Observe que a condição sobre os pesos $s_k^{\np+1}$ e $s_{\til{k}}^{\np+1}$
garante que todo complemento de $s_{\til{k}}$ é também um complemento de
$s_k$.
A seguinte proposição enuncia que $\relII$ é de fato uma relação de dominância.

\begin{myprop}[Relação $\relII$]
  \noindent
  $\relII$ é uma relação de dominância
  %\begin{enumerate}
  %  \item[(a)] $\relII$ é uma relação de dominância
  %  \item[(b)] $\relII$ é transitiva
  %  \item[(c)] $\relII$ admite ordem de preservação de dominância
  %  \item[(c)] $\relII = \dom$
%\end{enumerate}
\end{myprop}

\begin{proof}
  \noindent
  %\begin{enumerate}
    %\item[(a)]{
      Considere estados $s_k$ e $s_{\til{k}}$ tal que $s_k \relII s_{\til{k}}$.
      Uma vez que $s_k \relII s_{\til{k}}$, tem-se que $s_k \dom s_{\til{k}}$, e consequentemente
      $s_k^{\np+1} \leq s_{\til{k}}^{\np+1}$.
      Portanto, qualquer subconjunto       $J \cup \{k+1, \ldots, n\}$ complemento de $s_{\til{k}}$ é também complemento de $s_k$.
      Assim, para todo $e_{\til{n}} \in Ext(s_{\til{n}})$, existe
      $e_{n} \in Ext(s_{n})$, baseado no mesmo complemento de $s_{\til{n}}$.
      O que estabelece que $\relII$ satisfaz a Equação~\ref{eq:domrel}
      da Definição~\ref{def:domrel}.
    %}
    %\item[(b)]{Trivial.}
    %\item[(c)]{Pela Proposição~\ref{prop:domorder}, uma vez que $\relII$ é transitiva.}
    %\item[(d)]{Por definição.} \qedhere
  %\end{enumerate}
\end{proof}

\missingf{frase mal formulada:  Isto implica que $s_k \dom s_{\til{k}}$ ????? Além disso, uma vez que... Corrigir!!!

\resp Reformulei.
}


A relação $\relII$ é uma relação de dominância poderosa, uma vez que um estado
pode possivelmente dominar todos os outros estados de maior peso.
Esta relação requer no máximo $\np+1$ comparações para ser estabelecida entre
dois estados.

\subsubsection{A Relação $\rel^b$}

A terceira relação de dominância é baseada na comparação entre extensões
específicas de um estado e um limite superior de extensões de outros estados.
Um limite superior de um estado é definido a seguir.

\begin{mydef}[Limite superior]
  Um vetor objetivo $u = (u^1, \ldots, u^\np)$ é um \emph{limite superior}
  de um estado $s_k \in S_k$ se, e somente se, $\forall s_n \in Ext(s_k)$
  tem-se que $u^i \geq s_n^i, i = 1, \ldots, \np$.
\end{mydef}

Um tipo geral de relação de dominância pode ser derivada da maneira seguinte.
Considere dois estados $s_k, s_{\til{k}} \in S_k$. Se existe um complemento
$I$ de $s_k$ e um limite superior $\tilde{u}$ de $s_{\til{k}}$ tal que
$s_k^j + \sum_{i \in I} p_i^j \geq \tilde{u^j},\; j = 1, \ldots, \np$,
então $s_k$ domina $s_{\til{k}}$.

Para se implementar este tipo de relação de dominância é necessário
especificar os complementos e os limites superiores a serem utilizados.
Na implementação do algoritmo em questão, são considerados
dois complementos específicos $I'$ e $I''$ obtidos pelo simples algoritmo
de preenchimento \emph{guloso} definido a seguir.
Após rerotular os itens $k+1, \ldots, \np$ de acordo com a ordenação
$\ord^{sum}$ (e $\ord^{max}$ respectivamente), o complemento $I'$
(e $I''$ respectivamente) são obtidos através a inserção sequencial
dos itens restantes na respectiva solução, de forma que a restrição
de capacidade é respeitada.

Como definição do limite superior $u$ é aplicado a cada valor de objetivo
a definição de limite superior proposta por \cite{martello1990knapsack} para o caso escalar.
Considere $\overline{W}(s_k) = W - s_k^{\np+1}$ a capacidade residual associada
ao estado $s_k \in S_k$.
Denota-se por
$c_j = min\big\{l_j \in \{k_1, \ldots, n\} \; \big| \; \sum_{i=k+1}^{l_j} w_i > \overline{W}(s_k)\big\}$
a posição do primeiro item que não pode ser adicionado ao estado $s_k \in S_k$
quando os itens $k+1, \ldots, n$ são rerotulados de acordo com a ordenação $\ord^j$.
Desde modo, segundo~\cite{martello1990knapsack}, quando os itens $k+1, \ldots, n$
são rerotulados de acordo com a ordenação $\ord^j$, um limite superior
para o $j$-ésimo valor objetivo de $s_k \in S_k$ é para $j = 1, \ldots, \np$:
\begin{equation}
  u^j = s_k^j + \sum_{i=k+1}^{c_j-1} p_i^j +
    max\left\{ \left\lfloor\overline{W}(s_k)\frac{p^j_{c_j+1}}{w_{c_j+1}} \right\rfloor ,
     \left\lfloor p^j_{c_j} - \big(w_{c_j} - \overline{W}(s_k)\big).\frac{p^j_{c_{j-1}}}{w_{c_j-1}}
     \right\rfloor \right\}
  \label{eq:upperb}
\end{equation}
Finalmente definimos $\relIII$ uma relação de dominância como caso particular
do tipo geral para $k = 1, \ldots, n$ por:
\begin{displaymath}
  \forall s_k, s_{\til{k}} \in S_k, \; s_k \relIII s_{\til{k}}
    \Leftrightarrow
    \begin{cases}
      s_k^j + \sum_{i \in I'} p_i^j \geq \tilde{u^i}, & j = 1, \ldots, \np \\
      \text{ou} \\
      s_k^j + \sum_{i \in I''} p_i^j \geq \tilde{u^i}, & j = 1, \ldots, \np
    \end{cases}
\end{displaymath}
onde $\tilde{u} = (\tilde{u}_1, \ldots, \tilde{u}_\np)$ é o limite superior
para $s^{\til{k}}$ computado de acordo com a Equação~\ref{eq:upperb}.

A seguinte proposição mostra que $\relIII$ é de fato uma relação de dominância.
\begin{myprop}[Relação $D_k^b$]
  \noindent
  $D_k^b$ é uma relação de dominância
  %\begin{enumerate}
  %  \item[(a)] $D_k^b$ é uma relação de dominância
  %  \item[(b)] $D_k^b$ é transitiva
  %  \item[(c)] $D_k^b$ admite ordem de preservação de dominância
  %  \item[(c)] $D_n^b = \dom$
  %\end{enumerate}
\end{myprop}

\begin{proof}
  \noindent
  %\begin{enumerate}
  %  \item[(a)]{
      Considere os estados $s_k$ e $s_{\til{k}}$ tal que $s_k \relIII s_{\til{k}}$.
      Isto implica que existe $I \in {I', I''}$ capaz de gerar um estado $s_n \in Ext(s_{\til{k}})$.
      Além disso, uma vez que $\tilde{u}$ é um limite superior de $s_{\til{k}}$,
      temos $\tilde{u} \dom s_{\til{n}}, \forall s_{\til{n}} \in Ext(s_{\til{n}})$.
      Portanto, por transitividade de $\dom$, temos que $s_n \dom s_{\til{n}}$,
      o que estabelece que $\relIII$ satisfaz a condição da Equação~\ref{eq:domrel}
      da Definição~\ref{def:domrel}.
  %    }
  %  \item[(b)]{ Considere os estados $s_k, s_{\til{k}}$ e $s_{-k}$ tal que
  %    $s_k \relIII s_{\til{k}}$ e $s_{\til{k}} \relIII s_{-k}$.
  %    Isto implica que, por um lado, existe $I_1 \in \{I', I''\}$ tal que
  %    $s_k^j + \sum_{i \in I_1} p^j_i \geq \tilde{u}_j (j = 1, \ldots, \np)$
  %    por outro lado, existe $I_2 \in \{I', I''\}$ tal que
  %    $s_{\til{k}}^j + \sum_{i \in I_2} p^j_i \geq \bar{u}_j (j = 1, \ldots, \np)$
  %    Uma vez que $\tilde{u}$ é um limite superior para $s_{\tilde{k}}$ temos que
  %    $\tilde{u}_j \geq s_{\til{k}}^i + \sum_{j \in J_2} p^j_i (i = 1, \ldots, \np)$.
  %    Portanto temos que $s_k \relIII s_{-k}$. }
    %\item[(c)]{Pela Proposição~\ref{prop:domorder}, uma vez que $\relIII$ é transitiva.}
  %  \item[(d)]{Por definição.} \qedhere
  %\end{enumerate}
\end{proof}
A relação $\relIII$ é mais difícil de ser verificada do que as relações
$\relI$ e $\relII$ uma vez que requer uma maior quantidade de
comparações e informações sobre outros estados.

Obviamente, a relação $\relIII$ seria mais forte se utilizados complementos
adicionais (de acordo com outras ordenações) para $s_k$ e computado,
ao invés de apenas um limite superior $u$, um conjunto de limites superiores,
como, por exemplo, a proposta apresentada em \cite{ehrgott2007bound}.
Porém, no contexto em questão, uma vez que se tem que verificar $\relIII$
para muitos estados, enriquecer $\relIII$ dessa forma demandaria um
esforço computacional alto demais.

%\subsection{Detalhes de implementação}

Para se ter uma implementação eficiente, são utilizadas as três relações de
dominância apresentadas a cada iteração.
Como dito anteriormente, uma relação de dominância demanda maior
esforço computacional do que outra para ser verificada.
Além disso, mesmo que sejam parcialmente complementares,
é frequente mais de uma relação de dominância ser válida para o mesmo par de estados.
Por esse motivo é natural aplicar primeiramente relações de dominância que
podem ser verificadas facilmente (como $\relI$ e $\relII$)
para depois verificar, em um conjunto reduzido de estados, relações mais
custosas (como $\relIII$).

\begin{algorithm}
  \footnotesize
  %\Kw{$C_{k-1} = \{ s_{k-1(1)}, \ldots, s_{k-1(r)}\}$
      tal que $s_{k-1(i)} \geq_{lex} s_{k-1(j)} \; \forall i < j (1 \leq i, j \leq r)$}
\Begin{
  $C_k \gets \emptyset; M_k \gets \emptyset; i \gets 1; j \gets 1$\;
  \While{$j \leq r$ \textbf{and} $s_{k-1(1)}^{\np+1} + \sum^n_{l=k}w_l \leq W$}{
    $j \gets j + 1$\;
  }
  \While{$i \leq r$ \textbf{and} $s_{k-1(1)}^{\np+1} + w_k \leq W$}{
    $s_k \gets (s_{k-1(1)}^1 + p_k^1,
      \ldots,
      s_{k-1(1)}^{\np} + p_k^{\np},
      s_{k-1(1)}^{\np+1} + w_k)$\;
    \While{$j \leq r$ \textbf{and} $s_{k-1(j)} \leq_{lex} s_k$}{
      \texttt{MantainNonDominated($s_{k-1(j)}, M_k, C_k$)}; $j \gets j + 1$\;
    }
    \texttt{MantainNonDominated($s_k, M_k, C_k$)}; $i \gets i + 1$\;
  }
  \While{$ j \leq r$}{
    \texttt{MantainNonDominated($s_{k-1(j)}, M_k, C_k$)}; $j \gets j + 1$\;
  }
  \eIf{ k = n }{
    $C_n \gets M_n$\;
  }{
    $F \gets \emptyset$\;
    \For{ $\ord \in \{\ord^{sum}, \ord^{max}\}$}{
      \For{$s_k \in M_k$}{
        Rerotular itens $k+1, \ldots, n$ segundo ordenação $\ord$\;
        $s_n \gets s_k$\;
        \For{$j \gets k + 1, \ldots, n$}{
          \If{$s_n^{\np+1} + w_j \leq W$}{
            $s_n \gets (s_n^1 + p_j^1, \ldots, s_n^{\np} + p_j^{\np}, s_n^{\np+1} + w_j)$\;
          }
        }
        $F \gets \texttt{KeepNonDominated}(s_n, F)$\;
      }
    }
    $i \gets 1; remove \gets true; F \gets \{s_{n(1)}, \ldots, s_{n(h)}\}$\;
    \While{$i \leq c$ \textbf{and} $remove$}{
      Computar limite superior $u$ para $s_{k(i)}$ conforme Equação~\ref{eq:upperb}\;
      $j \gets 1; remove \gets false$\;
      \While{$j \leq h$ \textbf{and} $s_{n(j)} \dom_{lex} u$ \textbf{and} $\neg remove$}{
        \eIf{$s_{n(j)} \dom u$}{
          $remove \gets true$\;
        }{
          $j \gets j + 1$\;
        }
      }
      \If{$remove$}{
        $C_k \gets C_k\backslash \{s_{k(i)}\}; i \gets i + 1$\;
      }
    }
  }
  \textbf{return} $C_k$\;
}

  \begin{algorithmic}[1]
    \Function{BazDP}{$\bsym{p}, \bsym{w}, W$}
    \State $S^0 = [\emptyset ]$
    \For{$k \gets 1, n$}
      \State $S^{k-1} = [\sol{s_1}, \ldots, \sol{s_p}]$
      \State $w_{left} \gets (w_{k} + \ldots + w_n)$
      \State $i \gets min\{i \;| \; \weight{s_i} \geq w_{left}\}$
        \Comment{Deficient solution avoidance}
      \State $j \gets 1$
      \While{ $\bigweight{s^{k-1}_i} < w_{left}$ }
        \State $\ldots$
        \State $i \gets i + 1$
      \EndWhile
      \State $1$
    \EndFor
  %\State $P = \{\sol{x} \:|\: \nexists \sol{a} \in S^n: \dom{a}{x} \}$
  \State \Return $P$
  \EndFunction
\end{algorithmic}

  \caption{Algoritmo Bazgan.}
  \label{alg:baz3}
\end{algorithm}

O Algoritmo~\ref{alg:baz3} apresenta o algoritmo Bazgan que aplica as três relações
de dominâncias, definidas nesta Seção, ao algoritmo~\ref{alg:baz1new}.
Primeiramente o conjunto de estados inicial $S_0$ é definido contendo apenas o estado
vazio (linha 2).
Na linha 3 é definida a ordem com que os itens serão considerados nas iterações
para serem introduzidos aos estados.
A cada execução do laço das linhas 4 a 8 é gerado o conjunto $S_k$ de estados,
contendo apenas soluções viáveis, cujos itens é um subconjunto dos primeiros $k$ itens
sendo ainda $S_k$ conjunto independente com respeito as relações de dominância
$\relI$, $\relII$ e $\relIII$.
Na linha 5 é definido o conjunto $S_k^{*}$ que consiste de
todos os estados contidos em $S_{k-1}$ adicionados do item $k$,
desde que sejam viáveis (linha 5) e também dos estados
contidos no conjunto $S_{k-1}$ desde que sua capacidade disponível não exceda a soma
dos pesos dos itens restantes (linha 6), o que corresponde à aplicação da relação $\relI$.
Na linha 7 são selecionados do conjunto $S_k^{*}$ as soluções que não tem nenhuma
outra que as dominem (aplicação da relação $\relII$).
Na linha 8 são selecionados do conjunto $S_k^{**}$ as soluções cujo limite superior não é
dominado pelo limite inferior de nenhuma outra solução (aplicação da relação $\relIII$).
Ao final do algoritmo (linha 9) é retornado o conjunto $S_n$ de estados não dominados
representando o \paretoset{}.

Como dito anteriormente o Algoritmo~\ref{alg:baz3} será utilizado para verificar
a eficácia da indexação multidimensional como proposta de aceleração da operação
de verificação de dominância no contexto exato.
Pode-se observar que a operação de verificação de dominância é utilizada
nesse caso em dois pontos:
\begin{enumerate}[itemsep=-1pt, topsep=1pt]
  \item{ verificação da condição $u \dom s$ (linha 7);}
  \item{ verificação da condição $lb(u) \dom ub(s)$ (linha 8).}
\end{enumerate}
\noindent Nesses dois pontos do algoritmo será aplicada a proposta,
permanecendo inalterado os demais pontos do algoritmo.
Os impactos na performance serão examinados através de experimentos
computacionais posteriormente apresentados e discutidos, no Capítulo~\ref{cap:exp}.

%A seguir será descrito os detalhes da implementação e aplicação
%das três relações de dominância.
%O Algoritmo~\ref{alg:baz3}, que computa, na $k$-ésima iteração, o subconjunto
%de estados candidatos $C_k$ a partir de um subconjunto
%$C_{k-1} (k = 1, \ldots, n)$, substitui as linhas 4 a 6 do Algoritmo~\ref{alg:baz1}.

%A utilização das dominâncias $\relI$ e $\relII$ é descrita nas linhas 2-11 do algoritmo
%e a utilização da dominância $\relIII$ é descrita nas linhas 12-34.
%O algoritmo utiliza dois sub-procedimentos: \texttt{MantainNonDominated},
%que remove estados $\relII$-dominados, e procedimento \texttt{KeepNonDominated},
%que é utilizado durante a aplicação da relação $\relIII$.

A Seção seguinte abordará o contexto heurístico para o MOKP,
no qual será proposta a implementação de uma heurística para o MOKP, baseada
em um algoritmo evolucionário.
Os principais algoritmos da literatura serão mencionados, e posteriormente o
algoritmo evolucionário será apresentado para então ser aplicado ao MOKP.