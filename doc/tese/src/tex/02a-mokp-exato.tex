O algoritmo de Nemhauser e Ullmann é um algorimto de programação dinâmica
que resolve problemas da mochila de forma genérica aplicando o conceito de
dominância da mochila para remover soluções parciais que não resultarão em
soluções eficientes, ou seja, soluções que irão compor o \paretoset (conjunto solução).

\begin{algorithm}
  \caption{O algoritmo de Nemhauser e Ullmann para o \mokp.}
  \label{alg:nemull}
  \Kw{$\bsym{p}, \bsym{w}, W$}
\Begin{
  \SetAlgoLined
  $S^0 = \big\{\emptyset\big\}$\;
  \For{$k \gets 1, n$}{
    $S_*^k = S^{k-1} \cup \{\sol{x} \cup k \:|\: \sol{x} \in S^{k-1}\}$\;
    TODO...\;
  }
  $P = \{\sol{x} \:|\: \nexists \sol{a} \in S^n: \dom{a}{x} \;|\; \weight{x} \leq W \}$\;
  \textbf{return} $P$\;
}
\end{algorithm}

O algoritmo inicia definindo uma solução inicial $S^0$ contendo apenas a solução
vazia (linha 2).
Na $k$-ésimo iteração o algoritmo recebe um conjunto $S^{k-1}$ contendo
soluções exclusivamente compostas pelos primeiros ${k-1}$ itens,
ou seja, $\forall\sol{x} \in S^{k-1}, \sol{x} \subseteq \{1, \ldots, k-1\}$.
O conjunto $S^{k-1}$ é então expandido adicionando-se uma cópia de cada uma
das suas soluções mas desta vez incluindo o $k$-ésimo item (linha 4),
formando o conjunto $S^k_*$, o qual possui o dobro da cardinalidade de $S^{k-1}$.
O conjunto $S^k_*$..

Apesar de sua simplicidade o Algoritmo~\ref{alg:nemull} é consideravelmente poderoso.
Contudo o potencial crescimento exponencial do \paretoset para o \mokp
compromete severamente o seu desempenho.
Uma forma de atacar este problema é tentar reduzir ainda mais a quantidade de
soluções parciais manuseadas durante as iterações do algoritmo.
Três propostas...
