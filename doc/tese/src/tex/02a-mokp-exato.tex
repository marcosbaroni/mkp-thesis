\missing{Comentar sobre background de propostas de métodos exatos.}

\missing{Comentar sobre o método da Bazgan como sendo considerado o melhor, citar melhorias propostas.}

O algoritmo de Nemhauser e Ullmann é um algorimto de programação dinâmica
que resolve problemas da mochila de forma genérica aplicando o conceito de
dominância da mochila para remover soluções parciais que não resultarão em
soluções eficientes, ou seja, soluções que irão compor o \paretoset{} (conjunto solução).

\begin{algorithm}
  \caption{O algoritmo de Nemhauser e Ullmann para o \mokp.}
  \label{alg:nemull}
  \Kw{$\bsym{p}, \bsym{w}, W$}
\Begin{
  \SetAlgoLined
  $S^0 = \big\{\emptyset\big\}$\;
  \For{$k \gets 1, n$}{
    $S_*^k = S^{k-1} \cup \{\sol{x} \cup k \:|\: \sol{x} \in S^{k-1}\}$\;
    TODO...\;
  }
  $P = \{\sol{x} \:|\: \nexists \sol{a} \in S^n: \dom{a}{x} \;|\; \weight{x} \leq W \}$\;
  \textbf{return} $P$\;
}
\end{algorithm}

O algoritmo inicia definindo uma solução inicial $S^0$ contendo apenas a solução
vazia (linha 2).
Na $k$-ésimo iteração o algoritmo recebe um conjunto $S^{k-1}$ contendo
soluções exclusivamente compostas pelos primeiros ${k-1}$ itens,
ou seja, $\forall\sol{x} \in S^{k-1}, \sol{x} \subseteq \{1, \ldots, k-1\}$.
O conjunto $S^{k-1}$ é então expandido adicionando-se uma cópia de cada uma
das suas soluções mas desta vez incluindo o $k$-ésimo item (linha 4),
formando o conjunto $S^k_*$, o qual possui o dobro da cardinalidade de $S^{k-1}$.
O conjunto $S^k_*$ é então reduzido, retirado-se todas as soluções que são dominadas
(segundo a dominância da mochila) por alguma outra (linha 5).
Após a conclusão das iterações um passo final remove as soluções inviáveis
e também as dominadas por alguma outra, dessa vez porém, considerando apenas os
valores de lucro (linha 7).

Apesar de sua simplicidade o Algoritmo~\ref{alg:nemull} é consideravelmente poderoso.
Contudo o potencial crescimento exponencial do \paretoset{} para o \mokp
compromete severamente o seu desempenho.
Uma forma de atacar este problema é tentar reduzir ainda mais a quantidade de
soluções parciais manuseadas durante as iterações do algoritmo.
\missing{ Falar sobre as 3 propostas de redução de número de soluções da bazgan
e justificar as definições seguintes. }


\missing{Explicar que o algoritmo Bazgan considera um conceito generalizado
de dominância aplicado a cada iteração.}

O processo sequencial executado pelo algoritmo de programação dinâmica
consiste de $n$ iterações.
A cada $k$-ésima iteração é gerado o conjunto de estados $S^k$,
que representa todas as soluções viáveis compostas de itens exclusivamente
pertencentes aos $k$ primeiros itens ($k = 1, \ldots, n$).
Um estado $s_k = (s_k^1, \ldots, s_\np^k, s_{\np-1}^k) \in S_k$ represena uma solução
viável que tem valor $s^i_k$ como $i$-ésimo objetivo ($i = 1, \ldots, \np$)
e $s^{\np-1}_k$ de peso.
Portanto, temos $S_k = S_{k-1} \cup \{(s^1_{k-1}+p^1_k)\}$

\missing{Comentar sobre as estratégias de redução dos conjuntos
de estados, motivando as definições a seguir.}

\missing{ Definir conjunto cobertura, conjunto independente, etc.}

\begin{mydef}[Extensão, Restrição e Complemento]
Considere o Algoritmo~\ref{alg:nemull} e qualquer estado $s_k \in S_K (k < n)$.
Um \emph{complemento} de $s_k$ é qualquer subconjunto  $J \subseteq \{k+1, \ldots, n\}$
tal que $s_k^{\np+1} + \sum_{j \in J} w_j \leq W$.
Assumiremos que qualquer estado $s_n \in S_n$ admite o conjunto vazio como único complemento.
Um estado $s_n \in S_n$ é uma \emph{extensão} de $s_k \in s_k (k \leq n )$ se, e somente se,
existe um complemento $J$ de $s_k$ tal que $s_n^i = s_k^i + \sum_{j \in J} p_j^i) para \; i = 1, \ldots, \np$
e $s_n^{p+1} = s_k^{p+1} + \sum_{j \in J} w_j$.
O conjunto de extenções de $s_k$ é denotado por $Ext(s_k) (k \leq n)$.
Um estado $s_k \in S_k (k \leq n)$ é uma \emph{restrição} do estado $s_n \in S_n$
se, e somente se, $s_n$ é uma extenão de $s_k$.
\end{mydef}

\missing{Introdução às relações de dominância?}

\begin{mydef}[Relação de dominância entre soluções]
Uma relação $\relk$ sobre $S_k, k = i,\ ldots, n$, é uma relação de dominância
se, e somente se, para todo $s_k, s_{k'} \in S_k$,
\begin{equation}
  \relkargs{s_k}{s_{\til{k}}} \Rightarrow \forall s_{\til{n}} \in Ext(s_{\til{n}}),
    \exists s_n \in Ext(s_k), \dom{s_n}{s_{\til{n}}}
\end{equation}
\end{mydef}

Apesar das relações de dominância não serem transitivas por definição,
costumam ser transitivas por construção, como é o caso
das três relações de dominância da Seção~\ref{sec:domrel}.
Vale notar que se $\relk^i, \; i = 1, \ldots, \nrel$ são relações de dominância
então $\relk = \bigcup^{\nrel}_{i = 1} \relk^i$ é também uma relação
de dominância, geralmente não transitiva mesmo se $\relk^i, \; i = 1, \ldots, \nrel$
forem transitivas.

Para se ter uma implementação eficiência do algoritmo de programação dinâmica
é recomendável utilizar múltiplas relações de dominância
$\relk^1, \ldots, \relk^\nrel (\nrel \leq 1)$ a cada execução da $k$-ésima iteração
$k = 1, \ldots, n$ uma vez que cada relação $\relk^i$ explora características
específicas.

\begin{algorithm}
  \caption{Algoritmo de programação dinâmica utilizando múltiplas relações de dominância.}
  \label{alg:baz1}
  \begin{algorithmic}[1] % The number tells where the line numbering should start
    \Function{DP}{}
    \State $C^0 \gets \{(0, \ldots, 0)\}$;
    \For{$k \gets 1, n$}
      \State $C_k^0 \gets C_{k-1} \cup \big\{(s_{k-1}^1 + p_k^1, \ldots, s_{k-1}^\np + p_k^\np, s_{k-1}^{\np+1} + w_k)
        \; | \; s_{k-1}^{\np+1} + w_k \leq W, s_{k-1} \in C_{k-1} \big\};$
      \For{$i \gets 1, \nrel$}
        \State {Determinar um conjunto $C_k^i$ cobertura do conjunto $C_{k-1}^i$
          com respeito a $\relk^i$};
      \EndFor
      \State {$C_k \gets C_k^\nrel$};
    \EndFor
  \State \Return $C_n$;
  \EndFunction
\end{algorithmic}

\end{algorithm}

\begin{myprop}
  Para quaisquer relações de dominância $\relk^1, \ldots, \relk^\nrel (\nrel \leq 1)$
  sobre $S_k$, o conjunto $C_k^\nrel$ obtido pelo Algoritmo~\ref{alg:baz1}
  em cada iteração é uma cobertura de $C_k^0$ com respeito a
  $\relk = \bigcup_{i=1}^\nrel \relk^i \; (k = 1, \ldots, n)$.
\end{myprop}

\begin{proof}
Considere $s_k \in C_k^0\backslash C_k^\nrel$, este foi removido quando
selecionado um conjunto cobertura na iteração da linha 6.
Seja $i_1 \in \{1, \ldots, \nrel \}$ a iteração da linha 6, tal que
$s_k \in C_k^{i_1 - 1}\backslash C_k^{i_1}$.
Uma vez que $C_k^{i_1}$ é um conjunto cobertura de $C_k^{i_1-1}$ com respeito
a $\relk^{i-1}$, existe $s_\til{k}^{(1)} \in C_k^{i_1}$ tal que...
\end{proof}

\subsection{As relações de dominância}
\missing{Relações de dominância}
\label{sec:domrel}
