\missing{Comentar sobre background de propostas de métodos exatos.}

\missing{Comentar sobre o método da Bazgan como sendo considerado o melhor, citar melhorias propostas.}

\begin{algorithm}
  \caption{O algoritmo de Nemhauser e Ullmann para o \mokp.}
  \label{alg:dp1}
  \Kw{$\bsym{p}, \bsym{w}, W$}
\Begin{
  \SetAlgoLined
  $S_0 = \{(0, \ldots, 0)\}$\;
  \For{$k \gets 1, n$}{
    $S_k \gets S_{k-1} \cup \big\{(s^1 + p_k^1, \ldots, s^\np + p_k^\np, s^{\np+1} + w_k)\;$\
    $\phantom{S_k \gets S_{k-1} \cup \;} \big| \; s^{\np+1} + w_k \leq W, \: s \in S_{k-1} \big\}$\;
  }
  $P = \{ s \in S_n \:|\: \nexists (a \in S_n \;|\; a \dom s)\}$\;
  \textbf{return} $P$\;
}

\end{algorithm}

\missing{Explicar que o algoritmo Bazgan considera um conceito generalizado
de dominância aplicado a cada iteração.}

\missing{Dizer que a explicação do desenvolvimento do algoritmo será reproduzido
segundo o artigo original.}

O processo sequencial executado pelo algoritmo de programação dinâmica
consiste de $n$ iterações.
A cada $k$-ésima iteração é gerado o conjunto de estados $S^k$,
que representa todas as soluções viáveis compostas de itens exclusivamente
pertencentes aos $k$ primeiros itens ($k = 1, \ldots, n$).
Um estado $s_k = (s_k^1, \ldots, s_\np^k, s_{\np-1}^k) \in S_k$ represena uma solução
viável que tem valor $s^i_k$ como $i$-ésimo objetivo ($i = 1, \ldots, \np$)
e $s^{\np-1}_k$ de peso.
Portanto, temos $S_k = S_{k-1} \cup \{(s^1_{k-1}+p^1_k)\}$

\missing{Comentar sobre as estratégias de redução dos conjuntos
de estados, motivando as definições a seguir.}

\missing{ Definir conjunto cobertura, conjunto independente,
  conjunto eficiente reduzido, etc., definir parte simétrica/assimétrica de $\dom$?}

\begin{mydef}[Extensão, Restrição e Complemento]
Considere o Algoritmo~\ref{alg:dp1} e qualquer estado $s_k \in S_K (k < n)$.
Um \emph{complemento} de $s_k$ é qualquer subconjunto  $J \subseteq \{k+1, \ldots, n\}$
tal que $s_k^{\np+1} + \sum_{j \in J} w_j \leq W$.
Assumiremos que qualquer estado $s_n \in S_n$ admite o conjunto vazio como único complemento.
Um estado $s_n \in S_n$ é uma \emph{extensão} de $s_k \in s_k (k \leq n )$ se, e somente se,
existe um complemento $J$ de $s_k$ tal que $s_n^i = s_k^i + \sum_{j \in J} p_j^i) para \; i = 1, \ldots, \np$
e $s_n^{p+1} = s_k^{p+1} + \sum_{j \in J} w_j$.
O conjunto de extenções de $s_k$ é denotado por $Ext(s_k) (k \leq n)$.
Um estado $s_k \in S_k (k \leq n)$ é uma \emph{restrição} do estado $s_n \in S_n$
se, e somente se, $s_n$ é uma extenão de $s_k$.
\end{mydef}

\subsection{As relações de dominância}
\label{sec:domrel}

\missing{Introdução às relações de dominância?}

\begin{mydef}[Relação de dominância entre soluções]
Uma relação $\rel_k$ sobre $S_k, k = i,\ ldots, n$, é uma relação de dominância
se, e somente se, para todo $s_k, s_{k'} \in S_k$,
\begin{equation}
  s_k \rel_k s_{\til{k}} \Rightarrow \forall s_{\til{n}} \in Ext(s_{\til{n}}),
    \exists s_n \in Ext(s_k), s_n \dom s_{\til{n}}
  \label{eq:domrel}
\end{equation}
\label{def:domrel}
\end{mydef}

Apesar das relações de dominância não serem transitivas por definição,
costumam ser transitivas por construção, como é o caso
das três relações de dominância da Seção~\ref{sec:domrel}.
Vale notar que se $\rel_k^i, \; i = 1, \ldots, \nrel$ são relações de dominância
então $\rel_k = \bigcup^{\nrel}_{i = 1} \rel_k^i$ é também uma relação
de dominância, geralmente não transitiva mesmo se $\rel_k^i, \; i = 1, \ldots, \nrel$
forem transitivas.

Para se ter uma implementação eficiência do algoritmo de programação dinâmica
é recomendável utilizar múltiplas relações de dominância
$\rel_k^1, \ldots, \rel_k^\nrel (\nrel \leq 1)$ a cada execução da $k$-ésima iteração
$k = 1, \ldots, n$ uma vez que cada relação $\rel_k^i$ explora características
específicas.

\begin{algorithm}
  \caption{Algoritmo de programação dinâmica utilizando múltiplas relações de dominância.}
  \label{alg:baz1}
  \Begin{
  $C_0 \gets \{(0, \ldots, 0)\}$\;
  \For{$k \gets 1, n$}{
    $C_k^0 \gets C_{k-1} \cup \big\{(s^1 + p_k^1, \ldots, s^\np + p_k^\np, s^{\np+1} + w_k)
      \; | \; s^{\np+1} + w_k \leq W, s \in C_{k-1} \big\}$\;
    \For{$i \gets 1, \nrel$}{
      Determinar conjunto $C_k^i$ cobertura do conjunto $C_{k-1}^i$
        com respeito a $\relk^i$\;
    }
    $C_k \gets C_k^\nrel$\;
  }
  \textbf{return} $C_n$\;
}
\end{algorithm}

\begin{myprop}
  Para quaisquer relações de dominância $\rel_k^1, \ldots, \rel_k^\nrel (\nrel \geq 1)$
  sobre $S_k$, o conjunto $C_k^\nrel$ obtido pelo Algoritmo~\ref{alg:baz1}
  em cada iteração é um conjunto cobertura de $C_k^0$ com respeito a
  $\rel_k = \bigcup_{i=1}^\nrel \rel_k^i \; (k = 1, \ldots, n)$.
  \label{prop:coverset}
\end{myprop}

\begin{proof}
Considere $s_k \in C_k^0\backslash C_k^\nrel$, este foi removido quando
selecionado um conjunto cobertura na iteração da linha 5.
Seja $i_1 \in \{1, \ldots, \nrel \}$ a iteração da linha 5, tal que
$s_k \in C_k^{i_1 - 1}\backslash C_k^{i_1}$.
Uma vez que $C_k^{i_1}$ é um conjunto cobertura de $C_k^{i_1-1}$ com respeito
a $\rel^{i_1}_{\til{k}}$, existe $s_{\til{k}}^{(1)} \in C_k^{i_1}$ tal que $ s_{\til{k}}^{(1)} \rel_k^{i_1} {s_k}$.
Se $s_{\til{k}}^{(1)} \in C_k^m$ então as propriedades de cobertura são válidas,
uma vez que $\rel_k^{i_1} \subseteq \rel_k$.
Caso contrário, existe uma iteração $i_2 > i_1$, correspondente à iteração
da linha 5 tal que $s_{\til{k}}^{(1)} \in C_k^{i_2-1}\backslash C_k^{i_2}$.
Como anteriormente, estabelecemos que existe  $s_{\til{k}}^{(2)} \in C_k^{i_2}$
tal que $s_{\til{k}}^{(2)} \rel_k^{i_2} s_{\til{k}}^{(1)}$.
Uma vez que $\rel_k^{i_2} \subseteq \rel_k$ temos que
$s_{\til{k}}^{(2)} \rel_k s_{\til{k}}^{(1)} \rel_k s_k$ e por transitividade de $\rel_k$
garantimos que $s_{\til{k}}^{(2)} \rel_k s_k$.
Pela repetição desde processo podemos garantir que a existência de um estado
$s_{\til{k}} \in C_k^m$, tal que $s_{\til{k}} \rel_k s_k$.
\qedhere
\end{proof}

Temos agora condições para garantir que o Algoritmo~\ref{alg:baz1} gera o conjunto de
vetores objetivos não dominados.

\begin{myteo}
  Para qualquer família de relações de dominância
  $\rel_k^i (i = 1, \ldots, \nrel; k = 1, \ldots, n)$,
  o Algoritmo~\ref{alg:baz1} retorna $C_n$ que é um conjunto cobertura de $S_n$
  com respeito a $\dom$.
  Além disso, se na iteração $n$ utilizarmos ao menos uma relação
  $\rel_n^i = \dom$ e garantirmos que o conjunto cobertura $C_n^i$
  é também independente com respeito a $\rel_n^i$ então $C_n$ representa
  o conjunto $ND$ de vetores objetivos não dominados.
\end{myteo}

\begin{proof}
  Considere $s_{\til{n}} \in S_n\backslash C_n$, todas as restrições
  foram removidas ao reter-se o conjunto cobertura com respeito a
  $\rel_k = \cup_{i=1}^m \rel_k^i$ durante as iterações $k \leq n$.
  Seja $k_1$ a iteração mais alta em que $C_{k_1}^0$ ainda
  contém restrições de $s_{\til{n}}$, as quais serão removidas
  aplicando uma das relações $\rel_{k_1}^i (i = 1, \ldots, \nrel)$.
  Considere alguma destas restrições, denotada por $s_{\til{k_1}}^{(n)}$.
  Uma vez que $s_{\til{k_1}}^{n} \in C^0_{k_1}\backslash C_{k_1}$ sabemos,
  pela Proposição~\ref{prop:coverset}, que existe $s_{k_1} \in C_{k_1}$
  tak que $s_{k_1} \rel_{k_1} s_{\til{k_1}}^{(n)}$.
  Pela Equação~\ref{eq:domrel}, uma vez que $\rel_{k}$ é uma relação de
  dominância, temos que todas as extensões de $s_{k_1}^{(n)}$, e
  particularmente para $s_{\til{n}}$, existe $s_{n_1} \in Ext(s_{k_1})$
  tal que $s_{n_1} \dom s_{\til{n}}$.
  Se $s_{n_1} \in C_n$, então a propriedade de cobertura é válida.
  Caso contrário, existe uma iteração $k_2 > k_1$, correspondente
  a iteração mais alta em que $C_0^{k_2}$ ainda contém restrições de $s_{n_1}$,
  que serão removidas aplicando-se uma das relações $\rel_{k_2}^i (i = 1, \ldots, \nrel)$.
  Considere uma destas restrições, denotada por $s_{k_2}^{(n_1)}$.
  Como anteriormente, estabelecemos a existência de um estado $s_{k_2} \in C_{k_2}$
  tal que existe $s_{n_2} \in Ext(s_{k_2})$ tal que $s_{n_2} \dom s_{n_1}$.
  A transitividade de $\dom$ garante que $s_{n_2} \dom s_{\til{n}}$.
  Pela repetição deste processo, estabelecemos a existência de um estado
  $s_{k_2} \in C_{k_2}$ tal que existe $s_{n_2} \in Ext(s_{k_2})$
  tal que $s_n \dom s_{\til{n}}$.

  Além disso, selecionando-se um conjunto $C_n^i$ que é conjunto independente
  com relação a $\rel_n^i = \dom$, esta propriedade se mantém válida para $C_n^\nrel$
  o qual é subconjunto de $C_n^i$.
  Dessa forma $C_n$, que corresponde ao conjunto eficiente reduzido,
  representa o conjunto de vetores objetivos não dominados.
  \qedhere
\end{proof}

O teorema anterior apenas querer que um dos $n . \nrel$ conjuntos coberturas
seja independente com respeito a sua relação de dominância.
Mesmo se todos os conjunto $C_k^i$....

\subsection{Geração de conjuntos cobertura e independente}

Apresentamos no Algoritmo~\ref{alg:baz2} uma maneira de produzir $C_k^i$ um
conjunto cobertura e independente de $C_k^{i-1}$ com respeito a relação
transitiva $\rel_k^i$ (linha ??? do Algoritmo~\ref{alg:baz1}).

\begin{algorithm}
  \caption{Computação do conjunto $C_k^i$ cobertura e independente do conjunto
  $C_k^{i-1}$ com respeito a relação transitiva $\rel_k^i$.}
  \label{alg:baz2}
  \begin{algorithmic}[1] % The number tells where the line numbering should start
    \Function{}{$C_k^{i-1} = \{ s_{k(1)}, \ldots, s_{k(r)}\}$}
    \State $C_k^i \gets \{s_{k(1)}\}$;
    \For{$h \gets 2, r$}
      \State $C_k^i = \{s_{\til{k}(1)}, \ldots, s_{\til{k}(l_h)}\}$;
      \State $dominated \gets false;$
      \State $dominates \gets false;$
      \State $j \gets 1;$
      \While{$ j \leq l_h$ \textbf{and} $\neg dominated$ \textbf{and} $\neg dominates$}
        \If{$s_{\til{k}(j)} \rel_k^i s_{k(h)}$}
          \State $dominated \gets true;$
        \ElsIf{$s_{k(h)} \rel_k^i s_{\til{k}(j)}$}
          \State $C_k^i \gets C_k^i\backslash\{s_{\til{k}(j)}\};$
          \State $dominates \gets true;$
        \EndIf
        \State $j \gets j + 1;$
      \EndWhile
      \If{$\neg dominated$}
        \While{$j \leq l_h$}
          \If{$s_{k(h)} \rel_k^i s_{\til{k}(j)}$}
            \State $C_k^i \gets C_k^i \backslash \{s_{\til{k}(j)}\};$
          \EndIf
          \State $j \gets j + 1;$
        \EndWhile
        \State $C_k^i \gets C_k^i \cup \{s_{k(h)}\};$
      \EndIf
    \EndFor
  \State \Return $C_k$
  \EndFunction
\end{algorithmic}

\end{algorithm}

\begin{myprop}
  Para qualquer relação de dominância $\rel_k^i$ sobre $S_k$, o
  Algoritmo~\ref{alg:baz2} retorna $C_k^i$ um conjunto cobertura e independente
  de $C_k^{i-1}$ com respeito a $\rel_k^i (j = 1, \ldots, n; i = 1, \ldots, \nrel)$.
\end{myprop}

\begin{proof}
  Claramente $C_k^i$ é independente com relação a $\rel_k^i$, uma vez que
  inserimos o estado $s_k$ ao conjunto $C_k^i$ na linha ????
  apenas se não for dominado por nenhum outro estado em $C_k^i$ (linha ???)
  e todos os estados dominados por $s_k$ tenham sido removidos de $C_k^i$
  (linha ??? e ???).

  Mostramos agora que $C_k^i$ é um conjunto covertura de $C_k^{i-1}$
  com respeito a $\rel_k^i$.
  Considere $s_{\til{k}} \in C_k^{i-1} \backslash C_k^i$.
  Isto ocorre tanto porque não passa no teste da linha ??? quanto porque
  foi removido na linha ??? ou ???.
  Isto é devido respectivamente a um estado $s_{-k}$ já existente em $C_k^i$
  ou a ser incluído em $C_k^i$ (na linha ???) tal que $s_{-k}\rel_k^i s_{\til{k}}$.
  Pode ser que $s_{-k}$ seja removido de $C_k^i$ em uma iteração posterior
  da linha ??? se existir um novo estado $s^{\hat k} \in C_k^{i-1}$ a ser
  incluído em $C_k^i$, tal que $s_{\hat k} \rel_k^i s_{-k}$.
  Entretanto, a transitividade de $\rel_k^i$ garante a existência, ao fim da
  iteração $k$, de um estado $s_k \in C_k^i$ tal que $s_k \rel_k^i s_{\til{k}}$.
  \qedhere
\end{proof}

O Algoritmo~\ref{alg:baz2} pode ser aprimorado uma vez que geralmente é possível
gerar estados de $C_k^{i-1} = \{s_{k(1)}, \ldots, s_{k(r)}\}$ conforme
uma \emph{ordem de preservação de dominância} para $\rel_k^i$ de forma que
todo $l < j (1 \leq l, j \leq r)$ temos tanto $s_{k(l)} \rel_k^i s_{k(j)}$
ou $\neg (s_{k(j)} \rel_k^i s_{k(l)})$.
A proposição seguinte provê a condição necessária e suficiente para estabelecer
a existência da ordem de preservação de dominância para uma relação de dominância.

\begin{myprop}
  Seja $D_k$ uma relação de dominância sobre $S_k$.
  Existe uma ordem de preservação de dominância para $D_k$ se, e somente se,
  $\rel_k$ não admite ciclos em sua parte assimétrica.
  \label{prop:domorder}
\end{myprop}

\begin{proof}
  \noindent \\ $\Rightarrow$ A existência de um ciclo na parte assimétrica de $D^k$
  implicaria na existência de dois estados consecutivos $s_{k(j)}$ e $s_{k(l)}$
  neste ciclo sendo $j > l$, o que é uma contradição.\\
  $\Leftarrow$ Qualquer ordem topológica baseada na parte assimétrica de $D_k$
  é uma ordem de preservação de dominância para $D_k$.
  \qedhere
\end{proof}

Na Seção seguinte é apresentado um exemplo de ordem de preservação de dominância.
Se os estados em $C_k^{i-1}$ são gerados conforme uma ordem de preservação de
dominância para $D_k^i$, a linha ??? e o laço ???-??? do Algoritmo~\ref{alg:baz2}
podem ser omitidos.

\section{Detalhes de implementação}

Primeiramente é apresentada a ordem na qual serão considerados os itens no
processo sequencial do algoritmo.
Posteriormente são apresentados a três relações de dominância utilizadas
no algoritmo ??? e a maneira com que serão utilizadas.

\subsection{Ordem dos itens}

A ordem na qual os itens são considerados é uma questão crucial na implementação
do algoritmo.
Sabe-se que no caso do problema da mochila unidimensional, para se
obter uma boa solução, os itens devem ser geralmente considerados em
ordem decrescente segundo a proporção
$p_i/w_i$~\cite{ehrgott2013multicriteria, martello1990knapsack}.
Porém, para o caso multi-objetivo não existe uma ordem natural.

São introduzidas duas ordenações $\ord^{sum}$ e $\ord^{max}$
derivadas da agregação das ordens $\ord^j$ inferidas pelas proporções
$p_i^j/w_i$ para cada objetivo $(j = 1, \ldots, \np)$.
Considere $r_i^l$ o rank (ou posição) do item $l$ na ordenação $\ord^j$.
$\ord^{sum}$ denota uma ordenação segundo valores crescentes da soma dos
ranks dos itens nas $\np$ ordenações $(i = 1, \ldots, \np)$.
$\ord^{max}$ denota uma ordenação segundo valores crescentes de máximo ou
piores ranks de itens nas $\np$ ordenações $\ord^j (j = 1, \ldots, \np)$,
onde o pior rank do item $l$ nas $\np$ ordenações $\ord^j (j = 1, \ldots, \np)$
é calculado como $max_{i=1}^{\np}\{r_l^i\} + \frac{1}{\np n} \sum_{i = 1}^{\np} r_l^i$
para distinguir itens com o mesmo rank máximo.

\missing{Dizer que o artigo original conclui atraves de testes qual é a melhor
ordenação para ser utilizada nas iterações do algoritmo.}

\subsection{Relações de dominância}

Cada relação de dominância exploca uma consideração específica.
Por isto recomenda-se utilizar relações de dominância que sejam complementares.
Além disso, para se escolher uma relação de dominância é necessário considerar
sua potencial capacidade de descarte de estados diante do
custo computacional requerido.

A seguir serão apresentas as três relações de dominância utilizadas no algoritmo.
As primeiras duas relações são reazoavelmente triviais de se estabelecer sendo
a última ainda considerada, mesmo sendo um tanto mais complexa, devido à sua
complementaridade diantes das duas primeiras.

A primeira relação de dominância se estabelece segundo a seguinte observação.
Quando a capacidade residual de uma solução associada a um estado $s_k$
da iteração $k$ é maior ou igual a soma dos pesos dos itens restantes,
i.e. itens $k+1, \ldots, n$, o único complemento de $s_k$ que pode resultar
em uma solução eficiente é o complemento máximo $J = \{k+1, \ldots, n\}$.
Portanto, neste contexto, não se faz necessário gerar as extensões de $s_k$
que não contenham todos os itens restants.
Define-se então a relação de dominância $\rel_k^r$ sobre $S^k$ para
$k = 1, \ldots, n$ como:
\begin{displaymath}
  \forall s_k, s_{\til{k}} \in S_k, \enskip
  s_k \rel_k^r s_{\til{k}}
    \Leftrightarrow
    \begin{cases}
      s_{\til{k}} \in S_{k-1}, \\
      s_k = (s_{\til{k}}^1 + p_k^1, \ldots, s_{\til{k}}^\np + p_k^\np, s_{\til{k}}^{\np+1} + w_k), \; \text{e} \\
      s_{\til{k}}^{\np+1} \leq W - \sum_{i=k}^n w_i
    \end{cases}
\end{displaymath}
A seguinte proposição mostra que $\rel_k^r$ é de fato uma relação de dominância
e apresenta propriedades adicionais de $\rel_k^r$.

\begin{myprop}[Relação $D_k^r$]
  \noindent
  \begin{enumerate}
    \item[(a)] $D_k^r$ é uma relação de dominância
    \item[(b)] $D_k^r$ é transitiva
    \item[(c)] $D_k^r$ admite ordem de preservação de dominância
\end{enumerate}
\end{myprop}

\begin{proof}
  \noindent
  \begin{enumerate}
    \item[(a)]{
      Considere dois estados $s_k$ e $s_{\til{k}}$ tal que $s_k \rel_k^r s_{\til{k}}$.
      Isto implica que $s_k \dom s_{\til{k}}$.
      Além disso, uma vez que
      $s_k^{\np+1} = s_{\til{k}^{\np+1}} + w_k \leq W - \sum_{i=k+1}^n w_i$,
      qualquer subconjunto $J \subseteq \{k+1, \ldots, n\}$ é um complemento
      de $s_{\til{k}}$ e $s_k$.
      Portanto, para todo estado $s_{\til{n}} \in Ext(s_{\til{k}})$,
      existe $s_n \in Ext(s_k)$, baseado no mesmo complemento que $s_{\til{n}}$,
      tal que $s_n \dom s_{\til{n}}$.
      Isto estabelece que $\rel_k^r$ satisfaz a Equação~\ref{eq:domrel} da
      Definição~\ref{def:domrel}. }
    \item[(b)]{Trivial.}
    \item[(c)]{Pela Proposição~\ref{prop:domorder}, uma vez que $\rel_k^r$ é transitiva.}
  \end{enumerate}
\end{proof}
