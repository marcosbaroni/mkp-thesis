\missing{Comentar sobre background de propostas de métodos exatos.}

\missing{Comentar sobre o método da Bazgan como sendo considerado o melhor, citar melhorias propostas.}

\begin{algorithm}
  \caption{O algoritmo de Nemhauser e Ullmann para o \mokp.}
  \label{alg:dp1}
  \Kw{$\bsym{p}, \bsym{w}, W$}
\Begin{
  \SetAlgoLined
  $S^0 = \big\{\emptyset\big\}$\;
  \For{$k \gets 1, n$}{
    $S_*^k = S^{k-1} \cup \{\sol{x} \cup k \:|\: \sol{x} \in S^{k-1}\}$\;
    TODO...\;
  }
  $P = \{\sol{x} \:|\: \nexists \sol{a} \in S^n: \dom{a}{x} \;|\; \weight{x} \leq W \}$\;
  \textbf{return} $P$\;
}
\end{algorithm}

\missing{Explicar que o algoritmo Bazgan considera um conceito generalizado
de dominância aplicado a cada iteração.}

\missing{Dizer que a explicação do desenvolvimento do algoritmo será reproduzido
segundo o artigo original.}

O processo sequencial executado pelo algoritmo de programação dinâmica
consiste de $n$ iterações.
A cada $k$-ésima iteração é gerado o conjunto de estados $S^k$,
que representa todas as soluções viáveis compostas de itens exclusivamente
pertencentes aos $k$ primeiros itens ($k = 1, \ldots, n$).
Um estado $s_k = (s_k^1, \ldots, s_\np^k, s_{\np-1}^k) \in S_k$ represena uma solução
viável que tem valor $s^i_k$ como $i$-ésimo objetivo ($i = 1, \ldots, \np$)
e $s^{\np-1}_k$ de peso.
Portanto, temos $S_k = S_{k-1} \cup \{(s^1_{k-1}+p^1_k)\}$

\missing{Comentar sobre as estratégias de redução dos conjuntos
de estados, motivando as definições a seguir.}

\missing{ Definir conjunto cobertura, conjunto independente,
  conjunto eficiente reduzido, etc., definir parte simétrica/assimétrica de $\dom$?}

\begin{mydef}[Extensão, Restrição e Complemento]
Considere o Algoritmo~\ref{alg:dp1} e qualquer estado $s_k \in S_K (k < n)$.
Um \emph{complemento} de $s_k$ é qualquer subconjunto  $J \subseteq \{k+1, \ldots, n\}$
tal que $s_k^{\np+1} + \sum_{j \in J} w_j \leq W$.
Assumiremos que qualquer estado $s_n \in S_n$ admite o conjunto vazio como único complemento.
Um estado $s_n \in S_n$ é uma \emph{extensão} de $s_k \in s_k (k \leq n )$ se, e somente se,
existe um complemento $I$ de $s_k$ tal que $s_n^j = s_k^j + \sum_{i \in I} p_i^j) para \; j = 1, \ldots, \np$
e $s_n^{p+1} = s_k^{p+1} + \sum_{i \in I} w_i$.
O conjunto de extenções de $s_k$ é denotado por $Ext(s_k) (k \leq n)$.
Um estado $s_k \in S_k (k \leq n)$ é uma \emph{restrição} do estado $s_n \in S_n$
se, e somente se, $s_n$ é uma extensão de $s_k$.
\end{mydef}

\subsection{As relações de dominância}
\label{sec:domrel}

\missing{Introdução às relações de dominância?}

\begin{mydef}[Relação de dominância entre soluções]
Uma relação $\rel_k$ sobre $S_k, k = i,\ ldots, n$, é uma relação de dominância
se, e somente se, para todo $s_k, s_{k'} \in S_k$,
\begin{equation}
  s_k \rel_k s_{\til{k}} \Rightarrow \forall s_{\til{n}} \in Ext(s_{\til{n}}),
    \exists s_n \in Ext(s_k), s_n \dom s_{\til{n}}
  \label{eq:domrel}
\end{equation}
\label{def:domrel}
\end{mydef}

Apesar das relações de dominância não serem transitivas por definição,
costumam ser transitivas por construção, como é o caso
das três relações de dominância da Seção~\ref{sec:domrel}.
Vale notar que se $\rel_k^i, \; i = 1, \ldots, \nrel$ são relações de dominância
então $\rel_k = \bigcup^{\nrel}_{i = 1} \rel_k^i$ é também uma relação
de dominância, geralmente não transitiva mesmo se $\rel_k^i, \; i = 1, \ldots, \nrel$
forem transitivas.

Para se ter uma implementação eficiência do algoritmo de programação dinâmica
é recomendável utilizar múltiplas relações de dominância
$\rel_k^1, \ldots, \rel_k^\nrel (\nrel \leq 1)$ a cada execução da $k$-ésima iteração
$k = 1, \ldots, n$ uma vez que cada relação $\rel_k^i$ explora características
específicas.

\begin{algorithm}
  \caption{Algoritmo de programação dinâmica utilizando múltiplas relações de dominância.}
  \label{alg:baz1}
  \begin{algorithmic}[1] % The number tells where the line numbering should start
    \Function{DP}{}
    \State $C^0 \gets \{(0, \ldots, 0)\}$;
    \For{$k \gets 1, n$}
      \State $C_k^0 \gets C_{k-1} \cup \big\{(s_{k-1}^1 + p_k^1, \ldots, s_{k-1}^\np + p_k^\np, s_{k-1}^{\np+1} + w_k)
        \; | \; s_{k-1}^{\np+1} + w_k \leq W, s_{k-1} \in C_{k-1} \big\};$
      \For{$i \gets 1, \nrel$}
        \State {Determinar um conjunto $C_k^i$ cobertura do conjunto $C_{k-1}^i$
          com respeito a $\relk^i$};
      \EndFor
      \State {$C_k \gets C_k^\nrel$};
    \EndFor
  \State \Return $C_n$;
  \EndFunction
\end{algorithmic}

\end{algorithm}

\begin{myprop}
  Para quaisquer relações de dominância $\rel_k^1, \ldots, \rel_k^\nrel (\nrel \geq 1)$
  sobre $S_k$, o conjunto $C_k^\nrel$ obtido pelo Algoritmo~\ref{alg:baz1}
  em cada iteração é um conjunto cobertura de $C_k^0$ com respeito a
  $\rel_k = \bigcup_{i=1}^\nrel \rel_k^i \; (k = 1, \ldots, n)$.
  \label{prop:coverset}
\end{myprop}

\begin{proof}
Considere $s_k \in C_k^0\backslash C_k^\nrel$, este foi removido quando
selecionado um conjunto cobertura na iteração da linha 5.
Seja $i_1 \in \{1, \ldots, \nrel \}$ a iteração da linha 5, tal que
$s_k \in C_k^{i_1 - 1}\backslash C_k^{i_1}$.
Uma vez que $C_k^{i_1}$ é um conjunto cobertura de $C_k^{i_1-1}$ com respeito
a $\rel^{i_1}_{\til{k}}$, existe $s_{\til{k}}^{(1)} \in C_k^{i_1}$ tal que $ s_{\til{k}}^{(1)} \rel_k^{i_1} {s_k}$.
Se $s_{\til{k}}^{(1)} \in C_k^m$ então as propriedades de cobertura são válidas,
uma vez que $\rel_k^{i_1} \subseteq \rel_k$.
Caso contrário, existe uma iteração $i_2 > i_1$, correspondente à iteração
da linha 5 tal que $s_{\til{k}}^{(1)} \in C_k^{i_2-1}\backslash C_k^{i_2}$.
Como anteriormente, estabelecemos que existe  $s_{\til{k}}^{(2)} \in C_k^{i_2}$
tal que $s_{\til{k}}^{(2)} \rel_k^{i_2} s_{\til{k}}^{(1)}$.
Uma vez que $\rel_k^{i_2} \subseteq \rel_k$ temos que
$s_{\til{k}}^{(2)} \rel_k s_{\til{k}}^{(1)} \rel_k s_k$ e por transitividade de $\rel_k$
garantimos que $s_{\til{k}}^{(2)} \rel_k s_k$.
Pela repetição desde processo podemos garantir que a existência de um estado
$s_{\til{k}} \in C_k^m$, tal que $s_{\til{k}} \rel_k s_k$.
\qedhere
\end{proof}

Temos agora condições para garantir que o Algoritmo~\ref{alg:baz1} gera o conjunto de
vetores objetivos não dominados.

\begin{myteo}
  Para qualquer família de relações de dominância
  $\rel_k^i (i = 1, \ldots, \nrel; k = 1, \ldots, n)$,
  o Algoritmo~\ref{alg:baz1} retorna $C_n$ que é um conjunto cobertura de $S_n$
  com respeito a $\dom$.
  Além disso, se na iteração $n$ utilizarmos ao menos uma relação
  $\rel_n^i = \dom$ e garantirmos que o conjunto cobertura $C_n^i$
  é também independente com respeito a $\rel_n^i$ então $C_n$ representa
  o conjunto $ND$ de vetores objetivos não dominados.
\end{myteo}

\begin{proof}
  Considere $s_{\til{n}} \in S_n\backslash C_n$, todas as restrições
  foram removidas ao reter-se o conjunto cobertura com respeito a
  $\rel_k = \cup_{i=1}^m \rel_k^i$ durante as iterações $k \leq n$.
  Seja $k_1$ a iteração mais alta em que $C_{k_1}^0$ ainda
  contém restrições de $s_{\til{n}}$, as quais serão removidas
  aplicando uma das relações $\rel_{k_1}^i (i = 1, \ldots, \nrel)$.
  Considere alguma destas restrições, denotada por $s_{\til{k_1}}^{(n)}$.
  Uma vez que $s_{\til{k_1}}^{n} \in C^0_{k_1}\backslash C_{k_1}$ sabemos,
  pela Proposição~\ref{prop:coverset}, que existe $s_{k_1} \in C_{k_1}$
  tak que $s_{k_1} \rel_{k_1} s_{\til{k_1}}^{(n)}$.
  Pela Equação~\ref{eq:domrel}, uma vez que $\rel_{k}$ é uma relação de
  dominância, temos que todas as extensões de $s_{k_1}^{(n)}$, e
  particularmente para $s_{\til{n}}$, existe $s_{n_1} \in Ext(s_{k_1})$
  tal que $s_{n_1} \dom s_{\til{n}}$.
  Se $s_{n_1} \in C_n$, então a propriedade de cobertura é válida.
  Caso contrário, existe uma iteração $k_2 > k_1$, correspondente
  a iteração mais alta em que $C_0^{k_2}$ ainda contém restrições de $s_{n_1}$,
  que serão removidas aplicando-se uma das relações $\rel_{k_2}^i (i = 1, \ldots, \nrel)$.
  Considere uma destas restrições, denotada por $s_{k_2}^{(n_1)}$.
  Como anteriormente, estabelecemos a existência de um estado $s_{k_2} \in C_{k_2}$
  tal que existe $s_{n_2} \in Ext(s_{k_2})$ tal que $s_{n_2} \dom s_{n_1}$.
  A transitividade de $\dom$ garante que $s_{n_2} \dom s_{\til{n}}$.
  Pela repetição deste processo, estabelecemos a existência de um estado
  $s_{k_2} \in C_{k_2}$ tal que existe $s_{n_2} \in Ext(s_{k_2})$
  tal que $s_n \dom s_{\til{n}}$.

  Além disso, selecionando-se um conjunto $C_n^i$ que é conjunto independente
  com relação a $\rel_n^i = \dom$, esta propriedade se mantém válida para $C_n^\nrel$
  o qual é subconjunto de $C_n^i$.
  Dessa forma $C_n$, que corresponde ao conjunto eficiente reduzido,
  representa o conjunto de vetores objetivos não dominados.
  \qedhere
\end{proof}

O teorema anterior apenas querer que um dos $n . \nrel$ conjuntos coberturas
seja independente com respeito a sua relação de dominância.
Mesmo se todos os conjunto $C_k^i$....

\subsection{Geração de conjuntos cobertura e independente}

Apresentamos no Algoritmo~\ref{alg:baz2} uma maneira de produzir $C_k^i$ um
conjunto cobertura e independente de $C_k^{i-1}$ com respeito a relação
transitiva $\rel_k^i$ (linha ??? do Algoritmo~\ref{alg:baz1}).

\begin{algorithm}
  \caption{Computação do conjunto $C_k^i$ cobertura e independente do conjunto
  $C_k^{i-1}$ com respeito a relação transitiva $\rel_k^i$.}
  \label{alg:baz2}
  \Kw{$C_k^{i-1} = \{ s_{k(1)}, \ldots, s_{k(r)}\}$}
\Begin{
  $C_k^i \gets \{s_{k(1)}\}$\;
  \For{$h \gets 2, r$}{
    $C_k^i = \{s_{\til{k}(1)}, \ldots, s_{\til{k}(l_h)}\}$\;
    $dominated \gets false$\;
    $dominates \gets false$\;
    $j \gets 1$\;
    \While{$ j \leq l_h$ \textbf{and} $\neg dominated$ \textbf{and} $\neg dominates$}{
      \uIf{$s_{\til{k}(j)} \rel_k^i s_{k(h)}$}{
        $dominated \gets true$\;
      }
      \ElseIf{$s_{k(h)} \rel_k^i s_{\til{k}(j)}$}{
        $C_k^i \gets C_k^i\backslash\{s_{\til{k}(j)}\}$\;
        $dominates \gets true$\;
      }
      $j \gets j + 1$\;
    }
    \If{$\neg dominated$}{
      \While{$j \leq l_h$}{
        \If{$s_{k(h)} \rel_k^i s_{\til{k}(j)}$}{
          $C_k^i \gets C_k^i \backslash \{s_{\til{k}(j)}\}$\;
        }
        $j \gets j + 1;$
      }
      $C_k^i \gets C_k^i \cup \{s_{k(h)}\};$
    }
  }
  \textbf{return} $C_k$
}
\end{algorithm}

\begin{myprop}
  Para qualquer relação de dominância $\rel_k^i$ sobre $S_k$, o
  Algoritmo~\ref{alg:baz2} retorna $C_k^i$ um conjunto cobertura e independente
  de $C_k^{i-1}$ com respeito a $\rel_k^i (j = 1, \ldots, n; i = 1, \ldots, \nrel)$.
\end{myprop}

\begin{proof}
  Claramente $C_k^i$ é independente com relação a $\rel_k^i$, uma vez que
  inserimos o estado $s_k$ ao conjunto $C_k^i$ na linha ????
  apenas se não for dominado por nenhum outro estado em $C_k^i$ (linha ???)
  e todos os estados dominados por $s_k$ tenham sido removidos de $C_k^i$
  (linha ??? e ???).

  Mostramos agora que $C_k^i$ é um conjunto covertura de $C_k^{i-1}$
  com respeito a $\rel_k^i$.
  Considere $s_{\til{k}} \in C_k^{i-1} \backslash C_k^i$.
  Isto ocorre tanto porque não passa no teste da linha ??? quanto porque
  foi removido na linha ??? ou ???.
  Isto é devido respectivamente a um estado $s_{-k}$ já existente em $C_k^i$
  ou a ser incluído em $C_k^i$ (na linha ???) tal que $s_{-k}\rel_k^i s_{\til{k}}$.
  Pode ser que $s_{-k}$ seja removido de $C_k^i$ em uma iteração posterior
  da linha ??? se existir um novo estado $s^{\hat k} \in C_k^{i-1}$ a ser
  incluído em $C_k^i$, tal que $s_{\hat k} \rel_k^i s_{-k}$.
  Entretanto, a transitividade de $\rel_k^i$ garante a existência, ao fim da
  iteração $k$, de um estado $s_k \in C_k^i$ tal que $s_k \rel_k^i s_{\til{k}}$.
  \qedhere
\end{proof}

O Algoritmo~\ref{alg:baz2} pode ser aprimorado uma vez que geralmente é possível
gerar estados de $C_k^{i-1} = \{s_{k(1)}, \ldots, s_{k(r)}\}$ conforme
uma \emph{ordem de preservação de dominância} para $\rel_k^i$ de forma que
todo $l < j (1 \leq l, j \leq r)$ temos tanto $s_{k(l)} \rel_k^i s_{k(j)}$
ou $\neg (s_{k(j)} \rel_k^i s_{k(l)})$.
A proposição seguinte provê a condição necessária e suficiente para estabelecer
a existência da ordem de preservação de dominância para uma relação de dominância.

\begin{myprop}
  Seja $D_k$ uma relação de dominância sobre $S_k$.
  Existe uma ordem de preservação de dominância para $D_k$ se, e somente se,
  $\rel_k$ não admite ciclos em sua parte assimétrica.
  \label{prop:domorder}
\end{myprop}

\begin{proof}
  \noindent \\ $\Rightarrow$ A existência de um ciclo na parte assimétrica de $D^k$
  implicaria na existência de dois estados consecutivos $s_{k(j)}$ e $s_{k(l)}$
  neste ciclo sendo $j > l$, o que é uma contradição.\\
  $\Leftarrow$ Qualquer ordem topológica baseada na parte assimétrica de $D_k$
  é uma ordem de preservação de dominância para $D_k$.
  \qedhere
\end{proof}

Na Seção seguinte é apresentado um exemplo de ordem de preservação de dominância.
Se os estados em $C_k^{i-1}$ são gerados conforme uma ordem de preservação de
dominância para $D_k^i$, a linha ??? e o laço ???-??? do Algoritmo~\ref{alg:baz2}
podem ser omitidos.

\section{Detalhes de implementação}

Primeiramente é apresentada a ordem na qual serão considerados os itens no
processo sequencial do algoritmo.
Posteriormente são apresentados a três relações de dominância utilizadas
no algoritmo ??? e a maneira com que serão utilizadas.

\subsection{Ordem dos itens}

A ordem na qual os itens são considerados é uma questão crucial na implementação
do algoritmo.
Sabe-se que no caso do problema da mochila unidimensional, para se
obter uma boa solução, os itens devem ser geralmente considerados em
ordem decrescente segundo a proporção
$p_i/w_i$~\cite{ehrgott2013multicriteria, martello1990knapsack}.
Porém, para o caso multi-objetivo não existe uma ordem natural.

São introduzidas duas ordenações $\ord^{sum}$ e $\ord^{max}$
derivadas da agregação das ordens $\ord^j$ inferidas pelas proporções
$p_i^j/w_i$ para cada objetivo $(j = 1, \ldots, \np)$.
Considere $r_i^l$ o rank (ou posição) do item $l$ na ordenação $\ord^j$.
$\ord^{sum}$ denota uma ordenação segundo valores crescentes da soma dos
ranks dos itens nas $\np$ ordenações $(i = 1, \ldots, \np)$.
$\ord^{max}$ denota uma ordenação segundo valores crescentes de máximo ou
piores ranks de itens nas $\np$ ordenações $\ord^j (j = 1, \ldots, \np)$,
onde o pior rank do item $l$ nas $\np$ ordenações $\ord^j (j = 1, \ldots, \np)$
é calculado como $max_{i=1}^{\np}\{r_l^i\} + \frac{1}{\np n} \sum_{i = 1}^{\np} r_l^i$
para distinguir itens com o mesmo rank máximo.

\missing{Dizer que o artigo original conclui atraves de testes qual é a melhor
ordenação para ser utilizada nas iterações do algoritmo.}

\subsection{Relações de dominância}
\label{sec:reldom}

Cada relação de dominância exploca uma consideração específica.
Por isto recomenda-se utilizar relações de dominância que sejam complementares.
Além disso, para se escolher uma relação de dominância é necessário considerar
sua potencial capacidade de descarte de estados diante do
custo computacional requerido.

A seguir serão apresentas as três relações de dominância utilizadas no algoritmo.
As primeiras duas relações são reazoavelmente triviais de se estabelecer sendo
a última ainda considerada, mesmo sendo um tanto mais complexa, devido à sua
complementaridade diantes das duas primeiras.

A primeira relação de dominância se estabelece segundo a seguinte observação.
Quando a capacidade residual de uma solução associada a um estado $s_k$
da iteração $k$ é maior ou igual a soma dos pesos dos itens restantes,
i.e. itens $k+1, \ldots, n$, o único complemento de $s_k$ que pode resultar
em uma solução eficiente é o complemento máximo $I = \{k+1, \ldots, n\}$.
Portanto, neste contexto, não se faz necessário gerar as extensões de $s_k$
que não contenham todos os itens restants.
Define-se então a relação de dominância $\relI$ sobre $S^k$ para
$k = 1, \ldots, n$ como:
\begin{displaymath}
  \forall s_k, s_{\til{k}} \in S_k, \enskip
  s_k \relI s_{\til{k}}
    \Leftrightarrow
    \begin{cases}
      s_{\til{k}} \in S_{k-1}, \\
      s_k = (s_{\til{k}}^1 + p_k^1, \ldots, s_{\til{k}}^\np + p_k^\np, s_{\til{k}}^{\np+1} + w_k), \; \text{e} \\
      s_{\til{k}}^{\np+1} \leq W - \sum_{i=k}^n w_i
    \end{cases}
\end{displaymath}
A seguinte proposição mostra que $\relI$ é de fato uma relação de dominância
e apresenta propriedades adicionais de $\relI$.

\begin{myprop}[Relação $D_k^r$]
  \noindent
  \begin{enumerate}
    \item[(a)] $D_k^r$ é uma relação de dominância
    \item[(b)] $D_k^r$ é transitiva
    \item[(c)] $D_k^r$ admite ordem de preservação de dominância
\end{enumerate}
\end{myprop}

\begin{proof}
  \noindent
  \begin{enumerate}
    \item[(a)]{
      Considere dois estados $s_k$ e $s_{\til{k}}$ tal que $s_k \relI s_{\til{k}}$.
      Isto implica que $s_k \dom s_{\til{k}}$.
      Além disso, uma vez que
      $s_k^{\np+1} = s_{\til{k}^{\np+1}} + w_k \leq W - \sum_{i=k+1}^n w_i$,
      qualquer subconjunto $I \subseteq \{k+1, \ldots, n\}$ é um complemento
      de $s_{\til{k}}$ e $s_k$.
      Portanto, para todo estado $s_{\til{n}} \in Ext(s_{\til{k}})$,
      existe $s_n \in Ext(s_k)$, baseado no mesmo complemento que $s_{\til{n}}$,
      tal que $s_n \dom s_{\til{n}}$.
      Isto estabelece que $\relI$ satisfaz a Equação~\ref{eq:domrel} da
      Definição~\ref{def:domrel}. }
    \item[(b)]{Trivial.}
    \item[(c)]{Pela Proposição~\ref{prop:domorder}, uma vez que $\relI$ é transitiva.}
    \qedhere
  \end{enumerate}
\end{proof}

Esta relação de dominância é um tanto fraca, uma vez que em cada $k$-ésima
iteração ela pode apenas.......
Apesar disso, é uma relação de dominância extremamente fácil de ser verificada,
uma vez que, ao ser o valor de $W - \sum_{i=k}^n$ computado, a relação $\relI$
requer apenas uma comparação para ser estabelecida entre dois estados.

A relação de dominância seguinte é a generalização para o caso multi-objetivo
da relação de dominância proposta por Weingartner e Ness~\cite{weingartner1967methods}
e utilizada no clássico algoritmo de Nemhauser e Ulmann~\cite{nemhauser1969discrete}.
Esta segunda relação de dominância é definida sobre $s_k$ para $k = 1, \ldots, n$ por:
\begin{displaymath}
  \forall s_k, s_{\til{k}} \in S_k, \; s_k \relII s_{\til{k}}
    \Leftrightarrow
    \begin{cases}
      s_k \dom s_{\til{k}} \; & \text{e} \\
      s_k^{\np+1} \leq s_{\til{k}}^{\np+1} \; & \text{se} \; k < n
    \end{cases}
\end{displaymath}
Observe que a condição sobre os pesos $s_k^{\np+1}$ e $s_{\til{k}}^{\np+1}$
garante que todo complemento de $s_{\til{k}}$ é também um complemento de
$s_k$.
A seguinte proposição mostra que $\relII$ é de fato uma relação de
dominância e apresenta propriedades adicionais de $\relII$.

\begin{myprop}[Relação $\relII$]
  \noindent
  \begin{enumerate}
    \item[(a)] $\relII$ é uma relação de dominância
    \item[(b)] $\relII$ é transitiva
    \item[(c)] $\relII$ admite ordem de preservação de dominância
    \item[(c)] $\relII = \dom$
\end{enumerate}
\end{myprop}

\begin{proof}
  \noindent
  \begin{enumerate}
    \item[(a)]{ TODO.... }
    \item[(b)]{Trivial.}
    \item[(c)]{Pela Proposição~\ref{prop:domorder}, uma vez que $\relII$ é transitiva.}
    \item[(d)]{Por definição.} \qedhere
  \end{enumerate}
\end{proof}
A terceira relação de dominância é baseada na comparação entre extensões
específicas de um estado e um limite superior de extensões de outros estados.
Um limite superior de um estado é definido segundo no contexto a seguir.

\begin{mydef}[Limite superior]
  Um vetor objetivo $u = (u^1, \ldots, u^\np)$ é um limite superior
  de um estado $s_k \in S_k$ se, e somente se, $\forall s_n \in Ext(s_k)$
  tem-se que $u^i \geq s_n^i, i = 1, \ldots, \np$.
\end{mydef}

Um tipo geral de relação de dominância pode ser derivada da seguinte maneira:
considere dois estados $s_k, s_{\til{k}} \in S_k$, se existe um complemento
$I$ de $s_k$ e um limite superior $\tilde{u}$ de $s_{\til{k}}$ tal que
$s_k^j + \sum_{i \in I} p_i^j \geq \tilde{u^j},\; j = 1, \ldots, \np$,
então $s_k$ domina $s_{\til{k}}$.

Este tipo de relação de dominância pode ser implementara apenas para
complementos específicos e limites superiores.
Na implementação do algoritmo em questão são apenas considerados
dois complementos específicos $I'$ e $I''$ obtidos pelo simples algoritmo
de preenchimento \emph{guloso} definido a seguir.
Após rerotular os itens $k+1, \ldots, \np$ de acordo com a ordenação
$\ord^{sum}$ (e $\ord^{max}$ respectivamente), o complemento $I'$
(e $I''$ respectivamente) são obtidos através a inserção sequencial
dos itens restantes na respectiva solução de forma que a restrição
de capacidade é respeitada.

Para computar $u$, utiliza-se o limite inferior proposto em
\cite{martello1990knapsack} para cada valor de objetivo.
Considere $\overline{W}(s_k) = W - s_k^{\np+1}$ a capacidade residual associada
ao estado $s_k \in S_k$.
Denota-se por
$c_j = min\big\{l_j \in \{k_1, \ldots, n\} \; | \; \sum_{i=k+1}^{l_j} w_i > \overline{W}(s_k)\big\}$
a posição do primeiro item que não pode ser adicionado ao estado $s_k \in S_k$
quando os itens $k+1, \ldots, n$ são rerotulados de acordo com a ordenação $\ord^j$.
Desde modo, segundo~\cite{martello1990knapsack}, quando os itens $k+1, \ldots, n$
são rerotulados de acordo com a ordenação $\ord^j$, um limite inferior
para o $j$-ésimo valor objetivo de $s_k \in S_k$ é para $j = 1, \ldots, \np$:
\begin{equation}
  u^j = s_k^j + \sum_{i=k+1}^{c_j-1} p_i^j +
    max\left\{ \left\lfloor\overline{W}(s_k)\frac{p^j_{c_j+1}}{w_{c_j+1}} \right\rfloor ,
     \left\lfloor p^j_{c_j} - \big(w_{c_j} - \overline{W}(s_k)\big).\frac{p^j_{c_{j-1}}}{w_{c_j-1}}
     \right\rfloor \right\}
  \label{eq:upperb}
\end{equation}
Finalmente definimos $\relIII$ uma relação de dominância como caso particular
do tipo geral para $k = 1, \ldots, n$ por:
\begin{displaymath}
  \forall s_k, s_{\til{k}} \in S_k, \; s_k \relIII s_{\til{k}}
    \Leftrightarrow
    \begin{cases}
      s_k^j + \sum_{i \in I'} p_i^j \geq \tilde{u^i}, & j = 1, \ldots, \np \\
      \text{ou} \\
      s_k^j + \sum_{i \in I''} p_i^j \geq \tilde{u^i}, & j = 1, \ldots, \np
    \end{cases}
\end{displaymath}
onde $\tilde{u} = (\tilde{u}_1, \ldots, \tilde{u}_\np)$ é o limite superior
para $s^{\til{k}}$ computado de acordo com a Equação~\ref{eq:upperb}.

A seguinte proposição mostra que $\relIII$ é de fato uma relação de dominância
e apresenta propriedades adicionais de $\relIII$.
\begin{myprop}[Relação $D_k^b$]
  \noindent
  \begin{enumerate}
    \item[(a)] $D_k^b$ é uma relação de dominância
    \item[(b)] $D_k^b$ é transitiva
    \item[(c)] $D_k^b$ admite ordem de preservação de dominância
    \item[(c)] $D_n^b = \dom$
\end{enumerate}
\end{myprop}

\begin{proof}
  \noindent
  \begin{enumerate}
    \item[(a)]{ Considere os estados $s_k$ e $s_{\til{k}}$ tal que $s_k \relIII s_{\til{k}}$.
      Isto implica que existe $I \in {I', I''}$ que capaz de gerar um estado $s_n \in Ext(s_{\til{k}})$.
      Além disso, uma vez que $\tilde{u}$ é um limite superior de $s_{\til{k}}$,
      temos $\tilde{u} \dom s_{\til{n}}, \forall s_{\til{n}} \in Ext(s_{\til{n}})$.
      Portanto, por transitividade de $\dom$, temos que $s_n \dom s_{\til{n}}$,
      o que estabelece que $\relIII$ satisfaz a condição da Equação~\ref{eq:domrel}
      da Definição~\ref{def:domrel}.
      }
    \item[(b)]{ Considere os estados $s_k, s_{\til{k}}$ e $s_{-k}$ tal que
      $s_k \relIII s_{\til{k}}$ e $s_{\til{k}} \relIII s_{-k}$.
      Isto implica que, por um lado, existe $I_1 \in \{I', I''\}$ tal que
      $s_k^j + \sum_{i \in I_1} p^j_i \geq \tilde{u}_j (j = 1, \ldots, \np)$
      por outro lado, existe $I_2 \in \{I', I''\}$ tal que
      $s_{\til{k}}^j + \sum_{i \in I_2} p^j_i \geq \bar{u}_j (j = 1, \ldots, \np)$
      Uma vez que $\tilde{u}$ é um limite superior para $s_{\tilde{k}}$ temos que
      $\tilde{u}_j \geq s_{\til{k}}^i + \sum_{j \in J_2} p^j_i (i = 1, \ldots, \np)$.
      Portanto temos que $s_k \relIII s_{-k}$. }
    \item[(c)]{Pela Proposição~\ref{prop:domorder}, uma vez que $\relIII$ é transitiva.}
    \item[(d)]{Por definição.} \qedhere
  \end{enumerate}
\end{proof}
A relação $\relIII$ é mais difícil de ser verificada do que as relações
$\relI$ e $\relII$ uma vez que requer uma maior quantidade de
comparações e informações sobre outros estados.

Obviamente, a relação $\relIII$ seria mais forte se utizados complementos
adicionais (de acordo com outras ordenações) para $s_k$ e computado,
ao invés de apenas um limite superior $u$, um conjunto de limites superiores,
como por exemplo, a proposta apresentada em \cite{ehrgott2007bound}.
Porém, no contexto em questão, uma vez que temos que verificar $\relIII$
para muitos estados, enriquecer $\relIII$ dessa forma demandaria um
esforço computacional alto demais.

\subsection{Detalhes de implementação}

Para se ter uma implementação eficiente, são utilizadas as três relações de
dominância apresentadas na Seção~\ref{sec:reldom} em cada iteração.
Como dito na Seção anterior uma relação de dominância demanda maior
esforço computacional do que outra para ser vericada.
Além disso, mesmo que sejam parcialmente complementares,
é frequente mais de uma relação de dominância ser válida para o mesmo par de estados.
Por esse motivo é natural aplicar primeiramente relações de dominância que
podem ser verificadas facilmente (como $\relI$ e $\relII$)
para então verificar num conjunto reduzido de estados uma relações mais
custosas (como $\relIII$).

A seguir será descrito os detalhes da implementação e amplicação
das três relações de dominância.
O Algoritmo~\ref{alg:baz3}, que computa, na $k$-ésima iteração, o subconjunto
de estados candidatos $C_k$ a partir de um subconjunto
$C_{k-1} (k = 1, \ldots, n)$, substitui as linhas ??? a ??? do Algoritmo~\ref{alg:baz2}.

A utilização das dominâncias $\relI$ e $\relII$ é descrita nas linhas ???-???
e a utilização da dominância $\relIII$ é descrita nas linhas ???-???.
O algoritmo utiliza dois sub-procedimentos: \texttt{MantainNonDominated},
que remove estados $\relII$-dominados, e procedimento \texttt{KeepNonDominated},
que é utilizado durante a aplicação da relação $\relIII$.

\begin{algorithm}
  \caption{O algoritmo de Nemhauser e Ullmann para o \mokp.}
  \label{alg:baz3}
  \Kw{$C_{k-1} = \{ s_{k-1(1)}, \ldots, s_{k-1(r)}\}$
      tal que $s_{k-1(i)} \geq_{lex} s_{k-1(j)} \; \forall i < j (1 \leq i, j \leq r)$}
\Begin{
  $C_k \gets \emptyset; M_k \gets \emptyset; i \gets 1; j \gets 1$\;
  \While{$j \leq r$ \textbf{and} $s_{k-1(1)}^{\np+1} + \sum^n_{l=k}w_l \leq W$}{
    $j \gets j + 1$\;
  }
  \While{$i \leq r$ \textbf{and} $s_{k-1(1)}^{\np+1} + w_k \leq W$}{
    $s_k \gets (s_{k-1(1)}^1 + p_k^1,
      \ldots,
      s_{k-1(1)}^{\np} + p_k^{\np},
      s_{k-1(1)}^{\np+1} + w_k)$\;
    \While{$j \leq r$ \textbf{and} $s_{k-1(j)} \leq_{lex} s_k$}{
      \texttt{MantainNonDominated($s_{k-1(j)}, M_k, C_k$)}; $j \gets j + 1$\;
    }
    \texttt{MantainNonDominated($s_k, M_k, C_k$)}; $i \gets i + 1$\;
  }
  \While{$ j \leq r$}{
    \texttt{MantainNonDominated($s_{k-1(j)}, M_k, C_k$)}; $j \gets j + 1$\;
  }
  \eIf{ k = n }{
    $C_n \gets M_n$\;
  }{
    $F \gets \emptyset$\;
    \For{ $\ord \in \{\ord^{sum}, \ord^{max}\}$}{
      \For{$s_k \in M_k$}{
        Rerotular itens $k+1, \ldots, n$ segundo ordenação $\ord$\;
        $s_n \gets s_k$\;
        \For{$j \gets k + 1, \ldots, n$}{
          \If{$s_n^{\np+1} + w_j \leq W$}{
            $s_n \gets (s_n^1 + p_j^1, \ldots, s_n^{\np} + p_j^{\np}, s_n^{\np+1} + w_j)$\;
          }
        }
        $F \gets \texttt{KeepNonDominated}(s_n, F)$\;
      }
    }
    $i \gets 1; remove \gets true; F \gets \{s_{n(1)}, \ldots, s_{n(h)}\}$\;
    \While{$i \leq c$ \textbf{and} $remove$}{
      Computar limite superior $u$ para $s_{k(i)}$ conforme Equação~\ref{eq:upperb}\;
      $j \gets 1; remove \gets false$\;
      \While{$j \leq h$ \textbf{and} $s_{n(j)} \dom_{lex} u$ \textbf{and} $\neg remove$}{
        \eIf{$s_{n(j)} \dom u$}{
          $remove \gets true$\;
        }{
          $j \gets j + 1$\;
        }
      }
      \If{$remove$}{
        $C_k \gets C_k\backslash \{s_{k(i)}\}; i \gets i + 1$\;
      }
    }
  }
  \textbf{return} $C_k$\;
}

\end{algorithm}






