

\begin{resumo}
Diversos problemas reais envolvem a otimização simultânea de múltiplos critérios,
os quais são, geralmente, conflitantes entre si.
Estes problemas são denominados multiobjetivo e
não possuem uma única solução, mas um conjunto de soluções de interesse, denominadas soluções
eficientes ou não dominadas.
Um dos grande desafios a serem enfrentados na resolução deste tipo de problema é o
tamanho do conjunto solução, que tende a crescer rapidamente dado o tamanho da instância,
degradando a performance dos algoritmos.
Dentre os problemas multiobjetivos mais estudados está o problema da mochila multiobjetivo,
pelo qual diversos problemas reais podem ser modelados.
Este trabalho propõe a aceleração do processo de solução do problema da mochila multiobjetivo,
através da utilizando da \kdtree{} como estrutura de indexação multidimensional
para auxiliar a manipulação das soluções.
A performance da abordagem é analisada através de experimentos
computacionais, realizados no contexto exato utilizando um algoritmo estado da arte.
Testes também são realizados no contexto heurístico, utilizando a adaptação
de uma meta-heurística para o problema em questão, sendo esta também uma contribuição do presente trabalho.
Segundo os resultados, para o contexto exato a proposta foi eficaz, apresentam speedup de até $2.3$
para casos bi-objetivo e $15.5$ em casos 3-objetivo, não sendo porém
eficaz no contexto heurístico, apresentando pouco impacto no tempo computacional.
Em todos os casos, porém, houve considerável redução no número de avaliações de soluções.

\vspace{\onelineskip}

\noindent
\textbf{Palavras Chave}:
Problema da Mochila Multiobjetivo,
Indexação Multidimensional,
Meta-heurística,
Algoritmo Exato.
\end{resumo}


\begin{resumo}[Abstract]
\begin{otherlanguage*}{english}
Several real problems involve the simultaneous optimization of multiple criteria,
which are generally conflicting with each other.
These problems are called multiobjective and
do not have a single solution, but a set of solutions of interest, called efficient solutions
or non-dominated solutions.
One of the great challenges to be faced in solving this type of problem is the
size of the solution set, which tends to grow rapidly given the size of the instance,
degrading algorithms performance.
Among the most studied multiobjective problems is the multiobjective knapsack problem,
by which several real problems can be modeled.
This work proposes the acceleration of the resolution process of the multiobjective knapsack problem,
through the use of a \kdtree {} as a multidimensional index structure
to assist the manipulation of solutions.
The performance of the approach is analyzed through computational experiments,
performed in the exact context using a state-of-the-art algorithm.
Tests are also performed in the heuristic context, using the adaptation
of a meta-heuristic for the problem in question, being also a contribution of the present work.
According to the results, the proposal was effective for the exact context, presenting a speedup up to $2.3$
for bi-objective cases and $15.5$ for 3-objective cases, but not
effective in the heuristic context, presenting little impact on computational time.
In all cases, however, there was a considerable reduction in the number of solutions evaluations.
\vspace{\onelineskip}

\noindent

\textbf{Keywords}:
Multiobjective Knapsack Problem,
Multidimensional Indexing,
Metaheuristic,
Exact Algorithm.
\end{otherlanguage*}
\end{resumo}

% Falar brevemente sobre o MOKP
% Falar brevemente sore a estratégia de indexação
% Resumir os resultados obtidos

%\missingt{
%Observar as 5 regras:\\
%1. A general statement introducing the broad research area of the particular topic being investigated;\\
%2. An explanation of the specific problem (difficulty, obstacle, challenge) to be solved;\\
%3. A review of existing or standard solutions to this problem and their limitations;\\
%4. An outline of the proposed new solution;\\
%5. A summary of how the solution was evaluated and what the outcomes of the evaluation were.
%}

% 1. Uma declaração geral introduzindo a área de pesquisa;
% 2. Uma explicação espeficica do problema
% 3.
%
%