\section{Contribuições}

O presente trabalho propõe a indexação multidimensional das soluções
do problema da mochila multiobjetivo,
como proposta de aceleração das operações de verificação de dominância de solução.
A operação de verificação de dominância é uma das principais
operações necessárias para a resolução do problema.
A indexação multidimensional das soluções tem por objetivo
reduzir o número de avaliações de soluções necessárias para a execução da operação,
diminuindo consequentemente o tempo computacional demandado.

Há na literatura a proposta de utilização da quadtree como estrutura de dados
de indexação de soluções de problemas multiobjetivo.
Porém a estratégia mostrou-se eficaz apenas em casos bi-objetivos de conjuntos Paretos consideravelmente extensos.
Conjectura-se que essa ineficiência é decorrente
do alto \emph{overhead} da estrutura, especialmente em casos com mais de dois objetivos.

Para que a proposta de indexação fosse possível,
foi definido no presente trabalho um mapeamento entre a operação
de verificação de dominância e o problema da busca de faixa.
O problema de busca de faixa consiste em verificar a existência de pontos em uma determinada
região do espaço multidimensional.
Esse problema é bem conhecido em áreas como computação gráfica, geometria computacional e jogos,
onde a \kdtree{} é geralmente utilizada como estrutura de dados auxiliar.
Sendo assim, a proposta do presente trabalho foi a utilização da \kdtree{}
como estrutura auxiliar da operação de verificação de dominância.

A aplicação e o comportamento da \kdtree{} junto à operação
foram discutidos, bem como os da lista encadeada e da árvore AVL,
estruturas até então utilizadas pela literatura para este fim.
O desempenho da proposta de indexação foi testado no contexto exato e heurístico.

No contexto exato, a proposta foi aplicada ao algoritmo Bazgan,
considerado pela literatura como o mais eficiente atualmente para o problema da mochila multiobjetivo.
A performance do algoritmo foi comparada através de experimentos computacionais.
Para tanto, foi considerada a mesma proposta de instâncias empregada pela literatura,
sendo quatro tipos de instâncias bi-objetivo e dois tipos de instâncias 3-objetivo.
Para o caso 3-objetivo foram ainda propostas mais dois tipos de instâncias,
sendo estas generalizações de seus respectivos casos bi-objetivos.

Os resultados dos experimentos computacionais utilizando o algoritmo Bazgan,
mostraram que a proposta de indexação multidimensional possibilitou o speedup
de até $2.3$ para os casos bi-objetivo, e até $15.5$ para casos 3-objetivo.
A proposta não se mostrou eficiente apenas em instâncias consideradas fáceis,
cujos conjuntos Pareto têm tamanhos consideravelmente reduzidos em relação
às demais instâncias.
Nesses casos, onde se dá a manipulação de poucas soluções,
o \emph{overhead} computacional da \kdtree{} torna sua utilização ineficaz.

No contexto heurístico, a indexação multidimensional
foi aplicada ao algoritmo SCE para o MOKP, proposto também neste trabalho.
Primeiramente o SCE foi adaptado para resolver problemas multiobjetivo.
Para isso, a aptidão das soluções foi definida com base na ordenação
em frontes não dominados.
A aproximação do \paretoset{} foi gerada com o auxílio de um arquivo externo.
Em seguida, foi estabelecida a implementação específica para o MOKP,
definindo-se a operação de geração de solução viável
aleatória e a operação de cruzamento de soluções.

A validade da implementação do algoritmo SCE para o MOKP foi primeiramente
avaliada e comparada com os principais algoritmos da literatura,
não tendo melhores resultados que as
heurísticas de estado-da-arte, porém tendo resultados superiores aos
das heurísticas mais antigas.

Os experimentos computacionais no contexto heurístico
foram realizados sobre o conjunto de 6 instâncias,
utilizado pela literatura para avaliar a performance de heurísticas para o MOKP.
Apesar de reduzir consideravelmente o número de avaliações de soluções,
a utilização da \kdtree{} não apresentou eficiência relevante quanto ao tempo computacional
na abordagem heurística, tendo pior performance nas instâncias com \paretoset{}
relativamente pequenos.

Pode-se concluir que a utilização da \kdtree{} é capaz de reduzir consideravelmente
o número de avaliações de soluções na maioria dos casos, além de também reduzir o tempo computacional
demandado em casos em que é necessário executar a verificação de dominância em grandes
conjuntos de solução.
Os resultados evidenciam ser indispensável
a utilização de uma estrutura de indexação multidimensional,
nos casos de problemas com mais de 2 objetivos com grandes
conjuntos de solução.
Vale ressaltar que os algoritmos não foram alterados, somente
a forma de indexação das soluções, a qual pode ser aplicada a qualquer algoritmo
que faz utilização da operação de verificação de dominância de solução.

%\missingf{afirmacao forte: "Segundo os resultados, a utilização de uma estrutura de indexação multidimensional
%mostra-se indispensável no caso de problemas com mais de 2 objetivos com grandes
%conjuntos de solução." em vez de mostra, alterei para evidencia... 
%
%\resp Ok. Melhor.}

Notou-se porém, que em casos onde o número de soluções manipuladas é pequeno, a utilização
da \kdtree{} não é recomendável, pois apesar de apresentar redução no número de
avaliações, o \emph{overhead} da estrutura dificulta uma redução no tempo computacional demandado.

\section{Trabalhos Futuros}

Como trabalho futuro pretende-se verificar a performance da \kdtree{} em outros problemas
multiobjetivos, uma vez que o crescimento do \paretoset{} não é
uma característica específica do problema em questão, mas observado
em problemas multiobjetivo de forma geral, visto que as possibilidades de soluções
eficientes aumentando dado o tamanho da instância,
especialmente para casos com mais dois objetivos~\cite{ehrgott2013multicriteria}.

Outro ponto a ser explorado é a consideração de outras estruturas de dados
para auxílio à operação de verificação de dominância.
Uma das estruturas a ser considerada é a ND-Tree, proposta
em~\cite{jaszkiewicz2017} como estrutura de dados para auxiliar
a operação de verificação de dominância.
Vale também considerar a proposta de uma estrutura dedicada ao
problema de verificação de dominância, considerando que as
soluções pertencentes a um \paretoset{} possuem uma distribuição espacial
característica, na qual nenhuma é dominada por outra.

Pretende-se também aprimorar a implementação do SCE para o MOKP.
Uma possibilidade seria o aperfeiçoamento das soluções
através da utilização de conhecimentos específicos do problema.
Outra possibilidade seria realizando um pré-processamento sobre o problema inicial,
transformando-o em um problema mais simples ou dividindo-o em sub-problemas
menores.

Além das propostas de contribuição, pretende-se investigar a causa da
presente implementação do algoritmo Bazgan demandar mais tempo computacional
que o relatado pela literatura.


\missingf{Sugiro expandir a discussão de trabalhos futuros}

\missingf{ na verdade acho que a Conclusao tem que ser expandida como um todo. Tá parecendo conclusao de artigo, que tem limitacao de espaço. Vc poderia falar mais de cada um dos itens comentados, até apresentar exemplos para esclarecer os leitores.}