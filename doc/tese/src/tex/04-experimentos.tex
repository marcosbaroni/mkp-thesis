Vários experimentos computacionais foram realizados com o objetivo de veirficar
a eficiência da indexação multi-dimensional, especialmente em instâncias com
mais de duas dimensões.

\missingt{
  Explicar que são os mesmos tipo de instâncias, mas não o mesmo conjunto, pois
  não foi disponibilizado pelos autores.\\
  Dizer por que da métrica de comparação (hypervolume).\\
}


\missing{Introduzir comentário sobre as instâncias da Bazgan.}
Quatro tipos de instâncias bi-objetivo são consideradas:
\begin{enumerate}
  \item[A)] Aleatórias: $
    p^j_i \in [1, 1000],
    w_i \in [1,1000]$.
  \item[B)] Não-conflitantes: $
    p^1_i \in [111, 1000],\\
    p^2_i \in [p^1_i - 100, p^1_i + 100],\\
    w_i \in [1,1000]$.
  \item[C)] Conflitantes: $
    p^1_i \in [1, 1000],\\
    p^2_i \in [max\{900-p^1_i;1\}, min\{1100-p^1_i, 1000\}],\\
    w_i \in [1,1000]$.
  \item[D)] Conflitantes com pesos correlacionados: $
    p^1_i \in [1, 1000],\\
    p^2_i \in [max\{900-p^1_i;1\}, min\{1100-p^1_i, 1000\}],\\
    w_i \in [p^1_i+p^2_i-200, p^1_i+p^2_i+200]$.
\end{enumerate}
onde $\in [a,b]$ denota uma distribuição uniforme aleatória no intervalo
$[\,b]$.

\missingt{
  Esclarecer os critérios de generalização dos tipos de instância.
}

Para os experimentos com $3$-objetivo considerou-se
a generalização introduzida por~\cite{bazgan2009}
para os tipos $A$ e $C$ e também duas propostas de generalização
para os tipo $B$ e $D$:
\begin{enumerate}
  \item[A)] Aleatórias: $
    p^j_i \in [1, 1000]\\
    w_i \in [1,1000]$
  \item[B)] Não-conflitantes: $
    p^1_i \in [111, 1000],\\
    p^2_i \in [p^1_i - 100, p^1_i + 100],\\
    p^3_i \in [p^1_i - 100, p^1_i + 100],\\
    w_i \in [1,1000]$.
  \item[C)] Conflitantes: $
    p^1_i \in [1, 1000], \;
    p^2_i \in [1, 1001 - p^1_i] \\
    p^3_i \in [max\{900-p^1_i-p^2_i;1\}, min\{1100-p^1_i-p^2_i, 1001-p^1_i\}]\\
    w_i \in [1,1000]$.
  \item[D)] Conflitantes com pesos correlacionados: $
    p^1_i \in [1, 1000]\\
    p^2_i \in [1, 1001 - p^1_i] \\
    p^3_i \in [max\{900-p^1_i-p^2_i;1\}, min\{1100-p^1_i-p^2_i, 1001-p^1_i\}]\\
    w_i \in [p^1_i+p^2_i+p^3_i-200, p^1_i+p^2_i+p^3_i+200]$.
\end{enumerate}
Instâncias do tipo $B$ são consideradas mais fáceis enquanto instâncias do
tipo $D$ são consideradas as mais difíceis.
Para todas as instâncias, atribui-se $W=\frac{1}{2}\floor{\sum^n_{k=1} w^k}$.
Para cada tipo e valor de $n$ dez instâncias foram geradas.


\missingt{Descrever mais as tabelas e resultados.}

\missingt{
Dizer que a seleção dos parametros para o SCE sãos os recomendados pelo autor.\\
Dizer também que varição dos parametros não produzir melhorias consistentes.
}