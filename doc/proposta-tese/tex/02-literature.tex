Several exact approaches have been proposed in the literature for
solving the MOKP.
Examples of such approaches are
the theoretical work
on~\cite{klamroth2000dynamic} where several
dynamic programming formulations are presented,
a $\varepsilon$-constraint method presented in \cite{chankong2008multiobjective},
the two-phase method including a branch and bound algorithm proposed
in~\cite{visee1998two}, a labeling method presented
in~\cite{captivo2003solving} and the dynamic programming
algorithm proposed in~\cite{bazgan2009} which is the most
efficient exact algorithm according to literature.
Some later contributions for the dynamic programming approach
are an algorithmic improvement for bi-objective
cases~\cite{figueira2013algorithmic},
techniques for reducing its memory usage~\cite{correia2018} and
a multi-dimensional solution indexing strategy for performance
improvement~\cite{baroni2017}.

One of the main difficulties on multi-objective optimization problems
is the large cardinality of the set of non-dominated (or \emph{efficient}) solutions,
which has motivated research to provide an approximation of the solution
set~\cite{bazgan2015approximate, vanderpooten2017covers}.
Indeed, it is well-known, in particular, that most multi-objective combinatorial
optimization problems are \emph{intractable}, in the sense that
the number of non-dominated points is exponential in the size of the instance
\cite{ehrgott2013multicriteria}, which motivates the use of heuristics.

Most of proposed multi-objective metaheuristic are adaptations
of metaheuristics originally proposed for single-objective cases.
Among these we may mention
a simulated annealing algorithm~\cite{czyzzak1998pareto},
a scatter search based method~\cite{da2006scatter,da2007integrating},
a tabu search-based method~\cite{gandibleux2000tabu},
%a cooperative swarm intelligence~\cite{zouache2018cooperative},
a quantum inspired artificial immune system~\cite{gao2014quantum},
a hybrid genetic algorithm~\cite{abdelaziz1999hybrid} and
an estimation of distribution method~\cite{martins2017hybrid}.

One of the most popular heuristic algorithm for the MOKP
is the nondominated sorting genetic algorithm (NSGA-II)
proposed in~\cite{deb2002fast}.
This algorithm uses sorts based on the dominance concept
to obtain the convergence towards the Pareto optimal set.
It also employs the crowding distance to maintain the diversity
of the solution set.

Another common heuristic algorithm is the Strength Pareto
Evolutionary Algorithm (SPEA-II), proposed in~\cite{zitzler2001spea2}.
SPEA-II is a genetic algorithm that, in contrast
to NSGA-II, is based on the use of a external population
called archive.
This external population contains a limited number of
non-dominated solutions during the optimization phase.
At any iteration, the new non-dominated solutions of the
population are compared to members of the archive
with respect to dominance.

Recently a cooperative swarm intelligence algorithm called
MOFPA has been proposed in~\cite{zouache2018cooperative},
combing a firefly algorithm and a particle swarm
optimization for cooperation on the exploration of the search space.
A transfer function is used for discretization of the solutions.
The results are compared with NSGA-II, SPEA-II and other 3
algorithms.
The experiments showed that this novel approach 
out-performed the other algorithms in terms of convergence
to optimal while preserving good diversity of solutions.