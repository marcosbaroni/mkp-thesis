To verify the efficiency of the proposed metaheuristic, several computational
experiments was executed considering the original SCE for the MKP
without using the problem reduction procedure -- as proposed in \cite{baroni2015shuffled} --
and the SCE with the reduction procedure.
For brevity the SCE with the reduction procedure will be referred to as \scecore.

For the experiments, two sets of MKP instances was used:
a first set composed by 270 instances provided by Chu and Beasley (\cite{Chu-Beasley-1998})
and a second set composed by 11 instances provided by Glover and Kochenberger in
\cite{glover1996critical}.
These instances are all available at \cite{ORLibrary}.

The best known solution for all instances were taken from the literature and
were found by different algorithms which we had no access to the implementation.
In mosts cases those are exact algorithms which took
minutes or hours of execution time.

The set of MKP instances provided by Chu and Beasley was generated using a
procedure suggested by Freville and Plateau~\cite{freville1994efficient}, which
attempts to generate instances hard to solve.
The number of constraints $m$ varies among $5$, $10$ and $30$, and the number
of variables $n$ varies among $100$, $250$ and $500$.

The $w_{ij}$ were integer numbers drawn from the discrete uniform distribution
$U(0, 1000)$.
The capacity coefficient $c_i$ were set using
$b_i = \alpha\sum_{j=1}^{n} w_{ij}$ where $\alpha$ is a tightness ratio and
varies among $0.25$, $0.5$ and $0.75$.
For each combination of $(m,n,\alpha)$ parameters, $10$ random problems was generated,
totaling $270$ problems.
The profit $p_j$ of the items were correlated to $w_{ij}$ and generated as follows:
\begin{equation}
  p_j = \sum_{i=1}^m \frac{w_{ij}}{m} + 500q_j \qquad j = 1, \ldots, n
\end{equation}

%\subsection{Experimental Results}

All the experiments were run on a Intel$^R$ Core i5-3570 CPU @3.40GHz computer
with 4GB of RAM.
The original SCE and \scecore was both implemented in C programming language.
%For the set of random instance all best known solution was found by the solver
%SCIP 3.0.1 running for at least 10 minutes.

For the variable fixing procedure used on \scecore, the range size of the approximate core was
$|C| = m+\frac{n}{10}$ for all instances.
In all instances the parameters used for SCE and \scecore were the same recommended
in \cite{baroni2015shuffled} which was found after a batch test using Chu and Beasley instances:
\begin{itemize}
  \item $N = 20$: number of complexes;
  \item $M = 20$: number of individuals in each complex;
  \item $P = 5$: number of individuals in each subcomplex;
  \item $K = 300$: number of algorithm iterations;
  \item $K' = 20$: number of iterations used in the complex evolving process;
  \item $c = n/5$: number of genes carried from parent in crossing process.
\end{itemize}

Table~\ref{tab:chu} shows the performance of the SCE and \scecore on the Chu-Beasley set of instances.
Each instance in the set was executed 10 times for each algorithm.
Columns \textit{n}, \textit{m} and \textit{$\alpha$} shows the parameters used
on each instance generation.
The \textit{time} column shows the average execution time of the algorithms (lower is better).
The \textit{quality} column shows the average ratio of the solution found and
the best known solution from literature (\cite{vimont2008reduced, della2012improved}) of each instance (higher is better).
Best values are in bold.

It can be observed that the \scecore had faster convergence speed, achieving higher
quality solutions in all cases, achieving at least $97.96\%$ of best known, in less than $2$ seconds
for every instance.

It can be also noticed that \scecore executed in much less processing time than original
SCE.
This is due the variable fixing procedure which reduced the problem size,
resulting in less genes operations during the evolving procedures.
The variable fixing procedure also brought robustness for the method, as the quality
of the solution found increased in case of larger instances while on original
SCE the quality decreased considerably.

Table~\ref{tab:gk} shows the performance of SCE and \scecore on the Glover-Kochenberger set of instances.
Each instance in the set was executed 10 times for each algorithm.
Columns \textit{n} and \textit{m} indicate the size of each instance.
The \textit{time} column shows the average execution (lower is better).
The \textit{quality} column shows the average ratio of the solution found and
the best known solution of each instance.
Best values are in bold.
It can be noticed that SCEcr achieved high quality solutions, at least $98.22\%$
of best known solution, spending small amount of processing time, compared
to the time taken to find the best known solutions.

\begin{table}[hb]
{
\renewcommand{\arraystretch}{1}%
\fontsize{10.5pt}{1em}\selectfont 
\begin{center}
  \begin{tabular}{rrr|cc|cc} \hline
  \multirow{2}{*}{\bf n} &
  \multirow{2}{*}{\bf m} &
  \multirow{2}{*}{\textbf{$\alpha$}} &
    \multicolumn{2}{c|}{\textbf{time} (s)} &
    \multicolumn{2}{c}{\textbf{quality}(\%)} \\
  &
    &
    &
    \textbf{SCE} &
    \textbf{\scecore} &
    \textbf{SCE} &
    {\bf \scecore}  \\ \hline
100   &  5 & 0.25 & 1.22\fvar{0.04} & \textbf{0.17}\fvar{0.00} & 96.51\fvar{0.92} & \textbf{99.73}\fvar{0.04} \\
        &    & 0.50 & 1.34\fvar{0.02} & \textbf{0.18}\fvar{0.00} & 97.42\fvar{0.55} & \textbf{99.86}\fvar{0.01} \\
        &    & 0.75 & 1.37\fvar{0.03} & \textbf{0.17}\fvar{0.00} & 98.87\fvar{0.20} & \textbf{99.91}\fvar{0.00} \\ \cline{2-7}
        & 10 & 0.25 & 1.32\fvar{0.04} & \textbf{0.25}\fvar{0.00} & 95.68\fvar{1.28} & \textbf{99.53}\fvar{0.09} \\
        &    & 0.50 & 1.51\fvar{0.04} & \textbf{0.25}\fvar{0.00} & 96.65\fvar{0.49} & \textbf{99.76}\fvar{0.03} \\
        &    & 0.75 & 1.46\fvar{0.04} & \textbf{0.27}\fvar{0.00} & 98.54\fvar{0.19} & \textbf{99.96}\fvar{0.00} \\ \cline{2-7}
        & 30 & 0.25 & 1.74\fvar{0.06} & \textbf{1.20}\fvar{0.03} & 95.38\fvar{1.01} & \textbf{97.96}\fvar{0.22} \\
        &    & 0.50 & 1.79\fvar{0.08} & \textbf{0.89}\fvar{0.06} & 96.41\fvar{0.63} & \textbf{99.18}\fvar{0.06} \\
        &    & 0.75 & 1.72\fvar{0.09} & \textbf{0.95}\fvar{0.04} & 98.18\fvar{0.33} & \textbf{99.52}\fvar{0.04} \\ \hline
\multirow{2}{*}{250} & 5 & 0.25 & 2.87\fvar{0.07} & \textbf{0.69}\fvar{0.01} & 93.22\fvar{0.64} & \textbf{99.86}\fvar{0.00} \\
        &    & 0.50 & 2.82\fvar{0.11} & \textbf{0.70}\fvar{0.01} & 94.88\fvar{0.21} & \textbf{99.94}\fvar{0.00} \\
        &    & 0.75 & 2.93\fvar{0.08} & \textbf{0.69}\fvar{0.01} & 97.57\fvar{0.10} & \textbf{99.96}\fvar{0.00} \\ \cline{2-7}
        & 10 & 0.25 & 3.08\fvar{0.09} & \textbf{0.87}\fvar{0.01} & 93.14\fvar{0.67} & \textbf{99.58}\fvar{0.01} \\
        &    & 0.50 & 3.03\fvar{0.09} & \textbf{0.79}\fvar{0.02} & 94.55\fvar{0.26} & \textbf{99.79}\fvar{0.00} \\
        &    & 0.75 & 3.12\fvar{0.09} & \textbf{0.84}\fvar{0.01} & 97.16\fvar{0.13} & \textbf{99.88}\fvar{0.00} \\ \cline{2-7}
        & 30 & 0.25 & 3.74\fvar{0.12} & \textbf{1.52}\fvar{0.04} & 93.10\fvar{0.74} & \textbf{98.42}\fvar{0.08} \\
        &    & 0.50 & 3.74\fvar{0.16} & \textbf{1.36}\fvar{0.06} & 94.20\fvar{0.30} & \textbf{99.33}\fvar{0.02} \\
        &    & 0.75 & 3.99\fvar{0.13} & \textbf{1.48}\fvar{0.04} & 96.64\fvar{0.14} & \textbf{99.59}\fvar{0.01} \\ \hline
\multirow{2}{*}{500} & 5 & 0.25 & 5.62\fvar{0.10} & \textbf{1.25}\fvar{0.02} & 91.37\fvar{0.50} & \textbf{99.77}\fvar{0.00} \\
        &    & 0.50 & 5.72\fvar{0.19} & \textbf{1.24}\fvar{0.01} & 93.39\fvar{0.27} & \textbf{99.88}\fvar{0.00} \\
        &    & 0.75 & 5.88\fvar{0.14} & \textbf{1.20}\fvar{0.02} & 96.42\fvar{0.06} & \textbf{99.92}\fvar{0.00} \\ \cline{2-7}
        & 10 & 0.25 & 5.97\fvar{0.17} & \textbf{1.41}\fvar{0.02} & 91.62\fvar{0.50} & \textbf{99.51}\fvar{0.01} \\
        &    & 0.50 & 6.11\fvar{0.23} & \textbf{1.36}\fvar{0.03} & 93.09\fvar{0.20} & \textbf{99.77}\fvar{0.00} \\
        &    & 0.75 & 5.47\fvar{0.61} & \textbf{1.21}\fvar{0.03} & 96.24\fvar{0.06} & \textbf{99.84}\fvar{0.00} \\ \cline{2-7}
        & 30 & 0.25 & 6.20\fvar{1.14} & \textbf{1.96}\fvar{0.22} & 91.37\fvar{0.82} & \textbf{98.76}\fvar{0.02} \\
        &    & 0.50 & 6.26\fvar{1.07} & \textbf{1.82}\fvar{0.14} & 92.56\fvar{0.13} & \textbf{99.42}\fvar{0.01} \\
        &    & 0.75 & 6.05\fvar{1.16} & \textbf{1.73}\fvar{0.20} & 95.97\fvar{0.06} & \textbf{99.67}\fvar{0.00} \\ \hline
\end{tabular}

\end{center}
}
 \caption{SCE and \scecore  performance on Chu-Beasley problems.}
 \label{tab:chu}
\end{table}

\begin{table}[hb]
{
\renewcommand{\arraystretch}{1}%
\fontsize{10.5pt}{1em}\selectfont 
\begin{center}
  \begin{tabular}{rrr|cc|rr} \hline
  \multirow{2}{*}{\textbf{\#}} &
  \multirow{2}{*}{\textbf{n}} &
  \multirow{2}{*}{\textbf{m}} &
    \multicolumn{2}{c|}{\textbf{time}(s)} &
    \multicolumn{2}{c}{\textbf{quality}(\%)} \\
  &
    &
    &
    \textbf{SCE} &
    \textbf{SCEcr} &
    \textbf{SCE} &
    \textbf{SCEcr} \\ \hline
  01   &  100 &  15 &   1.47\fvar{0.00} & \textbf{0.08}\fvar{0.0} & 97.66\fvar{0.03} & \textbf{99.24}\fvar{0.02} \\ \hline
    02 &  100 &  25 &   1.61\fvar{0.00} & \textbf{0.09}\fvar{0.0} & 97.94\fvar{0.04} & \textbf{98.94}\fvar{0.09} \\ \hline
    03 &  150 &  25 &   2.51\fvar{0.01} & \textbf{0.09}\fvar{0.0} & 97.22\fvar{0.04} & \textbf{99.09}\fvar{0.02} \\ \hline
    04 &  150 &  50 &   3.56\fvar{0.03} & \textbf{0.09}\fvar{0.0} & 97.40\fvar{0.04} & \textbf{98.52}\fvar{0.02} \\ \hline
    05 &  200 &  25 &   3.55\fvar{0.01} & \textbf{0.09}\fvar{0.0} & 96.88\fvar{0.03} & \textbf{99.28}\fvar{0.01} \\ \hline
    06 &  200 &  50 &   4.81\fvar{0.09} & \textbf{0.10}\fvar{0.0} & 97.68\fvar{0.02} & \textbf{98.90}\fvar{0.03} \\ \hline
    07 &  500 &  25 &   7.30\fvar{0.09} & \textbf{0.10}\fvar{0.0} & 97.12\fvar{0.01} & \textbf{99.54}\fvar{0.00} \\ \hline
    08 &  500 &  50 &  12.20\fvar{0.47} & \textbf{0.11}\fvar{0.0} & 97.27\fvar{0.01} & \textbf{99.33}\fvar{0.01} \\ \hline
    09 & 1500 &  25 &  24.61\fvar{1.73} & \textbf{0.12}\fvar{0.0} & 95.40\fvar{0.01} & \textbf{98.22}\fvar{0.00} \\ \hline
    10 & 1500 &  50 &  33.79\fvar{2.44} & \textbf{0.13}\fvar{0.0} & 97.50\fvar{0.00} & \textbf{99.64}\fvar{0.00} \\ \hline
    11 & 2500 & 100 & 121.28\fvar{194.74} & \textbf{0.15}\fvar{0.0} & 97.95\fvar{0.00} & \textbf{99.70}\fvar{0.00} \\ \hline
\end{tabular}

\end{center}
}
 \caption{\scecore performance on Glover-Kochenberger problems.}
 \label{tab:gk}
\end{table}