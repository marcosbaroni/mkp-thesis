In opposition to single objective optimization,
multi-objective problems
involve optimizing multiple criteria that are usually conflicting.
These problems has typically no optimal solution, i.e.,
one that is best for all the objectives, but the solutions of interest
-- called efficient solutions -- are those such
that any solution which is better on one
objective is necessarily worse on at least one other objective.
These set of efficient solutions is called \emph{Pareto optimal set}.

One of the most important multi-objective problem is the
multi-dimensional knapsack problem (MOKP).
Many real applications like
project selection~\cite{teng1996multiobjective},
capital budgeting~\cite{rosenblatt1989generating},
cargo loading~\cite{teng1996multiobjective}
and low shop scheduling~\cite{ishibuchi2015behavior}
can be modeled as MOKP.
The MOKP is considered a \nphard{} problem since it is a generalization
of the well-known $0-1$ knapsack problem.
This work proposes the development of a hybrid heuristic algorithm
for solving the MOKP.

\section{Motivation}

Several exact approaches have been proposed in the literature for
solving the MOKP.
However, no exact method has proved to be effective
for large multi-objective problems with more than two objectives.
Even for the bi-objective case, some medium sized instances
has shown to be hard to solve exactly.
This reason has motivated the development of heuristic methods,
which allow the approximations of the
Pareto optimal solutions in a reasonable computational time.

%One of the main difficulties on multi-objective optimization problems
%is the large cardinality of the set of non-dominated (or \emph{efficient})
%solutions.
%Even for the bi-objective cases admit families
%of instances for which the number of efficient solutions is
%exponential in the size of the instance~\cite{ehrgott2013multicriteria}.

The heuristic proposed in this work will be based on an
evolutionary algorithm called shuffled complex evolution (SCE)
which combines the ideas of a controlled random search with the concepts
of competitive evolution and shuffling.
Several applications of the SCE have been proposed for successfully
solving optimization problems, among then,
the multi-dimensional knapsack problem (MKP), which is
also a contribution of this work.
As a performance improvement for the approach,
a multi-dimensional indexing strategy
will be used for handling the large amount of solutions.
This indexing strategy  has already presented
considerable performance improvement
when applied to an exact algorithm for the MOKP~\cite{baroni2017}
and is a contribution of this work as well.
%The development of the hybrid SCE will be discussed
%on Chapter~\ref{cap:scemokp}.

\section{Contributions and Publications}

The thesis consists of the following contributions:
\begin{enumerate}
\item{
\textbf{Efficient indexing strategy for MOKP solutions:}
a multi-dimensional indexing strategy for handling large amount
of solutions for the MOKP.
Several computational experiments show the applicability and efficiency
of the strategy on a dynamic programming algorithm for exactly solving
the problem.
The approach considerably reduces the number of solution
comparisons which resulted in an algorithm speedup of $2.3$ for 
bi-dimensional cases and up to $15.5$ for 3-dimensional cases.
Publication:
\begin{itemize}
 \item[{\tiny$\bullet$}] { BARONI, M. D. V; VAREJ\~AO, F. M. Multi-dimensional indexing on dynamic programming for multi-objective knapsack problem. \textit{International Transactions in Operational Research}, Wiley Online Library, 2017, Submitted. }
\end{itemize}
}
\item{
\textbf{A SCE algorithm for the MKP:}
a shuffled complex evolution algorithm using a
problem reduction technique for heuristically solving the multi-dimensional
knapsack problem.
The approach demanded small amount of time (less than 2 seconds)
for solving the hardest instances from literature,
although achieving an average solution quality of $99.0\%$
of the best known solution.
Publications:
\begin{itemize}
  \item[{\tiny$\bullet$}] { BARONI, M. D. V; VAREJ\~AO, F. M. A shuffled complex evolution algorithm
  for the multidimensional knapsack problem. In: \textit{Progress in Pattern Recognition, Image Analysis, Computer Vision, and Applications.} [S.l.]: Springer, 2015. p. 768-775. }
 \item[{\tiny$\bullet$}] { BARONI, M. D. V; VAREJ\~AO, F. M. A shuffled complex evolution algorithm
  for the multidimensional knapsack problem using core concept. In: IEEE. \textit{Evolutionary Computation (CEC), 2016 IEEE Congress on.} [S.l.], 2016 p. 2718-2723. }
\end{itemize}
}
\item{
\textbf{Hybrid heuristic for MOKP:}
a shuffled complex evolution algorithm for
heuristically solving the MOKP, which uses the multi-dimensional
indexing strategy for efficiently handling solutions.
Publication:
\begin{itemize}
 \item[{\tiny$\bullet$}] { BARONI, M. D. V; VAREJ\~AO, F. M. An efficient shuffled complex evolution algorithm for the multi-objective knapsack problem. In: IEEE \textit{Evolutionary Computation (CEC), 2018 IEEE Congress on.} [S.l.], 2018, In Preparation }
\end{itemize}
}
\end{enumerate}

\nocite{baroni2015shuffled}
\nocite{baroni2016shuffled}
\nocite{baroni2017}
\nocite{baroni2018}

\newpage
\section{Structure}
This work is structured as following.
Chapter~\ref{cap:mokp} presents the multi-objective knapsack problem
and briefly describes its existing literature.
Chapter~\ref{cap:kdtree} presents the multi-dimensional
indexing for MOKP solutions and its use case application
on a state of art exact algorithm.
Chapter~\ref{cap:sce} introduces the SCE 
meta-heuristic and its use case for solving
the multi-dimensional knapsack problem.
Chapter~\ref{cap:scemokp} presents the proposal for the development
of a SCE hybrid algorithm for heuristically the MOKP.
