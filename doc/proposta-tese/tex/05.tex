As exposed in previous chapters,
exact methods have difficulties solving several MOKP instances,
especially with more than 2 objectives.
This motivates the development of heuristic methods.
This chapter will discuss the development of
the proposed heuristic algorithm, how tests will be
conducted and analyzed.
A final schedule is presented in the last section.

\section{Algorithm Proposal}
For designing a multi-objective heuristic, one still take into account
two important criteria: (a) how to allow the diversity of solutions
and (b) how to ensure the convergence towards the Pareto optimal set.
The SCE algorithm (a) allows diversity of solutions by evolving independent
sets of solutions which are constructed by shuffling the 
population and (b) ensures convergence by prioritizing
the crossing of individuals with better fitness, which
indicates the potential to compute a good approximate Pareto optimal set.

As seen in previous sections
the SCE is easily applied to any optimization problem.
The fact that the MKP shares the same solution representation
as the MOKP
-- set of items included in the knapsack --
allow us to use the same procedures
presented in Section~\ref{sec:scemkp}
for (a) creating a new random solution
and (b) crossing two solutions.

A crucial point for the successful application of a metaheuristic
on a multi-objective problem is the definition of the fitness function.
One of the most popular multi-objective heuristic algorithms successfully uses sorts
based on the dominance relation for measuring the quality of a solution.
The method is called \emph{non-dominated sort}~\cite{deb2002fast}:
given a set $Q$ of solution, it first selects those which are not dominated
by any other solution in $Q$ to compose the set $S_1$, denoted as
\emph{first non-dominated front}.
Now, all select solutions are removed from $Q$ set and the process
is repeated to make set $S_2$.
The process continues until all fronts are identified.
Solutions in the first fronts are considered the ones with the best fitness.
For tie breaking the aggregation method can be used, which consist of using
a weight vector to compute a scalar over the multiple objective values
of a solution.

A performance issue on the non-dominated sort procedure is
checking if a solutions is dominated by another.
Our propose is to minimize this issue with the assistance
of the indexing strategy proposed in Chapter~\ref{cap:kdtree}.

\section{Computational Experiments and Analyzes}
The proposed algorithm will be tested on the same types
of instances presented  in Chapter~\ref{sec:dynprogcomp},
this time considering 2, 3 and 4 objective cases
and compared with the results that will be obtained from
the MOFPA proposed in~\cite{zouache2018cooperative}.

Two performance metrics will be used to evaluate the
approximation quality of the generated solutions:
(a) set coverage metric~\cite{zitzler1998multiobjective}
and (b) spacing metric~\cite{schott1995fault}:
\begin{itemize}
  \item[(a)]{Set coverage metric ($SC$):
  This metric is intended to be used for comparing
    two non-dominated Pareto sets.
    It defines a semi-distance between two solutions set $A$ and $B$ by:
    \begin{displaymath}
      SC(A, B) = \frac{\big|\{ b \in B
        \;|\; \exists a \in A : dom(a, b)\}\big|}{|B|}
    \end{displaymath}
    This quantity corresponds to the proportion of the solutions
    of the set $B$, that are dominated by at least one solution
    of $A$. It should be noted that $C(A,B)$ is not necessarily
    equal to $1-C(B, A)$. To compare $A$ and $B$ sets, both values
    of  $C(A,B)$ and $C(B,A)$ must be calculated.}
  \item[(b)]{Spacing metric ($SP$): This metric evaluates the uniformity
    of the distribution of non-dominated solutions:
    \begin{displaymath}
      SP(A) = \sqrt{\frac{1}{n-1} \sum^n_{i=1} (\bar{d}-d_i)^2}
    \end{displaymath}
    where $d_i$ is defined as:
    \begin{displaymath}
     d_i = min \sum^M_{m=1} | f^i_m - f^k_m | \qquad k \in A, k \neq i
    \end{displaymath}
    and $\bar{d}$ denotes the average of the distances $d_i$ for
    $i = 1, \ldots, n$.
    $SP$ represents the standard deviation of distance values between
    two consecutive solutions of the set $A$.
    The smaller $SP$, the better distribution of the solutions.
    Moreover, a null value of SP indicates that the solutions are
    equidistant. }
\end{itemize}

The computational results will be reported on a paper to be submitted
to IEEE Congress on Evolutionary Computation 2018 (CEC 2018) that
will be held on July 80-14 in Rio de Janeiro, Brazil.
Its paper submission deadline is January 15.

\newpage
\section{Final Doctoral Schedule}
Table~\ref{tab:sched} presents the final doctoral schedule.

\begin{table}[hb]
\centering
\bgroup
\def\arraystretch{1.1}%
\begin{tabular}{|l|c:c:c|c:c:c:c|c:c:c:c|}
 \hline
  \multirow{3}{*}{\textbf{ \phantom{aaaa} Activities}}
  & \multicolumn{11}{c|}{\textbf{Weeks}} \\ \cline{2-12}
   & \multicolumn{3}{c|}{\textbf{ November }}
   & \multicolumn{4}{c|}{\textbf{December}}
   & \multicolumn{4}{c|}{\textbf{January}} \\ \cline{2-12}
  & \;2º\; & 3º & 4º & 1º & 2º & 3º & 4º & 1º & 2º & 3º & 4º \\ \hline
 Literature review
  & $\bbt$ & $\bbt$ & $\bbt$ & & & & & & & & \\ \hline
 Implementation and adjustments
  & $\bbt$ & $\bbt$ & $\bbt$ & $\bbt$ & & & & & & & \\ \hline
 Computational experiments
  & & & $\bbt$ & $\bbt$ & $\bbt$ & & & & & & \\ \hline
 CEC Paper writing
  & & & & & & $\bbt$ & $\bbt$ & $\bbt$ & & & \\ \hline
 CEC Paper submission
  & & & & & & & & & $\bbt$ & & \\ \hline
 Thesis writing
  & & & & $\bbt$ & $\bbt$ & $\bbt$ & $\bbt$ & $\bbt$ & $\bbt$ & $\bbt$ & \\ \hline
\end{tabular}
\egroup
\caption{Final doctoral schedule.}
\label{tab:sched}
\end{table}

%\vspace{15pt}
%\emph{Métricas de avaliação de solucao, instancias de testes e analises.}
