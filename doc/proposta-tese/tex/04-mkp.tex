The multi-dimensional knapsack problem (MKP) is a strongly NP-hard combinatorial
optimization problem which can be viewed as a resource allocation problem and
defined as follows:
\begin{align}
  \text{maximize} & \sum_{j=1}^n p_j x_j \\
  \text{subject to} & \sum_{j=1}^n w_{ij} x_j \leqslant c_i \quad i \in \{1, \ldots, m\}\\
   & x_j \in \{0, 1\}, \quad j \in \{1, \ldots, n\}.
\end{align}

% Define the MKP
The problem can be interpreted as a set of $n$ items with profits $p_j$
and a set of $m$ resources with capacities $c_i$.
Each item $j$ consumes an amount $w_{ij}$ from each resource $i$, if selected.
The objective is to select a subset of items with maximum total profit,
not exceeding the defined resource capacities.
The decision variable $x_j$ indicates if $j$-th item is selected.
It is considered an integer programming problem (IP) since its variables $x_i$
are restricted to be integers.

The multi-dimensional knapsack problem can be applied on budget planning 
scenarios and project selections~\cite{mcmillan1973resource},
cutting stock problems~\cite{Gilmore-Gomory-1966}, loading problems~\cite{Shih-1979},
allocation of processors and databases in distributed computer programs~\cite{Gavish-Pirckul-1982}.

The problem is a generalization of the well-known knapsack problem (KP) in which
$m = 1$.
However it is a NP-hard problem significantly harder to solve in practice than the KP.
%Despite the existence of a fully polynomial approximation scheme (FPAS) for the KP,
%finding a FPAS for the MKP is NP-hard for $m \geqslant 2$~\cite{magazine1984note}.
Due its simple definition but challenging difficulty of solving, the MKP is often used to
to verify the efficiency of novel metaheuristics.

Its well known that the hardness of a NP-hard problem grows exponentially over
its size.
Thereupon, a suitable approach for tackling NP-hard problems is to reduce their size
through some variable fixing procedure.
Despite not guaranteeing optimality of the solution, an efficient variable
fixing procedure may provide near optimal solutions through a small computational effort.

%A metaheuristic is a set of concepts that can be used to define heuristic methods
%that can be applied to a wide set of different problems.
%In other words, a metaheuristic can be seen as a general algorithmic framework which can be applied to
%different optimization problems with relatively few modifications to make them adapted to a specific problem.”
As it can be noted in its description the SCE is easily applied to any
optimization problem.
The only steps needed to be specified is (a) the creation of a new random
solution and (b) the crossing procedure of two solutions.
The procedures presented in this section are based
on the work in \cite{baroni2015shuffled}.
These two procedures for the MKP are respectively presented by algorithms on
Figure~\ref{alg:new} and Figure~\ref{alg:cross}.

To construct a new random solution (Figure~\ref{alg:new}) the items are
at first shuffled in random order and stored in a list (line 2).
A new empty solution is then defined (line 3).
The algorithm iteratively tries to fill the solution's knapsack with 
the an item taken from the list (lines 4-9).
The feasibility of the solution is then checked: if the item insertion let
the solution unfeasible (line 6) its removed from knapsack (line 7).
After trying to place all available items the new solution is returned.

\begin{figure}
\begin{algorithmic}[1]
  \Procedure{Crossing}{$x^w:$ worst individual, $x^b:$ better individual, $c$}
    \State $v \leftarrow $ shuffle($1, 2, \ldots, n$)
    \For{$ i \leftarrow 1:c$ }
	  \State $j \leftarrow v_i$
	  \State $x^w_j \leftarrow x^b_j$ \Comment{gene carriage}
	\EndFor
	\If{$s^w$ is not feasible}
	  \State repair $s^w$
	\EndIf
	\State update $s^w$ fitness
  \State return $s^w$
  \EndProcedure
\end{algorithmic}
\caption{Crossing procedure used on SCE algorithm.}
\label{alg:cross}
\end{figure}

The crossing procedure (Figure~\ref{alg:cross}) takes as input the worst
solution taken from the subcomplex $x^w = (x^w_1, x^w_2, \ldots, x^w_n$),
the selected better solution $x_b = (x^b_1, x^b_2, \ldots, x^b_n$)
and the number $c$ of genes that will be carried from the better solution.
The $c$ parameter will control how similar the worst individual will be from the
given better individual.
At first the items are shuffled in random order and stored in a list (line 2).
Then $c$ randomly chosen genes are carried from the better individual to the worst
individual (line 5).
At the end of steps the feasibility of the solution is checked (line 7) and
the solution is repaired if needed.
The repair stage is a greedy procedure that iteratively removes the item that less
decreases the objective function.
Finally the fitness of the generated solution is updated (line 10) and
returned (line 11).

The following section will present the core concept for the MKP.
The concept will be used as a problem reduction
technique that will assist the SCE method as a pre-processing
step on the algorithm.
Its application on the SCE algorithm for the MKP is
also a contribution of this work.