\documentclass[brazil]{article}
\usepackage[utf8]{inputenc}
\usepackage[portuges,brazilian]{babel}
\usepackage[T1]{fontenc}
\usepackage{cite} % para quebrar citações que estrapolam a margem

\usepackage[dvipdf]{graphicx}
\usepackage{amsmath}
\usepackage[breaklinks,colorlinks=true,linkcolor=black,citecolor=black,urlcolor=black]{hyperref}

\usepackage[portuges]{babel}

\usepackage[ruled,longend,lined,linesnumbered]{algorithm2e}

%\renewcommand{\rmdefault}{put} % Utopia Font
%\renewcommand{\sfdefault}{put} % Utopia Font

\newcommand{\bigbullet}{\,\begin{picture}(-1,1)(-1,-2)\circle*{5}\end{picture}\ }

% Escrita dos algoritmos
\SetKwInput{KwIn}{Entrada}
\SetKwInput{KwOut}{Saída}
\SetKwFor{While}{enquanto}{faça:}{fim}
\SetKwFor{For}{para}{faça:}{fim}
\SetKwBlock{Begin}{início}{fim}
\SetKwIF{If}{ElseIf}{Else}{se}{então}{senão se}{senão}{fim}
\SetKw{KwTo}{até}

\pagestyle{plain}

\begin{document}

\begin{titlepage}

\begin{center}

\LARGE{Marcos Daniel Valadão Baroni}

\vspace{40mm}

\LARGE{Algoritmo Híbrido \textit{Intelligent Water Drops} \\ para o Problema da Mochila Multidimensional} \\
\vspace{5mm}
\Large{Projeto para Curso de Doutorado em Ciência da Computação}

\vspace{65mm}

\normalsize{Vitória, \today} \\
%\normalsize{Vitória, 20 de maio de 2013} \\

%\vspace{25mm}
\vfill

\begin{small}
Programa de Pós-Graduação em Informática \\
Centro Tecnológico -- Universidade Federal do Espírito Santo \\
Av. Fernando Ferrari, 514, Goiabeiras - Vitória - ES \\
\end{small}

\end{center}

\end{titlepage}

\newpage

\tableofcontents{}

\section{Introdução}

A otimização combinatória é o ramo da computação que
trata de pro\-ble\-mas cujo objetivo é encontrar um elemento ótimo
dentre um conjunto finito de elementos~\cite{Schrijver2003}.
Nos últimos anos o estudo da otimização combinatória tem crescido
de\-vi\-do ao aparecimento de complexos problemas deste caráter em
várias áreas, como produção industrial e economia.

O \textit{problema da mochila multidimensional} (ou, em inglês, \textit{multidimensional knapsack problem}
ou \textit{MKP}) é um problema de otimização combina\-tó\-ria que pode ser aplicado a
vários problemas práticos, como por exemplo, carregamento de carga~\cite{Shih-1979},
corte de estoque~\cite{Gilmore-Gomory-1966}, selecão de projetos~\cite{Li-Chiu-Cox-1999}
e alocação de processadores
em sistemas distribuídos~\cite{Gavish-Pirckul-1982}.

Por ser de grande relevância e difícil resolução, o MKP tem
sido bastante estudado.
Este, pode ser definido pelo problema de otimização seguinte~\cite{Puchinger2010}:
\begin{align}
  \textrm{maximizar}
    \quad & z = \sum_{j=1}^n p_j x_j \\
  \textrm{sujeito a}
  	\quad & \sum_{j=1}^n w_{ij}x_j \leq c_i, \quad i=1,\cdots,m, \\
          & x_j \in \{0,1\}, \quad j=1,\cdots,n.
\end{align}
O problema define $n$ elementos e $m$ recursos, cada um com capacidade $c_i > 0$.
Para a seleção de um elemento $j$ há um lucro $p_j$, mas também um gasto $w_{ij}$ do recurso $i$.
O objetivo é selecionar um subconjunto de elementos que totalize o maior lucro,
não ultrapassando as capacidades dos recursos.

No caso em que $m = 1$ o problema é chamado simplesmente de \textit{problema
da mochila} (knapsack problem).
Sabe-se que o problema está na classe \textit{NP-completo}, por ser um dos
21 problemas NP-completos apresentados por Richard Karp em 1972~\cite{Karp-1972, Garey1979}.

Com o crescimento em larga escala dos problemas de otimização combinatória,
métodos simples utilizados até então para a sua resolução
tornaram-se inviáveis, devido à complexidade, exigindo assim o desenvolvimento de métodos mais e\-fi\-ci\-en\-tes.

A utilização de métodos exatos para problemas NP-completos garantem otimalidade
da solução, porém tornam-se inviáveis, na maioria das vezes, mesmo para instâncias de tamanho médio,
devido ao tempo exponencial necessário para a execução dos métodos.
Algoritmos de aproximação são capazes de dar garantia na qualidade da solução,
demandando um menor tempo de execução, porém, são geralmente algoritmos difíceis
de serem formulados e a garantia de solução dada por vezes não é satisfatória.

Por estes motivos cada vez mais têm-se recorrido às \textit{heurísticas}: métodos não exatos que
demandam um menor tempo de execução.
Apesar de não proverem garantia na qualidade de solução, a utilização de heurísticas,
após os devidos ajustes, tem apresentado em geral resultados satisfatórios.

Além das heurísticas desenvolvidas expecificamente para um determinado problema
existem ainda as chamadas \textit{metaheurísticas} que são heurísticas com
um caráter genérico.
Este caráter genérico possibilita a sua aplicação em vários tipos de problemas,
geralmente através de poucas adaptações~\cite{Dorigo2013}.
Podem possuir ainda variações em suas aplicações ou mesmo serem aplicadas
em conjunto com uma outra heurística, gerando um método híbrido, mais efetivo
por aliarem as qualidades presentes em cada uma delas~\cite{Blum2003}.

Recentente foi proposta por Hamed Shah-Hosseini uma metaheurística cha\-ma\-da
\textit{intelligent water drops (IWD)} que baseia-se no comportamento das
águas ao fluírem através de um rio~\cite{Hosseini-2007}.
Cada porção de água percorre um caminho no espaço que depende, não só do
ambiente ao redor, mas também do tamanho e velocidade da porção d'água.
Ao passar por um caminho, a porção de água modifica o ambiente, podendo
ganhar velocidade e tamanho ao carregarem um pouco do solo.
Tem-se no final do processo um caminho traçado da nascente ao ponto em que o rio desagua.

A proposta desta nova metaheurística foi seguida pela sua aplicação ao MKP,
feita pelo mesmo autor~\cite{Hosseini-2008}, cujo experimento alcançou soluções
bem próximas ao ótimo e bem competitivas quando comparadas à ou\-tras metaheurísticas
aplicadas ao MKP por trabalhos conhecidos da literatura.

Devido a sua importância e fácil modelagem o MKP têm sido utilizado ultimamente
como um \textit{problema teste} na avaliação de novas metaheurísticas.
O objetivo principal desta proposta é desenvolver uma heurística híbrida
para a resolução do MKP, aliando a metaheurística IWD a um \textit{algoritmo
de estimação de distribuição}.

Um algoritmo de estimação de distribuição (AED) é um algoritmo probabilístico
que propõe encontrar boas soluções para problemas de otimização, com o auxílio
de um modelo probabilístico, proposto a partir de uma amostra de soluções.
Detalhes da utilização do AED à metaheurística IWD são apresentados nas próximas seções.

%\textit{algoritmo genético},
%\textit{colônia de formigas}~\cite{Dorigo-1991} e \textit{ar\-re\-fe\-ci\-men\-to simulado}.
%Pretendemos analisar e comparar a aplicação de diversas metaheurísticas
%ao MKP realizando ajustes, propondo melhorias e com\-pa\-ran\-do-as também aos outros métodos
%de resolução existentes.

Também pretende-se aplicar a heurística desenvolvida aqui a um problema real da
indústria de distribução de energia elétrica brasileira.

Na Seção seguinte apresentaremos os objetivos deste trabalho.
Na Seção~\ref{sec:justificativa} apresentaremos a justificativa para a pesquisa.
Na Seção~\ref{sec:metodologia} trataremos da metodologia que será utilizada no trabalho.
Na Seção~\ref{sec:cronograma} apresentaremos o cronograma do projeto.

\section{Objetivos}
\label{sec:objetivos}

%\subsection{Objetivos Gerais}

Esta proposta tem como objetivo o aperfeiçoamento da metaheurística IWD
aplicado ao MKP, através da utilização de um algoritmo de estimação de distribuição,
gerando uma heurística híbrida para o problema em questão.
A heurística será implementada e ajustes e melhorias serão feitas.
Testes computacionais seguirão a implementação com o objetivo de comparar sua
eficácia na resolução do MKP à outros métodos já existentes.

Os objetivos específicos do projeto são os seguintes:

\begin{itemize}
  \item{Realizar ampla revisão bibliográfica sobre o problema tratado (MKP) e os algoritmos envolvidos (IWD e AED);}
  \item{Implementar a heurística híbrida à resolução do problema, realizando os devidos ajustes;}
  \item{Propor melhorias à heurística implementada;}
  \item{Realizar comparações entre a heurística implementada e os métodos de resolução já existentes através de testes computacionais;}
  \item{Aplicar a heurística a um problema real da indústria de distribuição de e\-ner\-gia elétrica brasileira e comparar com os resultados obtidos pelo método já existente.}
\end{itemize}

\section{Justificativa}
\label{sec:justificativa}

\nocite{Martello-1990}

O MKP também pode ser interpretado como uma formulação geral para pro\-ble\-mas de
programação inteira 0-1 com coeficientes não ne\-ga\-ti\-vos, como é possível
concluir observando sua definição.
Dessa forma o MKP pode ser resolvido utilizando-se os métodos exatos já
conhecidos para tratar o problema de programação inteira 0-1, como o \textit{método
aditivo de Balas}~\cite{Balas-1965}.
Porém, devido a complexidade computacional dos métodos exatos, utilizá-los
torna-se inviável para instâncias do problema de tamanho médio,
justificando a utilização de heurísticas para sua resolução.

Grande parte das metaheurísticas são inspiradas em comportamentos na\-tu\-rais.
Dentre as mais conhecidas temos a \textit{busca tabu}~\cite{Glover-2003},
baseada em um comportamento social, a colônia de formigas, inspirada na
forma com que as formigas traçam o caminho da colônia até o alimento~\cite{Dorigo-1991},
o algoritmo genético, este inspirado na forma com que acontece a evolução genética
nos seres vivos~\cite{Holland-1975} e o ar\-re\-fe\-ci\-men\-to simulado, inspirado numa
téc\-ni\-ca de aquecimento e resfriamento controlado utilizada na produção de
alguns ma\-te\-ri\-ais~\cite{Kirkpatrick-1983}.

Em 1998 foi proposto por P. Chu e J. Beasley um algoritmo genético para o MKP~\cite{Chu-Beasley-1998}.
Neste algoritmo, operadores heurísticos utilizam também conhecimento específico
do problema para auxiliar no processo evolutivo.
Um algoritmo de arrefecimento simulado para o MKP é proposto em~\cite{Qian2007}.
Neste, testes computacionais mostraram que o algoritmo obteve considerável redução
no tempo de execução, não alcançando porém a mesma
qualidade de solução, quando comparado ao algoritmo genético proposto por P. Chu e J. Beasley.

Uma das abordagens que têm obtido me\-lho\-res resultados é a abordagem híbrida
proposta em~\cite{Vasquez2001} e aperfeiçoada em~\cite{Vasquez2005}.
Nessa abordagem o MKP tem o seu espaço de solução reduzido e particionado,
através da adição de algumas restrições, as quais fixam o número má\-xi\-mo
de itens a serem selecionados.
Limites para estas restrições são calculados, solucionando-se uma versão relaxada
do problema original.
Para cada parte do espaço de busca, uma busca tabu é aplicada de forma independente,
tendo como solução inicial a solução encontrada na versão relaxada do pro\-ble\-ma~\cite{Puchinger2010}.

Em 2002 foi proposto por Stefka Fidanova um algoritmo de colônia de formigas
para o MKP. Neste, várias informações heurísticas são utilizadas na cons\-tru\-ção
das soluções~\cite{Fidanova2002}.
Um outro algoritmo de colônia de formigas foi proposto em~\cite{Liangjun2008},
associado-o a um procedimento de busca local.
Já o trabalho apresentado em~\cite{Kong2008} propõe a resolução do MKP, utilizando
uma versão do algoritmo de colônia de formigas desenvolvida especialmente
para estruturas de soluções binárias.
Segundo o relato do trabalho, este apresentou melhores resultados que
os algoritmos de colônia de formigas convencionais aplicados ao MKP.

Em~\cite{Hanafi2010} um método híbrido é proposto para o MKP, utilizando
uma abordagem de programação matemática aliada a uma busca por decomposição
de vizinhança.
A cada iteração, uma versão relaxada do problema é solucionada para
guiar a busca por vizinhos.
Segundo os testes computacionais relatados no trabalho, os resultados
se comparam às heurísticas relacionadas até o momento como \textit{estado-da-arte}.

Beliczynski Bartlomiej propôs em 2007 uma implementação da metaheurística por 
\textit{enxame de partículas} para o MKP, alcançando resultados próximos e,
em alguns casos, iguais as melhores soluções conhecidas para as instâncias
testadas~\cite{Bartlomiej2007}.

Uma heurística híbrida, utilizando um algoritmo de estimação de distribuição
em conjunto com uma busca local já foi proposta para a versão multiobjetivo
do problema da mochila multidimensional~\cite{Li2004}.
Segundo o trabalho, experimentos computacionais indicaram que a heurística
híbrida proposta superou vários outros algoritmos, considerados até então
como \textit{estado-da-arte}.
Em~\cite{Larra2011} é apresentado, para fins didáticos, um algoritmo de
estimação de distribuição aplicado ao pro\-ble\-ma da mochila.

Como dito anteriormente o objetivo deste projeto é propor uma heurística híbrida
para a resolução do MKP, aliando a metaheurística \textit{intelligent water drops} a um
algoritmo de estimação de distribuição, esperando assim melhores resultados que os
observados em~\cite{Hosseini-2008}.

O algoritmo \textit{intelligent water drops} é uma metaheurística de caráter
po\-pu\-la\-cio\-nal, ou seja, a cada passo do algoritmo, um conjunto (ou população) de
soluções é gerado.
É também uma heurística construtiva por construir por si só
soluções completas, não necessitando de receber como entrada uma solução inicial.
O IWD é inspirado na forma com que porções de águas
traçam o seu caminho ao saírem de uma nascente até um ponto final onde desaguam.
Se não houvessem obstáculos no caminho até o seu destino, as porções de águas
em um rio fariam um caminho em linha reta, por ser o menor caminho de sua origem até o destino.
Porém, devido aos vários obstáculos existentes no ambiente,
o caminho real tomado pelas águas é diferente do ideal, possuindo assim voltas e torções.
É interessante observar que este caminho, de aparência tortuosa, parece ser o
menor quando considerados os obstáculos existentes no ambiente.

No algoritmo, o ambiente em que as gotas fluem é discreto, sendo modelado como um grafo, cujos
vértices representam as soluções parciais para o problema e cada aresta, um caminho
ligando duas soluções.
A cada iteração do algoritmo, cada gota d'água constrói uma solução, saindo de um
vértice inicial e chegando a um vértice final.

Cada aresta neste grafo possui uma quantidade de solo.
Ao passar pela aresta a gota d'água retira parte do solo, carregando-a agora consigo.
Cada gota d'água possui também uma velocidade: gotas d'água mais velozes
retiram uma maior quantidade de solo dos caminhos em que passam.
Simultaneamente, caminhos que possuem em si pouco solo acrescentam mais velocidade
às gotas que por ele passam.
Já caminhos que possuem muito solo acrescentam menos velocidade às gotas.
Ao longo do processo as gotas tendem a escolher caminhos mais fáceis, ou seja,
que possuam uma menor quantidade de solo.

Uma iteração do algoritmo termina quando todas as gotas d'água chegam numa solução completa.
A melhor solução construída na iteração é então selecionada para que, a partir dela,
a quantidade de solo nos caminhos sejam atualizadas.
O algoritmo se encerra quando alcança uma determinada quantidade de iterações.
A melhor solução encontrada é então dada como resposta.

Por sua vez, os algoritmos de estimação de distribuição (AED) são algoritmos pro\-ba\-bi\-lís\-ti\-cos
que trabalham gerando populações de soluções, baseando-se em um modelo
probabilístico estimado a partir de amostras de soluções selecionadas de
populações anteriores~\cite{bengoetxeaPHD02}. O algoritmo abaixo descreve um AED.

\begin{algorithm}[H]
%\setstretch{1.0}
%\trocafonte
\DontPrintSemicolon
	%\SetLine
	\caption{\bf O algoritmo de estimação de distribuição.}\label{powermetalg}
	\KwIn{ Número de iterações: $k$}
	\KwOut{ Solução: $s^*$}
	\Begin{
		$M_0 \leftarrow \textrm{distribuição\_uniforme}$ \;
		$D_0 \leftarrow \textrm{gera\_população}(M_0, R)$ \;
		$s^* \leftarrow \textrm{melhor\_solução}(D_0)$ \;
		\For{$i\leftarrow 1$ \KwTo $k$ }{
			$D^N_{i-1} \leftarrow \textrm{seleciona\_indivíduos}(D_{i-1}, N)$ \;
			$M_i \leftarrow \textrm{induz\_modelo}(D^N_{i-1})$ \;
			$D_i \leftarrow \textrm{gera\_população}(M_i, R)$ \;
			$s^* \leftarrow \textrm{atualiza\_melhor\_solução}(D_i)$ \;
		}
		retorna $s^*$ \;
	}
\label{alg:aed}
\end{algorithm}

Primeiramente o algoritmo gera uma população inicial contendo $R$ soluções (linha 3).
Esta geração inicial é geralmente feita assumindo uma distribuição uniforme
sobre as variáveis que compõem a solução do problema (linha 2).
Dentre esta população inicial, a melhor solução é selecionada como melhor solução global (linha 4).
Num segundo passo, para gerar uma nova população, são selecionadas da população
anterior $N$ soluções (sendo $N < R$) seguindo um determinado critério (linha 6).
No próximo passo, é induzido o modelo probabilístico que melhor representa as
interdependências entre as variáveis da solução (linha 7).
Os últimos passos são gerar uma nova população, contendo $R$ soluções, a partir do modelo
probabilístico estimado no passo anterior (linha 8), e atualizar a melhor solução global (linha 9).
Geralmente, nesta nova população a melhor solução é mantida, sendo necessária
a geração de apenas $R-1$ soluções.
O algoritmo então volta ao segundo passo até que o número máximo de iterações é
alcançado, dando como resposta a melhor solução encontrada (linha 11).

A indução do modelo probabilístico, considerado um procedimento de aprendizagem de máquina,
é o passo mais importante e crucial para o algoritmo, pois uma representação
apropriada de tais interdependências é essencial para a ge\-ra\-ção de boas soluções viáveis.
Assim, ao implementar um AED é importante decidir com cautela o modelo probabilístico a ser utilizado.
%como, por exemplo \textit{rede bayesiana} ou \textit{rede gaussiana}.

A vantagem do AED sobre outros métodos é registrar explicitamente as interdependências
entre as diferentes variáveis que compõem a solução do problema, através de um modelo
probabilístico, ao contrário da maioria das metaheurísticas, que que mantém
estas interdependências de forma implícita.
Este registro explícito tende a guiar o processo de busca de forma mais
\textit{inteligente}, levando a melhores soluções.

A proposta para o desenvolvimento da heurística híbrida para o MKP
é utilizar um AED para auxiliar a construção das soluções geradas
por cada gota d'água.
Após uma primeira iteração completa do IWD,
algumas melhores soluções são selecionadas da população gerada,
para que um modelo probabilístico seja proposto.
Na próxima iteração do IWD, cada gota d'água, ao escolher os caminhos a serem tomados
para a construção da solução, não só utilizaria informações do ambiente
(quantidade de solo nos caminhos) mas também informações do modelo probabilístico
proposto na iteração anterior.
Desta forma esperamos alcançar melhores soluções para o MKP.

%A busca tabu é uma metaheurística que parte
%de uma solução inicial e realiza uma busca local através da seleção de alguns movimentos.
%Após a busca a melhor solução é selecionada e os movimentos utilizados
%são tempo\-ra\-ria\-men\-te proibidos, evitando o confinamento próximo a um ótimo local.
%A lista dos movimentos proibidos é chama lista tabu.
%Em seguida novos movimentos são selecionados e uma nova busca local é executada e assim sucessivamente.
%Após algumas iterações os movimentos antes proibidos são permitidos novamente.
%A busca então continua até o fim das iterações.
%Seu nome faz referência a um comportamento observado no  âmbito social que se
%assemelha ao procedimento adotado pelo algoritmo.
%Na sociedade alguns conceitos considerados proibidos (\textit{tabus})
%tornam-se aceitáveis com o passar do tempo (\textit{quebra de tabu}).

%A metaheurística algotimo genético é inspirada
%no processo de evolução na\-tu\-ral, em que cada solução é considerada um \textit{indivíduo}.
%O algoritmo parte de um conjunto de soluções iniciais considerando-o uma \textit{população} inicial.
%Um novo conjunto de soluções é gerado através de combinações entre as soluções
%do conjunto corrente (\textit{cruzamentos}) e também pequenas mudanças
%nas soluções (\textit{mutações}).
%A cada iteração do algoritmo uma nova população é gerada como sendo uma
%\textit{evolução} da população anterior.
%Ao final das i\-te\-ra\-ções a melhor solução encontrada é selecionada.

%A metaheurística da colônia de formigas é inspirada no comportamento
%das formigas ao planejarem o menor caminho entre a colônia e o alimento.
%Inicialmente as formigas utilizam um caminho aleatório, deixam um rastro de feromônio,
%que funciona como um atrativo para outras formigas.
%Com o passar do tempo o feromônio evapora dos caminhos mais longos, nos quais
%há um menor trânsito de formigas, mas torna-se mais forte nos caminhos mais curtos.
%Por fim, atraídas pelo feromônio, as formigas tenderão a utilizar o caminho que
%possuir o rastro mais forte de feromônio, assumindo este ser o mais curto caminho encontrado.

%No algoritmo, cada formiga constrõe uma solução para o problema.
%Após a construção, as formigas depositam feromônio sobre os movimentos utilizados
%na construção da solução.
%A quantidade de feromônio depositada é baseada na qualidade da solução obtida.
%Após a primeira iteração as formigas passam a utilizar a quantidade de feromônio
%existentes nos movimentos para guiar a construção das soluções.
%Ao fim das iterações a melhor solução encontrada é selecionada.

%A metaheurística chamada arrefecimento simulado é inspirada numa téc\-ni\-ca de
%aquecimento e resfriamento controlado utilizada na produção de alguns materiais,
%visando a redução de irregularidades estruturais.
%A técnica consiste em aquecer o material a uma temperatura bem elevada,
%de forma que seus átomos gozem de livre movimento devido a elevada energia termodinâmica.
%Ao resfriar o material de forma lenta e controlada, os átomos do material
%aos poucos acomodam-se de forma organizada até chegar num estado de precisa
%organização, quando o material está totamlmente resfriado.

%A partir de uma solução inicial o algoritmo realiza uma busca iterativa por
%soluções vizinhas, baseada na ``temperatura'' corrente.
%A cada iteração o algoritmo decide adotar ou não uma solução vizinha como solução
%corrente com base na diferença entre a qualidade desta solução vizinha e a temperatura corrente.
%Nos estados iniciais do algoritmo, o sistema, por possuir alta temperatura, aceita
%eventualmente ``descer'' à uma solução bem pior que a corrente.
%Porém, à medida que as iterações vão sendo executadas, a temperatura do sistema
%diminui ficando cada vez mais difícil aceitar uma solução pior.
%Por fim o algoritmo responde como melhor solução a solução corrente.

%Dammeyer e Voss~\cite{Dammeyer-1993} apresentaram em 1993 uma busca tabu
%para o MKP, utilizando uma list tabu dinâmica.~\nocite{Chu-Beasley-1998}
%A implementação foi testada utilizando uma base padrão retirada da literatura.
%Das 57 configurações do problema, o método encontrou a melhor solução para 41 delas.
%Aboudi e Jörnsten~\cite{Aboudi-1994} em 1994 combinaram a busca tabu com um método
%heurístico proposto por Balas e Martin~\cite{Balas-1980} para o problema
%de programação 0-1, alcançando a melhor solução em 49 das 57 configurações do problema.
%Uma abordagem semelhante foi apresentada por 
%Lokketangen, Jörnsten and Storoy~\cite{Lokketangen-1994}, alcançando a melhor
%solução para 39 das instâncias.
%Glover e Kochenberger~\cite{Glover-1996} apresentaram em 1996 uma heurística
%baseada na busca tabu, utilizando um esquema de variação entre fases construtivas
%(solução corrente viável) e fases destrutivas (solução corrente inviável).
%O método alcançou a melhor solução em todos os 57 casos.
%
%Khuri, Back e Heitkotter~\cite{Khuri-1994} apresentaram em 1994 uma implementação do algoritmo
%genético para o MKP em que soluções inviáveis eram aceitas na busca.
%O trabalho divulgou apenas alguns resultados moderados.
%Thiel e Voss apresentaram~\cite{Thiel-1994} também em 1994 apresentaram um método híbrido
%entre algoritmo genético e uma heurística baseada em busca local, porém
%os resultados não foram computacionalmente competitivos.


%Resolução do problema, algoritmos exatos, aproximação, heuríticas, metaheurísticas.

%Tempo X Qualidade

Outra motivação para este trabalho é aplicar a solução encontrada a um problema
real da indústria de distribuição de energia elétrica brasileira.
O problema em questão é a otimização de investimentos em ações de
combate a perda de energia na distribuição de energia elétrica.
Neste problema deve-se maximizar a recuperação de energia em um cenário plurianual
de investimentos, no qual há uma disponibilidade de orçamento para cada ano do plano de investimentos.
Cada ação disponível possui um custo e produz uma determinada recuperação de energia.
Cada orçamento anual corresponde a uma mochila que deve ser preenchida com um conjunto de ações
de modo a maximizar a energia recuperada globalmente.

\section{Metodologia}
\label{sec:metodologia}

O início da pesquisa se dará com uma ampla revisão bibliográfica sobre o assunto.
Serão relacionados trabalhos abordando os principais métodos já propostos para a
resolução do problema, bem como trabalhos abordando os métodos utilizados (metaheurística IWD e AED).
Assim que alguns principais trabalhos forem relacionados, será iniciada
a implementação do método híbrido proposto e dos métodos pré-existentes.

Pequenas baterias de testes seguirão a implementação com o objetivo de
realizar os ajustes necessários.
Finalizada a implementação e ajustes do método proposto, os resultados
serão analisados para que melhorias sejam propostas.

Em seguida, serão reunidas as bases de testes utilizadas
na literatura, bem como instâncias reais do pro\-ble\-ma.
Os experimentos computacionais finais serão extensos, visando a me\-lhor observação
do comportamento da heurística proposta.
Para isso, novas bases de testes poderão ser geradas, caso seja necessário.
Os resultados obtidos no trabalho serão comparados aos resultados obtidos pelos métodos
pré-existentes e aos divulgados na literatura.
Uma análise estatística será feita sobre os resultados obtidos.

Assim que obtidos e analisados os resultados dos experimentos, estes serão
relatados em artigos submetidos a eventos e periódicos.
Os resultados parciais serão apresentados no Exame de Qualificação e na
Proposta de Doutorado.
A conclusão do trabalho será apresentada na Tese de Doutorado.

\section{Cronograma}
\label{sec:cronograma}

A Tabela~\ref{tab:cronograma} apresenta o cronograma do projeto de pesquisa.

\begin{table}[!h]
\begin{tabular}{l|c|c|c|c|c|c|c|c|}
  \cline{2-9}
  & \multicolumn{8}{|c|}{\textbf{Semestres}} \\
  \hline
  \multicolumn{1}{|c|}{\textbf{Atividades}}
    & 1º & 2º & 3º & 4º & 5º & 6º & 7º & 8º \\ \hline
  \multicolumn{1}{|l|}{Cumprimento de Créditos}
    & $\bigbullet$ & $\bigbullet$ & $\bigbullet$ & & & & & \\ \hline
  \multicolumn{1}{|l|}{Revisão Bibliográfica}
    & $\bigbullet$ & $\bigbullet$ & $\bigbullet$ & & & & & \\ \hline
  \multicolumn{1}{|l|}{Implementação e Ajustes}
    & $\bigbullet$ & $\bigbullet$ & $\bigbullet$ & $\bigbullet$ & $\bigbullet$ & $\bigbullet$ & & \\ \hline
  \multicolumn{1}{|l|}{Obtenção de Bases para Testes}
    & & $\bigbullet$ & $\bigbullet$ & $\bigbullet$ & & & & \\ \hline
  \multicolumn{1}{|l|}{Testes Computacionais}
    & & $\bigbullet$ & $\bigbullet$ & $\bigbullet$ & $\bigbullet$ & $\bigbullet$ & $\bigbullet$ & \\ \hline
  \multicolumn{1}{|l|}{Análise dos Resultados}
    & & & $\bigbullet$ & $\bigbullet$ & $\bigbullet$ & $\bigbullet$ & $\bigbullet$ & \\ \hline
  \multicolumn{1}{|l|}{Redação de Artigos}
    & & & & $\bigbullet$ & $\bigbullet$ & $\bigbullet$ & $\bigbullet$ & $\bigbullet$ \\ \hline
  \multicolumn{1}{|l|}{Exame de Qualificação}
    & & & & $\bigbullet$ & & & & \\ \hline
  \multicolumn{1}{|l|}{Proposta de Doutorado}
    & & & & & & $\bigbullet$ & & \\ \hline
  \multicolumn{1}{|l|}{Tese de Doutorado}
    & & & & & & & & $\bigbullet$ \\ \hline
\end{tabular}
\caption{Cronograma do projeto de pesquisa.}
\label{tab:cronograma}
\end{table}

\clearpage

{
\bibliographystyle{apalike}
\bibliography{proposta}
}

\end{document}

